
\section{The \openshmem Effort}

\openshmem is a specification that aims at consolidating the different
SHMEM versions provided by vendors, laboratories and universities 
into a single widely accepted specification. There are currently various vendor-specific SHMEM implementations available for
different platforms. These versions have subtle differences from one
another, and generally, code written using any one of these
implementations is not directly portable to the others. Additionally, \openshmem aims
to produce a portable specification enabling programmers to write
\openshmem code that can run on as many different platforms
as possible. To know more about the history of \openshmem, see the section on \hyperref[sec:openshmem_history]{History of \openshmem}.

The \openshmem specification is widely accepted by the community and is the de-facto
standard for SHMEM.  For vendors interested in implementing OpenSHMEM, there is an open source reference implementation
that can serve as a starting point to develop their own libraries or to experiment with extensions. The reference implementation is available at www.openshmem.org.

\rcomment{Manju: 1) We are introducing \openshmem{} without describing what it
is. 2) Second paragraph, IMO, this gives a impression that \openshmem{} effort is all about
consolodiation. Rather, it should be that only 1.0 and 1.1 have this aim. 
Future version is not aimed at consolodition. How does these two paragraphs 
look ?}

\openshmem is a \ac{PGAS} library interface specification. In the \ac{PGAS} model, each process has access to 
local private and globally shared memory, where portions of the shared memory may have affinity to a particular process. \openshmem
defines interfaces for defining partitioned global address space, and
interfaces for communication and synchronization on the address space. 


The \openshmem{} 1.0 and 1.1 are the first open source SHMEM specifications,
which are a consolidation of various SHMEM specifications. 
The main goal of these versions is to consolidate different SHMEM versions
provided by the vendors, laboratories, and universities into a single open
source specification. The different SHMEM specifications have a subtle
difference, as a conseuquence, SHMEM programs written using any one of these 
implementations is not necessarily portable to the others. 
More details on the history of the \openshmem are available in this
\hyperref[sec:openshmem_history]{section}.

The \openshmem{} 1.0 and 1.1 standardization efforts are driven by \ac{ORNL} and \ac{UH} 
with significant input from the \openshmem{} community. Besides the
specification, the effort included providing a reference implementation,
 and building infrastructure for the standardization process.
The specifications and the reference implementations are available here: 
www.openshmem.org.

\rcomment{Manju: Do we have to add a paragraph about the role of OSS ?} 
