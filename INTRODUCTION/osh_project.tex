\section{The \openshmem Effort}

%\openshmem is a SHMEM specification that aims to consolidate the different
%SHMEM implementations provided by vendors, laboratories and universities 
%into a single widely accepted specification. 
\openshmem is a \ac{PGAS} library interface specification. \openshmem aims to provide a standard \ac{API} for SHMEM libraries to aid portability and facilitate uniform predictable results of \openshmem applications by explicitly stating the behavior and semantics of the \openshmem library calls. Through the different versions, \openshmem will continue to address the requirements of the \ac{PGAS} community. 
As of this specification, existing vendors are moving towards \openshmem compliant implementations and new vendors are developing \openshmem library implementations to help the users write portable \openshmem code. This ensures that applications can run on multiple platforms without having to deal with subtle vendor-specific implementation differences. For more details on the history of 
\openshmem please refer to \hyperref[sec:openshmem_history]{The History of \openshmem} section.  

The \openshmem effort is driven by the \ac{ESSC} at \ac{ORNL} and the \ac{UH} 
with significant input from the \openshmem{} community. Besides the
specification, the effort also includes providing a reference \openshmem implementation,
validation and verification suites, tools, a mailing list and website infrastructure to support
specification activities. For more information please refer to: \url{http://www.openshmem.org}.
%  specifications and the reference implementations are available at: 
%\url{http://www.openshmem.org}. 

%\rcomment{Manju: 1) We are introducing \openshmem{} without describing what it
%is. 2) Second paragraph, IMO, this gives a impression that \openshmem{} effort is all about
%consolodiation. Rather, it should be that only 1.0 and 1.1 have this aim. 
%Future version is not aimed at consolodition. How does these two paragraphs 
%look ?}
%\rcomment{ \\ Oscar: 1) OpenSHMEM does not provide a global view of memory that is addressable by all PEs (like UPC, Chapel, where there is a PGAS address space). Instead
%it provides partioned-view of data, where symmetric data objects are accessible by all PEs. We need to avoid saying that OpenSHMEM provides a shared addressable space. 
%}
%
%\openshmem is a \ac{PGAS} library interface specification. In the \ac{PGAS} model, each process has access to 
%local private and globally shared memory, where portions of the shared memory may have affinity to a particular process. \openshmem
%defines interfaces for implementing a \ac{PGAS} program 111 access symmetric data objects ,  communication and synchronization on the address space. 
%
%\rcomment{Oscar: 2 OpenSHMEM I'm not sure if a spec can be defined as open source}
%\openshmem is the first community-driven SHMEM specification,
%which is a consolidation of various SHMEM specifications that addresses SHMEM community needs. 
%One of the goals of \openshmem is to consolidate different SHMEM versions
%provided by the vendors, laboratories, and universities into a single open
%source specification. In the past, different SHMEM implementations have a subtle
%differences, and as a consequence, SHMEM programs written using any one of these 
%implementations was portable across systems. 
%More details on the history of the \openshmem are available in this
%\hyperref[sec:openshmem_history]{section}.
%
%The \openshmem standardization effort is driven by \ac{ORNL} and \ac{UH} 
%with significant input from the \openshmem{} community. Besides the
%specification, the effort included providing a reference implementation,
% and building infrastructure for the standardization process.
%The specifications and the reference implementations are available here: 
%www.openshmem.org.
%
%\rcomment{Manju: Do we have to add a paragraph about the role of OSS ?}
%\rcomment{Probably not, up-to Steve} 
