\section{Language Bindings and Conformance}

%\openshmem is available with \Clang{} and \Fortran{} bindings.  The \Cpp{}
%interface is currently the same as that for \Clang.  An \openshmem implementation can be conformant to one or both of the
%interfaces.  An implementation that provides e.g.\ only a \Clang{} interface may claim to conform to the \openshmem specification with respect to
%the \Clang{} language, but not to \Fortran{} and should make this clear in its documentation.  An implementation that provides both \Clang{} and \Fortran{} bindings may claim
%complete conformance.

\openshmem provides ISO \Clang{} and \Fortran{} 90 language
bindings. Any implementation that provides both \Clang{} and \Fortran{} bindings 
can claim conformance to the specification. An implementation that provides e.g.\ only a \Clang{} interface may claim to conform to the \openshmem specification with respect to
the \Clang{} language, but not to \Fortran{}, and should make this clear in its documentation. The \openshmem header files for \Clang{} and \Fortran{} must contain only the interfaces 
and constant names defined in this specification.

\openshmem{} \ac{API}s can be implemented as either 
functions or macros. However, implementing the interfaces using macros is
strongly discouraged as this could severely limit the use of external profiling tools 
and high-level compiler optimizations. An \openshmem{} program should avoid defining function names, variables, or
identifiers with the prefix \shmemprefix{} (for \Clang{} and \Fortran{}), \shmemprefixC{} (for \Clang{}) or with \openshmem \ac{API} names.

%The \openshmem{} constants and environment variables are in all capital letters. 
%All \openshmem{} functions are prefixed with \shmemprefix{}, besides these 
%expections : start\_pes{}, shfree{}, shpalloc{}, shpclmove{}, shpdellc{}. 
%\begin{itemize}
%\item start\_pes{}
%\item shfree{}
%\item shpalloc{}
%\item shpclmove{}
%\item shpdellc{}
%\end{itemize}

 
%
%The \openshmem{}
%functions does not return any error code. 

