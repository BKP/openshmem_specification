\section{Compiling and Running Applications}

As of now the \openshmem{} specification is silent regarding how \openshmem{} programs
are compiled, linked and run. This section shows some examples of
how wrapper programs are utilized in the \openshmem{} Reference Implementation to compile and launch applications.
%The commands are styled after wrapper programs found in many MPI implementations.
\subsection{Compilation}
\subsubsection*{Applications written in \Clang}

The  \openshmem{} Reference Implementation provides a wrapper program named \textbf{oshcc}, to aid in the compilation of \Clang{} applications, the wrapper
could be called as follows:

\begin{lstlisting}[language=bash]
oshcc <compiler options> -o myprogram myprogram.c
\end{lstlisting}
Where the $\langle\mbox{compiler options}\rangle$ are options understood by the underlying \Clang{} compiler.


\subsubsection*{Applications written in \Cpp}

The  \openshmem{} Reference Implementation provides a wrapper program named \textbf{oshCC}, to aid in the compilation of \Cpp{} applications, the
wrapper could be called as follows:

\begin{lstlisting}[language=bash]
oshCC <compiler options> -o myprogram myprogram.cpp
\end{lstlisting}
Where the $\langle\mbox{compiler options}\rangle$ are options understood by the underlying \Cpp{} compiler called by \textbf{oshCC}.


\subsubsection*{Applications written in \Fortran}

The  \openshmem{} Reference Implementation provides a wrapper program named \textbf{oshfort}, to aid in the compilation of \Fortran{} applications,
the wrapper could be called as follows:

\begin{lstlisting}[language=bash]
oshfort <compiler options> -o myprogram myprogram.f
\end{lstlisting}
Where the $\langle\mbox{compiler options}\rangle$ are options understood by the underlying \Fortran{} compiler called by \textbf{oshfort}.

\subsection{Running Applications}

The  \openshmem{} Reference Implementation provides a wrapper program named \textbf{oshrun}, to launch \openshmem applications, the wrapper could
be called as follows:

\begin{lstlisting}[language=bash]
oshrun <additional options> -np <#> <program> <program arguments>
\end{lstlisting}
The program arguments for \textbf{oshrun} are:

\begin{tabular}{p{0.3\textwidth}p{0.6\textwidth}}
$\langle\mbox{additional options}\rangle$ & {Options passed to the underlying launcher.}\tabularnewline
-np $\langle\mbox{\#}\rangle$ & {The number of processing elements (PEs) to be used in the execution.}\tabularnewline
$\langle\mbox{program}\rangle$ & {The program executable to be launched.}\tabularnewline
$\langle\mbox{program arguments}\rangle$ & {Flags and other parameters to pass to the program.}\tabularnewline
\end{tabular}

%\begin{description}
%\item[$\langle\mbox{additional options}\rangle$] options passed to the underlying launcher
%\item[-np $\langle\mbox{\#}\rangle$] The number of processing elements (PEs) to be used
%in the execution.
%\item [$\langle\mbox{program}\rangle$] The program executable to be launched
%\item [$\langle\mbox{program arguments}\rangle$] flags and other parameters to pass to the program
%\end{description}
