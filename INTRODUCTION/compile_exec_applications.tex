
\section{Compiling and Running Applications}

The \openshmem specification is silent regarding how \openshmem programs
are compiled, linked and run. This section shows some examples of
how wrapper programs could be utilized to compile and launch applications.
The commands are styled after wrapper programs found in many MPI implementations.


\subsection{Compilation}


\subsubsection{Applications written in \Clang}

Assuming that the implementation provides a wrapper program named
\textbf{oshcc}, to aid in the compilation of \Clang{} applications, the wrapper
could be called as follows:

\begin{lstlisting}[language=bash]
oshcc <compiler options> -o myprogram myprogram.c
\end{lstlisting}


The program arguments for \textbf{oshcc} are:
\begin{description}
\item [{<compiler\ options>}] Options understood by the underlying \Clang{} compiler
called by \textbf{oshcc}.
\end{description}

\subsection{Applications written in \Cpp}

Assuming that the implementation provides a wrapper program named
\textbf{oshCC}, to aid in the compilation of \Cpp{} applications, the
wrapper could be called as follows:

\begin{lstlisting}[language=bash]
oshCC <compiler options> -o myprogram myprogram.cpp
\end{lstlisting}


The program arguments for \textbf{oshCC} are:
\begin{description}
\item [{<compiler\ options>}] Options understood by the underlying \Cpp{}
compiler called by \textbf{oshCC}.
\end{description}

\subsection{Applications written in \Fortran}

Assuming that the implementation provides a wrapper program named
\textbf{oshfort}, to aid in the compilation of \Fortran{} applications,
the wrapper could be called as follows:

\begin{lstlisting}[language=bash]
oshfort <compiler options> -o myprogram myprogram.f
\end{lstlisting}


The program arguments for \textbf{oshfort} are:
\begin{description}
\item [{<compiler\ options>}] Options understood by the underlying \Fortran{}
compiler called by \textbf{oshfort}.
\end{description}

\section{Running Applications}

Assuming that the implementation provides a wrapper program named
\textbf{oshrun}, to launch \openshmem applications, the wrapper could
be called as follows:

\begin{lstlisting}[language=bash]
oshrun <additional options> -np <#> <program> <program arguments>
\end{lstlisting}


The program arguments for \textbf{oshrun} are:
\begin{description}
\item [{<additional\ options>}] options passed to the underlying launcher
\item [{-np~<\#>}] The number of processing elements (PEs) to be used
in the execution.
\item [{<program>}] The program executable to be launched
\item [{<program\ arguments>}] flags and other parameters to pass to the program
\end{description}
