%Outline
%%Exectution model
%   *Define what is a OpenSHMEM program: a set of processes (either SPMD or MIMD?) where each process has its own 'local' (private) memory and symmetric memory regions that may be accessible by any PEs.
%   *Each OpenSHMEM process is called a processing element (PE)
%   *Each PE may be mapped to many to one hardware cores/threads or less.
%   *The number of PEs is specified at launch/runtime.
%   *Each PE must call startpe to initialize the OpenSHMEM runtime, before any other call for OpenSHMEM. There is an implicit barrier at startpe.
%   *Each PE executes asynchronously following Fortran or program execution in C [ISO/IEC00 Sec. 5.1.2.3]
%   *Each PE will have a unique global identifier and the execution of a program may depend on the PE id, if executed in SPMD.
%   *PE id may be used for library calls synchronizations, control flow constructs language  in C/Fortran
%   *PE may allocate symmetric data objects via a symmetric heap  during execution%SP: Does not cover global and static.
%   *As of now, PEs may finish execution at any time by returning from the main function. (no call to shmem_finalize yet!)
%   
%This comes from the UPC spec:
%The memory consistency model in a language defines the order in which the results of write operations may be observed through read operations.
%The behavior of a OpenSHMEM program may depend on the timing of accesses to symetric variables on PEs, so in general a program defines a set of possible executions, 
%rather than a single execution. The memory consistency model constrains the set of possible executions for a given program; the user may then rely 
%on properties that are true of all of those executions.
    

\section{Execution Model}
An OpenSHMEM program consists of a set of processes, called \ac{PE}s that execute and communicate with each other via \openshmem calls in a \ac{SPMD}-like execution model. In \openshmem different \ac{PE}s may have different execution paths and will execute asynchronously following \Fortran{} or program execution in \Clang.  All \ac{PE} may be mapped 1-to-1 to hardware cores/threads or less where the number of \ac{PE}s  is specified at runtime. An \openshmem program should start by calling the initialization function \FUNC{start \_pes}  before using any of the other \openshmem library routines. \ac{PE}s do not exist until after the call to \FUNC{start\_pes} returns. During execution, each \ac{PE} is assigned a unique global identifier for the duration of the program. These \ac{PE} identifiers are integers assigned in a monotonically increasing manner from zero to the total number of \ac{PE}s minus 1. \ac{PE} identifiers are used for other \openshmem library calls (i.e. to access symmetric objects from specific \ac{PE}s, collective synchronization, etc) or to dictate a control flow for \ac{PE}s using constructs of \Clang{} or \Fortran.  As of now, an \openshmem program finishes execution by returning from the main function. It is up to the implementation on how to handle the finalization of the \openshmem library and any other resources initialized by the library: there is currently no explicit finalization call defined in the \openshmem specification.

\subsection{Progress of \openshmem operations}
The \openshmem model assumes that computation and communication are
naturally overlapped.  High quality \openshmem implementations must insure that programs exhibit %SP: Changing MUST to may as per discussion on 01/31/2014
progression of communication both with and without \openshmem calls.
Consider a \ac{PE} that is engaged in a long computation with no \openshmem calls, other \ac{PE}s must be able to communicate (put/get,
collective, atomic) with that computationally-bound \ac{PE} without that \ac{PE}
issuing any explicit \openshmem calls. \openshmem communication calls involving that \ac{PE} must progress
regardless of when that \ac{PE} next engages in an \openshmem call.

\textbf{Note to implementers:} For example, progress will often be ensured through
the use of a dedicated progress thread in software, or through
network hardware that offloads communication handling from processors.
%SP: Why only communication ? Shouldn't t be for all openshmem calls?
%\subsection{Using the Symmetric \VAR{Work} and \VAR{pSync} Arrays}

%Multiple \VAR{pSync} arrays are often needed if a particular \ac{PE} calls a \openshmem
%collective  function twice without intervening barrier synchronization.
%Problems would occur if some \ac{PE}s in the \activeset{} for call 2 arrive at
%call 2 before processing of call 1 is complete by all \ac{PE}s in the call 1
%\activeset.  You can use  \FUNC{shmem\_barrier}  or  \FUNC{shmem\_barrier\_all}  to
%perform  a  barrier  synchronization between consecutive calls to \openshmem
%collective functions. There are two special cases:
%\begin{itemize}
%\item The \FUNC{shmem\_barrier} function allows the same \VAR{pSync} array to be used
%          on consecutive calls as long as the active \ac{PE} set does not change.
%\item  If the same collective function is called multiple times with the
%          same \activeset, the calls may alternate between two \VAR{pSync} arrays.
%          The \openshmem functions guarantee that a first call is completely finished by 
%          all \ac{PE}s by the time processing of a third  call  begins  on
%          any \ac{PE}.          
%\end{itemize}
%Because  the \openshmem functions restore \VAR{pSync} to its original contents,
%multiple calls that use the same \VAR{pSync} array do not require that \VAR{pSync}
%be reinitialized after the first call.

\subsection{Atomicity Guarantees}

\openshmem contains a number of routines that operate on symmetric data
atomically.  These routines guarantee that accesses by \openshmem's
atomic operations will be exclusive, but do not guarantee exclusivity
in combination with other routines, either inside \openshmem or
outside.

For example: during the execution of a remote integer increment
operation on a symmetric variable \VAR{x}, no other \openshmem atomic
operation may access \VAR{x}.  After the increment, \VAR{x} will have
increased its value by \CONST{1} on the target \ac{PE}, at which point other
atomic operations may then modify that \VAR{x}.

%  %Memory model
%    *Each OpenSHMEM PEs may have symmetric memory that is accessible by other PEs. 
%    *Symmetric memory is a region of memory where all the an instance of a data objects is replicated across PEs, have 
%     the same the same layout and relative offset.
%    *All PEs can allocate a symmetric data objects using the symmetric heap, but they must do so as a collective operation. (is there a barrier after shmalloc?)
%    *All writes to symmetric memory are relaxed (I'm not sure if this is the completion semantics) and are guaranteed to be visible to other PEs after a barrier_all, barrier(?), quiet, (what about wait? does it means iti sonly visible to me?) 
%    *Calls to barrier, barrier_all, quiet, wait, lock, atomics, are meant to guarantee memory consistency across PEs.
%    *Read/Writes to symmetic data object may appear after startpe or after a the symmetric data object has been allocated in the symmetric heap (if it is a dynamic).
%    *Operations like reduction, collect, etc guarantee memory consistency after completion(?)
%    *Data races are possible in OpenSHMEM if multiple PEs write/read a symmetric data object from a single PE without proper synchronization.  
