\section{Programming Model Overview}
The \openshmem programming model consists of library functions that provide
low-latency, high-bandwidth communication  for  use  in  highly  parallelized 
scalable programs. The functions in the \openshmem \ac{API} provide a programming 
model for exchanging data between cooperating parallel processes. The resulting programs are similar 
in style to \ac{MPI} programs. The \openshmem \ac{API} can be used either alone 
or in combination with \ac{MPI} functions in the same parallel program.

A \openshmem program is \ac{SPMD} in style. The
\openshmem  processes,  called  processing elements or \ac{PE}s, all start at the
same time, and they all run the same program.  Usually the \ac{PE}s  perform
computation on their own subdomains of the larger problem, and periodically 
communicate with other \ac{PE}s to exchange information on which the
next computation phase depends.The \openshmem functions minimize the overhead associated with data transfer
requests, maximize bandwidth, and minimize data latency.  Data latency
is  the  period  of  time that starts when a \ac{PE} initiates a transfer of data 
and ends when a \ac{PE} can use the data.

\openshmem functions support remote data transfer through \FUNC{put} operations, which  transfer data to a 
different \ac{PE}, get operations, which transfer data from a different \ac{PE}, and remote pointers, which 
allow direct  references  to  data objects owned by another \ac{PE}.  Other operations supported are \FUNC{collective} 
\FUNC{broadcast} and \FUNC{reduction}, \FUNC{barrier synchronization}, and \FUNC{atomic memory operations}. 
An atomic memory operation  is an atomic read-and-update operation, such as a fetch-and-increment, on a remote
or local data object.

