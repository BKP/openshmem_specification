SHMEM has a long history as a parallel programming
model, having been used extensively on a number of products since
1993, including Cray T3D, Cray X1E, the Cray XT3/4, SGI Origin, SGI
Altix, clusters based on the Quadrics interconnect, and to a very
limited extent, Infiniband based clusters.

\begin{itemize}
\item A SHMEM Timeline
  \begin{itemize}
  \item Cray SHMEM
    \begin{itemize}
    \item SHMEM first introduced by Cray Research Inc. in 1993 for Cray T3D
    \item Cray is acquired by SGI in 1996
    \item Cray is acquired by Tera in 2000 (MTA)
    \item Platforms: Cray T3D, T3E, C90, J90, SV1, SV2, X1, X2, XE, XMT, XT
    \end{itemize}
  \item SGI SHMEM
    \begin{itemize}
    \item SGI purchases  Cray Research Inc. and SHMEM was integrated into
      SGI's Message Passing Toolkit (MPT)
    \item SGI currently owns the rights to SHMEM and \openshmem
    \item Platforms: Origin, Altix 4700, Altix XE, Altix ICE, Altix UV
    \item SGI was purchased by Rackable Systems in 2009
    \item SGI and Open Source Software Solutions, Inc. (OSSS) signed a
      SHMEM trademark licensing agreement, in 2010
    \end{itemize}
  \item Other Implementations
    \begin{itemize}
    \item Quadrics (Vega UK, Ltd.)
    \item Hewlett Packard
    \item GPSHMEM
    \item IBM
    \item QLogic
    \item Mellanox
    % \item University of Houston
    \item University of Florida
    \end{itemize}
  \end{itemize}
\item OpenSHMEM Implementations 
 \begin{itemize}
  \item SGI \openshmem
  \item University of Houston - \openshmem Reference Implementation
  \item Mellanox ScalableSHMEM
  \item Portals-SHMEM
  \end{itemize} 
  \item Pending verification that supports \openshmem 
 \begin{itemize}
  \item IBM OpenSHMEM
  \end{itemize}
\end{itemize}


%Despite being supported by a variety of vendors there is no standard
%defining the SHMEM memory model or programming interface. Consistencies
%(where they exist) and extensions across the various implementations have
%been driven by the needs of an enthusiastic user community. The lack of a
%SHMEM standard has allowed each implementation to differ in both interface
%and semantics from vendor to vendor and even product line to product line,
%which has to this point limited broader acceptance.

%\begin{description}
%\item [{{Cray~SHMEM~(MP-SHMEM,~LC-SHMEM):}}] Cray first introduced
%SHMEM in 1993 for its Cray T3D systems. Cray SHMEM was also used in
%other models: T3E, PVP and XT series. 
%\item [{{SGI~SHMEM~(SGI-SHMEM):}}] Cray Research merged with Silicon
%Graphics (SGI) in February 1996. At this point SHMEM was incorporated
%into SGI's Message Passing Toolkit (MPT). The platforms supported
%were - SGI Irix, Origin and Altix. 
%\item [{{Quadrics~SHMEM~(Q-SHMEM):}}] an optimized API for the Quadrics
%QsNet interconnect. It included SGI extensions and provided non-blocking
%puts and gets. A joint effort from HCS Lab \& Quadrics incorporated
%a program profiling interface called PSHMEM that can aid in the execution
%analysis of SHMEM programs. 
%\end{description}
%The success of SHMEM's performance attracted several vendors to provide
%implementations (with varying names and features) for their systems.
%Some of them include: 
%\begin{description}
%\item [{{HP~SHMEM:}}] Based on the Quadrics API. It is included in the
%UPC product kit. 
%\item [{{Cyclops-64~SHMEM~(C64-SHMEM):}}] this SHMEM API supports the
%Cyclops-64 architecture. Most of the core features of Cray SHMEM are
%available with some additional interfaces specific to the Cyclops-64
%architecture. 
%
%\item [{{IBM~SHMEM:}}] An implementation created by IBM intended for
%internal use only. 
%\item [{{TurboSHMEM:}}] This implementation uses IBM's Low-Level API
%(LAPI) technology to obtain optimized one-sided communication for
%the put/get operations. This allows applications written with the
%SHMEM API to run on IBM platforms with minimal source code changes. 
%\item [{{GPSHMEM:}}] This implementation of SHMEM aims at providing full
%portability of applications. It is built mostly with Cray T3D components
%and functionalities and provides MPI and ARMCI support. This project
%is no longer maintained. 
%\end{description}
