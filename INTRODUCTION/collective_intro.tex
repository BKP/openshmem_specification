\section{Collective Operations}
Collective operation is defined as communication or synchronization operation 
on a group of \ac{PE}s called \activeset{}. The collective operations requires all
\ac{PE}s in the \activeset{} to simulatenously call the operation. 
A \ac{PE} that is not a part of \activeset{} calling the collective 
operation results in an undefined behavior.

The \activeset{} is defined by arguments PE\_start, logPE\_stride, 
and PE\_size. The PE\_start is the starting \ac{PE} number, a log (base 2) of logPE\_stride 
is the stride between \ac{PE}s, and PE\_size is the number of \ac{PE}s 
participating in the \activeset{}. All \ac{PE}s participating in the 
collective operations provides the same values for these arguments. 
 
Another argument important to the collective operation is pSync, a symmetric work 
array. All \ac{PE}s participating in a collective should pass the same
pSync array. On completion of a collective call, the pSync is restored to its 
original contents. The reuse of pSync array is allowed for a \ac{PE}, if all previous collective 
operations using the pSync array is completed by all participating 
\ac{PE}s. One can use a synchronization collective operation such as \barrier{}
to ensure completion of previous collective operations.

All collective operations defined in the specification are blocking. The 
collective operations return on completion. The collective operation 
The collective operations defined in the \openshmem{} specification 
are :

\begin{itemize}
\item[] \broadcast{} 
\item[] \barrier{}
\item[] \barrierall{}
\item[] \collect{}
\item[] \reduction{} 
\end{itemize} 

