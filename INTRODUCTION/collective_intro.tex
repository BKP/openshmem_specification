%Comments for Manju:
%Are Barrier_all, shmalloc, etc considered collectives? If yes, we need to state that all PEs belong to an implicit active set that contains all PES.
%Also describe the case where collective operations may be invoked by the implicit active set (all PEs) or active sets PEs
%State that collectives can be executed on statements that are not ordered .
%State that from the beginning to the end of the program, the sequence of collectives should be the same on a given active set (or implicit active set).
%Are arguments  the same for PEs that call the same collective? (i.e. target or source symmetric data?)
%Which collectives imply synchronization (i.e. barrier, quiet, etc) which ones not (i.e. broadcast on root?)

Collective operation is defined as communication or synchronization operation 
on a group of \ac{PE}s called \activeset{}. The collective operations requires all
\ac{PE}s in the \activeset{} to simultaneously call the operation. 
A \ac{PE} that is not a part of \activeset{} calling the collective 
operation results in an undefined behavior.

The \activeset{} is defined by arguments PE\_start, logPE\_stride, 
and PE\_size. The PE\_start is the starting \ac{PE} number, a log (base 2) of logPE\_stride 
is the stride between \ac{PE}s, and PE\_size is the number of \ac{PE}s 
participating in the \activeset{}. All \ac{PE}s participating in the 
collective operations provides the same values for these arguments. 
 
Another argument important to the collective operation is pSync, which is a symmetric work 
array. All \ac{PE}s participating in a collective must pass the same
pSync array. On completion of a collective call, the pSync is restored to its 
original contents. The reuse of pSync array is allowed for a \ac{PE}, if all previous collective 
operations using the pSync array is completed by all participating 
\ac{PE}s. One can use a synchronization collective operation such as \barrier{}
to ensure completion of previous collective operations. The two cases below
show the reuse of pSync array:

\begin{itemize}
\item The \FUNC{shmem\_barrier} function allows the same \VAR{pSync} array to be used
          on consecutive calls as long as the active \ac{PE} set does not change.
\item  If the same collective function is called multiple times with the
          same \activeset, the calls may alternate between two \VAR{pSync} arrays.
          The \openshmem functions guarantee that a first call is completely finished by 
          all \ac{PE}s by the time processing of a third  call  begins  on
          any \ac{PE}.          
\end{itemize}


All collective operations defined in the specification are blocking. The 
collective operations return on completion. The collective operation 
The collective operations defined in the \openshmem{} specification 
are :

\begin{itemize}
\item[] \broadcast{} 
\item[] \barrier{}
\item[] \barrierall{}
\item[] \collect{}
\item[] \reduction{} 
\end{itemize} 