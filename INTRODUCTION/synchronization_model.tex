\section{Synchronization and Communication Ordering in OpenSHMEM}

In the presence of the \openshmem's one-sided communication, synchronization becomes critical. The updates via shmem\_puts cannot be guaranteed untill some form of synchronization or ordering is introduced by the application programmer. The table below gives the different synchronization and ordering options and situations where they may be useful.\\

\begin{tabular}{|p{0.3\textwidth}|p{0.3\textwidth}|p{0.3\textwidth}|}
\hline 
\textbf{Synchronization Class} & \textbf{\openshmem API}& \textbf{Appropriate Situation}\tabularnewline
\hline 
\hline 
Point-to-point & shmem\_wait, shmem\_wait\_until & {Waits for a symmetric variable to be updated by a remote \ac{PE}. Should be used when computation at the local \ac{PE} cannot proceed without the value that the remote \ac{PE} is to update.}\tabularnewline
\hline 
Ordering puts issued by a local \ac{PE} & shmem\_fence & All puts issued before the fence operation by the local \ac{PE} are guaranteed to be delivered before puts issued after the fence call to the same remote \ac{PE}. This operation should be used when all remote writes by a local \ac{PE} to a remote \ac{PE} need to be visible (\rcomment{Swaroop: assuming visible == delivered}) before any new remote write operation to the same \ac{PE}.
\tabularnewline
\hline 
Ordering puts issued by all \ac{PE} & shmem\_quiet & {All puts issued by all \ac{PE}s are guaranteed to be delivered before the next local update or remote memory update via \openshmem (\rcomment{May change after SGI's input.}). This operation should be used when all remote writes by all \ac{PE}s need to be visible  to all other \ac{PE}s before any new local or remote memory update via OpenSHMEM library operation.} 
\tabularnewline
\hline 
Collective synchronization over an active set & shmem\_barrier & {All local and remote memory operations issued by all \ac{PE}s within the active set are guaranteed to be completed before any \ac{PE} in the active set returns from the call. Additionally no \ac{PE} my return from the barrier till all \ac{PE}s in the active set have called the same barrier call. This operation should be used when synchronization as well as completion of local stores and remote memory updates via OpenSHMEM is required over a sub-set of the executing \ac{PE}s.} \tabularnewline
\hline 
Collective synchronization over all \ac{PE}s & shmem\_barrier\_all & {All local and remote memory operations issued by all \ac{PE}s are guaranteed to be completed before any \ac{PE} returns from the call. Additionally no \ac{PE} my return from the barrier till all \ac{PE}s have called the same barrier call. This operation should be used when synchronization as well as completion of local stores and remote memory updates via OpenSHMEM is required over all \ac{PE}s.} \tabularnewline
\hline 
\end{tabular}