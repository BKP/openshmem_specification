\subsection{C Language Examples}

\lstinputlisting[caption={Program that is a trivial Hello World.},label={HelloW},language={C++}]{C/helloworld.c}

\lstinputlisting[caption={Program that implements a Circular Shift.},label={CircSh},language={C++}]{C/circ.c}

\pagebreak{}

\lstinputlisting[caption={Program that demonstrates the use of shmalloc.},label={shamalloc},language={C++}]{C/shmalloc.c}

\lstinputlisting[caption={Program that implements Ping.},label={ping},language={C++}]{C/ping.c}

\pagebreak{}

\lstinputlisting[caption={Program that uses the MAX reduction.},label={max},language={C++}]{C/reduce-max.c}

\pagebreak{}

\lstinputlisting[caption={Program that makes use of strided puts.},label={iput},language={C++}]{C/iput.c}

\pagebreak{}

\lstinputlisting[caption={Program that implements an ALL-2-ALL (header)},label={all2all_head},language={C++}]{C/bench.h}

\pagebreak{}

\lstinputlisting[caption={Program that implements an ALL-2-ALL (main)},label={all2allMain},language={C++}]{C/all2all_main.c}

\pagebreak{}

\lstinputlisting[caption={Program that implements an ALL-2-ALL (subs)},label={all2allSub},language={C++}]{C/all2all_subs.c}

\lstinputlisting[caption={Program that computes Pi},label={compute-pi},language={C++}]{C/pi.c}
