%Outline
%%Exectution model
%   *Define what is a OpenSHMEM program: a set of processes (either SPMD or MIMD?) where each process has its own 'local' (private) memory and symmetric memory regions that may be accessible by any PEs.
%   *Each OpenSHMEM process is called a processing element (PE)
%   *Each PE may be mapped to many to one hardware cores/threads or less.
%   *The number of PEs is specified at launch/runtime.
%   *Each PE must call startpe to initialize the OpenSHMEM runtime, before any other call for OpenSHMEM. There is an implicit barrier at startpe.
%   *Each PE executes asynchronously following Fortran or program execution in C [ISO/IEC00 Sec. 5.1.2.3]
%   *Each PE will have a unique global identifier and the execution of a program may depend on the PE id, if executed in SPMD.
%   *PE id may be used for library calls synchronizations, control flow constructs language  in C/Fortran
%   *PE may allocate symmetric data objects via a symmetric heap %SP: Does not cover global and static.
%   *As of now, PEs may finish execution at any time by returning from the main function. (no call to shmem_finalize yet!)
%   
%This comes from the UPC spec:
%The memory consistency model in a language defines the order in which the results of write operations may be observed through read operations.
%The behavior of a OpenSHMEM program may depend on the timing of accesses to symetric variables on PEs, so in general a program defines a set of possible executions, 
%rather than a single execution. The memory consistency model constrains the set of possible executions for a given program; the user may then rely 
%on properties that are true of all of those executions.
    

\section{Execution Model}
\openshmem uses a Single Process Multiple Data (SPMD) approach to express
parallelism.  An \openshmem application makes use of multiple processors,
referred to as Processing Elements or PEs, to complete operations
in parallel. Although all PEs execute the same program, they may have different execution paths and will execute asynchronously following Fortran or program execution in C. Each PE may be mapped to many to one hardware cores/threads or less. In \openshmem the number of PEs are specified at runtime.

\openshmem requires initialization before using any of the library
routines. To this end, the program must  issue a call to the \textbf{start\_pes()}
routine before any other OpenSHMEM library operations. \textbf{start\_pes()} performs any required initialization
steps, such as setting up the symmetric heap for every PE and creating
the PE numbers which act like unique global identifiers for the duration of the program. These PE numbers are integers assigned in a monotonically
increasing manner from zero to the total number of PEs minus 1. The PEs do not exist till after \textbf{start\_pes()} returns.

%The symmetric heap is one of the memory spaces
%that is remotely accessible by all PEs. The symmetric heap is discussed
%further in the Memory Model section. The PE numbers are the
%identifiers used to refer to each of the PEs involved in the execution.
Consistent with the SPMD nature of the \openshmem programming model  is  the concept of symmetric data objects.  These are arrays or variables that exist with the same size,  type,	 and  relative	address	 on  all  PEs. Another	term  for  symmetric data objects is "remotely accessible data objects."  In the interface definitions for \openshmem data  transfer	 functions,  one or more of the parameters are typically required to be symmetric or remotely accessible. The following kinds of data objects are symmetric:
\begin{itemize}
  \item Fortran data objects in common blocks or with the  SAVE  attribute. These data objects	must not be defined in a dynamic shared object (DSO).
  \item Non-stack C and C++ variables.   These  data	objects must  not  be defined in a DSO.
  \item Fortran arrays allocated with \textit{shpalloc} 
  \item C and C++ data allocated by \textit{shmalloc}
\end{itemize}       

Data transfer in \openshmem is possible through several one-sided put
(for write) and get (for read) operations, as well as various collective
routines such as broadcasts and reductions. Since the library provides the flexibility of one-sided operations the execution pattern is depends on the how the programmer decides to distribute work amongst different PEs and the synchronization and ordering operations used.

Query routines are available to gather information about the execution.
\openshmem also provides synchronization routines to coordinate data
transfers and other operations. 

It is up to the implementation how to handle the finalization of the
\openshmem library and any other resources initialized by the library:
there is currently no explicit call defined in the \openshmem specification.

\subsection{Communication Progress}

The \openshmem model assumes that computation and communication are
naturally overlapped.  \openshmem programs are expected to exhibit
progression of communication both with and without \openshmem calls.

Consider a PE that is engaged in a long computation with no \openshmem calls.
Other PEs must be able to communicate (put/get,
collective, atomic) with that computationally-bound PE without that PE
issuing any explicit \openshmem calls.

\openshmem communication calls involving that PE must progress
regardless of when that PE next engages in an \openshmem call.

\textbf{Note to implementers:} progress will often be ensured through
the use of a dedicated progress thread in software, or through
network hardware that offloads communication handling from processors.

\subsection{Atomicity Guarantees}

\openshmem contains a number of routines that operate on symmetric data
atomically.  These routines guarantee that accesses by \openshmem's
atomic operations will be exclusive, but do not guarantee exclusivity
in combination with other routines, either inside \openshmem's or
outside.

For example: during the execution of a remote integer increment
operation on a symmetric variable ``x'', no other \openshmem atomic
operation may access ``x''.  After the increment, ``x'' will have
increased its value by 1 on the target PE, at which point other
atomic operations may then modify that ``x''.

%  %Memory model
%    *Each OpenSHMEM PEs may have symmetric memory that is accessible by other PEs. 
%    *Symmetric memory is a region of memory where all the an instance of a data objects is replicated across PEs, have 
%     the same the same layout and relative offset.
%    *All PEs can allocate a symmetric data objects using the symmetric heap, but they must do so as a collective operation. (is there a barrier after shmalloc?)
%    *All writes to symmetric memory are relaxed (I'm not sure if this is the completion semantics) and are guaranteed to be visible to other PEs after a barrier_all, barrier(?), quiet, (what about wait? does it means iti sonly visible to me?) 
%    *Calls to barrier, barrier_all, quiet, wait, lock, atomics, are meant to guarantee memory consistency across PEs.
%    *Read/Writes to symmetic data object may appear after startpe or after a the symmetric data object has been allocated in the symmetric heap (if it is a dynamic).
%    *Operations like reduction, collect, etc guarantee memory consistency after completion(?)
%    *Data races are possible in OpenSHMEM if multiple PEs write/read a symmetric data object from a single PE without proper synchronization.  
