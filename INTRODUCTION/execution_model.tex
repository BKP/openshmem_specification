
\section{Execution Model}

This section describes the Execution Model of an \openshmem application.

\openshmem uses a Single Process Multiple Data (SPMD) approach to express
parallelism. An \openshmem application makes use of multiple processors,
referred to as Processing Elements or PEs, to complete operations
in parallel.

% OpenSHMEM applications may be launched using a dispatcher program,
% but it is not required. This dispatcher program may perform steps
% needed by the implementation to start the application on the desired
% processing elements. The following is an example of using a dispatcher:
% \lstinline[language=bash]!oshrun -np 4 myprogram!

\openshmem requires initialization before using any of the library
routines. To this end, the program issues a call to the \textbf{start\_pes()}
routine. \textbf{start\_pes()} performs any required initialization
steps, such as setting up the symmetric heap for every PE and creating
the PE numbers. The symmetric heap is one of the memory spaces
that is remotely accessible by all PEs. The symmetric heap is discussed
further in the Memory Model section. The PE numbers are the
identifiers used to refer to each of the PEs involved in the execution.
These PE numbers are integers assigned in a monotonically
increasing manner from zero to the total number of PEs minus 1.

Data transfer in \openshmem is possible through several one-sided put
(for write) and get (for read) operations, as well as various collective
routines such as broadcasts and reductions.

Query routines are available to gather information about the execution.
\openshmem also provides synchronization routines to coordinate data
transfers and other operations. 

It is up to the implementation how to handle the finalization of the
\openshmem library and any other resources initialized by the library:
there is currently no explicit call for the programmer.

% TONY: trying to keep the ``note to'' with the preceding text (so no
% pagebreak between text and its note).  Not sure how to do this ``for
% real'' so just forcing a break before the subsection for now.

\pagebreak

\subsection{Communication Progress}

The \openshmem model assumes that computation and communication are
naturally overlapped.  \openshmem programs are expected to exhibit
progression of communication both with and without \openshmem calls.

Consider a PE that is engaged in a long computation with no \openshmem calls.
Other PEs must be able to communicate (put/get,
collective, atomic) with that computationally-bound PE without that PE
issuing any explicit \openshmem calls.

\openshmem communication calls involving that PE must progress
regardless of when that PE next engages in an \openshmem call.

\textbf{Note to implementers:} progress will often be ensured through
the use of a dedicated progress thread in software, or through
network hardware that offloads communication handling from processors.

\subsection{Atomicity Guarantees}

\openshmem contains a number of routines that operate on symmetric data
atomically.  These routines guarantee that accesses by \openshmem's
atomic operations will be exclusive, but do not guarantee exclusivity
in combination with other routines, either inside \openshmem's or
outside.

For example: during the execution of a remote integer increment
operation on a symmetric variable ``x'', no other \openshmem atomic
operation may access ``x''.  After the increment, ``x'' will have
increased its value by 1 on the target PE, at which point other
atomic operations may then modify that ``x''.
