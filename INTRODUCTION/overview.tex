\section{Overview}
\openshmem is a \ac{PGAS} library interface specification. In the \ac{PGAS} model each process has a local and 
globally shared memory where portions of the shared memory may have affinity to a particular process. \openshmem 
implements \ac{PGAS} using symmetric memory to share information among processes or \ac{PE}s.   
It provides interfaces to perform communication and synchronization operations on both local and symmetric memory. 
\openshmem is a library and unlike UPC, CAF, Titanium, X10 and Chapel, which are all
PGAS languages, it relies on the programmer to use the library calls correctly.

The \openshmem specification is a result of effort to standardize widely available SHMEM implementations. 
The initial goal of the specification is to consolidate existing vendor SHMEM implementations, and  pave the way for 
a defacto standard for SHMEM with community involvement. This enables portability of SHMEM programs, and 
allows vendors to build and optimize their hardware architecture for OpenSHMEM.

The \openshmem specification defines routines, constants, variables, and language bindings for \Clang{} and \Fortran.
Some of important \openshmem operations are as follows:

\begin{enumerate}
\item \textbf{Data Transfers }

\begin{enumerate}
\item One-sided puts : the initiator \ac{PE} (active side) specifies the local
data to be written to the target \ac{PE}'s (passive side) memory. 
\item One-sided gets : an explicit fetch operation is used to copy a variable
amount of data from a remote process and store it locally.\end{enumerate}
\begin{description}
\item [{{Note:}}] By avoiding the need for matching send and receive
calls, \openshmem{}simplifies the communication process by reducing the
number of calls required to have one \ac{PE} interact with other \ac{PE}s. 
\end{description}
\item \textbf{Synchronization Mechanisms }

\begin{enumerate}
\item Fence: Ensures ordering of PUT operations to a specific \ac{PE}. 
\item Quiet: Ensures ordering of PUT operations to all \ac{PE}s. 
\item Barrier: A collective synchronization routine in which no \ac{PE} may leave
the barrier prior to all \ac{PE}s entering the barrier. 
\end{enumerate}
\item \textbf{Collective Communication}

\begin{enumerate}
\item Broadcast: Copy a block of data from one \ac{PE} to one or more remote
\ac{PE}s. 
\item Collection: Concatenate elements from the source array to a target
array over the specified \ac{PE}s. 
\item Reduction: Perform an associative binary operation over the specified
\ac{PE}s. 
\end{enumerate}
\item \textbf{Address Manipulation}

\begin{enumerate}
\item Allocating and deallocating memory blocks in the symmetric space.
\end{enumerate}
\item \textbf{Locks}

\begin{enumerate}
\item Implementation of mutual exclusion.
\end{enumerate}
\item \textbf{Atomic Memory Operations}

\begin{enumerate}
\item Swap, Conditional Swap, Add and Increment 
\end{enumerate}
\item \textbf{Data Cache control}

\begin{enumerate}
\item Implementation of mechanisms to exploit the capabilities of hardware
cache if available.
\end{enumerate}
\end{enumerate}
\begin{description}
\item [{{Note:}}] More information about \openshmem routines can be found
in the Library Routines section.
\end{description}

