\section{Overview}
\openshmem{} is a PGAS library interface specification. In the PGAS model each process has a local and 
global memory which it can access. \openshmem{} provides interfaces to 
perform communication and synchronization operations on both local and global memory. 
It is not a language, unlike UPC, CAF, Titanium, X-10 and Chapel, which are all
PGAS languages.

\openshmem{} specification is a result of effort to standardize widely available SHMEM implementations. 
The initial goal of the standard is to consolidate existing SHMEM specifications, and  pave the way for 
improvement of the specification with community involvement. The standardization of the specification 
would enable portability of SHMEM programs, and vendors to build and optimize their hardware 
architecture for one specification.

\openshmem{} specification has routines, constants, variables, and language bindings for C and Fortran.
Some of important \openshmem{} operations are as follows:

\begin{enumerate}
\item \textbf{Data Transfers }

\begin{enumerate}
\item One-sided puts : the initiator \ac{PE} (active side) specifies the local
data to be written to the target \ac{PE}'s (passive side) memory. 
\item One-sided gets : an explicit fetch operation is used to copy a variable
amount of data from a remote process and store it locally.\end{enumerate}
\begin{description}
\item [{{Note:}}] By avoiding the need for matching send and receive
calls, \openshmem simplifies the communication process by reducing the
number of calls required to have one \ac{PE} interact with other \ac{PE}s. 
\end{description}
\item \textbf{Synchronization Mechanisms }

\begin{enumerate}
\item Fence: Ensures ordering of PUT operations to a specific \ac{PE}. 
\item Quiet: Ensures ordering of PUT operations to all \ac{PE}s. 
\item Barrier: A collective synchronization routine in which no \ac{PE} may leave
the barrier prior to all \ac{PE}s entering the barrier. 
\end{enumerate}
\item \textbf{Collective Communication}

\begin{enumerate}
\item Broadcast: Copy a block of data from one \ac{PE} to one or more remote
PEs. 
\item Collection: Concatenate elements from the source array to a target
array over the specified \ac{PE}s. 
\item Reduction: Perform an associative binary operation over the specified
\ac{PE}s. 
\end{enumerate}
\item \textbf{Address Manipulation}

\begin{enumerate}
\item Allocating and deallocating memory blocks in the symmetric space.
\end{enumerate}
\item \textbf{Locks}

\begin{enumerate}
\item Implementation of mutual exclusion.
\end{enumerate}
\item \textbf{Atomic Memory Operations}

\begin{enumerate}
\item Swap, Conditional Swap, Add and Increment 
\end{enumerate}
\item \textbf{Data Cache control}

\begin{enumerate}
\item Implementation of mechanisms to exploit the capabilities of hardware
cache if available.
\end{enumerate}
\end{enumerate}
\begin{description}
\item [{{Note:}}] More information about \openshmem routines can be found
in the Library Routines section.
\end{description}

