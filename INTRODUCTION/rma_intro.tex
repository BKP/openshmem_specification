\subsection{Remote Memory Access Operations}
\label{sec:rma}
\ac{RMA} operations described in this section are one-sided communication
mechanisms of the \openshmem{} \ac{API}. While 
using these mechanisms, the programmer is required to provide parameters
only on the calling side. A characteristic of one-sided communication 
is that it decouples communication from
the synchronization. One-sided communication mechanisms transfer 
the data but do not synchronize the sender of the data with the receiver
of the data. 
%Oscar: I commented this because this sentence is useful but for background, but is out of scope for the spec API overview
%However, two-sided communication mechanisms such as Send/Recv have two effects: 
%communication of data, and synchronization of the sender with the receiver.

\openshmem{} \ac{RMA} operations are all performed on the symmetric objects. 
The initiator \ac{PE} of the call is designated as \source{}, and the 
\ac{PE} in which memory is accessed is designated as \target{}. In the case of the remote
update operation, \PUT{}, the origin is the \source{} \ac{PE} and the 
destination \ac{PE} is the \target{} PE. In the case of the remote read operation, \GET{}, 
the origin is the \target{} \ac{PE} and the destination is the \source{} \ac{PE}.

\openshmem{} provides three different types of one-sided communication interfaces. 
\FUNC{shmem\_put$<$bits$>$} interface transfers data in chunks 
of bits. \FUNC{shmem\_put32}, for example, copies data to a \target{} \ac{PE} in chunks of 
32 bits. \FUNC{shmem\_$<$datatype$>$\_put} interface copies elements of type 
\textit{datatype} from a \source{} \ac{PE} to a \target{} \ac{PE}. 
For example, \FUNC{shmem\_integer\_put}, copies elements
of type integer from a \source{} \ac{PE} to a \target{} \ac{PE}. 
\FUNC{shmem\_$<$datatype$>$\_p} interface is similar to \FUNC{shmem\_$<$datatype$>$\_put} 
except that it only transfers one element of type \VAR{datatype}.

\openshmem{} provides interfaces for transferring both contiguous and 
non-contiguous data. The non-contiguous data transfer interfaces are prefixed 
with ``\VAR{i}". \FUNC{shmem\_$<$datatype$>$\_iput} interface, for example, copies strided
data elements from the \source{} \ac{PE} to a \target{} \ac{PE}. 


