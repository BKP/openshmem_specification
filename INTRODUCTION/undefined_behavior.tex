\section{Undefined Behavior in \openshmem}

The specification provides guidelines to the expected behavior of
various library routines. In cases where routines are improperly used
or the input is not in accordance with the specification, undefined
behavior may be observed. Depending on the implementation there are
many interpretations of undefined behavior. 

$\;$

$ $%
\begin{tabular}{|>{\raggedright}p{0.3\textwidth}|>{\raggedright}p{0.6\textwidth}|}
\hline 
\textbf{Inappropriate Usage} & \textbf{Undefined Behavior}\tabularnewline
\hline 
\hline 
Uninitialized library & If \openshmem is not initialized through a call to \linebreak \textbf{start\_pes()},
subsequent accesses to \openshmem routines have undefined results.
An implementation may choose, for example, to try to continue or abort
immediately upon the first call to an uninitialized routine.\tabularnewline
\hline 
Accessing non-existent PEs & If a communications routine accesses a non-existent PE then the \openshmem
library can choose to handle this situation in an implementation-defined
way. For example, the library may issue an error message saying that
the PE accessed is outside the range of accessible PEs, or may exit
without a warning.\tabularnewline
\hline 
Use of non-symmetric variables & Some routines require remotely accessible variables to perform their
function. A {}``put'' to a non-symmetric variable can be trapped
where possible and the library can abort the program. Another implementation
may choose to continue either with a warning or silently.\tabularnewline
\hline 
Non-symmetric Variables & The symmetric memory management routines are collectives, which means
that all PEs in the program must issue the same shmalloc() call with
the same size request. \openshmem implementations should detect the
size mismatch and return error information to the caller. Implementations
may also produce an error message. Program behavior after a mismatched
shmalloc() call is undefined.\tabularnewline
\hline 
\end{tabular}
