\section{Version 1.1}
This section summarizes the changes from the \openshmem specification Version 1.0 to the Version 1.1.  
A major change in this version is that it provides an accurate description of \openshmem interfaces so that they are in agreement with the SGI specification.  This version also explains \openshmem’s programming, memory, and execution model.  The document was throughly changed to improve the readability of specification and usability of interfaces.  The code examples were added to demonstrate the usability of API. Additionally, diagrams were added to help understand the subtle semantic differences of various operations.

%This section summarizes the changes from the \openshmem specification version 1.0 to the version 1.1. 
%The major changes consist of more accurate descriptions of the \openshmem \acp{API} that 
%were part of the original SGI specification, but were omitted in version 1.0e. This new version also provides new
%diagrams that explain the correct behavior of \openshmem synchronization interfaces in terms of ordering, synchronization, delivery and
%completion of operations. Version 1.1 also does a better job at explaining the programming model, memory model and execution model  
%of \openshmem. 

The following list describes the specific changes in 1.1:%\rcomment{\\Eric: "better" is a subjective/ambiguous term, might need need a term to definitively describe how it is better.  i.e. is it more descriptive, more detailed, more accurate, easier to understand? etc.\\}

\begin{itemize}
\item Clarifications on the completion semantics of memory synchronization 
interfaces.\\See Section \ref{subsec:memory_order}.
\item Clarification about completion semantics of memory load and store 
operations in context of \FUNC{shmem\_barrier\_all} and \FUNC{shmem\_barrier} 
routines.\\See Section \ref{subsec:shmem_barrier_all} and \ref{subsec:shmem_barrier}.
\item Clarification about the completion and ordering semantics of \FUNC{shmem\_quiet} and \FUNC{shmem\_fence}.
\\See Section \ref{subsec:shmem_quiet} and \ref{subsec:shmem_fence}.
\item Clarifications about completion semantics of \ac{RMA} and \ac{AMO} routines.
\\See Sections \ref{sec:rma} and \ref{sec:amo}
\item Clarifications on the memory model and the memory alignment requirements for symmetric data objects.
\\See Section \ref{subsec:memory_model}.
\item Clarification on the execution model and the definition of a \ac{PE}.
\\See Section \ref{subsec:execution_model}
\item Clarifications of the semantics of \FUNC{shmem\_pe\_accessible} and \FUNC{shmem\_addr\_accessible}.
\\See Section \ref{subsec:shmem_pe_accessible} and \ref{subsec:shmem_addr_accessible}.
\item Added an annex on interoperability with \ac{MPI}.\\See Annex \ref{sec:mpi}.
% Pasha: this was not part of 1.0 % \item Removed redundant interface \FUNC{shmem\_barrier(void)} from \FUNC{shmem\_barrier\_all}.\\See Section \ref{subsec:shmem_barrier_all}.
\item Added examples to the different interfaces.
\item Clarification on the naming conventions for constant in \Clang{} and \Fortran{}.
\\See Section \ref{subsec:library_constants} and \ref{subsec:shmem_wait}.
\item Added \ac{API} calls: \FUNC{shmem\_char\_p}, \FUNC{shmem\_char\_g}. 
%These calls are part of the SGI specification.
\item Removed \ac{API} calls: \FUNC{shmem\_char\_put}, \FUNC{shmem\_char\_get}. 

\item The usage of \VAR{ptrdiff\_t}, \VAR{size\_t}, and \VAR{int} in the interface signature 
      was made consistent with the description in Sections \ref{subsec:coll}
      \ref{subsec:shmem_iput} \ref{subsec:shmem_iget}

%These calls are not part of the SGI specification. 
\end{itemize}
