\section{Version 1.1}
The aim of this section is to summarize the changes done in 1.1 to the \openshmem specification. 
The major changes consists of a more accurate descriptions of the \openshmem \acp{API} that 
were part of the original SGI specification but were omitted in version 1.0e. This new version also provides new
diagrams that explain their correct behavior of \openshmem synchronization interfaces in terms of ordering, synchronization, delivery and
completion of operations. Version 1.1 also does a better job at explaining the programming model, memory model and execution model  
of \openshmem. The following list describes the specific changes in 1.1:

\begin{itemize}
\item Clarifications on the completion semantics of memory synchronization 
interfaces.\\See Section \ref{subsec:memory_order}.
\item Clarification about completion semantics of memory load and store 
operations in context of \FUNC{shmem\_barrier} and \FUNC{shmem\_barrier\_all}
routines.\\See Section \ref{subsec:shmem_barrier_all} and \ref{subsec:shmem_barrier}.
\item Clarification about the completion and ordering semantics of \FUNC{shmem\_quiet} and \FUNC{shmem\_fence}.
\\See Section \ref{subsec:shmem_quiet} and \ref{subsec:shmem_fence}.
\item Clarifications about completion semantics of \ac{RMA} and \ac{AMO} routines.
\\See Sections \ref{sec:rma} and \ref{sec:amo}
\item Clarifications on the memory model and the memory alignment requirements for symmetric data objects.
\\See Section \ref{subsec:memory_model}.
\item Clarification on the execution model and the definition of a \ac{PE}.
\\See Section \ref{subsec:execution_model}
\item Clarifications of the semantics of \FUNC{shmem\_pe\_accessible} and \FUNC{shmem\_addr\_accessible}.
\\See Section \ref{subsec:shmem_pe_accessible} and \ref{subsec:shmem_addr_accessible}.
\item Added an annex on interoperability with \ac{MPI}.\\See Annex \ref{sec:mpi}.
\item Removed redundant interface \FUNC{shmem\_barrier(void)} from \FUNC{shmem\_barrier\_all}.\\See Section \ref{subsec:shmem_barrier_all}.
\item Added examples to the different interfaces.
\item Clarification on the naming conventions for constant in \Clang and \Fortran.
\\See Section \ref{subsec:library_constants}
\end{itemize}
