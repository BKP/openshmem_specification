
\section{Memory Model}

The \openshmem specification defines how data is stored in the memory
of each PE and how data objects are made remotely accessible to all
other PEs.

Data objects can be stored in a private local memory address or in
a remotely accessible memory address space. Objects in the private
address space can only be accessed by the PE itself; these data objects
cannot be accessed by other PEs via \openshmem routines. Remotely accessible
objects, however, can be accessed by remote PEs using \openshmem routines.
Remotely accessible data objects are also known as Symmetric Objects.
An object is symmetric if it has a corresponding object with the same
type, size and offset on all other PEs. Examples of Symmetric Objects
are static and global variables in C and C++, which are often allocated
at the same address on all PEs where the program is being executed
(\emph{e.g.} in the ELF executable format). See Figure \ref{fig:SymmetricHeap1}
for an example of how Symmetric Memory Objects may be arranged in
memory.

\openshmem routines allow the creation of dynamically allocated Symmetric
data objects. These objects are created in a special memory region
called the Symmetric Heap, which is created during execution at locations
determined by the implementation, meaning the Symmetric Heap may be
in different memory regions on different PEs. \openshmem has nothing
to say regarding the underlying memory layout; it is up to the implementation
to decide how to handle the Symmetric Heap.

%\begin{figure}[H]
%\noindent \begin{centering}
%\includegraphics{media/symmetric_heap}
%\par\end{centering}
%
%\caption{{\small Example of Symmetric Objects}\label{fig:SymmetricHeap1}}
%\end{figure}

