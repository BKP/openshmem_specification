\section{Writing \openshmem Programs}

\subsection{Incorporating \openshmem into Programs}

% C and C++ programs that use the \openshmem library must \texttt{\textbf{}}\lstinline[basicstyle={\ttfamily},language={C++}]!#include <mpp/shmem.h>!.
% Fortran programs should \lstinline[basicstyle={\ttfamily},language=Fortran]!include 'mpp/shmem.fh'!;
% and Fortran programs that use constants defined by \openshmem must
% \lstinline[basicstyle={\ttfamily},language=Fortran]!include 'mpp/shmem.fh'!. 

C and C++ programs that use the \openshmem library \emph{must}

\begin{lstlisting}[language=C++]
#include <shmem.h>
\end{lstlisting}

All Fortran \openshmem programs \emph{should}

\begin{lstlisting}[language=Fortran]
include 'shmem.fh'
\end{lstlisting}

and Fortran \openshmem programs that use constants defined by \openshmem
\emph{must}

\begin{lstlisting}[language=Fortran]
include 'shmem.fh'
\end{lstlisting}

\subsubsection{Compatibility Note}

Implementations \emph{must} also provide these header files so that
they can be referenced in C and C++ as

\begin{lstlisting}[language=C++]
#include <mpp/shmem.h>
\end{lstlisting}

and in Fortran as

\begin{lstlisting}[language=Fortran]
include 'mpp/shmem.fh'
\end{lstlisting}

for backward compatbility with OpensHMEM 1.0 and SGI SHMEM.

\subsection{Initialization}

An \openshmem program must call \textbf{start\_pes() }(See Section
\prettyref{sub:start_pes}) before any other \openshmem routine. If
\textbf{start\_pes()} is not called first, the subsequent behavior of
\openshmem is undefined.  Calling \textbf{start\_pes()} more than once
has no subsequent effect. The parameter to \textbf{start\_pes()} is
ignored; the number of PEs is taken from the invoking environment.
