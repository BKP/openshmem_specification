\bAPI{SHPALLOC}{Allocates a block of memory from the symmetric heap.}
\synF   %Synopsis for FORTRAN API

POINTER (addr, A(1))
INTEGER length, errcode, abort
CALL SHPALLOC(addr, length, errcode, abort) %*\synFE   %DO NOT DELETE. THIS LINE IS NOT A COMMENT  

% Arguments table. If no arguments you can use \argRow{None}{}{}
\desB{  
 \argRow{OUT}{addr}{First word address of the allocated block.}
 \argRow{IN}{length}{Number of words of memory requested. One word is 32 bits.}
 \argRow{OUT}{errcode}{Error code is \CONST{0} if no error was detected; otherwise, it is a negative integer code for the type of error.}
 \argRow{IN}{abort}{Abort code; nonzero requests abort on error; \CONST{0}  requests
		      an error code.}
}
 %API description
 {   
       \FUNC{SHPALLOC} allocates a block of memory from the program's symmetric heap
       that is greater than or equal to the size requested. To maintain symmetric heap 
       consistency, all \ac{PE}s in an program must call \FUNC{SHPALLOC} with the same value of length; if any  \ac{PE}s are missing, the program will hang.
       
       By using the \Fortran{} \CONST{POINTER} mechanism in the following manner, you can use array \VAR{A} to refer to the block allocated by \FUNC{SHPALLOC}: \CONST{POINTER} (\VAR{addr},
       \VAR{A}())
  }
 %API Description Table. 
{
{
 %Return Values     
\desR{ }
\cRow{Error Code} {Condition}
\cRow{ \CONST{-1} } {Length is not an integer greater than \CONST{0}}
\cRow{\CONST{-2}} { No more memory is available from the system (checked if the  request cannot be satisfied from the available blocks on the symmetric heap).}
}%end of DesR
\notesB{  
       The total size of the symmetric heap is determined at job startup.  One
       may adjust the size of the heap using the \CONST{SMA\_SYMMETRIC\_SIZE}
       environment variable (if available).	
}
        \notesImp{
        The symmetric heap allocation functions always return a pointer to
        corresponding symmetric objects across all PEs. The \openshmem{} 
        specification does not require that the virtual addresses are equal across
        all \acp{PE}. Nevertheless, the implementation must avoid costly address
        translation operations in the communication path, including order $N$ (where
        $N$ is the number of \acp{PE}) memory translation tables. 
        In order to avoid address translations, the implementation may
        re-map the allocated block of memory based on agreed virtual address. 
        Additionally, some operating systems provide an option to disable 
        virtual address randomization, which enables predictable allocation
        of virtual memory addresses.
        }
}
%Example
%\exampleB{}
\eAPI 
