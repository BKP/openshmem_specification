\bAPI{SHMEM\_INFO\_GET\_NAME}{This routine returns the vendor defined character string.}
\synC   %Synopisis for C API

void shmem_info_get_name(char *name); %*\synCE    %DO NOT DELETE. THIS LINE IS NOT A COMMENT

\synF   %Synopsis for FORTRAN API

SHMEM_INFO_GET_NAME(NAME)   
CHARACTER *(*)NAME %*\synFE   %DO NOT DELETE. THIS LINE IS NOT A COMMENT 

% Arguments table. If no arguments you can use \argRow{None}{}{} 
\desB{  
    \argRow{OUT}{name}{The vendor defined string.}
 }
%API description
{
  This routine returns the vendor defined character string of size defined by the 
  constant SHMEM\_MAX\_NAME\_LEN. The program calling this function 
  prepares the memory of size SHMEM\_MAX\_NAME\_LEN, and the implementation copies the 
  string of size at most SHMEM\_MAX\_NAME\_LEN. In C, the string is terminated by a null character. 
  In Fortran, the string of size less than SHMEM\_MAX\_NAME\_LEN is padded with blank characters up to 
  size SHMEM\_MAX\_NAME\_LEN. The implementation copying a string of size greater than SHMEM\_MAX\_NAME\_LEN results 
  in an undefined behavior.
  Multiple invocations of the 
  routine in an \openshmem{} program always return the same string. 
  For a given library implementation, the major and minor version returned by these calls is consistent with the compile-time constants defined in its shmem.h.
}
%This newline is required 
{
%API Description Table.
\desR{
    %Return Values    
    None.
}
% Notes. If there are no notes, this field can be left empty.
\notesB{ None.
 }
}
\eAPI

