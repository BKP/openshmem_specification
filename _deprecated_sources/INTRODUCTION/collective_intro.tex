%Comments for Manju:
%Are Barrier_all, shmalloc, etc considered collectives? If yes, we need to state that all PEs belong to an implicit active set that contains all PES.
%Also describe the case where collective operations may be invoked by the implicit active set (all PEs) or active sets PEs
%State that collectives can be executed on statements that are not ordered .
%State that from the beginning to the end of the program, the sequence of collectives should be the same on a given active set (or implicit active set).
%Are arguments  the same for PEs that call the same collective? (i.e. target or source symmetric data?)
%Which collectives imply synchronization (i.e. barrier, quiet, etc) which ones not (i.e. broadcast on root?)

\emph{Collective routines} are defined as communication or synchronization operations 
on a group of \acp{PE} called an \activeset{}. The collective routines require all
\acp{PE} in the \activeset{} to simultaneously call the routine. 
A \ac{PE} that is not part of the \activeset{} calling the collective 
routines results in an undefined behavior.  All
collective routines have an \activeset{} as an input parameter except
\barrierall{}. The \barrierall{} is called by all \acp{PE} of the \openshmem{} program. 

The \activeset{} is defined by the arguments \VAR{PE\_start}, \VAR{logPE\_stride}, 
and \VAR{PE\_size}.  \VAR{PE\_start} is the starting \ac{PE} number, a log (base
2) of \VAR{logPE\_stride} is the stride between \acp{PE}, and \VAR{PE\_size} is
the number of \acp{PE} participating in the \activeset{}.  All \acp{PE} participating in the 
collective routines provide the same values for these arguments. 
 
Another argument important to collective routines is \VAR{pSync}, which is a
symmetric work array.  All \acp{PE} participating in a collective must pass the same
\VAR{pSync} array.  On completion of a collective call, the \VAR{pSync} is restored to its 
original contents.  The user is permitted to reuse a \VAR{pSync} array if
all previous collective routines using the \VAR{pSync} array have been completed by all participating 
\acp{PE}.  One can use a synchronization collective routine such as \barrier{}
to ensure completion of previous collective routines. The \FUNC{shmem\_barrier} routine allows the same \VAR{pSync} array to be used on consecutive calls as long as the \ac{PE} \activeset{} does not change. 

%The two cases below
%show the reuse of \VAR{pSync} array:
%
%\begin{itemize}
%\item The \FUNC{shmem\_barrier} function allows the same \VAR{pSync} array to be used on consecutive calls as long as the active \ac{PE} set does not change.
%\item  If the same collective function is called multiple times with the
%          same \activeset, the calls may alternate between two \VAR{pSync} arrays.
%          The \openshmem functions guarantee that a first call is completely finished by 
%          all \ac{PE}s by the time processing of a third  call  begins  on any \ac{PE}.          
%\end{itemize}


All collective routines defined in the specification are blocking.  The 
collective routines return on completion.  The collective routines defined in the \openshmem{} specification 
are:

\begin{itemize}
\item[] \broadcast{} 
\item[] \barrier{}
\item[] \barrierall{}
\item[] \collect{}
\item[] \reduction{}
\end{itemize} 
