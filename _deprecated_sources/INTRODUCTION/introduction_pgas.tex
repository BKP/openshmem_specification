
\section{What is \openshmem?}

This section is an introduction to previous work on \openshmem. We begin
with a quick overview of the Partitioned Global Address Space model,
which is the basis for \openshmem's data sharing strategy.


\subsection{Partitioned Global Address Space}

Conventional Parallel Programming Models can be broadly classified
into 2 types: 
\begin{description}
\item [{{Shared-Memory~Model:}}] in this model all processors interact
with a globally available memory space. 
\item [{{Distributed-Memory~Model:}}] in this model each processor has
its own memory to work with and can only directly access the data
that resides in its memory. When a processor needs data from another
processor an explicit function call must be made to communicate with
the target processor. 
\end{description}
The current high performance computing architectures prefer a combination
of the above mentioned memory models, which is referred to \textbf{Partitioned
Global Address Space} or PGAS for short. In PGAS, each processing
element (\ac{PE}) has access to its own private local memory and also to
a shared memory space. This programming model enhances performance
by exposing data/thread locality. PGAS programming languages include
\textbf{Unified Parallel \Clang (UPC)}, \textbf{Co-Array \Fortran (CAF)},
\textbf{Titanium}, \textbf{X-10} and \textbf{Chapel}.

More information about PGAS can be found at the PGAS Forum website.\cite{pgasfor}


\subsection{\openshmem}

% SGI asked for this to be removed to protect the trademark
%
%      SHMEM stands for \textbf{SH}ared \textbf{MEM}ory.

\openshmem is a library \ac{API} that allows its participating processes (the
places where work occurs are called Processing Elements or \ac{PE}s) to
view a Partitioned Global Address Space. Each \ac{PE} is able to see
variables with a common name, but each \ac{PE} has its own local copy
of the variable.

The \openshmem library provides inter-processor communication using data
passing and one-sided communication techniques. \openshmem differs from
the Message Passing Interface (MPI), currently the most widely used
communication model, in that the latter generally uses two-sided communication
(MPI now also includes one-sided calls). In two-sided communication,
both sides of the exchange (source and destination) are required to
participate actively. The one-sided communication mechanism decouples
data transfer and synchronization, reducing communication overhead,
resulting in faster communication patterns. Figure \ref{fig:Communication-Scheme}
shows diagrams for one-sided and two-sided communications.\medskip{}


%\begin{center}
%\begin{figure}[H]
%\begin{centering}
%\includegraphics[scale=0.7]{media/communication} 
%\par\end{centering}
%
%\caption{Communication Scheme\label{fig:Communication-Scheme}}
%\end{figure}
%
%\par\end{center}

\medskip{}
The following are some of the communication operations available in
\openshmem:
\begin{enumerate}
\item \textbf{Data Transfers }

\begin{enumerate}
\item One-sided puts : the initiator \ac{PE} (active side) specifies the local
data to be written to the target \ac{PE}'s (passive side) memory. 
\item One-sided gets : an explicit fetch operation is used to copy a variable
amount of data from a remote process and store it locally.\end{enumerate}
\begin{description}
\item [{{Note:}}] By avoiding the need for matching send and receive
calls, \openshmem simplifies the communication process by reducing the
number of calls required to have one \ac{PE} interact with other \ac{PE}s. 
\end{description}
\item \textbf{Synchronization Mechanisms }

\begin{enumerate}
\item Fence: Ensures ordering of PUT operations to a specific \ac{PE}. 
\item Quiet: Ensures ordering of PUT operations to all \ac{PE}s. 
\item Barrier: A collective synchronization routine in which no \ac{PE} may leave
the barrier prior to all \ac{PE}s entering the barrier. 
\end{enumerate}
\item \textbf{Collective Communication}

\begin{enumerate}
\item Broadcast: Copy a block of data from one \ac{PE} to one or more target
\ac{PE}s. 
\item Collection: Concatenate elements from the source array to a target
array over the specified \ac{PE}s. 
\item Reduction: Perform an associative binary operation over the specified
\ac{PE}s. 
\end{enumerate}
\item \textbf{Address Manipulation}

\begin{enumerate}
\item Allocating and deallocating memory blocks in the symmetric space.
\end{enumerate}
\item \textbf{Locks}

\begin{enumerate}
\item Implementation of mutual exclusion.
\end{enumerate}
\item \textbf{Atomic Memory Operations}

\begin{enumerate}
\item Swap, Conditional Swap, Add and Increment 
\end{enumerate}
\item \textbf{Data Cache control}

\begin{enumerate}
\item Implementation of mechanisms to exploit the capabilities of hardware
cache if available.
\end{enumerate}
\end{enumerate}
\begin{description}
\item [{{Note:}}] More information about \openshmem routines can be found
in the Library Routines section.
\end{description}

\subsection{History of \openshmem}
\begin{description}
\item [{{Cray~SHMEM~(MP-SHMEM,~LC-SHMEM):}}] Cray first introduced
SHMEM in 1993 for its Cray T3D systems. Cray SHMEM was also used in
other models: T3E, PVP and XT series. 
\item [{{SGI~SHMEM~(SGI-SHMEM):}}] Cray Research merged with Silicon
Graphics (SGI) in February 1996. At this point SHMEM was incorporated
into SGI's Message Passing Toolkit (MPT). The platforms supported
were - SGI Irix, Origin and Altix. 
\item [{{Quadrics~SHMEM~(Q-SHMEM):}}] an optimized API for the Quadrics
QsNet interconnect. It included SGI extensions and provided non-blocking
puts and gets. A joint effort from HCS Lab \& Quadrics incorporated
a program profiling interface called PSHMEM that can aid in the execution
analysis of SHMEM programs. 
\end{description}
The success of SHMEM's performance attracted several vendors to provide
implementations (with varying names and features) for their systems.
Some of them include: 
\begin{description}
\item [{{HP~SHMEM:}}] Based on the Quadrics API. It is included in the
UPC product kit. 
\item [{{Cyclops-64~SHMEM~(C64-SHMEM):}}] this SHMEM API supports the
Cyclops-64 architecture. Most of the core features of Cray SHMEM are
available with some additional interfaces specific to the Cyclops-64
architecture. %
%\begin{comment}
%Forcing new line, can't get \LaTeX{} to do it automatically.
%\end{comment}

\item [{{IBM~SHMEM:}}] An implementation created by IBM intended for
internal use only. 
\item [{{TurboSHMEM:}}] This implementation uses IBM's Low-Level \ac{API}
(LAPI) technology to obtain optimized one-sided communication for
the put/get operations. This allows applications written with the
SHMEM API to run on IBM platforms with minimal source code changes. 
\item [{{GPSHMEM:}}] This implementation of SHMEM aims at providing full
portability of applications. It is built mostly with Cray T3D components
and functionalities and provides MPI and ARMCI support. This project
is no longer maintained. \end{description}
