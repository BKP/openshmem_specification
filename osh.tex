% Version as of April 8, 1997

%   ----------------------------------------------------------------------
%   osh.tex inspired by mpi-macs.tex  --- man page macros,
%                discuss, missing, mpifunc macros
%
% ----------------------------------------------------------------------

% TeX if definitions.  These are defined here so (a) there is one place to
% change the defaults and (b) so that they can be changed by reading a
% configuration file
% Control the use of color
\newif\ifusecolor
\usecolortrue
% Control whether changes are highlighted
\newif\ifshowchange
\showchangetrue
% Control whether deleted text is shown
\newif\ifshowdelete
\showdeletetrue
% Control the use of change "bars" (really markers).  Note that
% both changetrue and changebarstrue must not be set
\newif\ifchangebars
\changebarsfalse
% Control whether tickets are indicated in the margins (or inline, when 
% they occur in inner mode.  Note that this has no effect if \showchangetrue
% is not set
\newif\ifshowtickets
\showticketstrue
% For publisher's additions in the printed book
\newif\ifbookprinting
\bookprintingfalse

%
% There are a number of features that are controlled by LaTeX if commands.
% This step allows you to control these through a configuration file
% This file should contain valid LaTeX commands, including comments (using
% the standard % character for comments)
%
% Known commands include:
% \showchangetrue   - Show changes/additions (from a previous version)
% \showdeletetrue   - Show deleted text (deleted from a previous version)
% \changebarstrue   - Show change "bars" around changes (really begin/end
%                     markers in the output file)
% \usecolortrue     - Use color to show changes
% 
\newread\cfgin

%
% General text color update macros
% These permit the use of nesting of the color changes, as well as
% a "do not change" option
%
\ifusecolor
\definecolor{orange}{rgb}{1,0.5,0}
\definecolor{purple}{rgb}{0.8,0,1}
\let\XA=\expandafter
% Definition for "Use current color"
\def\ColorSame{same}
% Create a stack that is 5 deep (no arrays in TeX)
\def\ColorS{black}
\def\ColorSi{same}
\def\ColorSii{same}
\def\ColorSiii{same}
\def\ColorSiiii{same}
% Stack pointer
\newcount\ColorStackP 
\ColorStackP=0
% Push a new color
\def\ColorPush#1{%
\global\advance\ColorStackP 1\relax%
\ifnum\ColorStackP>4\message{Font Color stack too deep}\fi%
\ifnum\ColorStackP=1\global\def\ColorSi{#1}\else
\ifnum\ColorStackP=2\global\def\ColorSii{#1}\else
\ifnum\ColorStackP=3\global\def\ColorSiii{#1}\else
\ifnum\ColorStackP=4\global\def\ColorSiiii{#1}%
\fi
\fi
\fi
\fi
\def\ColorCur{#1}%
\ifx \ColorCur\ColorSame \relax\else \color{#1}\fi
}
% Pop that color
\def\ColorPop{\global\advance\ColorStackP -1\relax%
\ifnum\ColorStackP<0\message{Font Color stack < 0}\fi%
\ifnum\ColorStackP=0\def\ColorCur{\ColorS}\else
\ifnum\ColorStackP=1\def\ColorCur{\ColorSi}\else
\ifnum\ColorStackP=2\def\ColorCur{\ColorSii}\else
\ifnum\ColorStackP=3\def\ColorCur{\ColorSiii}\else
\ifnum\ColorStackP=4\def\ColorCur{\ColorSiiii}%
\fi
\fi
\fi
\fi
\fi%\typeout{cur = \ColorCur and same = \ColorSame}%
\ifx\ColorCur\ColorSame\relax\else\color{\ColorCur}\fi%
}
% A synonym for color that is controlled by the \usecolortrue command
\def\Color#1{\color{#1}}
\else
\def\ColorPush#1{}
\def\ColorPop{}
\def\Color#1{}
\fi % \ifusecolor

\ifshowdelete
\def\nocomment{\catcode`\%=9}
\def\restorecomment{\catcode`\%=14}
\else
\let\nocomment=\relax
\let\restorecomment=\relax
\fi

\ifchangebars
\def\BegChange{\begchange}
\def\EndChange{\endchange}
\else
\let\BegChange=\relax
\let\EndChange=\relax
\fi

\ifshowchange
%
% Use margin par for the ticket number unless marginpar won't work, in 
% which case inline the ticket number.  To do more would require special
% code for the TeX output routine, which isn't worth it for what we need 
% here.
% Note that the changebars and the ticket both use marginpar, and if both 
% are used at the same time, LaTeX may run out of floats
\ifchangebars
\def\ticket#1{\relax}
\else
\ifshowtickets
\def\ticket#1{\ifinner[ticket#1.]\else\protect\marginpar[\mbox{\hbox to \marginparwidth{\hss ticket#1.\hspace{30pt}}}]{\hbox to \marginparwidth{\hspace{30pt}ticket#1.\hss}}\fi}
\else
\def\ticket#1{\relax}
\fi % showtickets
\fi % changebars
\fi

% fancyvrb defines Verbatim, which is a slightly better Verbatim environment
% Regrettably, the key feature needed, commandchars, does not work correctly
% in our environment (it doesn't accept arbitrary grouping characters).
% See README-2.2 for instructions on using Verbatim along with the above 
% update macros.
%\usepackage{fancyvrb}


\def\snir{\relax}
\def\rins{\relax}


% To make the changes without showing the location or old source: 
%    \newcommand{\CHANGE}[2]{}
%    \newcommand{\INTO}[1]{#1}
%    \newcommand{\ADD}[2]{#2}
 %   \newcommand{\DELETE}[2]{}

 
% available: red green blue cyan yellow magenta
 \def\RVWCAP/{}                                                       % shortcut for Review item 23.a - capitalization of titles (in \section}
% \def\RVWcap/{\mpiiidotiMergeFromREVIEWbegin{23.a}}                   % shortcut for Review item 23.a - capitalization of titles
\def\RVWcap/{}                                                       % shortcut for Review item 23.a - capitalization of titles
 
\def\OnlyForAutomaticAnnexGeneration#1{}% deleting the content were the macro is used; but preserving it for the Annex

 
% This macro enables that all "_" (underscore) characters in the pfd
% file are searchable, and that cut&paste will copy the "_" as underscore. 
% Without the following macro, the \_ is treated in searches and cut&paste
% as a " " (space character). 
% This macro does not modify the behavior of _ in math or in verbatim 
% environments. In verbatim environments, the "_" is always treated
% as a searchable character.
%
\DeclareRobustCommand{\_}{\texttt{\char`\_}} 
% 

% From MPI-2.0
% ------------
 
% Place some penalty for doing the break
% The penalty for a ``\gb'' should be greater than a \hyphenpenalty.
% \hyphenpenalty is 50 in plain.tex.
\def\gb{\penalty10000\hskip 0pt plus 8em\penalty4800\hskip 0pt plus-8em%
\penalty10000}

% A theorem-like environment for code Examples (S. Otto) see Lamport, pg 58
%
% Note that because we use a theorem environment that resets the counter
% with every chapter, pdflatex will issue a warning for each example that
% has the same number as an example in another chapter.  This is too hard
% to fix (the easist way is to not use the theorem environment and roll
% a custom environment, including adding the necessary low-level commands
% for the pdf link support.
%\newtheorem{example}{Example}[chapter]
% Theorems have \em text; we want \rm.  The easiest way to fix this, 
% since we are not using Theorems, is to change the @begintheorem macro
\makeatletter
\def\@begintheorem#1#2{\rm \trivlist \item[\hskip \labelsep{\bf #1\ #2}]}
\makeatother
% Use \exindex{MPI\_FUNC} to generate an index entry for MPI
% functions/constants 
%\newcommand{\exindex}[1]{\relax}
%\newcommand{\exindex}[1]{\index{EXAMPLES:#1}}

% a couple of commands from Marc Snir, modified S. Otto

%\newlength{\discussSpace}
%\setlength{\discussSpace}{.4cm}

%\newenvironment{funcdef}[1]{
%    \vspace{\codeSpace}
%    \noindent
%    \samepage
%    \hangindent 7em\hangafter=1
%    {\funcNoIndex{{\prefix}#1}}\mpifuncmainindex{#1}
%    \MPIfunclist
%}{\end{list} \vspace{\codeSpace}}

%\newcommand{\function}[1]{{\raggedright \hangindent 7em\hangafter=1\tt #1 \par \vspace{0.1in}}}
%\newcommand{\CM}{Communication Middleware}
% Watermark
%
% For release please remove waternark and lino
% DRAFT !!!
% \usepackage{draftwatermark}

% max depth for table of content
\setcounter{tocdepth}{4}
% set path to figures
\graphicspath{{./figures/}}

\usepackage{xspace}
%\sloppy
%\newcommand{\openshmem} {\mbox{OpenSHMEM}\xspace % \textsuperscript{{\small \texttrademark}}
%}
\newcommand{\openshmem} {{Open\-SHMEM}\xspace}
\newcommand{\insertDocVersion}{1.1 DRAFT}
\newcommand{\deprecate}[1]{\textcolor{Gray}{#1 (deprecated)}}
%\renewcommand\linenumberfont{\normalfont\scriptsize\sffamily}

\hypersetup{pdftitle={\openshmem Specification Draft},
 pdfauthor={HPC Tools, University of Houston},
 pdfkeywords={Specification, Draft, \openshmem, SHMEM, PGAS, Partitioned, Global, Address, Gasnet, Parallel}}
 
\definecolor{gray}{rgb}{0.92,0.92,0.92}
\lstset{ %
  breakatwhitespace=false,         % sets if automatic breaks should only happen at whitespace
  basicstyle=\ttfamily\footnotesize,
  %identifierstyle=\ttfamily\itshape,
  breaklines=true,                 % sets automatic line breaking
  escapeinside={\%*}{*)},          % if you want to add LaTeX within your code
  extendedchars=true,              % lets you use non-ASCII characters; for 8-bits encodings only, does not work with UTF-8
  %frame=single,                    % adds a frame around the code
  keepspaces=true,                 % keeps spaces in text, useful for keeping indentation of code (possibly needs columns=flexible)
  %keywordstyle=\color{blue},       % keyword style
  %language=Octave,                 % the language of the code
  morekeywords={*,...},            % if you want to add more keywords to the set
 % numbers=left,                    % where to put the line-numbers; possible values are (none, left, right)
  %numbersep=5pt,                   % how far the line-numbers are from the code
  showspaces=false,                % show spaces everywhere adding particular underscores; it overrides 'showstringspaces'
  showstringspaces=false,          % underline spaces within strings only
  showtabs=false,                  % show tabs within strings adding particular underscores
  %tabsize=2,                       % sets default tabsize to 2 spaces
  %title=\lstname,                   % show the filename of files included with \lstinputlisting; also try caption instead of title
}\newcommand{\discuss}[1]{}

% long discussion
\bgroup\catcode`\{=10\catcode`\}=10\catcode`\[=1\catcode`\]=2\long\gdef\Eatdiscussion#1end{discussion}[\relax\end[discussion]]\relax\egroup\newenvironment{discussion}{\bgroup\def\do##1{\catcode`##1=10}\dospecials\Eatdiscussion}{\egroup}

\newcommand{\missing}[1]{}

\newcommand{\alter}[1]{}

\newcommand{\status}[1]{}

% special comment command for last round
%\newcommand{\question}[1]{\vspace{\discussSpace} {\small {\bf Question:} #1} \vspace{\discussSpace}}
\newcommand{\question}[1]{}
%%%%%%%%%%%%%%%%%%%%%%%%%%%%%%%%%%%%%%%%%%%%%%%%%%%%%%%%%
%
% Use this to make pages start on the right side
%\newcommand{\startchap}[0]{\relax}

%\newlength{\codeSpace}
%\setlength{\codeSpace}{.4cm}


\def\RMA/{\textsf{RMA}}    % RMA macro -- see TEX Book, pg 8,204 bloody TEX!!

\newcommand{\uu}[1]{\underline{\hyperpage{#1}}} 
\newcommand{\typedefindex}[1]{\index{TYPEDEF:#1}}
\newcommand{\cdeclindex}[1]{\index{CONST:#1}}        %  for index entry of C declarations like MPI_Comm
\newcommand{\cdeclmainindex}[1]{\index{CONST:#1|uu}}
 
\def\MPIfunclist{
    \begin{list}{}{                     % see pg 113 of Lamport's book
        \setlength{\leftmargin}{105pt}
        \setlength{\labelwidth}{80pt}
        \setlength{\labelsep}{10pt}
        \setlength{\itemindent}{0pt}
        \setlength{\itemsep}{0pt}
        \setlength{\topsep}{2pt}
    }
}

\newlength{\codeSpace}
\setlength{\codeSpace}{.4cm}

\newcommand{\IN}[0]{{\small IN}}
\newcommand{\OUT}[0]{{\small OUT}}
\newcommand{\INOUT}[0]{{\small INOUT}}
\newcommand{\prefix}[0]{{LLC\_}}

\newcommand{\mpiarg}[1]{\gb\textsf{#1}} 

\newcommand{\funcarg}[3]{\item[\hbox to 30pt{\textsf{#1} \hfill} \mpiarg{#2}\hfill]{\small #3}}

\def\gb{\penalty10000\hskip 0pt plus 8em\penalty4800\hskip 0pt plus-8em%
\penalty10000}
\newcommand{\funcNoIndex}[1]{\gb\textsf{#1}}
\newcommand{\mpifuncmainindex}[1]{\index{#1|uu}}

% special for functions that you don't want listed in index.
% This was added for showing corrections to functions already listed.
\newenvironment{funcdefnolist}[1]{      
    \vspace{\codeSpace}
    \vspace{\codeSpace}
    \noindent
    \samepage
    {\func{#1}}
    \begin{list}{}{                     % see pg 113 of Lamport's book
        \setlength{\leftmargin}{200pt} 
        \setlength{\labelwidth}{180pt} 
        \setlength{\labelsep}{10pt} 
        \setlength{\itemindent}{0pt}
        \setlength{\itemsep}{0pt}
        \setlength{\topsep}{5pt}
    }
}{\end{list} \vspace{\codeSpace}}


\newenvironment{ffuncdef}[1]{   
    \vspace{\codeSpace}
    \noindent
    Fortran binding:

    \noindent
    \samepage
    {\ffunc{#1}}
    \begin{list}{}{                     % see pg 113 of Lamport's book
        \setlength{\leftmargin}{200pt} 
        \setlength{\labelwidth}{180pt} 
        \setlength{\labelsep}{10pt} 
        \setlength{\itemindent}{0pt}
        \setlength{\itemsep}{0pt}
        \setlength{\topsep}{5pt}
    }
}{\end{list} \vspace{\codeSpace}}

                                        % see page 77, the TeX book.
\newcommand{\cfunc}[1]{\gb\textsf{#1}}
\newcommand{\ffunc}[1]{\gb\textsf{#1}}
\newcommand{\const}[1]{\protect\gb\protect{\textsf{\small #1}}\index{CONST:#1}}
\newcommand{\constskip}[1]{\protect\gb\protect{\textsf{\small #1}}}
% for ones from  MPI-1
\newcommand{\consti}[1]{\protect\gb\protect{\textsf{\small #1}}\index{CONST:#1}} % constants/handles - language independent
%\newcommand{\consti}[1]{\protect\gb\protect{\small\sf #1}\index{CONST:#1}} % constants/handles - language independent
\newcommand{\constiskip}[1]{\protect\gb\protect{\textsf{\small #1}}}             % ... same, but not in the Constant Index
\newcommand{\constitemtwo}[3]{\item[\const{#1}, \const{#2}\hfill]{#3}}
\newcommand{\constitemthree}[4]{\item[\const{#1}, \const{#2}, \const{#3}\hfill]{#4}}
% for ones that don't go in index
\newcommand{\constskipitem}[2]{\item[\constskip{#1}\hfill]{#2}}
% \newcommand{\carg}[1]{\gb\textsf{#1}}                  % currently not used
% \newcommand{\farg}[1]{\gb\textsf{#1}}                  % currently not used
\newcommand{\type}[1]{\gb\textsf{#1}\index{CONST:#1}}  % datatype handles
%
\newcommand{\gtype}[1]{\textsf{#1}} % generic (language independent) type
\newcommand{\shorttype}[1]{\textsf{#1}\index{CONST:#1}}  % ... same but without \gb panelty
\newcommand{\ctype}[1]{\gb\texttt{#1}}                 % - and corresponding C type
\newcommand{\ftype}[1]{\gb\texttt{#1}}                 % - and corresponding Fortran type
%
% Info is for MPI_Info predefined strings.  \infokey{keyname} and
% \infoval{keyvaluename}
\newcommand{\info}[1]{\protect\gb\protect{\small\sf #1}\index{CONST:#1}}
\newcommand{\infoval}[1]{\protect\gb\protect{\small\sf #1}\index{CONST:#1}}
\let\infokey=\infoval
\newcommand{\infoskip}[1]{\protect\gb\protect{\small\sf #1}}
%
\newcommand{\error}[1]{\protect\gb\protect{\small\sf #1}\index{CONST:#1}}
\newcommand{\errorskip}[1]{\protect\gb\protect{\small\sf #1}}
\newcommand{\errori}[1]{\protect\gb\protect{\small\sf #1}}


\def\class{$\langle$CLASS$\rangle$}

\newenvironment{constlist}[0]{  
    \vspace{\codeSpace}
    \noindent
    \begin{list}{}{                     % see pg 113 of Lamport's book
        \setlength{\leftmargin}{200pt} 
        \setlength{\labelwidth}{190pt} 
        \setlength{\labelsep}{10pt} 
        \setlength{\itemindent}{10pt}
        \setlength{\itemsep}{-5pt}
        \setlength{\topsep}{-5pt}
    }
}{\end{list} \vspace{\codeSpace}}

%       some commands from Bill Gropp

\def\code#1{\texttt{#1}}
\def\setmargin#1{\begingroup\leftmargin #1 \advance\leftmargin\labelsep 
                 \leftmargini #1 \advance\leftmargini\labelsep}
\def\esetmargin{\endgroup}
\def\ibamount{3.0cm\relax}
\def\ibaamount{4.0cm}
\def\ibdamount{4.5cm}
\def\ibcamount{2.0cm}
\def\ib#1{\hbox to \ibamount{#1\hfil}}
\def\iba#1{\hbox to \ibaamount{#1\hfil}}
\def\ibd#1{\hbox to \ibdamount{#1\hfil}}
\def\ibc#1{\hbox to \ibcamount{#1\hfil}}

% Use \code{...} for code fragments
%\def\code#1{\texttt{#1}}
% Use \df{name} for a definition of name in the text
\def\df#1{{\bf #1}}
% Use \note{text} for marginal notes
\def\note#1{\marginpar{\bf #1}}

%
% Get line numbers in the gutters.  Thanks to Guy Steele and HPFF!
%

\makeatletter
%
% This is used to put line numbers on plain pages.  Used in draft.tex
%
\def\withlinenumbers{\relax
  \def\@evenfoot{\hbox to 0pt{\hss\LineNumberRuler\hskip 1.5pc}\hfil}\relax
  \def\@oddfoot{\hfil\hbox to 0pt{\hskip 1.5pc\LineNumberRuler\hss}}}

\def\LineNumberRuler{\vbox to 0pt{\vss\normalsize \baselineskip13.6pt
    \lineskip 1pt \normallineskip 1pt \def\baselinestretch{1}\relax
    \LNR{1}\LNR{2}\LNR{3}\LNR{4}\LNR{5}\LNR{6}\LNR{7}\LNR{8}\LNR{9}
    \LNR{10}\LNR{11}\LNR{12}\LNR{13}\LNR{14}
        \LNR{15}\LNR{16}\LNR{17}\LNR{18}\LNR{19}
    \LNR{20}\LNR{21}\LNR{22}\LNR{23}\LNR{24}
        \LNR{25}\LNR{26}\LNR{27}\LNR{28}\LNR{29}
    \LNR{30}\LNR{31}\LNR{32}\LNR{33}\LNR{34}\LNR{35}
        \LNR{36}\LNR{37}\LNR{38}\LNR{39}
    \LNR{40}\LNR{41}\LNR{42}\LNR{43}\LNR{44}
        \LNR{45}\LNR{46}\LNR{47}\LNR{48}
    \vskip 31pt}}
\def\LNR#1{\hbox to 1pc{\hfil\tiny#1\hfil}}

% jmm; merge the withlinenumbers stuff into
% the centered page numbers that tex defines by default
\def\ps@plainwithlinenumbers{\let\@mkboth\@gobbletwo
     \def\@oddhead{}
     \def\@oddfoot{\hfil\rm\thepage\hfil
       \hbox to 0pt{\hskip 1.5pc\LineNumberRuler\hss}}
     \def\@evenhead{}
     \def\@evenfoot{\hbox to 0pt{\hss
     \LineNumberRuler\hskip 1.5pc}\rm\hfil\thepage\hfil}}

% The old version; uncommenting the withlinenumbers part didn't
% work because that macro replaced the footer definition that
% did page numbering.
%\def\ps@plainwithlinenumbers{\ps@plain}%\withlinenumbers}
% end jmm changes

%
% 1st page of a chapter has its own page style, so we have to put line
% numbers in here also.
%
\newwrite\chappages
\immediate\openout\chappages=chappage.txt
\def\writespace{ }
%
% Contents is done with \chapter*{Contents}, so we need to turn off the
% line numbers in this case.  Easiest to look at def
%
\def\incontents{0}
\newif\ifcontents
\contentsfalse
\def\chapter{\clearpage \ifcontents\else\thispagestyle{plainwithlinenumbers}\fi
        \write\chappages{Chapter \thechapter\writespace - \the\count0}
        \global\@topnum\z@ \@afterindentfalse \secdef\@chapter\@schapter}

%
% Change "Chapter" to "Chapter", "Appendix" to "Annex"
%
\renewcommand{\chaptername}{Chapter} 
\renewcommand{\appendixname}{Annex} 
% ... old code does not work correctly with pdflatex  
% \def\@chapapp{Chapter}
% \def\appendix{\par
%  \setcounter{chapter}{0}
%  \setcounter{section}{0}
%  \def\@chapapp{Annex}
%  \def\thechapter{\Alph{chapter}}}

\makeatother


%
% Also from HPFF.  These look potentially useful.
%

\newenvironment{rationale}{\begin{list}{}{}\item[]{\it Rationale.}
}{{\rm ({\it End of rationale.})} \end{list}}

\newenvironment{implementors}{\begin{list}{}{}\item[]{\it Advice
        to implementors.}
}{{\rm ({\it End of advice to implementors.})} \end{list}}

\newenvironment{users}{\begin{list}{}{}\item[]{\it Advice to users.}
}{{\rm ({\it End of advice to users.})} \end{list}}



%
% Use Sans Serif font for sections, etc.  S. Otto
%
\makeatletter
\def\section{\@startsection {section}{1}{\z@}{-3.5ex plus -1ex minus 
-.2ex}{2.3ex plus .2ex}{\Large\sf}}
\def\subsection{\@startsection{subsection}{2}{\z@}{-3.25ex plus -1ex minus 
-.2ex}{1.5ex plus .2ex}{\large\sf}}
\def\subsubsection{\@startsection{subsubsection}{3}{\z@}{-3.25ex plus 
-1ex minus -.2ex}{1.5ex plus .2ex}{\normalsize\sf\bf}}
\def\paragraph{\@startsection {paragraph}{4}{\z@}{3.25ex plus 1ex 
minus .2ex}{-1em}{\normalsize\sf}}
\makeatother
%
% An Editor's Note macro
%
\def\ednote#1{{\sl Editor's note: #1}}

% a way to comment out large sections of text
\newcommand{\commentOut}[1]{{}}

%
%  A few commands to help in writing MPI man pages
%
\def\twoc#1#2{
\begin{list}
{\hbox to95pt{#1\hfil}}
{\setlength{\leftmargin}{120pt}
 \setlength{\labelwidth}{95pt}
 \setlength{\labelsep}{0pt}
 \setlength{\partopsep}{0pt}
 \setlength{\parskip}{0pt}
 \setlength{\topsep}{0pt}
}
\item
{#2}
\end{list}
}
\outer\long\def\onec#1{
\begin{list}
{}
{\setlength{\leftmargin}{25pt}
 \setlength{\labelwidth}{0pt}
 \setlength{\labelsep}{0pt}
 \setlength{\partopsep}{0pt}
 \setlength{\parskip}{0pt}
 \setlength{\topsep}{0pt}
}
\item
{#1}
\end{list}
}
\def\manhead#1{\noindent{\bf{#1}}}

\makeatletter
%
% make our own index environment that can have a different
% title than just "Index" -- S. Otto
%
\def\@index{Index}
\def\@introtext{}  % MPI-2.1
\newif\if@restonecol
\def\myindex{\@restonecoltrue\if@twocolumn\@restonecolfalse\fi
\columnseprule \z@
%\columnsep 35pt\twocolumn[\@makeschapterhead{Index}]
 %\@mkboth{INDEX}{INDEX}\thispagestyle{plain}\parindent\z@
%\columnsep 35pt\twocolumn[\@makeschapterhead{\@index}]
\columnsep 35pt\twocolumn[\@makeschapterhead{\@index}\@introtext\vspace{15pt}] % MPI-2.1
 \@mkboth{\@index}{\@index}\thispagestyle{plain}\parindent\z@
 \parskip\z@ plus .3pt\relax\let\item\@idxitem}
\def\@idxitem{\par\hangindent 40pt}
\def\subitem{\par\hangindent 40pt \hspace*{20pt}}
\def\subsubitem{\par\hangindent 40pt \hspace*{30pt}}
\def\endmyindex{\if@restonecol\onecolumn\else\clearpage\fi}
\def\indexspace{\par \vskip 10pt plus 5pt minus 3pt\relax}
\makeatother

%macros for language binding: mpibind, mpifbind, and fargs:


\def\fargs{\\\advance\leftskip 2em}

\raggedbottom

% from binding chapter for appendix B.
% -*- latex -*-

\makeatletter
\newbox\arg@box

\def\separator{\rule{\linewidth}{0.5pt}}
\def\function#1{\texttt{#1}}
\def\variable#1{\texttt{#1}}

\def\subtitle{\pagebreak[3]\@ifstar{\@subtit@star}{\@subtit@norm}}
\def\@subtit@star#1{
  \item[\hbox{\normalsize\sf\begin{tabular}[t]{l}#1\end{tabular}}\hfill]
  \hfil\par
  \expandafter{\let\par=\space\ignorespaces\let\par=\endgraf}
}
\def\@subtit@norm#1{
  \setbox\arg@box=\hbox{\normalsize\sf\begin{tabular}[t]{l}#1\end{tabular}}
  \ifdim \wd\arg@box > \labelwidth \item[\copy\arg@box\hfill]\hfil\par
  \else \dp\arg@box=0pt \item[\copy\arg@box\hfill] \fi
  \expandafter{\let\par=\space\ignorespaces\let\par=\endgraf}
}

\newenvironment{manpage}[3]{\@beginManpage#1\@@#2\@@#3\@@}{\@endManpage}

\def\@beginManpage#1\@@#2\@@#3\@@{
  \addcontentsline{toc}{subsection}{#2}
  \clearpage    
  \begin{list}{}{
      \setlength\labelwidth{1.2in}
      \setlength\leftmargin{\labelwidth}
      \addtolength\leftmargin{\labelsep}
      \topsep  5pt plus 2pt minus 2pt
      \itemsep 5pt plus 2pt minus 2pt
      \parsep 10pt plus 2pt minus 2pt
      \raggedbottom
      }
    }

\def\@endManpage{\end{list} \clearpage \flushbottom}

\makeatother

% The next set of macros define change marks for the document
%----------------------------------------------------------------------
% intended for general change marks not associated with a certain version
%\def\begchange{\marginpar[\hspace*{-60pt}\mbox{\hspace*{10pt}
%$\top$ \tiny (General)}]{\mbox{$\top$ \tiny (General)}}}
%\def\endchange{\marginpar[\hspace*{-60pt}\mbox{\hspace*{10pt}
%$\bot$ \tiny (General)}]{\mbox{$\bot$ \tiny (General)}}}
%
% These versions are careful to generate no extraneous error messages 
% about overfull boxes
%
% Because they use marginpar, they can't be used everywhere.  If 
% marginpar isn't available, they do *not* add any marks.
%
\def\begchange{\ifinner\else\protect\marginpar[\mbox{\hbox to
    \marginparwidth{\hss\mbox{\hspace*{10pt}$\top$ \tiny
        (Fin2)}\hspace*{30pt}}}]{\hbox to \marginparwidth{\mbox{$\top$ \tiny (Fin2)}\hss}}\fi}
\def\endchange{\ifinner\else\protect\marginpar[\mbox{\hbox to \marginparwidth{\hss\mbox{\hspace*{10pt}$\bot$ \tiny (Fin2)}\hspace*{30pt}}}]{\hbox to \marginparwidth{\mbox{$\bot$ \tiny (Fin2)}\hss}}\fi}
%get rid of these change marks
%\def\begchange{}
%\def\endchange{}

%\def\begchange{\ifinner\else\protect\fi}
%\def\endchange{\ifinner\else\fi}

% change marks for June draft
\def\begchangejune{}
\def\endchangejune{}

% change marks for July draft
\def\begchangejuly{}
\def\endchangejuly{}

% change marks for Sept draft
\def\begchangesept{}
\def\endchangesept{}

% change marks for Oct draft
\def\begchangeoct{}
\def\endchangeoct{}

% change marks for Dec draft - for RT
\def\begchangedec{}
\def\endchangedec{}

% change marks for Jan draft
\def\begchangejan{}
\def\endchangejan{}

% change marks for February draft - for RT
\def\begchangefeb{}
\def\endchangefeb{}

% change marks for March draft
\def\begchangemar{}
\def\endchangemar{}
\def\begchangemarch{}
\def\endchangemarch{}

% change marks for April draft
\def\begchangeapr{}
\def\endchangeapr{}

% change marks for first final draft
\def\begchangefini{}
\def\endchangefini{}

% change marks for second final draft
\def\begchangefinii{}
\def\endchangefinii{}

%\def\begchangefiniii{\marginpar[\mbox{\hbox to
%    \marginparwidth{\hss\mbox{\hspace*{10pt}$\top$ \tiny
%        (Fin3)}\hspace*{30pt}}}]{\hbox to \marginparwidth{\mbox{$\top$ \tiny (Fin3)}\hss}}}
%\def\endchangefiniii{\marginpar[\mbox{\hbox to \marginparwidth{\hss\mbox{\hspace*{10pt}$\bot$ \tiny (Fin3)}\hspace*{30pt}}}]{\hbox to \marginparwidth{\mbox{$\bot$ \tiny (Fin3)}\hss}}}
%get rid of these change marks
\def\begchangefiniii{}
\def\endchangefiniii{}
%----------------------------------------------------------------------
\newcommand{\startchap}[0]{%\cleardoublepage
}

\newcommand{\OSH}{\emph{OpenSHMEM}}
\newcommand{\rcomment}[1]
    {{\color{red}\textsf{#1}}}

\newcommand{\bAPI}[2]{
\subsubsection{\bf #1}%\hfill
#2
\hfill %\\
\begin{description}
%\par\nobreak\vspace{-\parskip}
  \item[SYNOPSIS] \hfill \\ \\ 
  %\\ \\
}

\newcommand{\eAPI}{
\end{description}
}

\newcommand{\synC}{%\hfill \\
  %\vspace{-\parskip} \par
  \textbf{C/C++:}
  \begin{lstlisting} [language={C}, backgroundcolor=\color{gray}, lineskip=2pt, morekeywords={size_t}, aboveskip=0pt, belowskip=0pt]  
}

\newcommand{\synCE}{
\end{lstlisting} 
} 

\newcommand{\synFE}{
\end{lstlisting} 
} 

\newcommand{\synF}{%\hfill \\     
  %\vspace{-\parskip} \par
       \textbf{FORTRAN:}	 
	    \begin{lstlisting} [language={Fortran}, backgroundcolor=\color{gray}, lineskip=3pt, deletekeywords={TARGET,LEN}, aboveskip=0pt, belowskip=0pt]
}
\newcommand{\aC}[1] {\textit{#1}}
\newcommand{\sC}[1] {\textbf{#1}}
\newcommand{\aF}[1]{\textit{#1}}

 \newcommand{\desB}[3] {\hfill
   \item[DESCRIPTION] \hfill %\\
     \begin{description}
       \item[Arguments] \hfill \\
       #1
      \hfill %\\ \\
       \item[API description] \hfill \\ %\\
       #2 %\par
       \hfill \\ %\\
       #3 
       \hfill \\ %\\
       \end{description}
 }  
\newcommand{\argRow}[3] {
   \begin{tabular}{p{2cm} p{2cm} p{10cm}}
   \textbf{#1} & \textit{#2} & {#3} \\ 
   \end{tabular} 
}
\newcommand{\desTB}[2] {#1 \\ \\
    \begin{tabular}{p{5cm} p{9cm}}
       \hline
      Routine & Data Type of target and source\\
       \hline \tabularnewline
       \end{tabular}\\
        #2
        %\hfill
}  

\newcommand{\desTBC}[4] {#1 \\ \\
    \begin{tabular}{p{5cm} p{9cm}}
       \hline
      #2 & #3\\
       \hline \tabularnewline
       \end{tabular}\\
        #4
        \hfill
} 

\newcommand{\desR}[1]
{\hfill %\\ \\
  \item[Return Values] \hfill \\ %\\
       #1 
       \\
}

\newcommand{\cRow}[2]{
 \begin{tabular}{p{5cm} p{9cm}}
 #1 & #2 \tabularnewline
  \end{tabular}\\
}

\newcommand{\notesB}[1]{\hfill %\\ \\
\item[Notes] \hfill \\ %\\
#1
%\\
}

\newcommand{\exampleB}[1] {
\item[EXAMPLES] \hfill \\ \\
      #1
}

\newcommand{\exampleITEM}[3] {
 #1
 \lstinputlisting[language={C}, tabsize=2, basicstyle=\ttfamily\footnotesize, morekeywords={size_t}] {#2}	
 #3  
}

\newcommand{\exampleITEMF}[3] {
 #1
 \lstinputlisting[language={Fortran}, tabsize=2, basicstyle=\ttfamily\footnotesize, deletekeywords={TARGET}] {#2}
 #3 
}

\newcommand{\source}{\textit{source}}
\newcommand{\target}{\textit{target}}
\newcommand{\PUT}{\textit{Put}}
\newcommand{\GET}{\textit{Get}}
\newcommand{\OPR}[1]{\textit{#1}}


%\newcommand{\FUNC}[1] {\texttt{#1}}
%\newcommand{\FUNC}[1] {{\bf \texttt{#1}}}
%\newcommand{\FUNC}[1] {\texttt{#1}}
\newcommand{\FUNC}[1] {\textit{#1}}

\newcommand{\VAR}[1] {\textit{#1}}

\newcommand{\CONST}[1] {\textit{#1}}

\newcommand{\CorCpp}{\textit{C/C++}}

\newcommand{\Fortran}{\textit{Fortran}}

\newcommand{\Clang}{\textit{C}}

\newcommand{\Cpp}{\textit{C++}}

\newcommand{\barrier}{\FUNC{SHMEM\_BARRIER}}
\newcommand{\barrierall}{\FUNC{SHMEM\_BARRIER\_ALL}}
\newcommand{\broadcast}{\FUNC{SHMEM\_BROADCAST}}
\newcommand{\collect}{\FUNC{SHMEM\_COLLECT}}
\newcommand{\reduction}{\textit{Reduction Operations}}
\newcommand{\activeset}{\textit{Active~set}}
\newcommand{\shmemprefix}{\textit{SHMEM\_}}
\newcommand{\shmemprefixC}{\textit{\_SHMEM\_}}

\def\StandardListing {
  \lstset {
%%    basicstyle=\scriptsize\ttfamily,
%%    backgroundcolor=\color{ListingBG},
%%    showspaces=false,
%%    showstringspaces=false,
%%    showtabs=false,
%%    frame=tlBR,
%%    frameround=tttt,
%%    numbers=none,
%%    caption=\lstname
    breakatwhitespace=false,         % sets if automatic breaks should only happen at whitespace
    basicstyle=\ttfamily\footnotesize,
    breaklines=true,                 % sets automatic line breaking
    escapeinside={\%*}{*)},          % if you want to add LaTeX within your code
    extendedchars=true,              % lets you use non-ASCII characters; for 8-bits encodings only, does not work with UTF-8
    keepspaces=true,                 % keeps spaces in text, useful for keeping indentation of code (possibly needs columns=flexible)
    morekeywords={*,...},            % if you want to add more keywords to the set
    showspaces=false,                % show spaces everywhere adding particular underscores; it overrides 'showstringspaces'
    showstringspaces=false,          % underline spaces within strings only
    showtabs=false,                  % show tabs within strings adding particular underscores
    backgroundcolor=\color{gray}, 
  }
}

% annotated program source should be line numbered though

\def\ProgramNumberedListing {
  \StandardListing
  \lstset {
    numbers=left,
    numberstyle=\footnotesize
  }
}

% new command to show program listings

\newcommand{\numberedlisting}[2] {
  \ProgramNumberedListing
  \lstinputlisting[#1]{#2}
  \StandardListing
}

\lstdefinelanguage{OSH+C}[]{C}{
  classoffset=1,
  morekeywords={
    _SHMEM_BCAST_SYNC_SIZE, _SHMEM_SYNC_VALUE,
    start_pes,
    my_pe, _my_pe, shmem_my_pe,
    num_pes, _num_pes, shmem_n_pes,
    shmem_int_p, shmem_short_p, shmem_long_p,
    shmem_int_put, shmem_short_put, shmem_long_put,
    shmem_barrier_all, shmem_barrier,
    shmalloc,  shfree, shrealloc,
    shmem_broadcast32, shmem_broadcast64,
    shmem_short_inc, shmem_int_inc, shmem_long_inc,
    shmem_short_add, shmem_int_add, shmem_long_add,
    shmem_short_finc, shmem_int_finc, shmem_long_finc,
    shmem_short_fadd, shmem_int_fadd, shmem_long_fadd,
    shmem_set_lock, shmem_test_lock, shmem_clear_lock,
    shmem_long_sum_to_all,
    shmem_complexd_sum_to_all,
  },
  keywordstyle=\color{black}\textbf,
  classoffset=0,
  sensitive=true
}

\lstdefinelanguage{OSH2+C}[]{OSH+C}{
  classoffset=1,
  morekeywords={
    shmem_init,
    shmem_finalize,
    shmem_malloc,
    shmem_my_pe,
    shmem_error,
  },
  keywordstyle=\color{black}\textbf,
  classoffset=0,
  sensitive=true
}

\lstdefinelanguage{OSH+F}[]{Fortran}{
  classoffset=1,
  morekeywords={
    SHMEM_BCAST_SYNC_SIZE, SHMEM_SYNC_VALUE,
    start_pes,
    my_pe, shmem_my_pe,
    num_pes, shmem_n_pes,
    shmem_int_p, shmem_short_p, shmem_long_p,
    shmem_int_put, shmem_short_put, shmem_long_put,
    shmem_barrier_all, shmem_barrier,
    shpalloc,  shpdeallc, shpclmove,
    shmem_broadcast32, shmem_broadcast64,
    shmem_broadcast4, shmem_broadcast8,
    shmem_short_inc, shmem_int_inc, shmem_long_inc,
    shmem_short_add, shmem_int_add, shmem_long_add,
    shmem_short_finc, shmem_int_finc, shmem_long_finc,
    shmem_short_fadd, shmem_int_fadd, shmem_long_fadd,
    shmem_set_lock, shmem_test_lock, shmem_clear_lock,
    shmem_long_sum_to_all,
  },
  keywordstyle=\color{black}\textbf,
  classoffset=0,
  sensitive=false
}

\newcommand{\outputlisting}[2] {
\begin{minipage}{\linewidth}
\vspace{0.1in}
  \lstinputlisting[#1]{#2}
  \StandardListing
\vspace{0.1in}
\end{minipage}
}

\hyphenation{Open-SHMEM}
