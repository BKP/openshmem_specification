\definecolor{ListingBG}{rgb}{0.91,0.91,0.91}
\definecolor{shadecolor}{rgb}{0.92,0.92,0.92}

\hyphenation{Open-SHMEM}

\renewcommand{\chaptername}{Chapter} 
\renewcommand{\appendixname}{Annex} 

% Place some penalty for doing the break
% The penalty for a ``\gb'' should be greater than a \hyphenpenalty.
% \hyphenpenalty is 50 in plain.tex.
\def\gb{\penalty10000\hskip 0pt plus 8em\penalty4800\hskip 0pt plus-8em%
\penalty10000}

% This macro enables that all "_" (underscore) characters in the pfd
% file are searchable, and that cut&paste will copy the "_" as underscore. 
% Without the following macro, the \_ is treated in searches and cut&paste
% as a " " (space character). 
% This macro does not modify the behavior of _ in math or in verbatim 
% environments. In verbatim environments, the "_" is always treated
% as a searchable character.
%
\DeclareRobustCommand{\_}{\texttt{\char`\_}} 
% 

\def\colorswapnt{\colorlet{saved}{.}\color{ForestGreen}}
\def\colorswapot{\colorlet{saved}{.}\color{red}}
\def\prevcolor{\color{saved}}

\newcommand{\newtext}[1]{\textcolor{ForestGreen}{#1}}
\newcommand{\oldtext}[1]{\textcolor{magenta}{\sout{#1}}}
\newcommand{\insertDocVersion}{1.5}
\newcommand{\OSH}{\emph{OpenSHMEM}}
\newcommand{\openshmem}{{Open\-SHMEM}\xspace}
\newcommand{\FUNC}[1]{\textit{#1}}
\newcommand{\VAR}[1]{\textit{#1}}
\newcommand{\CONST}[1]{\textit{#1}}
\newcommand{\const}[1]{\protect\gb\protect{\textsf{\small #1}}\index{CONST:#1}}
\newcommand{\CorCpp}{\textit{C/C++}\xspace}
\newcommand{\CorCppFor}{\textit{C/C++/Fortran}\xspace}
\newcommand{\Fortran}{\textit{Fortran}}
\newcommand{\Clang}{\textit{C}}
\newcommand{\Cpp}{\textit{C++}}
\newcommand{\Celev}{\textit{C11}}
\newcommand{\TYPE}{\emph{TYPE}}
\newcommand{\TYPENAME}{\emph{TYPENAME}}
\newcommand{\SIZE}{\emph{SIZE}}

\newcommand{\source}{\textit{source}}
\newcommand{\target}{\textit{target}}
\newcommand{\PUT}{\textit{Put}}
\newcommand{\GET}{\textit{Get}}
\newcommand{\OPR}[1]{\textit{#1}}
\newcommand{\dest}{\textit{dest}}
\newcommand{\barrier}{\FUNC{SHMEM\_BARRIER}\xspace} % why here an not others?
\newcommand{\barrierall}{\FUNC{SHMEM\_BARRIER\_ALL}\xspace} % why here an not others?
\newcommand{\broadcast}{\FUNC{SHMEM\_BROADCAST}}
\newcommand{\collect}{\FUNC{SHMEM\_COLLECT}}
\newcommand{\fcollect}{\FUNC{SHMEM\_FCOLLECT}}
\newcommand{\reduction}{\textit{Reduction Operations}}
\newcommand{\alltoall}{\FUNC{SHMEM\_ALLTOALL}}
\newcommand{\alltoalls}{\FUNC{SHMEM\_ALLTOALLS}}
\newcommand{\activeset}{\textit{Active~set}\xspace} % why here and not others?
\newcommand{\shmemprefix}{\textit{SHMEM\_}}
\newcommand{\shmemprefixC}{\textit{\_SHMEM\_}}
\newcommand{\ith}{${\textit{i}^{\text{\tiny th}}}$}
\newcommand{\jth}{${\textit{j}^{\text{\tiny th}}}$}
\newcommand{\kth}{${\textit{k}^{\text{\tiny th}}}$}
\newcommand{\lth}{${\textit{l}^{\text{\tiny th}}}$}

\begin{acronym}
\acro{RMA}{\emph{Remote Memory Access}}
\acro{RMO}{\emph{Remote Memory Operation}}
\acro{AMO}{\emph{Atomic Memory Operation}}
\acro{PE}{\emph{Processing Element}}
\acrodefplural{PE}[PEs]{\emph{Processing Elements}}
\acro{PGAS}{\emph{Partitioned Global Address Space}}
\acro{API}{\emph{Application Programming Interface}}
\acro{MPI}{\emph{Message Passing Interface}}
\acro{SPMD}{\emph{Single Program Multiple Data}}
\acro{UH}{University of Houston}
\acro{UO}{University of Oregon}
\acro{ORNL}{Oak Ridge National Laboratory}
\acro{LANL}{Los Alamos National Laboratory}
\acro{ESSC}{Extreme Scale Systems Center}
\acro{OSSS}{Open Software System Solutions}
\acro{DoD}{U.S. Department of Defense}
\end{acronym}


%
% This is used to put line numbers on plain pages.  Used in draft.tex
%
\makeatletter

\def\withlinenumbers{\relax
  \def\@evenfoot{\hbox to 0pt{\hss\LineNumberRuler\hskip 1.5pc}\hfil}\relax
  \def\@oddfoot{\hfil\hbox to 0pt{\hskip 1.5pc\LineNumberRuler\hss}}}

\def\LineNumberRuler{\vbox to 0pt{\vss\normalsize \baselineskip13.6pt
    \lineskip 1pt \normallineskip 1pt \def\baselinestretch{1}\relax
    \LNR{1}\LNR{2}\LNR{3}\LNR{4}\LNR{5}\LNR{6}\LNR{7}\LNR{8}\LNR{9}
    \LNR{10}\LNR{11}\LNR{12}\LNR{13}\LNR{14}
        \LNR{15}\LNR{16}\LNR{17}\LNR{18}\LNR{19}
    \LNR{20}\LNR{21}\LNR{22}\LNR{23}\LNR{24}
        \LNR{25}\LNR{26}\LNR{27}\LNR{28}\LNR{29}
    \LNR{30}\LNR{31}\LNR{32}\LNR{33}\LNR{34}\LNR{35}
        \LNR{36}\LNR{37}\LNR{38}\LNR{39}
    \LNR{40}\LNR{41}\LNR{42}\LNR{43}\LNR{44}
        \LNR{45}\LNR{46}\LNR{47}\LNR{48}
    \vskip 31pt}}
\def\LNR#1{\hbox to 1pc{\hfil\tiny#1\hfil}}

\def\ps@plainwithlinenumbers{\let\@mkboth\@gobbletwo
     \def\@oddhead{}
     \def\@oddfoot{\hfil\rm\thepage\hfil
       \hbox to 0pt{\hskip 1.5pc\LineNumberRuler\hss}}
     \def\@evenhead{}
     \def\@evenfoot{\hbox to 0pt{\hss
     \LineNumberRuler\hskip 1.5pc}\rm\hfil\thepage\hfil}}

    % Contents is done with \chapter*{Contents}, so we need to turn off the
    % line numbers in this case.  Easiest to look at def

\newwrite\chappages
\immediate\openout\chappages=chappage.txt
\def\writespace{ }

\def\incontents{0}
\newif\ifcontents
\contentsfalse
\def\chapter{\clearpage \ifcontents\else\thispagestyle{plainwithlinenumbers}\fi
        \write\chappages{Chapter \thechapter\writespace - \the\count0}
        \global\@topnum\z@ \@afterindentfalse \secdef\@chapter\@schapter}

\makeatother

%
% End this is used to put line numbers on plain pages.  Used in draft.tex
%

%
% Use Sans Serif font for sections, etc.
%
\makeatletter
\def\section{\@startsection {section}{1}{\z@}{-3.5ex plus -1ex minus 
-.2ex}{2.3ex plus .2ex}{\Large\sf}}
\def\subsection{\@startsection{subsection}{2}{\z@}{-3.25ex plus -1ex minus 
-.2ex}{1.5ex plus .2ex}{\large\sf}}
\def\subsubsection{\@startsection{subsubsection}{3}{\z@}{-3.25ex plus 
-1ex minus -.2ex}{1.5ex plus .2ex}{\normalsize\sf\bf}}
\def\paragraph{\@startsection {paragraph}{4}{\z@}{3.25ex plus 1ex 
minus .2ex}{-1em}{\normalsize\sf}}
\makeatother
%
% End use Sans Serif font for sections, etc.  S. Otto
%


%
% This section is for example code listings
%
\definecolor{gray}{rgb}{0.92,0.92,0.92}

\lstset{ % set defaults for languages not otherwise defined
  breakatwhitespace=false,         % sets if automatic breaks should only happen at whitespace
  basicstyle=\ttfamily\footnotesize,
  breaklines=true,                 % sets automatic line breaking
  escapeinside={|}{|},          % if you want to add LaTeX within your code
  extendedchars=true,              % lets you use non-ASCII characters; for 8-bits 
                                   % encodings only, does not work with UTF-8
  keepspaces=true,                 % keeps spaces in text, useful for keeping indentation of code 
                                   % (possibly needs columns=flexible)
  morekeywords={*,...},            % if you want to add more keywords to the set
  showspaces=false,                % show spaces everywhere adding particular underscores; 
                                   % it overrides 'showstringspaces'
  showstringspaces=false,          % underline spaces within strings only
  showtabs=false,                  % show tabs within strings adding particular underscores
}

\def\StandardListing {
  \lstset {
    breakatwhitespace=false,         % sets if automatic breaks should only happen at whitespace
    basicstyle=\ttfamily\footnotesize,
    breaklines=true,                 % sets automatic line breaking
    escapeinside={\%*}{*)},          % if you want to add LaTeX within your code
    extendedchars=true,              % lets you use non-ASCII characters; for 8-bits 
                                     % encodings only, does not work with UTF-8
    keepspaces=true,                 % keeps spaces in text, useful for keeping
                                     % indentation of code (possibly needs columns=flexible)
    morekeywords={*,...},            % if you want to add more keywords to the set
    showspaces=false,                % show spaces everywhere adding particular underscores; 
                                     % it overrides 'showstringspaces'
    showstringspaces=false,          % underline spaces within strings only
    showtabs=false,                  % show tabs within strings adding particular underscores
    backgroundcolor=\color{gray}, 
  }
}

\def\ProgramNumberedListing {
  \StandardListing
  \lstset {
    numbers=left,
    numberstyle=\footnotesize
  }
}

\newcommand{\numberedlisting}[2] {
  \ProgramNumberedListing
  \lstinputlisting[#1]{#2}
  \StandardListing
}

\newcommand{\outputlisting}[2] {
\begin{minipage}{\linewidth}
\vspace{0.1in}
  \lstinputlisting[#1]{#2}
  \StandardListing
\vspace{0.1in}
\end{minipage}
}

\lstdefinelanguage{OSH+C}[]{C}{
  classoffset=1,
  morekeywords={
    SHMEM_BCAST_SYNC_SIZE, SHMEM_SYNC_VALUE,
    start_pes,
    my_pe, _my_pe, shmem_my_pe,
    num_pes, _num_pes, shmem_n_pes,
    shmem_int_p, shmem_short_p, shmem_long_p,
    shmem_int_put, shmem_short_put, shmem_long_put,
    shmem_barrier_all, shmem_barrier,
    shmalloc,  shfree, shrealloc,
    shmem_broadcast32, shmem_broadcast64,
    shmem_short_inc, shmem_int_inc, shmem_long_inc,
    shmem_short_add, shmem_int_add, shmem_long_add,
    shmem_short_finc, shmem_int_finc, shmem_long_finc,
    shmem_short_fadd, shmem_int_fadd, shmem_long_fadd,
    shmem_set_lock, shmem_test_lock, shmem_clear_lock,
    shmem_long_sum_to_all,
    shmem_complexd_sum_to_all,
  },
  keywordstyle=\color{black}\textbf,
  classoffset=0,
  sensitive=true
}

\lstdefinelanguage{OSH2+C}[]{OSH+C}{
  classoffset=1,
  morekeywords={
    shmem_init,
    shmem_finalize,
    shmem_malloc,
    shmem_my_pe,
    shmem_error,
    shmem_global_exit,
  },
  keywordstyle=\color{black}\textbf,
  classoffset=0,
  sensitive=true
}

\lstdefinelanguage{OSH+F}[]{Fortran}{
  classoffset=1,
  morekeywords={
    SHMEM_BCAST_SYNC_SIZE, SHMEM_SYNC_VALUE,
    start_pes,
    my_pe, shmem_my_pe,
    num_pes, shmem_n_pes,
    shmem_int_p, shmem_short_p, shmem_long_p,
    shmem_int_put, shmem_short_put, shmem_long_put,
    shmem_barrier_all, shmem_barrier,
    shpalloc,  shpdeallc, shpclmove,
    shmem_broadcast32, shmem_broadcast64,
    shmem_broadcast4, shmem_broadcast8,
    shmem_short_inc, shmem_int_inc, shmem_long_inc,
    shmem_short_add, shmem_int_add, shmem_long_add,
    shmem_short_finc, shmem_int_finc, shmem_long_finc,
    shmem_short_fadd, shmem_int_fadd, shmem_long_fadd,
    shmem_set_lock, shmem_test_lock, shmem_clear_lock,
    shmem_long_sum_to_all,
  },
  keywordstyle=\color{black}\textbf,
  classoffset=0,
  sensitive=false
}

\lstdefinelanguage{OSH2+F}[]{OSH+F}{
  classoffset=1,
  morekeywords={
    shmem_init,
    shmem_finalize,
    shmem_malloc,
    shmem_my_pe,
    shmem_error,
    shmem_global_exit,
  },
  keywordstyle=\color{black}\textbf,
  classoffset=0,
  sensitive=true
}

%
% End this section is for example code listings
%

%
% Library API description template commands
%

\newcommand{\apisummary}[1]{
    #1
\hfill
}

\newenvironment{apidefinition}{
\begin{description}
\item[SYNOPSIS] \hfill \\ \\ 
\vspace{-2em}
}
{
\end{description}
}

\lstnewenvironment{C11synopsis}
{ 
  \textbf{C11:} 
  \lstset{language={C++}, backgroundcolor=\color{gray}, lineskip=2pt,
  morekeywords={size_t, TYPE}, aboveskip=0pt, belowskip=0pt,}}{}

\lstnewenvironment{CsynopsisCol}
{ 
  \lstset{language={C}, backgroundcolor=\color{gray}, lineskip=2pt,
  morekeywords={size_t, TYPE, TYPENAME, SIZE}, aboveskip=0pt, belowskip=0pt}}{}


\lstnewenvironment{Csynopsis}
{ 
  \textbf{C/C++:} 
  \lstset{language={C}, backgroundcolor=\color{gray}, lineskip=2pt,
  morekeywords={size_t, TYPE, TYPENAME, SIZE}, aboveskip=0pt, belowskip=0pt}}{}

\lstnewenvironment{CsynopsisST}
{ 
  \textbf{C/C++:} 
  \color{red}  
  {\lstset{language={C}, backgroundcolor=\color{gray}, lineskip=2pt,
  morekeywords={size_t}, aboveskip=0pt, belowskip=0pt}}
  }
  {}
  
\lstnewenvironment{Fsynopsis}
{ \textbf{FORTRAN:} 
  \lstset{language={Fortran}, backgroundcolor=\color{gray}, lineskip=3pt,
  deletekeywords=[2]{STATUS},
  deletekeywords=[3]{LOG}, aboveskip=0pt,
  belowskip=0pt}}{}

\newenvironment{apiarguments}{
\newcommand{\apiargument}[3]{
\begin{tabular}{p{2cm} p{2cm} p{10cm}}
\textbf{##1} & \textit{##2} & {##3} \\ 
\end{tabular}
}
\hfill
\item[DESCRIPTION] \hfill 

\begin{description}
\item[Arguments] \hfill \\
}
{
\hfill
\end{description}
}

\newcommand{\apidescription}[1]{
\begin{description}
\vspace{-1em}
\item[API description] \hfill \\ 
    #1
\hfill
}

\newcommand{\apidesctable}[4] {\hfill \\ #1 \\ \\
    \begin{tabular}{p{5cm} p{9cm}}
       \hline
       #2 & #3 \\
       \hline \tabularnewline
       \end{tabular}\\
        #4
}  

\newcommand{\apireturnvalues}[1]{
\hfill 
\item[Return Values] \hfill \\
    #1
\\
\hfill
}

\newcommand{\apitablerow}[2]{
 \begin{tabular}{p{5cm} p{9cm}}
 #1 & #2 \tabularnewline
  \end{tabular}\\
}

\newcommand{\apinotes}[1]{
\item[Notes] \hfill \\
    #1
\hfill \\
\end{description}
}

\newcommand{\apiimpnotes}[1]{
\begin{description}
\item[Note to implementors] \hfill \\
    #1
\hfill \\
\end{description}
}

\newenvironment{apiexamples}{
\newcommand{\apicexample}[3]{
    ##1
    \lstinputlisting[language={C}, tabsize=2,
    basicstyle=\ttfamily\footnotesize, morekeywords={size_t}]{##2}
    ##3 }
\newcommand{\apifexample}[3]{
    ##1
    \lstinputlisting[language={Fortran}, tabsize=2,
    basicstyle=\ttfamily\footnotesize, deletekeywords={TARGET}]{##2}
    ##3 }
\vspace{-2pt}
\item[EXAMPLES] \hfill \\
\vspace{-2pt}
}
{
}

%
% End library API description template commands
%
