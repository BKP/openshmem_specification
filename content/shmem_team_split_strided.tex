\apisummary{
Create a new \openshmem team from a subset of the existing parent team \acp{PE},
where the subset is defined by the
\ac{PE} triplet (\VAR{start}, \VAR{stride}, and \VAR{size}) supplied to the routine.}

\begin{apidefinition}

\begin{Csynopsis}
int @\FuncDecl{shmem\_team\_split\_strided}@(shmem_team_t parent_team, int start, int stride, int size,
    const shmem_team_config_t *config, long config_mask, shmem_team_t *new_team);
\end{Csynopsis}

\begin{apiarguments}
\apiargument{IN}{parent\_team}{An \openshmem team.}

\apiargument{IN}{start}{The lowest \ac{PE} number of the subset of \acp{PE} from
the parent team that will form the new team.}

\apiargument{IN}{stride}{The stride between team \ac{PE}
numbers in the parent team that comprise the subset of \acp{PE} that will form
the new team.}

\apiargument{IN}{size}{The number of \acp{PE} from the parent team in the subset
of \acp{PE} that will form the new team. \VAR{size} must be a positive integer.}

\apiargument{IN}{config}{
  A pointer to the configuration parameters for the new team.}

\apiargument{IN}{config\_mask}{
  The bitwise mask representing the set of configuration parameters to use
  from \VAR{config}.}

\apiargument{OUT}{new\_team}{A new \openshmem team handle, representing a \ac{PE}
subset of all the \acp{PE} in the parent team that is created from
the \ac{PE} triplet provided.}

\end{apiarguments}

\apidescription{
The \FUNC{shmem\_team\_split\_strided} routine is a collective routine.
It creates a new \openshmem team from a subset of the existing parent team,
where the \ac{PE} subset is defined by the triplet of arguments
(\VAR{start}, \VAR{stride}, \VAR{size}).
A valid triplet is one such that:
\begin{equation*}
  start + stride \cdot i \in \mathbb{Z}_N
  \hspace{0.35em}
  \forall
  \hspace{0.35em}
  i \in \mathbb{Z}_{size}
\end{equation*}
where $N$ is the number of \acp{PE} in the parent team and $size$ is greater than zero.

This routine must be called by all \acp{PE} in the parent team.
All \acp{PE} must provide the same values for the \ac{PE} triplet.
This routine will return a \VAR{new\_team} containing the \ac{PE}
subset specified by the triplet and ordered by the existing global
\ac{PE} number.

On successful creation of the new team:
\begin{itemize}
\item The \VAR{new\_team} handle will reference a valid team for the
  subset of \acp{PE} in the parent team specified by the triplet.
\item Those \acp{PE} in the parent team that are not in the subset
  specified by the triplet will have \VAR{new\_team} assigned to
  \LibConstRef{SHMEM\_TEAM\_INVALID}.
\item \FUNC{shmem\_team\_split\_strided} will return zero to all
  \acp{PE} in the parent team.
\end{itemize}

If the new team cannot be created or an invalid \ac{PE} triplet is provided,
then \VAR{new\_team} will be assigned the value \LibConstRef{SHMEM\_TEAM\_INVALID} and
\FUNC{shmem\_team\_split\_strided} will return a nonzero value on all
\acp{PE} in the parent team.

The \VAR{config} argument specifies team configuration parameters, which are
described in Section~\ref{subsec:shmem_team_config_t}.

The \VAR{config\_mask} argument is a bitwise mask representing the set of
configuration parameters to use from \VAR{config}.
A \VAR{config\_mask} value of \CONST{0} indicates that the team
should be created with the default values for all configuration parameters.
See Section~\ref{subsec:shmem_team_config_t} for field mask names and
default configuration parameters.

If \VAR{parent\_team}
compares equal to \LibConstRef{SHMEM\_TEAM\_INVALID}, then no new team
will be created and \VAR{new\_team} will be assigned the value
\LibConstRef{SHMEM\_TEAM\_INVALID}.  If \VAR{parent\_team} is otherwise invalid, the behavior is undefined.
}

\apireturnvalues{
  Zero on successful creation of \VAR{new\_team}; otherwise, nonzero.
}

\apinotes{
  It is important to note the use of the less restrictive
  \VAR{stride} argument instead of \VAR{logPE\_stride}. This method of
  creating a team with an arbitrary set of \acp{PE} is inherently restricted
  by its parameters, but allows for many additional use-cases over using a
  \VAR{logPE\_stride} parameter, and may provide an easier transition for
  existing \openshmem programs to create and use \openshmem teams.

  See the description of team handles and predefined teams at the top of
  Section~\ref{subsec:team} for more information about semantics and usage.
}

\begin{apiexamples}

    \apicexample
    {The following example demonstrates the use of strided split in a
    \Cstd[11] program. The program creates a new team of all even number
    \acp{PE} from the default team, then retrieves the \ac{PE} number and
    team size on all \acp{PE} that are members of the new team.}
    {./example_code/shmem_team_split_strided.c}
    {}

\end{apiexamples}

\end{apidefinition}
