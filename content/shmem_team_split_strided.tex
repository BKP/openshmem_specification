\apisummary{
shmemx\_team\_split\_strided is a collective routine to partition the existing
parent team into a new SHMEM team based on the \ac{PE}triplet (PE\_start,
PE\_stride, and PE\_size) supplied to the function.
}

\begin{apidefinition}

\begin{Csynopsis}
void @\FuncDecl{shmemx\_team\_split\_strided}@(shmem_team_t parent_team,
int PE_start, int PE_stride, int PE_size, shmem_team_t *newteam);
\end{Csynopsis}

\begin{apiarguments}
\apiargument{IN}{parent\_team}{A valid SHMEM team. The predefined teams
SHMEM\_TEAM\_WORLD or SHMEM\_TEAM\_NODE may be used, or any team created by the
users.}

\apiargument{IN}{PE\_start}{The lowest virtual \ac{PE} number of the
parent\_team of \acp{PE}.}

\apiargument{IN}{PE\_stride}{The stride between consecutive virtual \ac{PE}
numbers in the parent\_team.}

\apiargument{IN}{PE\_size}{The number of \acp{PE} in the defined set.}

\apiargument{OUT}{newteam}{A new SHMEM team handle, representing a \ac{PE}
subset of all the \acp{PE}, that is created from the \ac{PE} triplet provided.}
\end{apiarguments}

\apidescription{
The shmemx\_team\_split\_strided function is a collective routine.
It partitions the existing parent team into a new SHMEM team based on
the \ac{PE} triplet (PE\_start, PE\_stride, and PE\_size) supplied to
the function. It is important to note the use of the less restrictive
PE\_stride argument instead of logPE\_stride. This method of
creating a team with an arbitrary set of \acp{PE} is inherently restricted by
its parameters, but allows for many additional use-cases over using a
logPE\_stride parameter, and may provide an easier transition for
existing SHMEM programs to create and use SHMEM teams. This function
must be called by all processes contained in the SHMEM triplet
specification. It may be called by additional \acp{PE} not included in the
triplet specification, but for those processes a newteam value of
SHMEM\_TEAM\_NULL is returned. All calling processes must provide the
same values for the \ac{PE} triplet. This function will return a newteam
containing the \ac{PE} subset specified by the triplet, and ordered by the
existing global \ac{PE} rank value. None of the parameters need to reside in
symmetric memory.

Error checking will be done to ensure a valid \ac{PE} triplet is provided,
and also to determine whether a valid team handle is provided for the
parent\_team.

All errors are considered fatal and will result in the job aborting with
an informative error message.
}

\apireturnvalues{
None.
}

\apinotes{
Note that SHMEM team handles have local semantics only. That is, team
handles should not be stored in shared variables and used across other
processes. Doing so will result in unpredictable behavior.
}

\begin{apiexamples}

\end{apiexamples}

\end{apidefinition}
