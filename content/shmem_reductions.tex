\apisummary{
    The following functions perform reduction operations across all
    \acp{PE} in a set of \acp{PE}.
}

\begin{apidefinition}


\begin{table}[h]
  \begin{center}
{\color{Green}
    \begin{tabular}{|l|l|l|l|l|}
      \hline
      \TYPE              & \TYPENAME  & \multicolumn{3}{c|}{Operations Supporting \TYPE}\\ \hline
      short              & short      & AND, OR, XOR & MAX, MIN & SUM, PROD \\ \hline
      int                & int        & AND, OR, XOR & MAX, MIN & SUM, PROD \\ \hline
      long               & long       & AND, OR, XOR & MAX, MIN & SUM, PROD \\ \hline
      long long          & longlong   & AND, OR, XOR & MAX, MIN & SUM, PROD \\ \hline
      float              & float      & & MAX, MIN & SUM, PROD \\ \hline
      double             & double     & & MAX, MIN & SUM, PROD \\ \hline
      long double        & longdouble & & MAX, MIN & SUM, PROD \\ \hline
      double \_Complex   & complexd   & & & SUM, PROD \\ \hline
      float  \_Complex   & complexf   & & & SUM, PROD \\ \hline
    \end{tabular}
    \TableCaptionRef{Reduction Types, Names and Supporting Operations}
    \label{reducetypes}
}
  \end{center} 
\end{table}


\paragraph{AND}
Performs a bitwise AND reduction across a set of \acp{PE}.\newline

%% C11
{\color{Green}
\begin{C11synopsis}
int @\FuncDecl{shmem\_and\_to\_all}@(TYPE *dest, const TYPE *source, int nreduce, shmem_team_t team);
\end{C11synopsis}
where \TYPE{} is one of the integer types supported for the AND operation as specified by Table \ref{reducetypes}.
}

%% C/C++
\begin{Csynopsis}
\end{Csynopsis}
{\color{Green}
\begin{CsynopsisCol}
int @\FuncDecl{shmem\_team\_\FuncParam{TYPENAME}\_and\_to\_all}@(TYPE *dest, const TYPE *source, int nreduce, shmem_team_t team);
\end{CsynopsisCol}
}
\begin{DeprecateBlock}
\begin{CsynopsisCol}
void @\FuncDecl{shmem\_\FuncParam{TYPENAME}\_and\_to\_all}@(TYPE *dest, const TYPE *source, int nreduce, int PE_start, int logPE_stride, int PE_size, short *pWrk, long *pSync);
\end{CsynopsisCol}
\end{DeprecateBlock}
\newtext{where \TYPE{} is one of the integer types supported for the AND operation and has a corresponding \TYPENAME{} as specified by Table \ref{reducetypes}.}

%% Fortran
\begin{Fsynopsis}
CALL @\FuncDecl{SHMEM\_INT4\_AND\_TO\_ALL}@(dest, source, nreduce, PE_start, logPE_stride, PE_size, pWrk, pSync)
CALL @\FuncDecl{SHMEM\_INT8\_AND\_TO\_ALL}@(dest, source, nreduce, PE_start, logPE_stride, PE_size, pWrk, pSync)
\end{Fsynopsis}
%%

\paragraph{OR}
Performs a bitwise OR reduction across a set of \acp{PE}.\newline

%% C11
{\color{Green}
\begin{C11synopsis}
int @\FuncDecl{shmem\_or\_to\_all}@(TYPE *dest, const TYPE *source, int nreduce, shmem_team_t team);
\end{C11synopsis}
where \TYPE{} is one of the integer types supported for the OR operation as specified by Table \ref{reducetypes}.
}

%% C/C++
\begin{Csynopsis}
\end{Csynopsis}
{\color{Green}
\begin{CsynopsisCol}
int @\FuncDecl{shmem\_team\_\FuncParam{TYPENAME}\_or\_to\_all}@(TYPE *dest, const TYPE *source, int nreduce, shmem_team_t team);
\end{CsynopsisCol}
}
\begin{DeprecateBlock}
\begin{CsynopsisCol}
void @\FuncDecl{shmem\_\FuncParam{TYPENAME}\_or\_to\_all}@(TYPE *dest, const TYPE *source, int nreduce, int PE_start, int logPE_stride, int PE_size, short *pWrk, long *pSync);
\end{CsynopsisCol}
\end{DeprecateBlock}
\newtext{where \TYPE{} is one of the integer types supported for the OR operation and has a corresponding \TYPENAME{} as specified by Table \ref{reducetypes}.}

%% Fortran
\begin{Fsynopsis}
CALL @\FuncDecl{SHMEM\_INT4\_OR\_TO\_ALL}@(dest, source, nreduce, PE_start, logPE_stride, PE_size, pWrk, pSync)
CALL @\FuncDecl{SHMEM\_INT8\_OR\_TO\_ALL}@(dest, source, nreduce, PE_start, logPE_stride, PE_size, pWrk, pSync)
\end{Fsynopsis}
%%

\paragraph{XOR}
Performs a bitwise exclusive OR (XOR) reduction across a set of \acp{PE}.\newline

%% C11
{\color{Green}
\begin{C11synopsis}
int @\FuncDecl{shmem\_xor\_to\_all}@(TYPE *dest, const TYPE *source, int nreduce, shmem_team_t team);
\end{C11synopsis}
where \TYPE{} is one of the integer types supported for the XOR operation as specified by Table \ref{reducetypes}.
}

%% C/C++
\begin{Csynopsis}
\end{Csynopsis}
{\color{Green}
\begin{CsynopsisCol}
int @\FuncDecl{shmem\_team\_\FuncParam{TYPENAME}\_xor\_to\_all}@(TYPE *dest, const TYPE *source, int nreduce, shmem_team_t team);
\end{CsynopsisCol}
}
\begin{DeprecateBlock}
\begin{CsynopsisCol}
void @\FuncDecl{shmem\_\FuncParam{TYPENAME}\_xor\_to\_all}@(TYPE *dest, const TYPE *source, int nreduce, int PE_start, int logPE_stride, int PE_size, short *pWrk, long *pSync);
\end{CsynopsisCol}
\end{DeprecateBlock}
\newtext{where \TYPE{} is one of the integer types supported for the XOR operation and has a corresponding \TYPENAME{} as specified by Table \ref{reducetypes}.}

%% Fortran
\begin{Fsynopsis}
CALL @\FuncDecl{SHMEM\_INT4\_XOR\_TO\_ALL}@(dest, source, nreduce, PE_start, logPE_stride, PE_size, pWrk, pSync)
CALL @\FuncDecl{SHMEM\_INT8\_XOR\_TO\_ALL}@(dest, source, nreduce, PE_start, logPE_stride, PE_size, pWrk, pSync)
\end{Fsynopsis}
%%

\paragraph{MAX}
Performs a maximum-value reduction across a set of \acp{PE}.\newline

%% C11
{\color{Green}
\begin{C11synopsis}
int @\FuncDecl{shmem\_max\_to\_all}@(TYPE *dest, const TYPE *source, int nreduce, shmem_team_t team);
\end{C11synopsis}
where \TYPE{} is one of the integer or real types supported for the MAX operation as specified by Table \ref{reducetypes}.
}

%% C/C++
\begin{Csynopsis}
\end{Csynopsis}
{\color{Green}
\begin{CsynopsisCol}
int @\FuncDecl{shmem\_team\_\FuncParam{TYPENAME}\_max\_to\_all}@(TYPE *dest, const TYPE *source, int nreduce, shmem_team_t team);
\end{CsynopsisCol}
}
\begin{DeprecateBlock}
\begin{CsynopsisCol}
void @\FuncDecl{shmem\_\FuncParam{TYPENAME}\_max\_to\_all}@(TYPE *dest, const TYPE *source, int nreduce, int PE_start, int logPE_stride, int PE_size, short *pWrk, long *pSync);
\end{CsynopsisCol}
\end{DeprecateBlock}
\newtext{where \TYPE{} is one of the integer or real types supported for the MAX operation and has a corresponding \TYPENAME{} as specified by Table \ref{reducetypes}.}

%% Fortran
\begin{Fsynopsis}
CALL @\FuncDecl{SHMEM\_INT4\_MAX\_TO\_ALL}@(dest, source, nreduce, PE_start, logPE_stride, PE_size, pWrk, pSync)
CALL @\FuncDecl{SHMEM\_INT8\_MAX\_TO\_ALL}@(dest, source, nreduce, PE_start, logPE_stride, PE_size, pWrk, pSync)
CALL @\FuncDecl{SHMEM\_REAL4\_MAX\_TO\_ALL}@(dest, source, nreduce, PE_start, logPE_stride, PE_size, pWrk, pSync)
CALL @\FuncDecl{SHMEM\_REAL8\_MAX\_TO\_ALL}@(dest, source, nreduce, PE_start, logPE_stride, PE_size, pWrk, pSync)
CALL @\FuncDecl{SHMEM\_REAL16\_MAX\_TO\_ALL}@(dest, source, nreduce, PE_start, logPE_stride, PE_size, pWrk, pSync)
\end{Fsynopsis}

\paragraph{MIN}
Performs a minimum-value reduction across a set of \acp{PE}.\newline

%% C11
{\color{Green}
\begin{C11synopsis}
int @\FuncDecl{shmem\_min\_to\_all}@(TYPE *dest, const TYPE *source, int nreduce, shmem_team_t team);
\end{C11synopsis}
where \TYPE{} is one of the integer or real types supported for the MIN operation as specified by Table \ref{reducetypes}.
}

%% C/C++
\begin{Csynopsis}
\end{Csynopsis}
{\color{Green}
\begin{CsynopsisCol}
int @\FuncDecl{shmem\_team\_\FuncParam{TYPENAME}\_min\_to\_all}@(TYPE *dest, const TYPE *source, int nreduce, shmem_team_t team);
\end{CsynopsisCol}
}
\begin{DeprecateBlock}
\begin{CsynopsisCol}
void @\FuncDecl{shmem\_\FuncParam{TYPENAME}\_min\_to\_all}@(TYPE *dest, const TYPE *source, int nreduce, int PE_start, int logPE_stride, int PE_size, short *pWrk, long *pSync);
\end{CsynopsisCol}
\end{DeprecateBlock}
\newtext{where \TYPE{} is one of the integer or real types supported for the MIN operation and has a corresponding \TYPENAME{} as specified by Table \ref{reducetypes}.}

%% Fortran
\begin{Fsynopsis}
CALL @\FuncDecl{SHMEM\_INT4\_MIN\_TO\_ALL}@(dest, source, nreduce, PE_start, logPE_stride, PE_size, pWrk, pSync)
CALL @\FuncDecl{SHMEM\_INT8\_MIN\_TO\_ALL}@(dest, source, nreduce, PE_start, logPE_stride, PE_size, pWrk, pSync)
CALL @\FuncDecl{SHMEM\_REAL4\_MIN\_TO\_ALL}@(dest, source, nreduce, PE_start, logPE_stride, PE_size, pWrk, pSync)
CALL @\FuncDecl{SHMEM\_REAL8\_MIN\_TO\_ALL}@(dest, source, nreduce, PE_start, logPE_stride, PE_size, pWrk, pSync)
CALL @\FuncDecl{SHMEM\_REAL16\_MIN\_TO\_ALL}@(dest, source, nreduce, PE_start, logPE_stride, PE_size, pWrk, pSync)
\end{Fsynopsis}

\paragraph{SUM}
Performs a sum reduction across a set of \acp{PE}.\newline

%% C11
{\color{Green}
\begin{C11synopsis}
int @\FuncDecl{shmem\_sum\_to\_all}@(TYPE *dest, const TYPE *source, int nreduce, shmem_team_t team);
\end{C11synopsis}
where \TYPE{} is one of the integer, real, or complex types supported for the SUM operation as specified by Table \ref{reducetypes}.
}

%% C/C++
\begin{Csynopsis}
\end{Csynopsis}
{\color{Green}
\begin{CsynopsisCol}
int @\FuncDecl{shmem\_team\_\FuncParam{TYPENAME}\_sum\_to\_all}@(TYPE *dest, const TYPE *source, int nreduce, shmem_team_t team);
\end{CsynopsisCol}
}
\begin{DeprecateBlock}
\begin{CsynopsisCol}
void @\FuncDecl{shmem\_\FuncParam{TYPENAME}\_sum\_to\_all}@(TYPE *dest, const TYPE *source, int nreduce, int PE_start, int logPE_stride, int PE_size, short *pWrk, long *pSync);
\end{CsynopsisCol}
\end{DeprecateBlock}
\newtext{where \TYPE{} is one of the integer, real, or complex types supported for the SUM operation and has a corresponding \TYPENAME{} as specified by Table \ref{reducetypes}.}

%% Fortran
\begin{Fsynopsis}
CALL @\FuncDecl{SHMEM\_COMP4\_SUM\_TO\_ALL}@(dest, source, nreduce, PE_start, logPE_stride, PE_size, pWrk, pSync)
CALL @\FuncDecl{SHMEM\_COMP8\_SUM\_TO\_ALL}@(dest, source, nreduce, PE_start, logPE_stride, PE_size, pWrk, pSync)
CALL @\FuncDecl{SHMEM\_INT4\_SUM\_TO\_ALL}@(dest, source, nreduce, PE_start, logPE_stride, PE_size, pWrk, pSync)
CALL @\FuncDecl{SHMEM\_INT8\_SUM\_TO\_ALL}@(dest, source, nreduce, PE_start, logPE_stride, PE_size, pWrk, pSync)
CALL @\FuncDecl{SHMEM\_REAL4\_SUM\_TO\_ALL}@(dest, source, nreduce, PE_start, logPE_stride, PE_size, pWrk, pSync)
CALL @\FuncDecl{SHMEM\_REAL8\_SUM\_TO\_ALL}@(dest, source, nreduce, PE_start, logPE_stride, PE_size, pWrk, pSync)
CALL @\FuncDecl{SHMEM\_REAL16\_SUM\_TO\_ALL}@(dest, source, nreduce, PE_start, logPE_stride, PE_size, pWrk, pSync)
\end{Fsynopsis}

\paragraph{PROD}
Performs a product reduction across a set of \acp{PE}.\newline

%% C11
{\color{Green}
\begin{C11synopsis}
int @\FuncDecl{shmem\_prod\_to\_all}@(TYPE *dest, const TYPE *source, int nreduce, shmem_team_t team);
\end{C11synopsis}
where \TYPE{} is one of the integer, real, or complex types supported for the PROD operation as specified by Table \ref{reducetypes}.
}

%% C/C++
\begin{Csynopsis}
\end{Csynopsis}
{\color{Green}
\begin{CsynopsisCol}
int @\FuncDecl{shmem\_team\_\FuncParam{TYPENAME}\_prod\_to\_all}@(TYPE *dest, const TYPE *source, int nreduce, shmem_team_t team);
\end{CsynopsisCol}
}
\begin{DeprecateBlock}
\begin{CsynopsisCol}
void @\FuncDecl{shmem\_\FuncParam{TYPENAME}\_prod\_to\_all}@(TYPE *dest, const TYPE *source, int nreduce, int PE_start, int logPE_stride, int PE_size, short *pWrk, long *pSync);
\end{CsynopsisCol}
\end{DeprecateBlock}
\newtext{where \TYPE{} is one of the integer, real, or complex types supported for the PROD operation and has a corresponding \TYPENAME{} as specified by Table \ref{reducetypes}.}

%% Fortran
\begin{Fsynopsis}
CALL @\FuncDecl{SHMEM\_COMP4\_PROD\_TO\_ALL}@(dest, source, nreduce, PE_start, logPE_stride, PE_size, pWrk, pSync)
CALL @\FuncDecl{SHMEM\_COMP8\_PROD\_TO\_ALL}@(dest, source, nreduce, PE_start, logPE_stride, PE_size, pWrk, pSync)
CALL @\FuncDecl{SHMEM\_INT4\_PROD\_TO\_ALL}@(dest, source, nreduce, PE_start, logPE_stride, PE_size, pWrk, pSync)
CALL @\FuncDecl{SHMEM\_INT8\_PROD\_TO\_ALL}@(dest, source, nreduce, PE_start, logPE_stride, PE_size, pWrk, pSync)
CALL @\FuncDecl{SHMEM\_REAL4\_PROD\_TO\_ALL}@(dest, source, nreduce, PE_start, logPE_stride, PE_size, pWrk, pSync)
CALL @\FuncDecl{SHMEM\_REAL8\_PROD\_TO\_ALL}@(dest, source, nreduce, PE_start, logPE_stride, PE_size, pWrk, pSync)
CALL @\FuncDecl{SHMEM\_REAL16\_PROD\_TO\_ALL}@(dest, source, nreduce, PE_start, logPE_stride, PE_size, pWrk, pSync)
\end{Fsynopsis}



\begin{apiarguments}

\apiargument{OUT}{dest}{A symmetric array, of length \VAR{nreduce} elements, to
    receive the result of the reduction routines.  The data type of \dest{} varies
    with the version of the reduction routine being called.  When calling from
    \CorCpp, refer to the SYNOPSIS section for data type information.}
\apiargument{IN}{source}{ A symmetric array, of length \VAR{nreduce} elements, that
    contains one element for each separate reduction routine.  The \source{}
    argument must have the same data type as \dest.}
\apiargument{IN}{nreduce}{The number of elements in the \dest{} and \source{}
    arrays.  \VAR{nreduce} must be of type integer.  When using \Fortran, it
    must be a default integer value.}

\newtext{%
\apiargument{IN}{team}{The team over which to perform the operation.}%
}

\begin{DeprecateBlock}
\apiargument{IN}{PE\_start}{The lowest \ac{PE} number of the active set of
    \acp{PE}.  \VAR{PE\_start} must be of type integer.  When using \Fortran,
    it must be a default integer value.}
\apiargument{IN}{logPE\_stride}{The log (base 2) of the stride between consecutive
    \ac{PE} numbers in the active set.  \VAR{logPE\_stride} must be of type integer.
    When using \Fortran, it must be a default integer value.}
\apiargument{IN}{PE\_size}{The number of \acp{PE} in the active set.
    \VAR{PE\_size} must be of type integer.  When using \Fortran, it must be a
    default integer value.}
\apiargument{IN}{pWrk}{
    A symmetric work array of size at least
    max(\VAR{nreduce}/2 + 1, \CONST{SHMEM\_REDUCE\_MIN\_WRKDATA\_SIZE})
    elements.}
\apiargument{IN}{pSync}{
    A symmetric work array of size \CONST{SHMEM\_REDUCE\_SYNC\_SIZE}.
    In \CorCpp, \VAR{pSync} must be an array of elements of type \CTYPE{long}.
    In \Fortran, \VAR{pSync} must be an array of elements of default integer type.
    Every element of this array must be initialized with the value
    \CONST{SHMEM\_SYNC\_VALUE} before any of the \acp{PE} in the active set
    enter the reduction routine.}
\end{DeprecateBlock}

\end{apiarguments}

\apidescription{
    \openshmem reduction routines \newtext{are collective routines over an active set or
    existing \openshmem team that} compute one or more reductions across symmetric
    arrays on multiple \acp{PE}.  A reduction performs an associative binary routine
    across a set of values.

    The \VAR{nreduce} argument determines the number of separate reductions to
    perform.  The \source{} array on all \acp{PE} \newtext{participating in the reduction}
    \oldtext{in the active set} %%
    provides one element for each reduction.  The results of the reductions are placed in the
    \dest{} array on all \acp{PE} \newtext{participating in the reduction.}
    \oldtext{in the active set.} %%
    
    The \source{} and \dest{} arrays may be the same array, but they may not be
    overlapping arrays. The same \dest{} and \source{} arrays
    must be passed to all \acp{PE} \newtext{participating in the reduction.}
    \oldtext{in the active set.} %%

{\color{Green}
    Team-based reduction routines operate over all \acp{PE} in the provided team argument. All
    \acp{PE} in the provided team must participate in the reduction. If an invalid team handle
    or \LibConstRef{SHMEM\_TEAM\_NULL} is passed to this routine, the behavior is undefined.

    Active-set-based sync routines operate over all \acp{PE} in the active set
    defined by the \VAR{PE\_start}, \VAR{logPE\_stride}, \VAR{PE\_size} triplet.
}

    As with all \oldtext{\openshmem} \newtext{active set-based} collective routines,
    each of these routines assumes
    that only \acp{PE} in the active set call the routine.  If a \ac{PE} not in
    the active set calls an \oldtext{\openshmem} \newtext{active set-based} collective routine,
    the behavior is undefined.

    The values of arguments \VAR{nreduce}, \VAR{PE\_start}, \VAR{logPE\_stride},
    and \VAR{PE\_size} must be equal on all \acp{PE} in the active set.
    The same \VAR{pWrk} and \VAR{pSync} work arrays must be passed to all
    \acp{PE} in the active set.

    Before any \ac{PE} calls a reduction routine, the following conditions must be ensured:
    \begin{itemize}
    \item The \dest{} array on all \acp{PE} \newtext{participating in the reduction}
      \oldtext{in the active set} %%
      is ready to accept the results of the \OPR{reduction}.
    \item \newtext{If using active-set-based routines,} the
      \VAR{pWrk} and \VAR{pSync} arrays on all \acp{PE} in the
      active set are not still in use from a prior call to a collective
      \openshmem routine.
    \end{itemize}
    Otherwise, the behavior is undefined.
    
    Upon return from a reduction routine, the following are true for the local
    \ac{PE}:
    \begin{itemize}
    \item The \dest{} array is updated and the \source{} array may be safely reused.
    \item \newtext{If using active-set-based routines,}
    the values in the \VAR{pSync} array are restored to the original values.
    \end{itemize}

    The complex-typed interfaces are only provided for sum and product reductions.
    When the \Cstd translation environment does not support complex types
    \footnote{That is, under \Cstd language standards prior to \Cstd[99] or under \Cstd[11]
    when \CONST{\_\_STDC\_NO\_COMPLEX\_\_} is defined to 1}, an \openshmem
    implementation is not required to provide support for these
    complex-typed interfaces.
}



%\deprecationstart
\apidesctable{
    When calling from \Fortran, the \dest{} date types are as follows:
}{Routine}{Data type}
    \apitablerow{shmem\_int8\_and\_to\_all}{Integer, with an element size of 8 bytes.}
    \apitablerow{shmem\_int4\_and\_to\_all}{Integer, with an element size of 4 bytes.}
    \apitablerow{shmem\_comp8\_max\_to\_all}{Complex, with an element size equal to two 8-byte real values.}
    \apitablerow{shmem\_int4\_max\_to\_all}{Integer, with an element size of 4 bytes.}
    \apitablerow{shmem\_int8\_max\_to\_all}{Integer, with an element size of 8 bytes.}
    \apitablerow{shmem\_real4\_max\_to\_all}{Real, with an element size of 4 bytes.}
    \apitablerow{shmem\_real16\_max\_to\_all}{Real, with an element size of 16 bytes.}
    \apitablerow{shmem\_int4\_min\_to\_all}{Integer, with an element size of 4 bytes.}
    \apitablerow{shmem\_int8\_min\_to\_all}{Integer, with an element size of 8 bytes.}
    \apitablerow{shmem\_real4\_min\_to\_all}{Real, with an element size of 4 bytes.}
    \apitablerow{shmem\_real8\_min\_to\_all}{Real, with an element size of 8 bytes.}
    \apitablerow{shmem\_real16\_min\_to\_all}{Real,with an element size of 16 bytes.}
    \apitablerow{shmem\_comp4\_sum\_to\_all}{Complex, with an element size equal to two 4-byte real values.}
    \apitablerow{shmem\_comp8\_sum\_to\_all}{Complex, with an element size equal to two 8-byte real values.}
    \apitablerow{shmem\_int4\_sum\_to\_all}{Integer, with an element size of 4 bytes.}
    \apitablerow{shmem\_int8\_sum\_to\_all}{Integer, with an element size of 8 bytes..}
    \apitablerow{shmem\_real4\_sum\_to\_all}{Real, with an element size of 4 bytes.}
    \apitablerow{shmem\_real8\_sum\_to\_all}{Real, with an element size of 8 bytes.}
    \apitablerow{shmem\_real16\_sum\_to\_all}{Real, with an element size of 16 bytes.}
    \apitablerow{shmem\_comp4\_prod\_to\_all}{Complex, with an element size equal to two 4-byte real values.}
    \apitablerow{shmem\_comp8\_prod\_to\_all}{Complex, with an element size equal to two 8-byte real values.}
    \apitablerow{shmem\_int4\_prod\_to\_all}{Integer, with an element size of 4 bytes.}
    \apitablerow{shmem\_int8\_prod\_to\_all}{Integer, with an element size of 8 bytes.}
    \apitablerow{shmem\_real4\_prod\_to\_all}{Real, with an element size of 4 bytes.}
    \apitablerow{shmem\_real8\_prod\_to\_all}{Real, with an element size of 8 bytes.}
    \apitablerow{shmem\_real16\_prod\_to\_all}{Real, with an element size of 16 bytes.}
    \apitablerow{shmem\_int8\_or\_to\_all}{Integer, with an element size of 8 bytes.}
    \apitablerow{shmem\_int4\_or\_to\_all}{Integer, with an element size of 4 bytes.}
    \apitablerow{shmem\_int8\_xor\_to\_all}{Integer, with an element size of 8 bytes.}
    \apitablerow{shmem\_int4\_xor\_to\_all}{Integer, with an element size of 4 bytes.}

%\deprecationend


\apireturnvalues{
  \newtext{Zero on successful local completion. Nonzero otherwise.}
}

\apinotes{
\newtext{%
    There are no specifically defined error codes for this routine.
    See section \ref{subsec:error_handling} for expected error checking and
    return code behavior specific to implementations. For portable
    error checking and debugging behavior, programs should do their own checks
    for invalid team handles or \LibConstRef{SHMEM\_TEAM\_NULL}
    }

    All \openshmem reduction routines reset the values in \VAR{pSync} before they
    return, so a particular \VAR{pSync} buffer need only be initialized the first
    time it is used. The user must ensure that the \VAR{pSync} array is not being updated on any \ac{PE}
    in the active set while any of the \acp{PE} participate in processing of an
    \openshmem reduction routine. Be careful to avoid the following situations: If
    the \VAR{pSync} array is initialized at run time, some type of synchronization
    is needed to ensure that all \acp{PE} in the working set have initialized
    \VAR{pSync} before any of them enter an \openshmem routine called with the
    \VAR{pSync} synchronization array. A \VAR{pSync} or \VAR{pWrk} array can be
    reused in a subsequent reduction routine call only if none of the \acp{PE} in
    the active set are still processing a prior reduction routine call that used
    the same \VAR{pSync} or \VAR{pWrk} arrays. In general, this can be assured only
    by doing some type of synchronization.
}

\begin{apiexamples}

\apifexample
    {This \Fortran reduction example statically initializes the \VAR{pSync} array
    and finds the logical \OPR{AND} of the integer variable \VAR{FOO} across all
    even \acp{PE}.}
    {./example_code/shmem_and_example.f90}
    {}

\apifexample
    {This \Fortran example statically initializes the \VAR{pSync} array and finds
    the \OPR{maximum} value of real variable \VAR{FOO} across all even \acp{PE}.}
    {./example_code/shmem_max_example.f90}
    {}

\apifexample
    { This \Fortran example statically initializes the \VAR{pSync} array and finds
    the \OPR{minimum} value of real variable \VAR{FOO} across all the even
    \acp{PE}.}
    {./example_code/shmem_min_example.f90}
    {}

\apifexample
    {This \Fortran example statically initializes the \VAR{pSync} array and finds
    the \OPR{sum} of the real variable \VAR{FOO} across all even \acp{PE}.}
    {./example_code/shmem_sum_example.f90}
    {}

\apifexample
    {This \Fortran example statically initializes the \VAR{pSync} array and finds
    the \OPR{product} of the real variable \VAR{FOO} across all the even \acp{PE}.}
    {./example_code/shmem_prod_example.f90}
    {}

\apifexample
    {This \Fortran example statically initializes the \VAR{pSync} array and finds
    the logical \OPR{OR} of the integer variable \VAR{FOO} across all even
    \acp{PE}.}
    {./example_code/shmem_or_example.f90}
    {}

\apifexample
    {This \Fortran example statically initializes the \VAR{pSync} array and
    computes the exclusive \OPR{XOR} of variable \VAR{FOO} across all even
    \acp{PE}.}
    {./example_code/shmem_xor_example.f90}
    {}

\end{apiexamples}

\end{apidefinition}
