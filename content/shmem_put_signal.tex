\apisummary{
    The \OPR{put-with-signal} routines provide a method for copying data from a
    contiguous local data object to a data object on a specified \ac{PE}
    and subsequently updating a remote flag to signal completion.
}

\begin{apidefinition}

\begin{C11synopsis}
void @\FuncDecl{shmem\_put\_signal}@(TYPE *dest, const TYPE *source, size_t nelems, uint64_t *sig_addr, uint64_t signal, int sig_op, int pe);
void @\FuncDecl{shmem\_put\_signal}@(shmem_ctx_t ctx, TYPE *dest, const TYPE *source, size_t nelems, uint64_t *sig_addr, uint64_t signal, int sig_op, int pe);
\end{C11synopsis}
where \TYPE{} is one of the standard \ac{RMA} types specified by Table \ref{stdrmatypes}.

\begin{Csynopsis}
void @\FuncDecl{shmem\_\FuncParam{TYPENAME}\_put\_signal}@(TYPE *dest, const TYPE *source, size_t nelems, uint64_t *sig_addr, uint64_t signal, int sig_op, int pe);
void @\FuncDecl{shmem\_ctx\_\FuncParam{TYPENAME}\_put\_signal}@(shmem_ctx_t ctx, TYPE *dest, const TYPE *source, size_t nelems, uint64_t *sig_addr, uint64_t signal, int sig_op, int pe);
\end{Csynopsis}
where \TYPE{} is one of the standard \ac{RMA} types and has a corresponding \TYPENAME{} specified by Table \ref{stdrmatypes}.

\begin{CsynopsisCol}
void @\FuncDecl{shmem\_put\FuncParam{SIZE}\_signal}@(void *dest, const void *source, size_t nelems, uint64_t *sig_addr, uint64_t signal, int sig_op, int pe);
void @\FuncDecl{shmem\_ctx\_put\FuncParam{SIZE}\_signal}@(shmem_ctx_t ctx, void *dest, const void *source, size_t nelems, uint64_t *sig_addr, uint64_t signal, int sig_op, int pe);
\end{CsynopsisCol}
where \SIZE{} is one of \CONST{8, 16, 32, 64, 128}.

\begin{CsynopsisCol}
void @\FuncDecl{shmem\_putmem\_signal}@(void *dest, const void *source, size_t nelems, uint64_t *sig_addr, uint64_t signal, int sig_op, int pe);
void @\FuncDecl{shmem\_ctx\_putmem\_signal}@(shmem_ctx_t ctx, void *dest, const void *source, size_t nelems, uint64_t *sig_addr, uint64_t signal, int sig_op, int pe);
\end{CsynopsisCol}

\begin{apiarguments}
    \apiargument{IN}{ctx}{A context handle specifying the context on which to
    perform the operation. When this argument is not provided, the operation is
    performed on the default context.}
    \apiargument{OUT}{dest}{Data object to be updated on the remote \ac{PE}.
    This data object must be remotely accessible.}
    \apiargument{IN}{source}{Data object containing the data to be copied.}
    \apiargument{IN}{nelems}{Number of elements in the \dest{} and \source{}
    arrays. \VAR{nelems} must be of type \VAR{size\_t} for \Cstd.}
    \apiargument{OUT}{sig\_addr}{signal data object to be updated on the remote
    \ac{PE} as a signal. This signal data object must be remotely accessible.}
    \apiargument{IN}{signal}{Unsigned 64-bit value that is used for updating the
    remote \VAR{sig\_addr} signal data object.}
    \apiargument{IN}{sig\_op}{Signal operator that represents the type of update
    to be performed on the remote \VAR{sig\_addr} signal data object.}
    \apiargument{IN}{pe}{\ac{PE} number of the remote \ac{PE}.}
\end{apiarguments}

\apidescription{
    The \OPR{put-with-signal} routines provide a method for copying data from a
    contiguous local data object to a data object on a specified \ac{PE}
    and subsequently updating a remote flag to signal completion. The routines
    return after the data has been copied out of the \source{} array on the
    local \ac{PE}.

    The \VAR{sig\_op} signal operator determines the type of update to be
    performed on the remote \VAR{sig\_addr} signal data object. The completion
    of signal update based on the \VAR{sig\_op} signal operator using the
    \VAR{signal} flag on the remote \ac{PE} indicates the delivery of its
    corresponding \dest{} data words into the data object on the remote \ac{PE}.

    An update to the \VAR{sig\_addr} signal data object through a
    \OPR{put-with-signal} routine completes as if performed atomically as described in
    Section~\ref{subsec:signal_atomicity}. The various options as described in
    Section~\ref{subsec:signal_operator} can be used as the \VAR{sig\_op} signal
    operator.
}

\apireturnvalues{
    None.
}

\apinotes{
    The \dest{} and \VAR{sig\_addr} data objects must both be remotely
    accessible. The \VAR{sig\_addr} and \dest{} could be of different kinds,
    for example, one could be a global/static \Cstd variable and the other could
    be allocated on the symmetric heap.

    \VAR{sig\_addr} and \dest{} may not be overlapping in memory.

    The completion of signal update using the \VAR{signal} flag on the remote
    \ac{PE} indicates only the delivery of its corresponding \dest{} data words
    into the data object on the remote \ac{PE}. Without a memory-ordering
    operation, there is no implied ordering between the signal update of a
    \OPR{put-with-signal} routine and another data transfer. For example, the
    completion of the signal update in a sequence consisting of a put routine
    followed by a \OPR{put-with-signal} routine does not imply delivery of the put
    routine's data.
}

\begin{apiexamples}

\apicexample
    {The following example demonstrates the usage of \FUNC{shmem\_put\_signal}.
    It shows the implementation of a broadcast operation from \ac{PE} 0 to
    itself and all other \acp{PE} in the job as a simple ring-based algorithm
    using \FUNC{shmem\_put\_signal}:}
    {./example_code/shmem_put_signal_example.c}
    {}
\end{apiexamples}

\end{apidefinition}
