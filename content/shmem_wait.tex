\apisummary{
    Wait for a variable on the local \ac{PE} to change.
}

\begin{apidefinition}

\begin{Csynopsis}
void shmem_int_wait(volatile int *ivar, int cmp_value);
void shmem_int_wait_until(volatile int *ivar, int cmp, int cmp_value);
void shmem_long_wait(volatile long *ivar, long cmp_value);
void shmem_long_wait_until(volatile long *ivar, int cmp, long cmp_value);
void shmem_longlong_wait(volatile long long *ivar, long long cmp_value);
void shmem_longlong_wait_until(volatile long long *ivar, int cmp, long long cmp_value);
void shmem_short_wait(volatile short *ivar, short cmp_value);
void shmem_short_wait_until(volatile short *ivar, int cmp, short cmp_value);
void shmem_wait(volatile long *ivar, long cmp_value);
void shmem_wait_until(volatile long *ivar, int cmp, long cmp_value);
\end{Csynopsis}

\begin{Fsynopsis}
CALL SHMEM_INT4_WAIT(ivar, cmp_value)
CALL SHMEM_INT4_WAIT_UNTIL(ivar, cmp, cmp_value)
CALL SHMEM_INT8_WAIT(ivar, cmp_value)
CALL SHMEM_INT8_WAIT_UNTIL(ivar, cmp, cmp_value)
CALL SHMEM_WAIT(ivar, cmp_value)
CALL SHMEM_WAIT_UNTIL(ivar, cmp, cmp_value)
\end{Fsynopsis}

\begin{apiarguments}

\apiargument{OUT}{ivar}{A remotely accessible integer variable that is being updated
    by another \ac{PE}.  If you are using \CorCpp, the type of \VAR{ivar} should
    match that implied in the SYNOPSIS section.} 
\apiargument{IN}{cmp}{The compare operator that compares \VAR{ivar} with
    \VAR{cmp\_value}.  \VAR{cmp} must be of type integer.  If you are using
    \Fortran, it must  be of  default kind.  If you are using \CorCpp, the type of
    \VAR{cmp} should match that implied in the SYNOPSIS section.}        
\apiargument{IN}{cmp\_value}{\VAR{cmp\_value} must be of type integer.  If you are
    using \CorCpp, the type of \VAR{cmp\_value} should match that implied in the
    SYNOPSIS section.  If  you are using \Fortran, cmp\_value must be an integer of
    the same size and kind as \VAR{ivar}.}

\end{apiarguments}

\apidescription{ 
    \FUNC{shmem\_wait} and \FUNC{shmem\_wait\_until} wait for \VAR{ivar} to be
    changed by a remote write or an atomic operation issued by a different \ac{PE}.
    These  routines can be used for point-to-point direct synchronization.  A call
    to \VAR{shmem\_wait} does not return until some other \ac{PE} writes  a value,
    not equal to \VAR{cmp\_value}, into \VAR{ivar} on the waiting \ac{PE}.  A call
    to \FUNC{shmem\_wait\_until} does not return until some  other \ac{PE} changes
    \VAR{ivar} to satisfy the condition implied by \VAR{cmp} and \VAR{cmp\_value}.
    This mechanism is useful when a \ac{PE} needs to tell another \ac{PE} that it
    has completed some action.  The \FUNC{shmem\_wait}  routines return when
    \VAR{ivar} is no  longer  equal  to \VAR{cmp\_value}. The
    \FUNC{shmem\_wait\_until} routines return when the compare condition is true.
    The compare condition is defined by the \VAR{ivar}  argument  compared with the
    \VAR{cmp\_value} using the comparison operator, \VAR{cmp}. 
}


\apidesctable{
    If you are using \Fortran, \VAR{ivar} must be a specific sized integer type
    according to the routine being called, as follows:
}{Routine}{Data type}

\apitablerow{shmem\_wait, shmem\_wait\_until}{default INTEGER}
\apitablerow{shmem\_int4\_wait, shmem\_int4\_wait\_until}{INTEGER*4}
\apitablerow{shmem\_int8\_wait, shmem\_int8\_wait\_until}{INTEGER*8}

\apidesctable{
    The following \VAR{cmp} values are supported:
}{CMP Value}{Comparison}

\CorCpp:\\
\apitablerow{SHMEM\_CMP\_EQ }{  Equal}
\apitablerow{SHMEM\_CMP\_NE}{Not equal}
\apitablerow{SHMEM\_CMP\_GT}{Greater than}
\apitablerow{SHMEM\_CMP\_LE}{Less than or equal to}
\apitablerow{SHMEM\_CMP\_LT}{Less than}
\apitablerow{SHMEM\_CMP\_GE}{Greater than or equal to}
\\
\Fortran:\\
\apitablerow{SHMEM\_CMP\_EQ }{  Equal}
\apitablerow{SHMEM\_CMP\_NE}{Not equal}
\apitablerow{SHMEM\_CMP\_GT}{Greater than}
\apitablerow{SHMEM\_CMP\_LE}{Less than or equal to}
\apitablerow{SHMEM\_CMP\_LT}{Less than}
\apitablerow{SHMEM\_CMP\_GE}{Greater than or equal to}

\apireturnvalues{
    None.
}

\apinotes{
    \newtext{
    The order in which updates performed through the OpenSHMEM interface become
    visible in local memory at the target \ac{PE} is not specified. This order
    is affected by multiple factors, including the memory model of the
    underlying architecture. The \FUNC{shmem\_wait} and
    \FUNC{shmem\_wait\_until} operations can be used to portably ensure
    visibility of updates that occurred and were ordered prior to the update
    satisfying the wait operation.}

}

\apiimpnotes{
    Implementations must ensure that \FUNC{shmem\_wait} and
    \FUNC{shmem\_wait\_until} do not return before the update of the memory
    indicated by \VAR{ivar} is fully complete.  Partial updates to the memory
    must not cause \FUNC{shmem\_wait} or \FUNC{shmem\_wait\_until} to return.
}


\begin{apiexamples}

\apifexample
{ The following call returns when variable ivar is not equal to 100:}
{./example_code/shmem_wait1_example.f90}
{}

\apifexample
{ The following call to \FUNC{SHMEM\_INT8\_WAIT\_UNTIL} is  equivalent to the
call to \FUNC{SHMEM\_INT8\_WAIT} in example 1:}
{./example_code/shmem_wait2_example.f90}
{}

\apicexample
{The following \CorCpp{} call waits until the sign bit in ivar is set by a
transfer from a remote PE:}
{./example_code/shmem_wait3_example.f90}
{}

\apifexample
{The following \Fortran{} example is in the context of a subroutine:}
{./example_code/shmem_wait4_example.f90}
{}

\end{apiexamples}

\end{apidefinition}
