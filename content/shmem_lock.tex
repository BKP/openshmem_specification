\apisummary{
    Releases, locks, and tests a mutual exclusion memory lock.
}
\begin{apidefinition}

\begin{Csynopsis}
void @\FuncDecl{shmem\_clear\_lock}@(long *lock);
void @\FuncDecl{shmem\_set\_lock}@(long *lock);
int @\FuncDecl{shmem\_test\_lock}@(long *lock);
\end{Csynopsis}

\begin{apiarguments}
\apiargument{IN}{lock}{Symmetric address of data object that is a scalar variable or an array
    of length \CONST{1}.  This data object must be set to \CONST{0} on all
    \acp{PE} prior to the first use.}
\end{apiarguments}

\apidescription{
    The \FUNC{shmem\_set\_lock} routine sets a mutual exclusion lock after
    waiting for the lock to be freed by any other \ac{PE} currently holding
    the lock.  Waiting \acp{PE} are guaranteed to set the lock in a
    first-come, first-served manner.  The \FUNC{shmem\_test\_lock} routine sets
    a mutual exclusion lock only if it is currently cleared.  By using this
    routine, a \ac{PE} can avoid blocking on a set lock.  If the lock is
    currently set, the routine returns without waiting.  The
    \FUNC{shmem\_clear\_lock} routine releases a lock previously set by
    \FUNC{shmem\_set\_lock} or \FUNC{shmem\_test\_lock} after performing a
    quiet operation on the default context to ensure that all symmetric memory
    accesses that occurred during the critical region are complete.  These
    routines are appropriate for protecting a critical region from simultaneous
    update by multiple \acp{PE}.

    The \openshmem lock \ac{API} provides a non-reentrant mutex.  Thus, a call to
    \FUNC{shmem\_set\_lock} or \FUNC{shmem\_test\_lock} when the calling \ac{PE}
    already holds the given lock will result in undefined behavior.  In a
    multithreaded \openshmem program, the user must ensure that such calls do
    not occur.
}

\apireturnvalues{
    The \FUNC{shmem\_test\_lock} routine returns \CONST{0} if the lock was
    originally cleared and this call was able to set the lock. A value of
    \CONST{1} is returned if the lock had been set and the call returned without
    waiting to set the lock.
}

\apinotes{
    The lock variable must be initialized to zero before any \ac{PE} performs an
    \openshmem lock operation on the given variable.  Accessing an in-use lock
    variable using any method other than the \openshmem lock \ac{API} (e.g, using
    local load/store, \ac{RMA}, or \ac{AMO} operations) results in undefined behavior.

    Calls to \FUNC{shmem\_ctx\_quiet} can be performed prior to calling the
    \FUNC{shmem\_clear\_lock} routine to ensure completion of operations issued
    on additional contexts.
}

\begin{apiexamples}

\apicexample
    {The following example uses \FUNC{shmem\_lock} in a \Cstd[11] program.}
    {./example_code/shmem_lock_example.c}
    {}

\end{apiexamples}

\end{apidefinition}
