\apisummary{
    Releases, locks, and tests a mutual exclusion memory lock.
}
\begin{apidefinition}

\begin{Csynopsis}
void shmem_clear_lock(long *lock);
void shmem_set_lock(long *lock);
int shmem_test_lock(long *lock);
\end{Csynopsis}

\begin{Fsynopsis}
INTEGER lock, SHMEM_TEST_LOCK
CALL SHMEM_CLEAR_LOCK(lock)
CALL SHMEM_SET_LOCK(lock)
I = SHMEM_TEST_LOCK(lock)
\end{Fsynopsis}

\begin{apiarguments}
\apiargument{IN}{lock}{A symmetric data object that is a scalar variable or an array
    of  length \CONST{1}.  This data  object  must  be set to \CONST{0} on all
    \ac{PE}s prior to the first use.  \VAR{lock}  must  be  of type \CONST{long}.
    If you are using \Fortran, it must be of default kind.}
\end{apiarguments}

\apidescription{
    The \FUNC{shmem\_set\_lock} routine sets a mutual exclusion lock after  waiting
    for  the lock  to be freed by any other \ac{PE} currently holding the lock.
    Waiting \ac{PE}s are assured of getting the lock in a first-come, first-served
    manner.  The \FUNC{shmem\_clear\_lock} routine releases a lock  previously set
    by \FUNC{shmem\_set\_lock} after ensuring that all local and remote	 stores
    initiated in the critical region are complete.  The \FUNC{shmem\_test\_lock}
    routine sets a mutual exclusion lock only if it is currently cleared.  By using
    this routine, a \ac{PE} can avoid blocking on a set lock.  If the lock is
    currently set, the routine returns without waiting.  These routines are
    appropriate for protecting a critical region from simultaneous update by
    multiple \ac{PE}s.	  
}

\apireturnvalues{
    The \FUNC{shmem\_test\_lock} routine returns \CONST{0} if  the lock  was
    originally cleared and  this  call was  able  to set the lock.  A value of
    \CONST{1} is returned if the lock had been set and the call returned without
    waiting to set the lock.
}

\apinotes{
    The term symmetric data object is defined in Section \ref{subsec:memory_model}.
    The lock variable should always be initialized to zero and accessed only by the \openshmem locking
    \ac{API}.  Changing the value of the lock variable by other means without using
    the \openshmem \ac{API}, can lead to undefined behavior.
}

\begin{apiexamples}

\apicexample
    {The following example uses \FUNC{shmem\_lock} in a \Cstd program.}
    {./example_code/shmem_lock_example.c}
    {}

\end{apiexamples}

\end{apidefinition}
