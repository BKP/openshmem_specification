\apisummary{
    Wait on an array of variables on the local \ac{PE} until any one variable meets the specified wait condition.
}

\begin{apidefinition}

\begin{C11synopsis}
size_t @\FuncDecl{shmem\_wait\_until\_any}@(TYPE * restrict ivars, size_t nelems, int * restrict status,
    int cmp, TYPE cmp_value);
\end{C11synopsis}
where \TYPE{} is one of the point-to-point synchronization types specified by
Table \ref{p2psynctypes}.

\begin{Csynopsis}
size_t @\FuncDecl{shmem\_\FuncParam{TYPENAME}\_wait\_until\_any}@(TYPE * restrict ivars, size_t nelems,
    int * restrict status, int cmp, TYPE cmp_value);
\end{Csynopsis}
where \TYPE{} is one of the point-to-point synchronization types and has a
corresponding \TYPENAME{} specified by Table~\ref{p2psynctypes}.

\begin{apiarguments}

  \apiargument{IN}{ivars}{A pointer to an array of remotely accessible data
    objects.}
  \apiargument{IN}{nelems}{The number of elements in the \VAR{ivars} array.}
  \apiargument{INOUT}{status}{An optional mask array of length \VAR{nelems}
    that indicates which elements in \VAR{ivars} are excluded from the wait set.}
  \apiargument{IN}{cmp}{A comparison operator from Table~\ref{p2p-consts}
    that compares elements of \VAR{ivars} with \VAR{cmp\_value}.}
  \apiargument{IN}{cmp\_value}{The value to be compared with the objects
    pointed to by \VAR{ivars}.}

\end{apiarguments}

\apidescription{ 
    The \FUNC{shmem\_wait\_until\_any} routine waits until any one entry in the
    wait set specified by \VAR{ivars} and \VAR{status} satisfies the wait
    condition at the calling \ac{PE}.  The order in which these elements are
    waited upon is unspecified.  If an entry $i$ in \VAR{ivars} within the wait
    set satisfies the wait condition, a series of calls to
    \FUNC{shmem\_wait\_until\_any} must eventually return $i$.

    The optional \VAR{status} is a mask array of length \VAR{nelems} where each
    element corresponds to the respective element in \VAR{ivars} and indicates
    whether the element is excluded from the wait set.  On input, elements of
    \VAR{status} set to 0 will be included in the wait set, and elements set to
    1 will be ignored.  If all elements in \VAR{status} are set to 1 or
    \VAR{nelems} is 0, the wait set is empty and this routine returns
    SIZE\_MAX.  On output, the element of \VAR{status} with the index given by
    the return value of \FUNC{shmem\_wait\_until\_any} is set to 1.  If
    \VAR{status} is a null pointer, it is ignored and all elements in
    \VAR{ivars} are included in the wait set.
}

\apireturnvalues{
    \FUNC{shmem\_wait\_until\_any} returns the index of an element in the
    \VAR{ivars} array that satisfies the wait condition. If the wait set is
    empty, this routine returns SIZE\_MAX.
}

\apinotes{
    None.
}

\apiimpnotes{
    Implementations must ensure that \FUNC{shmem\_wait\_until\_any} does not
    return before the update of the memory indicated by \VAR{ivars} is fully
    complete.  Partial updates to the memory must not cause
    \FUNC{shmem\_wait\_until\_any} to return.
}


\begin{apiexamples}
  \apicexample
      {The following \Cstd[11] example demonstrates the use of
      \FUNC{shmem\_wait\_until\_any} to process a simple all-to-all transfer
      of N data elements via a sum reduction.}
      {./example_code/shmem_wait_until_any_all2all_sum.c}
      {}

\end{apiexamples}

\end{apidefinition}

