\apisummary{
    Indicate whether at least one variable within an array of variables on the
    local \ac{PE} meets its specified test condition.
}

\begin{apidefinition}

\begin{C11synopsis}
size_t @\FuncDecl{shmem\_test\_some\_vector}@(TYPE *ivars, size_t nelems, size_t *indices, const int *status,
    int cmp, TYPE *cmp_values);
\end{C11synopsis}
where \TYPE{} is one of the standard \ac{AMO} types specified by
Table \ref{stdamotypes}.

\begin{Csynopsis}
size_t @\FuncDecl{shmem\_\FuncParam{TYPENAME}\_test\_some\_vector}@(TYPE *ivars, size_t nelems, size_t *indices,
    const int *status, int cmp, TYPE *cmp_values);
\end{Csynopsis}
where \TYPE{} is one of the standard \ac{AMO} types and has a
corresponding \TYPENAME{} specified by Table \ref{stdamotypes}.

\begin{apiarguments}

  \apiargument{IN}{ivars}{Symmetric address of an array of remotely accessible data
    objects.
    The type of \VAR{ivars} should match that implied in the SYNOPSIS section.}
  \apiargument{IN}{nelems}{The number of elements in the \VAR{ivars} array.}
  \apiargument{OUT}{indices}{Local address of an array of indices of length at least
    \VAR{nelems} into \VAR{ivars} that satisfied the test condition.}
  \apiargument{IN}{status}{Local address of an optional mask array of length \VAR{nelems}
    that indicates which elements in \VAR{ivars} are excluded from the test set.}
  \apiargument{IN}{cmp}{A comparison operator from Table~\ref{p2p-consts} that
    compares elements of \VAR{ivars} with elements of \VAR{cmp\_values}.}
  \apiargument{IN}{cmp\_values}{Local address of an array of length \VAR{nelems}
    containing values to be compared with the respective objects in \VAR{ivars}.
    The type of \VAR{cmp\_values} should match that implied in the SYNOPSIS section.}

\end{apiarguments}

\apidescription{
    The \FUNC{shmem\_test\_some\_vector} routine indicates whether at
    least one entry in the test set specified by \VAR{ivars} and \VAR{status}
    satisfies the test condition at the calling \ac{PE}.  The \VAR{ivars}
    objects at the calling \ac{PE} may be updated by an \ac{AMO} performed by a
    thread located within the calling \ac{PE} or within another \ac{PE}.
    This routine does not
    block and returns zero if no entries in \VAR{ivars} satisfied the test
    condition.  This routine compares each element of the \VAR{ivars}
    array in the test set with each respective value in \VAR{cmp\_values}
    according to the comparison operator \VAR{cmp} at the calling \ac{PE}.
    This routine tests all elements of \VAR{ivars} in the test set at least
    once, and the order in which the elements are tested is unspecified.

    Upon return, the \VAR{indices} array contains the indices of the elements
    in the test set that satisfied the test condition during the call to
    \FUNC{shmem\_test\_some\_vector}.  The return value of
    \FUNC{shmem\_test\_some\_vector} is equal to the total number of
    these satisfied elements.  If the return value is $N$, then the first $N$
    elements of the \VAR{indices} array contain those unique indices that
    satisfied the test condition.  These first $N$ elements of \VAR{indices}
    may be unordered with respect to the corresponding indices of \VAR{ivars}.
    The array pointed to by \VAR{indices} must be at least \VAR{nelems} long.
    If an entry $i$ in \VAR{ivars} within the test set satisfies the test
    condition, a series of calls to \FUNC{shmem\_test\_some\_vector}
    must eventually include $i$ in the \VAR{indices} array.

    The optional \VAR{status} is a mask array of length \VAR{nelems} where each element
    corresponds to the respective element in \VAR{ivars} and indicates whether
    the element is excluded from the test set.  Elements of \VAR{status} set to
    0 will be included in the test set, and elements set to a nonzero value will be ignored.  If all
    elements in \VAR{status} are nonzero or \VAR{nelems} is 0, the test set is
    empty and this routine returns 0.  If \VAR{status} is a null pointer, it is ignored and all
    elements in \VAR{ivars} are included in the test set.  The \VAR{ivars},
    \VAR{indices}, and \VAR{status} arrays must not overlap in memory.

    Implementations must ensure that \FUNC{shmem\_test\_some\_vector} does not
    return indices before the updates of the memory indicated by the
    corresponding \VAR{ivars} elements are fully complete.
}

\apireturnvalues{
    \FUNC{shmem\_test\_some\_vector} returns the number of indices returned in
    the \VAR{indices} array. If the test set is empty, this routine returns 0.
}

\end{apidefinition}
