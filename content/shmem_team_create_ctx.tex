\apisummary{
  Create a communication context from a team.
}

\begin{apidefinition}

\begin{Csynopsis}
int @\FuncDecl{shmem\_team\_create\_ctx}@(shmem_team_t team, long options, shmem_ctx_t *ctx);
\end{Csynopsis}

\begin{apiarguments}
  \apiargument{IN}{team}{A handle to the specified \ac{PE} team.}
  \apiargument{IN}{options}{
    The set of options requested for the given context.
    Multiple options may be requested by combining them with a bitwise OR
    operation; otherwise, \CONST{0} can be given if no options are requested.}
  \apiargument{OUT}{ctx}{A handle to the newly created context.}
\end{apiarguments}

\apidescription{
  The \FUNC{shmem\_team\_create\_ctx} routine creates a new communication
  context and returns its handle through the \VAR{ctx} argument.
  This context is created from the team specified by the \VAR{team} argument.

  In addition to the team, the \FUNC{shmem\_team\_create\_ctx} routine accepts
  the same arguments and provides all the same return conditions as the
  \FUNC{shmem\_ctx\_create} routine.

  The \FUNC{shmem\_team\_create\_ctx} routine may be called any number of times,
  but the total number of simultaneously existing contexts created from a team
  must be no more than were specified by the \VAR{num\_contexts} member of the
  \CTYPE{shmem\_team\_config\_t} configuration parameters that were specified
  when the team was created. Calling \FUNC{shmem\_team\_create\_ctx} on a
  team for which the maximum number of contexts currently exists results in a
  failure with nonzero return code.

  All explicitly created resources associated with a team must be destroyed
  before the \FUNC{shmem\_team\_destroy} routine is called. If a context
  returned from \FUNC{shmem\_team\_create\_ctx} is not explicitly
  destroyed before the team is destroyed, behavior is undefined.

  %% All \openshmem routines that operate on this context will do so with
  %% respect to the associated \ac{PE} team.
  %% That is, all point-to-point routines operating on this context will use
  %% team-relative \ac{PE} numbering.
}

\apireturnvalues{
  Zero on success and nonzero otherwise.
}

\apinotes{
  None.
}

\end{apidefinition}
