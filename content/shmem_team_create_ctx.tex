\apisummary{
  Create a communication context collectively.
}

\begin{apidefinition}

\begin{Csynopsis}
int @\FuncDecl{shmem\_team\_create\_ctx}@(shmem_team_t team, long options, shmem_ctx_t *ctx);
\end{Csynopsis}

\begin{apiarguments}
  \apiargument{IN}{team}{A handle to the specified \ac{PE} team.}
  \apiargument{IN}{options}{
    The set of options requested for the given context.
    Multiple options may be requested by combining them with a bitwise OR
    operation; otherwise, \CONST{0} can be given if no options are requested.}
  \apiargument{OUT}{ctx}{A handle to the newly created context.}
\end{apiarguments}

\apidescription{
  The \FUNC{shmem\_team\_create\_ctx} routine creates a new communication
  context and returns its handle through the \VAR{ctx} argument.
  This context is created collectively by all \acp{PE} in the team
  specified by the \VAR{team} argument.
  The specified team may not have the \LibConstRef{SHMEM\_TEAM\_NOCOLLECTIVE}
  option enabled; otherwise, the behavior is undefined.

  %% All \openshmem routines that operate on this context will do so with
  %% respect to the associated \ac{PE} team.
  %% That is, all point-to-point routines operating on this context will use
  %% team-relative \ac{PE} numbering.

  In addition to the team, the \FUNC{shmem\_team\_create\_ctx} routine accepts
  the same arguments and provides all the same return conditions as the
  \FUNC{shmem\_ctx\_create} routine.
  The call is either collectively successful or collectively fails across
  all \acp{PE} in the team.

  As \FUNC{shmem\_team\_create\_ctx} is collective, it includes a call to a
  procedure semantically equivalent to \FUNC{shmem\_team\_sync} on both entry
  and exit.
}

\apireturnvalues{
  Zero on success and nonzero otherwise.
}

\apinotes{
  Depending on the \openshmem implementation, system configuration, and
  application communication pattern, some applications may observe higher
  performance with collectively created contexts than with locally created
  contexts.
}

\end{apidefinition}
