\apisummary{
    Performs all operations described in the \FUNC{shmem\_sync\_all} interface
    but with respect to a subset of \acp{PE} defined by the active set.
}

\begin{apidefinition}

\begin{Csynopsis}
void @\FuncDecl{shmem\_sync}@(int PE_start, int logPE_stride, int PE_size, long *pSync);
\end{Csynopsis}

\begin{apiarguments}

\apiargument{IN}{PE\_start}{The lowest \ac{PE} number of the active set of
    \acp{PE}.  \VAR{PE\_start} must be of type integer.}
\apiargument{IN}{logPE\_stride}{The log (base 2) of the stride between
    consecutive \ac{PE} numbers in the active set.  \VAR{logPE\_stride} must be
    of type integer.}
\apiargument{IN}{PE\_size}{The number of \acp{PE} in the active set.
    \VAR{PE\_size} must be of type integer.}
\apiargument{IN}{pSync}{A symmetric work array. In \CorCpp, \VAR{pSync} must be
    of type \CTYPE{long} and size \CONST{SHMEM\_BARRIER\_SYNC\_SIZE}.  Every element of
    this array must be initialized to \CONST{SHMEM\_SYNC\_VALUE} before any of the
    \acp{PE} in the active set enter \FUNC{shmem\_sync} the first time.}

\end{apiarguments}

\apidescription{
    \FUNC{shmem\_sync} is a collective synchronization routine over an
    active set.  Control returns from \FUNC{shmem\_sync} after all \acp{PE} in
    the active set (specified by \VAR{PE\_start}, \VAR{logPE\_stride}, and
    \VAR{PE\_size}) have called \FUNC{shmem\_sync}.

    As with all \openshmem collective routines, each of these routines assumes
    that only \acp{PE} in the active set call the routine.  If a \ac{PE} not in
    the active set calls an \openshmem collective routine, the behavior is undefined.

    The values of arguments \VAR{PE\_start}, \VAR{logPE\_stride}, and
    \VAR{PE\_size} must be equal on all \acp{PE} in the active set.  The same
    work array must be passed in \VAR{pSync} to all \acp{PE} in the active set.

    In contrast with the \FUNC{shmem\_barrier} routine, \FUNC{shmem\_sync} only
    ensures completion and visibility of previously issued memory stores and does not ensure
    completion of remote memory updates issued via \openshmem routines.

    The same \VAR{pSync} array may be reused on consecutive calls to
    \FUNC{shmem\_sync} if the same active set is used.
}

\apireturnvalues{
    None.
}

\apinotes{
    If the \VAR{pSync} array is initialized at run time, another method of
    synchronization (e.g., \FUNC{shmem\_sync\_all}) must be used before
    the initial use of that \VAR{pSync} array by \FUNC{shmem\_sync}.

    If the active set does not change, \FUNC{shmem\_sync} can be called
    repeatedly with the same \VAR{pSync} array.  No additional synchronization
    beyond that implied by \FUNC{shmem\_sync} itself is necessary in this case.

    The \FUNC{shmem\_sync} routine can be used to portably ensure that
    memory access operations observe remote updates in the order enforced by the
    initiator \acp{PE}, provided that the initiator PE ensures completion of remote
    updates with a call to \FUNC{shmem\_quiet} prior to the call to the
    \FUNC{shmem\_sync} routine.
}

\begin{apiexamples}

\apicexample
    {The following \FUNC{shmem\_sync\_all} and \FUNC{shmem\_sync} example is
    for \Cstd[11] programs:}
    {./example_code/shmem_sync_example.c}
    {}

\end{apiexamples}

\end{apidefinition}
