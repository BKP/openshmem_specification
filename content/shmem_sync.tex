\apisummary{
    Performs all operations described in the \FUNC{shmem\_sync\_all} interface
    but with respect to a subset of \acp{PE} defined by the \activeset.
}

\begin{apidefinition}

\begin{Csynopsis}
void shmem_sync(int PE_start, int logPE_stride, int PE_size, long *pSync);
\end{Csynopsis}

\begin{apiarguments}

\apiargument{IN}{PE\_start}{The lowest \ac{PE} number of the \activeset of
    \acp{PE}.  \VAR{PE\_start} must be of type integer.}
\apiargument{IN}{logPE\_stride}{The log (base 2) of the stride between
    consecutive \ac{PE} numbers in the \activeset.  \VAR{logPE\_stride} must be
    of type integer.}
\apiargument{IN}{PE\_size}{The number of \acp{PE} in the \activeset.
    \VAR{PE\_size} must be of type integer.}
\apiargument{IN}{pSync}{A symmetric work array. In \CorCpp, \VAR{pSync} must be
    of type long and size \CONST{SHMEM\_BARRIER\_SYNC\_SIZE}.  Every element of
    this array must be initialized to \CONST{SHMEM\_SYNC\_VALUE} before any of the
    \acp{PE} in the \activeset enter \FUNC{shmem\_sync} the first time.}

\end{apiarguments}

\apidescription{
    \FUNC{shmem\_sync} is a collective synchronization routine over an
    \activeset.  Control returns from \FUNC{shmem\_sync} after all \acp{PE} in
    the \activeset (specified by \VAR{PE\_start}, \VAR{logPE\_stride}, and
    \VAR{PE\_size}) have called \FUNC{shmem\_sync}.

    As with all \openshmem collective routines, each of these routines assumes
    that only \acp{PE} in the \activeset call the routine.  If a \ac{PE} not in
    the \activeset calls an \openshmem collective routine, undefined behavior
    results.

    The values of arguments \VAR{PE\_start}, \VAR{logPE\_stride}, and
    \VAR{PE\_size} must be equal on all \acp{PE} in the \activeset.  The same
    work array must be passed in \VAR{pSync} to all \acp{PE} in the \activeset.

    In contrast with the \FUNC{shmem\_barrier} routine, \FUNC{shmem\_sync} only
    ensures completion of previously issued memory stores and does not ensure
    completion of remote memory updates issued via \openshmem routines.

    The same \VAR{pSync} array may be reused on consecutive calls to
    \FUNC{shmem\_barrier} if the same active \ac{PE} set is used.
}

\apireturnvalues{
    None.
}

\apinotes{
    If the \VAR{pSync} array is initialized at run time, be sure to use some type of
    synchronization, for example, a call to \FUNC{shmem\_sync\_all}, before
    calling \FUNC{shmem\_sync} for the first time.

    If the \activeset does not change, \FUNC{shmem\_sync} can be called
    repeatedly with the same \VAR{pSync} array.  No additional synchronization
    beyond that implied by \FUNC{shmem\_sync} itself is necessary in this case.

    The \FUNC{shmem\_sync\_all} routine can be used to portably ensure that
    memory access operations observe remote updates in the order enforced by
    initiator PEs, provided that the initiator PE ensures completion of remote
    updates with a call to \FUNC{shmem\_quiet} prior to the call to
    \FUNC{shmem\_sync\_all} routine.
}

%\begin{apiexamples}
%
%\apicexample
%	{The following barrier example is for \CorCpp programs:}
%	{./example_code/shmem_barrier_example.c}
%	{}
%
%\end{apiexamples}

\end{apidefinition}
