\apisummary{
    Concatenates blocks of data from multiple \acp{PE} to an array in every
    \ac{PE}.
}

\begin{apidefinition}

\begin{Csynopsis}
void shmem_collect32(void *dest, const void *source, size_t nelems, int PE_start, int logPE_stride, int PE_size, long *pSync);
void shmem_collect64(void *dest, const void *source, size_t nelems, int PE_start, int logPE_stride, int PE_size, long *pSync);
void shmem_fcollect32(void *dest, const void *source, size_t nelems, int PE_start, int logPE_stride, int PE_size, long *pSync);
void shmem_fcollect64(void *dest, const void *source, size_t nelems, int PE_start, int logPE_stride, int PE_size, long *pSync);
\end{Csynopsis}

\begin{Fsynopsis}
INTEGER nelems
INTEGER PE_start, logPE_stride, PE_size
INTEGER pSync(SHMEM_COLLECT_SYNC_SIZE)
CALL SHMEM_COLLECT4(dest, source, nelems, PE_start, logPE_stride, PE_size, pSync)
CALL SHMEM_COLLECT8(dest, source, nelems, PE_start, logPE_stride, PE_size, pSync)
CALL SHMEM_COLLECT32(dest, source, nelems, PE_start, logPE_stride, PE_size, pSync)
CALL SHMEM_COLLECT64(dest, source, nelems, PE_start, logPE_stride, PE_size, pSync)
CALL SHMEM_FCOLLECT4(dest, source, nelems, PE_start, logPE_stride, PE_size, pSync)
CALL SHMEM_FCOLLECT8(dest, source, nelems, PE_start, logPE_stride, PE_size, pSync)
CALL SHMEM_FCOLLECT32(dest, source, nelems, PE_start, logPE_stride, PE_size, pSync)
CALL SHMEM_FCOLLECT64(dest, source, nelems, PE_start, logPE_stride, PE_size, pSync)
\end{Fsynopsis}

\begin{apiarguments}

\apiargument{OUT}{dest}{A symmetric array. The \dest{} argument must be large enough
    to accept the concatenation of the \source{} arrays on all \acp{PE}.  The data
    types are as follows: For \FUNC{shmem\_collect8}, \FUNC{shmem\_collect64},
    \FUNC{shmem\_fcollect8}, and \FUNC{shmem\_fcollect64}, any data type with an
    element size of 64 bits.  \Fortran derived types, \Fortran character type,
    and \CorCpp{}  structures  are not permitted.  For \FUNC{shmem\_collect4},
    \FUNC{shmem\_collect32}, \FUNC{shmem\_fcollect4}, and \FUNC{shmem\_fcollect32},
    any data type with an element size of \CONST{32} bits.  \Fortran derived
    types, \Fortran character type, and \CorCpp{} structures are not permitted.}
\apiargument{IN}{source}{A symmetric data object that can be of any type permissible
    for the \dest{} argument.}
\apiargument{IN}{nelems}{The number of elements in the \source{} array. \VAR{nelems}
    must be of type \VAR{size\_t} for \Cstd. If you are using \Fortran, it must be
    a default integer value.}
\apiargument{IN}{PE\_start}{The lowest \ac{PE} number of the \activeset{} of
    \acp{PE}.  \VAR{PE\_start} must be of type integer.  If you are using \Fortran,
    it must be a default integer value.}
\apiargument{IN}{logPE\_stride}{The log (base \CONST{2}) of the stride between
    consecutive \ac{PE} numbers in the \activeset. \VAR{logPE\_stride} must be of
    type integer.  If you are using \Fortran, it must be a default integer value.}
\apiargument{IN}{PE\_size}{The number of \acp{PE} in the \activeset. \VAR{PE\_size}
    must be of type integer.  If you are using  \Fortran, it must be a default
    integer value.}
\apiargument{IN}{pSync}{A symmetric  work array.  In \CorCpp, \VAR{pSync} must be of
    type long and size \CONST{SHMEM\_COLLECT\_SYNC\_SIZE}.  In \Fortran,
    \VAR{pSync} must be of type integer and size \CONST{SHMEM\_COLLECT\_SYNC\_SIZE}.
    If you are using \Fortran, it must be a default integer value.  Every element of
    this array must be initialized with the value \CONST{SHMEM\_SYNC\_VALUE} in
    \CorCpp{} or \CONST{SHMEM\_SYNC\_VALUE} in \Fortran before any of the \acp{PE}
    in the \activeset{} enter \FUNC{shmem\_collect} or \FUNC{shmem\_fcollect}.}

\end{apiarguments}

\apidescription{   
    \openshmem{} \FUNC{collect} and \FUNC{fcollect} routines concatenate \VAR{nelems}
    \CONST{64}-bit or \CONST{32}-bit data items from the \source{} array into the
    \dest{} array, over the set of \acp{PE} defined by \VAR{PE\_start},
    \VAR{log2PE\_stride}, and \VAR{PE\_size}, in processor number order. The
    resultant \dest{} array contains the contribution from \ac{PE} \VAR{PE\_start}
    first, then the contribution from \ac{PE} \VAR{PE\_start} + \VAR{PE\_stride}
    second, and so on. The collected result is written to the \dest{} array for all
    \acp{PE} in the \activeset.
    
    The \FUNC{fcollect} routines require that \VAR{nelems} be the same value in all
    participating \acp{PE}, while the \FUNC{collect} routines allow \VAR{nelems} to
    vary from \ac{PE} to \ac{PE}.
    
    As with all \openshmem collective routines, each of these routines assumes that
    only \acp{PE} in the \activeset{} call the routine. If a \ac{PE} not in the
    \activeset{} and calls this collective routine, the behavior is undefined.
    
    The values of arguments \VAR{PE\_start}, \VAR{logPE\_stride}, and \VAR{PE\_size}
    must be equal on all \acp{PE} in the \activeset. The same \dest{} and \source{}
    arrays and the same \VAR{pSync} work array must be passed to all \acp{PE} in the
    \activeset.
    
    Upon return from a collective routine, the following are true for the local
    \ac{PE}: The \dest{} array is updated and the \source{} array may be safely reused. 
    The values in the \VAR{pSync} array are
    restored to the original values.
}

\apireturnvalues{
    None.
}

\apinotes{
    All \openshmem collective routines reset the values in \VAR{pSync} before they
    return, so a particular \VAR{pSync} buffer need only be initialized the first
    time it is used.
    
    You  must ensure that the \VAR{pSync} array is not being updated on any \ac{PE}
    in the \activeset{} while any of the \acp{PE} participate in processing of an
    \openshmem collective routine.  Be careful to avoid these situations: If the
    \VAR{pSync} array is initialized at run time, some type of synchronization is
    needed to ensure that all \acp{PE} in the working set have initialized
    \VAR{pSync} before any of them  enter an \openshmem routine called with the
    \VAR{pSync} synchronization array.  A \VAR{pSync} array can be reused on a
    subsequent \openshmem collective routine only if none of the \acp{PE} in the
    \activeset{}  are still processing a  prior \openshmem collective routine call
    that used the same \VAR{pSync} array.  In general, this may be ensured only by
    doing some type of synchronization.  
    
    The collective routines operate on active \ac{PE} sets that have a
    non-power-of-two \VAR{PE\_size} with some performance degradation.  They operate
    with no performance degradation when \VAR{nelems} is a non-power-of-two value.
}

\begin{apiexamples}

\apicexample
    {The following \FUNC{shmem\_collec}t example is for \CorCpp{} programs:}
    {./example_code/shmem_collect_example.c}
    {}

\apifexample
    {The following \FUNC{SHMEM\_COLLECT} example is for \Fortran programs:}
    {./example_code/shmem_collect_example.f90}
    {}

\end{apiexamples}

\end{apidefinition}
