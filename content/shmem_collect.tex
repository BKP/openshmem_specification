\apisummary{
    Concatenates blocks of data from multiple \acp{PE} to an array in every
    \ac{PE} participating in the collective routine.
}

\begin{apidefinition}

%% C11
\begin{C11synopsis}
int @\FuncDecl{shmem\_collect}@(shmem_team_t team, TYPE *dest, const TYPE *source, size_t nelems);
int @\FuncDecl{shmem\_fcollect}@(shmem_team_t team, TYPE *dest, const TYPE *source, size_t nelems);
\end{C11synopsis}
where \TYPE{} is one of the standard \ac{RMA} types specified by Table \ref{stdrmatypes}.

\begin{Csynopsis}
\end{Csynopsis}
\begin{CsynopsisCol}
int @\FuncDecl{shmem\_\FuncParam{TYPENAME}\_collect}@(shmem_team_t team, TYPE *dest, const TYPE *source, size_t nelems);
int @\FuncDecl{shmem\_\FuncParam{TYPENAME}\_fcollect}@(shmem_team_t team, TYPE *dest, const TYPE *source, size_t nelems);
\end{CsynopsisCol}
where \TYPE{} is one of the standard \ac{RMA} types and has a corresponding \TYPENAME{} specified by Table \ref{stdrmatypes}.

\begin{CsynopsisCol}
int @\FuncDecl{shmem\_collectmem}@(shmem_team_t team, void *dest, const void *source, size_t nelems);
int @\FuncDecl{shmem\_fcollectmem}@(shmem_team_t team, void *dest, const void *source, size_t nelems);
\end{CsynopsisCol}

\begin{DeprecateBlock}
\begin{CsynopsisCol}
void @\FuncDecl{shmem\_collect32}@(void *dest, const void *source, size_t nelems, int PE_start, int logPE_stride, int PE_size, long *pSync);
void @\FuncDecl{shmem\_collect64}@(void *dest, const void *source, size_t nelems, int PE_start, int logPE_stride, int PE_size, long *pSync);
void @\FuncDecl{shmem\_fcollect32}@(void *dest, const void *source, size_t nelems, int PE_start, int logPE_stride, int PE_size, long *pSync);
void @\FuncDecl{shmem\_fcollect64}@(void *dest, const void *source, size_t nelems, int PE_start, int logPE_stride, int PE_size, long *pSync);
\end{CsynopsisCol}
\end{DeprecateBlock}

\begin{apiarguments}

\apiargument{IN}{team}{A valid \openshmem team handle.}%

\apiargument{OUT}{dest}{Symmetric address of an array large enough
    to accept the concatenation of the \source{} arrays on all participating \acp{PE}.
    The type of \dest{} should match that implied in the SYNOPSIS section.}
\apiargument{IN}{source}{Symmetric address of the source data object.
    The type of \source{} should match that implied in the SYNOPSIS section.}
\apiargument{IN}{nelems}{The number of elements in the \source{} array.}

\begin{DeprecateBlock}
\apiargument{IN}{PE\_start}{The lowest \ac{PE} number of the active set of
    \acp{PE}.}
\apiargument{IN}{logPE\_stride}{The log (base \CONST{2}) of the stride between
    consecutive \ac{PE} numbers in the active set.}
    \apiargument{IN}{PE\_size}{The number of \acp{PE} in the active set.}
\apiargument{IN}{pSync}{
    Symmetric address of a work array of size at least \CONST{SHMEM\_COLLECT\_SYNC\_SIZE}.}
\end{DeprecateBlock}

\end{apiarguments}

\apidescription{
    \openshmem \FUNC{collect} and \FUNC{fcollect} routines perform a collective
    operation to concatenate \VAR{nelems}
    data items from the \source{} array into the
    \dest{} array, over an \openshmem team or active set
    in processor number order. The resultant \dest{} array contains the contribution from
    \acp{PE} as follows:
    
    \begin{itemize}
    \item For an active set, the data from \ac{PE} \VAR{PE\_start} is first, then the
    contribution from \ac{PE} \VAR{PE\_start} + \VAR{PE\_stride} second, and so on.
    \item For a team, the data from \ac{PE} number \CONST{0} in the team is first, then the
    contribution from \ac{PE} \CONST{1} in the team, and so on.
    \end{itemize}
    
    The collected result is written to the \dest{} array for all \acp{PE}
    that participate in the operation. The same \dest{} and \source{}
    arrays must be passed by all \acp{PE} that participate in the operation.
    
    The \FUNC{fcollect} routines require that \VAR{nelems} be the same value in all
    participating \acp{PE}, while the \FUNC{collect} routines allow \VAR{nelems} to
    vary from \ac{PE} to \ac{PE}.

    Team-based collect routines operate over all \acp{PE} in the provided team argument. All
    \acp{PE} in the provided team must participate in the operation.

    Active-set-based collective routines operate over all \acp{PE} in the active set
    defined by the \VAR{PE\_start}, \VAR{logPE\_stride}, \VAR{PE\_size} triplet.
    As with all active-set-based collective routines,
    each of these routines assumes that
    only \acp{PE} in the active set call the routine. If a \ac{PE} not in the
    active set and calls this collective routine, the behavior is undefined.
    
    The values of arguments \VAR{PE\_start}, \VAR{logPE\_stride}, and \VAR{PE\_size}
    must be the same value on all \acp{PE} in the active set. The same
    \VAR{pSync} work array must be passed by all \acp{PE} in the active set.
    
    Upon return from a collective routine, the following are true for the local
    \ac{PE}:
    \begin{itemize}
    \item The \dest{} array is updated and the \source{} array may be safely reused. 
    \item For active-set-based collective routines, the values in the \VAR{pSync} array are
    restored to the original values.
    \end{itemize}
}

\apireturnvalues{
    Zero on successful local completion. Nonzero otherwise.
}

\apinotes{
    The collective routines operate on active \ac{PE} sets that have a
    non-power-of-two \VAR{PE\_size} with some performance degradation.  They operate
    with no performance degradation when \VAR{nelems} is a non-power-of-two value.
}

\begin{apiexamples}

\apicexample
    {The following \FUNC{shmem\_collect} example is for \CorCpp{} programs:}
    {./example_code/shmem_collect_example.c}
    {}

\end{apiexamples}

\end{apidefinition}
