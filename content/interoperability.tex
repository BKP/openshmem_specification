\chapter{Interoperability with Other Programming Models}\label{sec:interoperability}

OpenSHMEM routines may be used in conjunction with the routines of other
communication libraries or parallel languages in the same program. This section
describes the interoperability with other programming models, including
clarification of undefined behaviors caused by mixed use of different models,
advice to \openshmem library users and developers that may improve the portability
and performance of hybrid programs, and definition of an OpenSHMEM
API that queries the interoperability features provided by an \openshmem library.


\section{MPI Interoperability}

\openshmem and MPI are two commonly used parallel programming models for
distributed-memory systems. The user can choose to utilize both models in the same program
to efficiently and easily support various communication patterns.

A vendor may implement the \openshmem and MPI libraries in different ways. For
instance, one may implement both \openshmem and MPI as standalone libraries,
each of which allocates and initializes fully isolated communication
resources.
Another common approach
is to implement both \openshmem and MPI interfaces within the
same software system in order to share a communication resource when possible.

To improve interoperability and portability in \openshmem + MPI hybrid
programming, we clarify the relevant semantics in the following subsections.


\subsection{Initialization}
In order to ensure that a hybrid program can be portably performed with different vendor
implementations, the \openshmem environment of the program must be initialized by
a call to \FUNC{shmem\_init} or \FUNC{shmem\_init\_thread} and be finalized by
a call to \FUNC{shmem\_finalize}; the MPI environment of the program must be initialized
by a call to \FUNC{MPI\_Init} or \FUNC{MPI\_Init\_thread} and be finalized by a
call to \FUNC{MPI\_Finalize}.

\apiimpnotes{
Portable implementations of OpenSHMEM and MPI must ensure that the initialization
calls can be made in an arbitrary order within a program; the same rule also
applies to the finalization calls. A software runtime that utilizes a shared
communication resource for \openshmem and MPI communication may maintain an
internal reference counter in order to ensure that the shared resource is
initialized only once and thus no shared resource is released until the last
finalization call is made.
}


\subsection{Dynamic Process Creation}
\label{subsec:interoperability:mpmd}

MPI defines a dynamic process model that allows creation of processes after
an MPI application has started (e.g., by calling \FUNC{MPI\_Comm\_spawn}) and
connection to independent processes (e.g., through \FUNC{MPI\_Comm\_accept}
and \FUNC{MPI\_Comm\_connect})
and provides a mechanism to establish communication
between the newly created processes and the existing MPI application (see
MPI standard version 3.1, Chapter 10).
Unlike MPI, \openshmem starts all processes at once and requires all PEs to
collectively allocate and initialize resources (e.g., symmetric heap) used by
the \openshmem library before any other \openshmem routine may
be called. \openshmem does not support communication with dynamically created
or connected processes. In such a scenario, MPI must be used to communicate
with these processes.


\subsection{Thread Safety}
\label{subsec:interoperability:thread}
Both \openshmem and MPI define the interaction with user threads in a program
with routines that can be used for initializing and querying the thread
environment. In a hybrid program, the user may request different thread levels
at the initialization calls of \openshmem and MPI environments; however, the
returned support level provided by the \openshmem or MPI library might be different
from that returned in an non-hybrid program. For instance, the former
initialization call in a hybrid program may initialize a resource with the
user-requested thread level, but the supported level cannot be updated by a subsequent
initialization call if the underlying software runtime of \openshmem and MPI
share the same internal communication resource.
The program should always check the \VAR{provided} thread level returned
at the corresponding initialization call or query the level of thread support
after initialization to portably ensure thread support in each communication
environment.

Both \openshmem and MPI define similar thread levels, namely, \VAR{THREAD\_SINGLE},
\VAR{THREAD\_FUNNELED}, \VAR{THREAD\_SERIALIZED}, and \VAR{THREAD\_MULTIPLE}.
When requesting threading support in a hybrid program, however,
the following additional rules are applied if the implementations of \openshmem
and MPI share the same internal communication resource.
Users are strongly advised to always follow these rules to ensure program
portability.

\begin{itemize}
    \item The \VAR{THREAD\_SINGLE} thread level requires a single-threaded program.
    Hence, users should not request \VAR{THREAD\_SINGLE} at the initialization
    call of either \openshmem or MPI but request a different thread level at the
    initialization call of the other model in the same program.

    \item The \VAR{THREAD\_FUNNELED} thread level allows only the main thread to
    make communication calls. A hybrid program using the \VAR{THREAD\_FUNNELED}
    thread level in both \openshmem and MPI should ensure that the same main thread
    is used in both communication environments.

    \item The \VAR{THREAD\_SERIALIZED} thread level requires the program to ensure
    that communication calls are not made concurrently by multiple threads. If a
    hybrid program uses \VAR{THREAD\_SERIALIZED} in one communication environment
    and \VAR{THREAD\_SERIALIZED} or \VAR{THREAD\_FUNNELED} in the other one, it
    should also guarantee that the \openshmem and MPI calls are not made concurrently
    from two distinct threads.
\end{itemize}

\subsection{Mapping Process Identification Numbers}
\label{subsec:interoperability:id}

Similar to the PE number in \openshmem, MPI defines rank as the
identification number of a process in a communicator. Both the \openshmem PE
and the MPI rank are unique integers assigned from zero to one less than the total
number of processes. In a hybrid program, the \openshmem
PE number in \LibHandleRef{SHMEM\_TEAM\_WORLD}
and the MPI rank in \VAR{MPI\_COMM\_WORLD} of a process can be equal.
This feature, however, may be provided by only some of the \openshmem and MPI
implementations (e.g., if both environments share the same underlying process
manager) and is not portably guaranteed. A portable program should always
use the standard functions in each model, namely, \FUNC{shmem\_team\_my\_pe} in \openshmem
and \FUNC{MPI\_Comm\_rank} in MPI, to query the process identification numbers
in each communication environment and manage the mapping of identifiers in the
program when necessary.

\subsubsection*{Example}
\label{subsubsec:interoperability:id:example}
The following example demonstrates how to manage the mapping between \openshmem
PE numbers and MPI ranks in \VAR{MPI\_COMM\_WORLD} in a hybrid \openshmem
and MPI program.

\lstinputlisting[language={C}, tabsize=2,
      basicstyle=\ttfamily\footnotesize]
      {example_code/hybrid_mpi_mapping_id.c}

\subsection{RMA Programming Models}
\label{subsec:interoperability:rma}

\openshmem and MPI each define similar one-sided communication models;
however, a portable program should not assume interoperability between these
models.
For instance, \openshmem guarantees the atomicity only of concurrent \openshmem AMO operations
that operate on symmetric data with the same datatype. Access to the same symmetric
object with MPI atomic operations, such as an \FUNC{MPI\_Fetch\_and\_op}, may
result in an undefined result. Users should avoid situations where MPI and
\openshmem operations perform concurrent accesses to the same memory location.

\subsection{Communication Progress}
\label{subsec:interoperability:progress}

\openshmem promises the progression of communication both with and without
\openshmem calls and requires the software progress mechanism in the implementation
(e.g., a progress thread) when the hardware does not provide asynchronous communication
capabilities. In MPI, however, a weak progress semantics is applied. That is,
an MPI communication call is guaranteed only to complete in finite time. For
instance, an \FUNC{MPI\_Put} may be completed only when the remote process makes an MPI
call that internally triggers the progress of MPI, if the underlying hardware
does not support asynchronous communication. A hybrid program
should not assume that the \openshmem library also makes progress for MPI.
A call to \FUNC{shmem\_query\_interoperability} with the \VAR{SHMEM\_PROGRESS\_MPI}
property (see definition in \ref{subsec:interoperability:query})
can be used to portably check whether the implementation provides asynchronous
progression also for MPI. If it is not provided, the user program may have to
explicitly manage the asynchronous communication in MPI in
order to prevent any deadlock or performance degradation.

\apiimpnotes{
Implementations that provide both \openshmem and MPI interfaces should try
to ensure progress for both models, when necessary and possible, for performance
reasons. For instance, an implementation
may utilize a software progress thread to process any software-handled
communication requests, after the user program has called
\FUNC{shmem\_init} and \FUNC{MPI\_Init} provided by the same system.
}


\section{Query Interoperability}

A hybrid user program can query the interoperability feature of an \openshmem
implementation in order to avoid unnecessary overhead and programming complexity.
For instance, the user program can eliminate manual progress polling for MPI
communication if the \openshmem implementation guarantees asynchronous
communication also for MPI.

\subsection{\textbf{SHMEM\_QUERY\_INTEROPERABILITY}}
\label{subsec:interoperability:query}
\apisummary{
    Determines whether an interoperability feature is supported by the \openshmem
    library implementation.
}
\begin{apidefinition}

\begin{Csynopsis}
int @\FuncDecl{shmem\_query\_interoperability}@(int property);
\end{Csynopsis}

\begin{apiarguments}
    \apiargument{IN}{property}{The interoperability property queried by the user.}
\end{apiarguments}

\apidescription{
    \FUNC{shmem\_query\_interoperability} queries whether an interoperability property
    is supported by the \openshmem library. One of the following properties can be 
    queried in an \openshmem program after finishing the
    initialization call to \openshmem and that of the relevant programming models
    being used in the program. An \openshmem library implementation may extend the
    available properties.

    \begin{itemize}
    \item \VAR{SHMEM\_PROGRESS\_MPI} Query whether the \openshmem
    implementation makes progress for the MPI communication used in the user program.
    \end{itemize}
}

\apireturnvalues{
    The return value is \CONST{1} if \VAR{property} is supported by the \openshmem library;
    otherwise, it is \CONST{0}.
}

\apinotes{
  None.
}

\begin{apiexamples}

\apicexample
    {The following example queries whether the \openshmem library supports asynchronous
progress for MPI. If it returns 1, the library guarantees the MPI nonblocking send
is processed while PE 0 is in the busy wait loop with repeated calls to
\FUNC{shmem\_int\_atomic\_fetch} so that deadlock will not occur.}
    {./example_code/shmem_query_mpi_progress.c}
    {}

\end{apiexamples}

\end{apidefinition}
