The \acp{PE} in an \openshmem program can communicate either using
point-to-point routines that specify the \ac{PE} number of the target
\ac{PE} or using collective routines which operate over some predefined
set of \acp{PE}. Teams in \openshmem allow programs to group subsets
of \acp{PE} for collective communications and provide a contiguous reindexing
of the \acp{PE} within that subset that can be used in point-to-point communication.

An \openshmem team is a set of \acp{PE} defined by calling a specific team
split routine with a parent team argument and other arguments to further
specify how the parent team is to be split into one or more new teams.
A team created by a \FUNC{shmem\_team\_split\_*} routine can be used as the parent team
for a subsequent call to a team split routine.  A team persists and can
be used for multiple collective routine calls until it is destroyed by
\FUNC{shmem\_team\_destroy}.

Every team must have a least one member. Any attempt to create a team over an
empty set of \acp{PE} will result in no new team being created.

A ``team handle'' is an opaque object with type \CTYPE{shmem\_team\_t} that is used
to reference a defined team.  Team handles are created by one of the team split
routines and destroyed by the team destroy routine. Team handles have local
semantics only. That is, team handles should not be stored in shared variables
and used across other \acp{PE}. Doing so will result in undefined behavior.

By default, \openshmem creates predefined teams that will be available
for use once the routine \FUNC{shmem\_init} has been called. See
Section~\ref{subsec:library_handles} for a description of all predefined team handles
provided by \openshmem. Predefined \CTYPE{shmem\_team\_t} handles can be used as
the parent team when creating new \openshmem teams.

Every \ac{PE} is a member of the default team, which may be referenced
through the team handle \LibHandleRef{SHMEM\_TEAM\_WORLD}.
The \ac{PE} number in the default team is equal to the
value of its \ac{PE} number as returned by \FUNC{shmem\_my\_pe}.

A special team handle value, \LibConstRef{SHMEM\_TEAM\_NULL}, may be used to
indicate that a returned team handle is not valid. This value can be tested
against to check for successful split operations and can be assigned to user
declared team handles as a sentinel value.

Teams that are created by a \FUNC{shmem\_team\_split\_*} routine may be
provided a configuration argument that specifies team creation options.
This configuration argument is of type \CTYPE{shmem\_team\_config\_t}, which
is detailed further in Section~\ref{subsec:shmem_team_config_t}.

Team creation is a collective operation. As such, team creation in a
multithreaded environment follows the same semantics as discussed in section
\ref{subsec:coll}. That is, while \openshmem routines are thread-safe as
per threading model (see section \ref{subsec:thread_support}),\openshmem
teams objects are not themselves thread-safe. For team creation, this means
that the program must ensure that there are no simultaneous split operations
occuring on the same parent team on a given \ac{PE}.

Like other collectives, team creation is matched across PEs based
on ordering. So, team creation events must occur in the same order on all \acp{PE}
in the resulting child teams. Additionally, there must not be team creation
operations from the same parent team simultaneously occuring that involve
the same \acp{PE} in any resulting child teams. In practice, this means that when a parent team
is split multiple times, and the resulting child teams have overlapping membership,
the program must call the \FUNC{shmem\_team\_sync} routine on the parent team
between subsequent calls to split routines.

Upon completion of a team creation operation, the resulting child teams will be
immediately usable for any team-based operations, including creating new child teams,
without any intervening synchronization.
