\apisummary{
    Performs an atomic swap to a remote data object.
}

\begin{apidefinition}

\begin{C11synopsis}
TYPE @\FuncDecl{shmem\_atomic\_swap}@(TYPE *dest, TYPE value, int pe);
TYPE @\FuncDecl{shmem\_atomic\_swap}@(shmem_ctx_t ctx, TYPE *dest, TYPE value, int pe);
\end{C11synopsis}
where \TYPE{} is one of the extended \ac{AMO} types specified by Table \ref{extamotypes}.

\begin{Csynopsis}
TYPE @\FuncDecl{shmem\_\FuncParam{TYPENAME}\_atomic\_swap}@(TYPE *dest, TYPE value, int pe);
TYPE @\FuncDecl{shmem\_ctx\_\FuncParam{TYPENAME}\_atomic\_swap}@(shmem_ctx_t ctx, TYPE *dest, TYPE value, int pe);
\end{Csynopsis}
where \TYPE{} is one of the extended \ac{AMO} types and has a corresponding \TYPENAME{} specified by Table \ref{extamotypes}.

\begin{DeprecateBlock}
\begin{C11synopsis}
TYPE @\FuncDecl{shmem\_swap}@(TYPE *dest, TYPE value, int pe);
\end{C11synopsis}
where \TYPE{} is one of \{\CTYPE{float}, \CTYPE{double}, \CTYPE{int},
\CTYPE{long}, \CTYPE{long long}\}.

\begin{Csynopsis}
TYPE @\FuncDecl{shmem\_\FuncParam{TYPENAME}\_swap}@(TYPE *dest, TYPE value, int pe);
\end{Csynopsis}
where \TYPE{} is one of \{\CTYPE{float}, \CTYPE{double}, \CTYPE{int},
\CTYPE{long}, \CTYPE{long long}\} and has a corresponding
\TYPENAME{} specified by Table~\ref{extamotypes}.
\end{DeprecateBlock}

\begin{Fsynopsis}
INTEGER SHMEM_SWAP, value, pe
ires = @\FuncDecl{SHMEM\_SWAP}@(dest, value, pe)
INTEGER*4 SHMEM_INT4_SWAP, value_i4, ires_i4
ires\_i4 = @\FuncDecl{SHMEM\_INT4\_SWAP}@(dest, value_i4, pe)
INTEGER*8 SHMEM_INT8_SWAP, value_i8, ires_i8
ires\_i8 = @\FuncDecl{SHMEM\_INT8\_SWAP}@(dest, value_i8, pe)
REAL*4 SHMEM_REAL4_SWAP, value_r4, res_r4
res\_r4 = @\FuncDecl{SHMEM\_REAL4\_SWAP}@(dest, value_r4, pe)
REAL*8 SHMEM_REAL8_SWAP, value_r8, res_r8
res\_r8 = @\FuncDecl{SHMEM\_REAL8\_SWAP}@(dest, value_r8, pe)
\end{Fsynopsis}

\begin{apiarguments}
  \apiargument{IN}{ctx}{\oldtext{The context on which to perform the operation.} \newtext{A context handle specifying the context on which to perform the operation.}
    When this argument is not provided, the operation is performed on
    \oldtext{\CONST{SHMEM\_CTX\_DEFAULT}} \newtext{the default context}.}
  \apiargument{OUT}{dest}{The  remotely accessible integer data object to be
    updated on the remote \ac{PE}.	 When using \CorCpp, the type of
    \dest{} should match that  implied in the SYNOPSIS section.}
  \apiargument{IN}{value}{The value to be atomically written to the remote
    \ac{PE}. \VAR{value}  is the same type as \dest.}
  \apiargument{IN}{pe}{ An integer that indicates the \ac{PE} number on which
    \dest{} is to be updated. When using \Fortran, it must be a default
    integer value.}
\end{apiarguments}

\apidescription{
    \FUNC{shmem\_atomic\_swap} performs an atomic swap operation.
    It writes \VAR{value} into \dest{} on \ac{PE} and returns the previous
    contents of \dest{} as an atomic operation.
    \newtext{
    If the context handle \VAR{ctx} does not correspond to a valid context,
    the behavior is undefined.
    }
}

\apidesctable{
  When using \Fortran, \VAR{dest} and \VAR{value} must be of the following type:
}{Routine}{Data type of \VAR{dest} and \VAR{value}}

\apitablerow{SHMEM\_SWAP}{Integer of default kind}
\apitablerow{SHMEM\_INT4\_SWAP}{\CONST{4}-byte integer}
\apitablerow{SHMEM\_INT8\_SWAP}{\CONST{8}-byte integer}
\apitablerow{SHMEM\_REAL4\_SWAP}{\CONST{4}-byte real}
\apitablerow{SHMEM\_REAL8\_SWAP}{\CONST{8}-byte real}

\apireturnvalues{
       The content that had been at the \dest{} address on the remote \ac{PE}
       prior to the swap is returned.
}

\apinotes{
    None.
}

\begin{apiexamples}

\apicexample
    {The example below swaps values between odd numbered \acp{PE} and
    their right (modulo) neighbor and outputs the result of swap.}
    {./example_code/shmem_atomic_swap_example.c}
    {}

\end{apiexamples}

\end{apidefinition}
