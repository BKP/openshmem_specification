All \openshmem \ac{RMA}, \ac{AMO}, and memory ordering routines must be
performed on a valid communication context.  The communication context defines an
independent ordering and completion environment, allowing users to manage the
overlap of communication with computation and also to manage communication
operations performed by separate threads within a multithreaded \ac{PE}.  For
example, in single-threaded environments, contexts may be used to pipeline
communication and computation.  In multithreaded environments, contexts may
additionally provide thread isolation, eliminating overheads resulting from
thread interference.

A specific communication context is referenced through a context handle, which is
passed as an argument in the \CTYPE{shmem\_ctx\_*} and type-generic \ac{API}
routines.  \ac{API} routines that do not accept a context handle argument operate on the
default context.  The default context can be used explicitly through the
\LibHandleRef{SHMEM\_CTX\_DEFAULT} handle.
Context handles are of type \CTYPE{shmem\_ctx\_t} and may be used for
language-level assignment and equality comparison.

The default context is valid for the duration of the \openshmem portion of
an application.
Contexts created by a successful call to \FUNC{shmem\_ctx\_create} remain
valid until they are destroyed.
A handle value that does not correspond to a valid context is considered
to be invalid, and its use in \ac{RMA} and \ac{AMO} routines results in
undefined behavior.
A context handle may be initialized to or assigned the value
\CONST{SHMEM\_CTX\_INVALID} to indicate that handle does not reference a
valid communication context.
When managed in this way, applications can use an equality comparison
to test whether a given context handle references a valid context.

Every communication context is associated with a team.
This association is established at context creation.
Communication contexts created by \FUNC{shmem\_ctx\_create} are
associated with the world team, while contexts created by
\FUNC{shmem\_team\_create\_ctx} are associated with and created from a team
specified at context creation.
The default context is associated with the world team.
A context's associated team specifies the set of \acp{PE} over which
\ac{PE}-specific routines that operate on a communication context,
explicitly or implicitly, are performed.
All point-to-point routines that operate on this context will do so with
respect to the team-relative \ac{PE} numbering of the associated team.
If the PE number passed to such a routine is invalid, being negative or greater
than or equal to the size of the \openshmem team, then the behavior is undefined.

By default, contexts are {\em shareable} and, when it is allowed by the
threading model provided by the \openshmem library, they can be used concurrently by
multiple threads within the PE where they were created.
%
The following options can be supplied during context creation to restrict
this usage model and enable performance optimizations.  When using a given
context, the application must comply with the requirements of all options
set on that context; otherwise, the behavior is undefined.
No options are enabled on the default context.

\apitablerow{\LibConstRef{SHMEM\_CTX\_SERIALIZED}}{
    The given context is shareable; however, it will not be used by multiple threads
    concurrently.  When the \CONST{SHMEM\_CTX\_SERIALIZED} option is
    set, the user must ensure that operations involving the given
    context are serialized by the application.}

\apitablerow{\LibConstRef{SHMEM\_CTX\_PRIVATE}}{
    The given context will be used only by the thread that created it.}

\apitablerow{\LibConstRef{SHMEM\_CTX\_NOSTORE}}{
    Quiet and fence operations performed on the given context are not
    required to enforce completion and ordering of memory store
    operations performed by the application.
    When ordering of store operations is needed, the application must
    perform a synchronization operation on a context without the
    \CONST{SHMEM\_CTX\_NOSTORE} option enabled.}
