\apisummary{
    A collective operation that allocates and initializes the resources used by
    the \openshmem library.
}

\begin{apidefinition}

\begin{Csynopsis}
void shmem_init(void);
\end{Csynopsis}

\begin{Fsynopsis}
CALL SHMEM_INIT()
\end{Fsynopsis}


\begin{apiarguments}
    \apiargument{None.}{}{}
\end{apiarguments}

\apidescription{
    \FUNC{shmem\_init} allocates and initializes resources used by the \openshmem
    library. It is a collective operation that all \acp{PE} must call before any
    other \openshmem routine may be called. At the end of the \openshmem program
    which it initialized, the call to \FUNC{shmem\_init} must be matched with a
    call to \FUNC{shmem\_finalize}. After the first call to \FUNC{shmem\_init}, a
    subsequent call to \FUNC{shmem\_init} in the same program results in undefined
    behavior.
}

\apireturnvalues{
    None.
}      

\apinotes{
    As of \openshmem Specification 1.2 the use of \FUNC{start\_pes} has been
    deprecated and is replaced with \FUNC{shmem\_init}. While support for
    \FUNC{start\_pes} is still required in \openshmem libraries, users are
    encouraged to use \FUNC{shmem\_init}. An important difference between
    \FUNC{shmem\_init} and \FUNC{start\_pes} is that multiple calls to
    \FUNC{shmem\_init} within a program results in undefined behavior, while in the
    case of \FUNC{start\_pes}, any subsequent calls to \FUNC{start\_pes} after the
    first one resulted in a no-op.
}

\begin{apiexamples}

\apifexample
    { This is a simple program that calls \FUNC{shmem\_init}: } 
    { example_code/shmem_init_example.f90 }
    {}

\end{apiexamples}

\end{apidefinition}
