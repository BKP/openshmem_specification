\apisummary{
    A collective operation that allocates and initializes the resources used by
    the \openshmem library.
}

\begin{apidefinition}

\begin{Csynopsis}
void @\FuncDecl{shmem\_init}@(void);
\end{Csynopsis}

\begin{apiarguments}
    \apiargument{None.}{}{}
\end{apiarguments}

\apidescription{
    \FUNC{shmem\_init} allocates and initializes resources used by the \openshmem
    library. It is a collective operation that all \acp{PE} must call before any
    other \openshmem routine may be called. At the end of the \openshmem program
    which it initialized, the call to \FUNC{shmem\_init} must be matched with a
    call to \FUNC{shmem\_finalize}. After the first call to \FUNC{shmem\_init}, a
    subsequent call to \FUNC{shmem\_init} or \FUNC{shmem\_init\_thread} in the
    same program results in undefined behavior.
}

\apireturnvalues{
    None.
}

\apinotes{
    As of \openshmem[1.2], the use of \FUNC{start\_pes} has been
    deprecated and calls to it should be replaced with calls to \FUNC{shmem\_init}.
    While support for \FUNC{start\_pes} is still required in \openshmem libraries,
    users are encouraged to use \FUNC{shmem\_init}. An important difference between
    \FUNC{shmem\_init} and \FUNC{start\_pes} is that multiple calls to
    \FUNC{shmem\_init} within a program results in undefined behavior, while in the
    case of \FUNC{start\_pes}, any subsequent calls to \FUNC{start\_pes} after the
    first one results in a no-op.
}

\begin{apiexamples}

\apicexample
    {The following \FUNC{shmem\_init} example is for \Cstd[11] programs:}
    {example_code/shmem_init_example.c}
    {}

\end{apiexamples}

\end{apidefinition}
