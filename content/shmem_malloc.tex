\apisummary{
    Symmetric heap memory management routines.
}

\begin{apidefinition}

\begin{Csynopsis}
void *shmem_malloc(size_t size);
void shmem_free(void *ptr);
void *shmem_realloc(void *ptr, size_t size);
void *shmem_align(size_t alignment, size_t size);
\end{Csynopsis}

\begin{apiarguments}
    \apiargument{IN}{size}{The size, in bytes, of a block to be
        allocated from the symmetric heap. This argument is of type \VAR{size\_t}}
    \apiargument{IN}{ptr}{Points to a block within the symmetric heap.}
    \apiargument{IN}{alignment}{Byte alignment of the block allocated from the
        symmetric heap.}
\end{apiarguments}


\apidescription{
    The \FUNC{shmem\_malloc} routine returns a pointer to a block of at least
    \VAR{size} bytes suitably aligned for any use.  This space is allocated from the
    symmetric heap (in contrast to \FUNC{malloc}, which allocates from the private
    heap).
    
    The \FUNC{shmem\_align} routine allocates a block in the symmetric heap that has
    a byte alignment specified by the alignment argument.
    
    The \FUNC{shmem\_free} routine causes the block to which \VAR{ptr} points to be
    deallocated, that is, made available for further allocation.  If \VAR{ptr} is a
    null pointer, no action occurs. 
           
    The \FUNC{shmem\_realloc} routine changes the size of the block to which
    \VAR{ptr} points to the size (in bytes) specified by \VAR{size}.  The contents
    of the block are unchanged up to the lesser of the new and old sizes. If the new
    size is larger, the newly allocated portion of the block is
    uninitialized.  If \VAR{ptr} is a \CONST{NULL} pointer, the
    \FUNC{shmem\_realloc} routine behaves like the \FUNC{shmem\_malloc} routine for
    the specified size.  If \VAR{size} is \CONST{0} and \VAR{ptr} is not a
    \CONST{NULL} pointer, the block to which it points is freed. If the space cannot
    be allocated, the block to which \VAR{ptr} points is unchanged.
    
    The \FUNC{shmem\_malloc}, \FUNC{shmem\_align}, \FUNC{shmem\_free}, and \FUNC{shmem\_realloc} routines
    are provided  so that multiple \acp{PE} in a program can allocate symmetric,
    remotely accessible memory blocks.  These memory blocks can then be used with
    \openshmem communication routines.  Each of these routines include at least one
    call to a procedure that is semantically equivalent to \FUNC{shmem\_barrier\_all}:
    \FUNC{shmem\_malloc} and \FUNC{shmem\_align} call a
    barrier on exit; \FUNC{shmem\_free} calls a barrier on entry; and
    \FUNC{shmem\_realloc} may call barriers on both entry and exit, depending on
    whether an existing allocation is modified and whether new memory is allocated.
    This ensures that all
    \acp{PE} participate in the memory allocation, and that the memory on other
    \acp{PE} can be used as  soon as the local \ac{PE} returns.  The user is
    responsible for calling these routines with identical argument(s) on all
    \acp{PE}; if differing \VAR{size} arguments are used, the behavior of the call
    and any subsequent \openshmem calls becomes undefined.
}

\apireturnvalues{
    The \FUNC{shmem\_malloc} routine returns a pointer to the allocated space;
    otherwise, it returns a \CONST{NULL} pointer.
    
    The \FUNC{shmem\_free} routine returns no value.
    
    The \FUNC{shmem\_realloc} routine returns a pointer to the allocated space
    (which may have moved); otherwise, it returns a null pointer.
    
    The \FUNC{shmem\_align} routine returns an aligned pointer to the allocated
    space; otherwise, it returns a \CONST{NULL} pointer.
}

\apinotes{ 
    As of Specification 1.2 the use of \FUNC{shmalloc}, \FUNC{shmemalign},
    \FUNC{shfree},  and \FUNC{shrealloc} has been deprecated. Although OpenSHMEM
    libraries are required to support the calls, program users are encouraged to use
    \FUNC{shmem\_malloc}, \FUNC{shmem\_align}, \FUNC{shmem\_free}, and
    \FUNC{shmem\_realloc} instead.  The behavior and signature  of the routines
    remains unchanged from the deprecated versions.
    					 
    The total size of the symmetric heap is determined at job startup.  One can
    adjust the size of the heap using the \CONST{SHMEM\_SYMMETRIC\_SIZE} environment
    variable (where available).	
    
    The \FUNC{shmem\_malloc}, \FUNC{shmem\_free}, and \FUNC{shmem\_realloc} routines
    differ from the private heap allocation routines in that all \acp{PE} in a
    program must call them (a barrier is used to ensure this).
}		

\apiimpnotes{
    The symmetric heap allocation routines always return a pointer to corresponding
    symmetric objects across all PEs. The \openshmem specification does not
    require that the virtual addresses are equal across all \acp{PE}. Nevertheless,
    the implementation must avoid costly address translation operations in the
    communication path, including order $N$ (where $N$ is the number of \acp{PE})
    memory translation tables.  In order to avoid address translations, the
    implementation may re-map the allocated block of memory based on agreed virtual
    address.  Additionally, some operating systems provide an option to disable
    virtual address randomization, which enables predictable allocation of virtual
    memory addresses.
}

\end{apidefinition}
