\emph{Collective routines} are defined as coordinated communication or synchronization
operations performed by a group of \acp{PE}.

\openshmem provides three types of collective routines:

\begin{enumerate}
\item Collective routines that operate on teams use a team handle parameter to determine
  which \acp{PE} will participate in the routine, and use resources encapsulated by the team object
  to perform operations. See Section~\ref{subsec:team} for details on team management.

\begin{DeprecateBlock}
\item Collective routines that operate on active sets use a set of parameters to determine
  which \acp{PE} will participate and what resources are used to perform operations.
\end{DeprecateBlock}

\item Collective routines that accept neither team nor active set
  parameters, which implicitly operate on the default team and, as
  required, the default context.
\end{enumerate}

Concurrent accesses to symmetric memory by an \openshmem collective
routine and any other means of access---where at least one updates the
symmetric memory---results in undefined behavior.
Since \acp{PE} can enter and exit collectives at different times,
accessing such memory remotely may require additional synchronization.

\subsubsection*{Team-based collectives}

The team-based collective routines are performed with respect to a valid
\openshmem team, which is specified by a team handle argument.
Team-based collective operations require all \acp{PE} in the team to call
the routine in order for the operation to complete. If an invalid team handle
or \LibConstRef{SHMEM\_TEAM\_INVALID} is passed to a team-based collective
routine, the behavior is undefined.

Team objects encapsulate the per \ac{PE} system resources required to complete
team-based collective routines.
All \openshmem teams-based collective calls are blocking routines which may use those
system resources. On completion of a team-based collective call, the \ac{PE} may
immediately call another collective on that same team without any other intervening
synchronization across the team.

While \openshmem routines provide thread support according to the
thread-support level provided at initialization (see
Section~\ref{subsec:thread_support}), team-based collective routines
may not be called simultaneously by multiple threads on a given team.

Collective operations are matched across a given team based on ordering. So for a given team,
collectives must occur in the same order across all PEs in a team.

The team-based collective routines defined in the \openshmem Specification are:

\begin{itemize}
\item \FUNC{shmem\_team\_sync}
\item \FUNC{shmem\_\{TYPE\_\}broadcast\{mem\}}
\item \FUNC{shmem\_\{TYPE\_\}collect\{mem\}}
\item \FUNC{shmem\_\{TYPE\_\}fcollect\{mem\}}
\item Reduction routines for the following operations: AND, OR, XOR, MAX, MIN, SUM, PROD
\item \FUNC{shmem\_\{TYPE\_\}alltoall\{mem\}}
\item \FUNC{shmem\_\{TYPE\_\}alltoalls\{mem\}}
\end{itemize}

In addition, all team creation functions are collective operations. In addition to the ordering
and thread safety requirements described here, there are additional synchronization requirements
on team creation operations. See Section~\ref{subsec:team} for more details.

\begin{DeprecateBlock}

\subsubsection*{Active-set-based collectives}

The active-set-based collective routines require all \acp{PE}
in the active set to simultaneously call the
routine.  A \ac{PE} that is not in the active set calling the collective
routine results in undefined behavior.

The active set is defined by the arguments \VAR{PE\_start}, \VAR{logPE\_stride},
and \VAR{PE\_size}.  \VAR{PE\_start} specifies the starting \ac{PE} number and
is the lowest numbered \ac{PE} in the active set.  The stride between successive
\acp{PE} in the active set is $2^{logPE\_stride}$ and \VAR{logPE\_stride} must
be greater than or equal to zero.  \VAR{PE\_size} specifies the number of
\acp{PE} in the active set and must be greater than zero.  The active set must
satisfy the requirement that its last member corresponds to a valid \ac{PE}
number, that is
$0 \le PE\_start + (PE\_size - 1) * 2^{logPE\_stride} < npes$.

All \acp{PE} participating in the active-set-based collective routine must provide the same
values for these arguments.  If any of these requirements are not met, the
behavior is undefined.

Another argument important to active-set-based collective routines is \VAR{pSync}, which is a
symmetric work array.  All \acp{PE} participating in an active-set-based collective must pass the
same \VAR{pSync} array.
Every element of the \VAR{pSync} array must be initialized to
\LibConstRef{SHMEM\_SYNC\_VALUE} before it is used as an argument to
any active-set-based collective routine.
On completion of such a collective call, the \VAR{pSync} is
restored to its original contents.  The user is permitted to reuse a \VAR{pSync}
array if all previous collective routines using the \VAR{pSync} array have
completed on all participating \acp{PE}. One can use a synchronization
collective routine such as \FUNC{shmem\_barrier} to ensure completion of previous active-set-based collective
routines. The \FUNC{shmem\_barrier} and \FUNC{shmem\_sync} routines allow the same
\VAR{pSync} array to be used on consecutive calls as long as the \acp{PE}
in the active set do not change.

All collective routines defined in the Specification are blocking.  The
collective routines return on completion.  The active-set-based collective
routines defined in the \openshmem Specification are:

\begin{itemize}
\item \FUNC{shmem\_barrier}
\item \FUNC{shmem\_sync}
\item \FUNC{shmem\_broadcast\{32, 64\}}
\item \FUNC{shmem\_collect\{32, 64\}}
\item \FUNC{shmem\_fcollect\{32, 64\}}
\item Reduction routines for the following operations: AND, OR, XOR, MAX, MIN, SUM, PROD
\item \FUNC{shmem\_alltoall\{32, 64\}}
\item \FUNC{shmem\_alltoalls\{32, 64\}}
\end{itemize}

\end{DeprecateBlock}


\subsubsection*{Team-implicit collectives}

The \FUNC{shmem\_sync\_all} routine synchronizes all \acp{PE} in the
computation through the default team. This routine is equivalent to a
call to \FUNC{shmem\_team\_sync} on the default team.

The \FUNC{shmem\_barrier\_all} routine synchronizes all \acp{PE} in
the default team and ensures completion of all local and remote memory
updates issued via the default context.  This routine is equivalent to
a call to \FUNC{shmem\_ctx\_quiet} on the default context followed by a
call to \FUNC{shmem\_team\_sync} on the default team.

\subsubsection*{Error codes returned from team-based collectives}

Collective operations involving multiple \acp{PE} may return values
indicating success while other \acp{PE} are still executing the
collective operation. Return values indicating success or failure of a
collective routine on one \ac{PE} may not indicate that all \acp{PE}
involved in the collective operation will return the same value. Some
operations, such as team creation, must return identical return codes
across multiple \acp{PE}.
