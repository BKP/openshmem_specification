\apisummary{
Create two new teams by splitting an existing parent team into two subsets based on a
2D Cartesian space defined by the \VAR{xrange} argument and a \VAR{y} dimension derived from \VAR{xrange}
and the parent team size. These ranges describe the Cartesian space in \emph{x}-
and \emph{y}-dimensions.}

\begin{apidefinition}

\begin{Csynopsis}
int @\FuncDecl{shmem\_team\_split\_2d}@(shmem_team_t parent_team, int xrange,
     shmem_team_config_t *xaxis_config, long xaxis_mask, shmem_team_t *xaxis_team,
     shmem_team_config_t *yaxis_config, long yaxis_mask, shmem_team_t *yaxis_team);
\end{Csynopsis}

\begin{apiarguments}
\apiargument{IN}{parent\_team}{A valid \openshmem team. Any predefined teams, such as
\LibHandleRef{SHMEM\_TEAM\_WORLD}, may be used, or any team created by the user.}

\apiargument{IN}{xrange}{A nonnegative integer representing the number of
elements in the first dimension.}

\apiargument{INOUT}{xaxis\_config}{
  A pointer to the configuration parameters for the new \VAR{x}-axis team.}

\apiargument{IN}{xaxis\_mask}{
  The bitwise mask representing the set of configuration parameters to use
  from \VAR{xaxis\_config}.}

\apiargument{OUT}{xaxis\_team}{A new \ac{PE} team handle representing a \ac{PE}
subset consisting of all the \acp{PE} that have the same coordinate along the \VAR{x}-axis
as the calling \ac{PE}.}

\apiargument{INOUT}{yaxis\_config}{
  A pointer to the configuration parameters for the new \VAR{y}-axis team.}

\apiargument{IN}{yaxis\_mask}{
  The bitwise mask representing the set of configuration parameters to use
  from \VAR{yaxis\_config}.}

\apiargument{OUT}{yaxis\_team}{A new \ac{PE} team handle representing a \ac{PE}
subset consisting of all the \acp{PE} that have the same coordinate along the \VAR{y}-axis
as the calling \ac{PE}.}
\end{apiarguments}

\apidescription{
The \FUNC{shmem\_team\_split\_2d} routine is a collective routine. It creates two
new teams by splitting an existing parent team into up to two subsets based on a
2D Cartesian space. The user provides the size of the \VAR{x} dimension, which is then
used to derive the size of the \VAR{y} dimension based on the size of the parent team.
The size of the \VAR{y} dimension will be equal to $\lceil N \div xrange \rceil$, where
\VAR{N} is the size of the parent team. In other words,
$xrange \times yrange \geq N$, so that every \ac{PE} in the parent team has a
unique \VAR{(x,y)} location the 2D Cartesian space.

The mapping of \ac{PE} number to coordinates is $(x, y) = ( pe \mod xrange, \lfloor pe \div xrange \rfloor )$,
where $pe$ is the \ac{PE} number in the parent team. So, if $xrange = 3$,
then the first 3 \acp{PE} in the parent team will form the first
\VAR{xteam}, the second three \acp{PE} in the parent team form the second \VAR{xteam},
and so on.

Thus, after the split operation, each of the new \VAR{xteam}s will contain all \acp{PE} that
have the same coordinate along the \VAR{y}-axis as the calling \ac{PE}. Each of the
new \VAR{yteam}s will contain all \acp{PE} with the same coordinate along the
\VAR{x}-axis as the calling \ac{PE}.

The \acp{PE} are numbered in the new teams based on the coordinate of the
\ac{PE} along the given axis. So, another way to think of the result of the split
operation is that the value returned by \FUNC{shmem\_team\_my\_pe}(\VAR(xteam)) is the
x-coordinate and the value returned by \FUNC{shmem\_team\_my\_pe}(\VAR(yteam))
is the y-coordinate of the calling \ac{PE}.

Any valid \openshmem team can be used as the parent team. This routine must be
called by all \acp{PE} in the parent team. The value of \VAR{xrange} must be
nonnegative and all \acp{PE} in the parent team must pass the same value for
\VAR{xrange}. None of the parameters need to reside in symmetric memory.

The \VAR{xaxis\_config} and \VAR{yaxis\_config} arguments specify team
configuration parameters for the \VAR{x}- and \VAR{y}-axis teams, respectively.
These parameters are described in Section~\ref{subsec:shmem_team_config_t}.
All \acp{PE} that will be in the same resultant team must specify the same
configuration parameters.
The \acp{PE} in the parent team \emph{do not} have to all provide the same
parameters for new teams.

The \VAR{xaxis\_mask} and\VAR{xaxis\_mask} arguments are a bitwise masks
representing the set of configuration parameters to use from
\VAR{xaxis\_config} and \VAR{yaxis\_config}, respectively.
A mask value of \CONST{0} indicates that the team
should be created with the default values for all configuration parameters.
See Section~\ref{subsec:shmem_team_config_t} for field mask names and
default configuration parameters.

If \VAR{parent\_team} is an invalid team handle, the behavior is undefined.

If \VAR{parent\_team} compares equal to \LibConstRef{SHMEM\_TEAM\_INVALID}, no new
teams will be created, and both \VAR{xaxis\_team} and \VAR{yaxis\_team}
will be assigned the value \LibConstRef{SHMEM\_TEAM\_INVALID}.

If either team cannot be created, that team will be assigned the value
\LibConstRef{SHMEM\_TEAM\_INVALID}.
}

\apireturnvalues{
  Zero on successful creation of both \VAR{xaxis\_team} and \VAR{yaxis\_team},
  nonzero otherwise.
}

\apinotes{
Since the split may result in a 2D space with more points than there are members of
the parent team, there may be a final, incomplete row of the 2D mapping of the parent
team. This means that the resultant \VAR{yteam}s may vary in size by up to 1 \ac{PE},
and that there may be one resultant \VAR{xteam} of smaller size than all of the other
\VAR{xteam}s.

The following grid shows the 12 teams that would result from splitting a parent team
of size 10 with \VAR{xrange} of 3. The numbers in the grid cells are the \ac{PE} numbers
in the parent team. The rows are the \VAR{xteam}s. The columns are the \VAR{yteam}s.

\begin{center}
\begin{tabular}{|l|l|l|l|}
 \hline
             & yteam & yteam & yteam \\
             & x=0 & x=1 & x=2  \\ \hline
 xteam, y=0  & 0   & 1   & 2  \\ \hline
 xteam, y=1  & 3   & 4   & 5  \\ \hline
 xteam, y=2  & 6   & 7   & 8  \\ \hline
 xteam, y=3  & 9  \\
 \cline{0-1}
\end{tabular}
\end{center}

It would be legal, for example, if \acp{PE} 0, 3, 6, 9 specified a different value
for \VAR{yaxis\_config} than all of the other \acp{PE}, as long as the configuration parameters match
for all \acp{PE} in each of the new teams.

See the description of team handles and predefined teams at the top of section
\ref{subsec:team} for more information about team handle semantics and usage.
}

\begin{apiexamples}

    \apicexample
    {The following example demonstrates the use of 2D Cartesian split in a
    \Cstd[11] program. This example shows how multiple 2D splits can be used
    to generate a 3D Cartesian split. This method can be extrapolated to
    generate splits of any number of dimensions.}
    {./example_code/shmem_team_split_2D.c}
    {
    The example above splits \LibHandleRef{SHMEM\_TEAM\_WORLD} into a 3D team
    with dimensions 3x4xN.  For example, if \VAR{npes} = 16,  \VAR{xdim} = 3,
    and \VAR{ydim} = 4, then the final dimensions are 3x4x2.  In this case, the
    first split of \LibHandleRef{SHMEM\_TEAM\_WORLD} results in 6 \VAR{xteams}
    and 3 \VAR{yzteams}:

    \begin{center}
    \begin{tabular}{|l|l|l|l|l|}
     \hline
      \multicolumn{2}{|c|}{} & \multicolumn{3}{c|}{\VAR{yzteam}} \\ \cline{3-5}
      \multicolumn{2}{|c|}{} & \VAR{x} = 0 & \VAR{x} = 1 & \VAR{x} = 2  \\ \hline
\multirow{6}{*}{\VAR{xteam}} & \VAR{yz} = 0  & 0   & 1   & 2  \\ \cline{2-5}
                             & \VAR{yz} = 1  & 3   & 4   & 5  \\ \cline{2-5}
                             & \VAR{yz} = 2  & 6   & 7   & 8  \\ \cline{2-5}
                             & \VAR{yz} = 3  & 9   & 10 &  11 \\ \cline{2-5}
                             & \VAR{yz} = 4  & 12  & 13 & 14  \\ \cline{2-5}
                             & \VAR{yz} = 5  & 15  \\
     \cline{0-2}
    \end{tabular}
    \end{center}

    The second split of \VAR{yzteam} for \VAR{x} = 0, \VAR{ydim} = 4 results in 2
    \VAR{yteams} and 4 \VAR{zteams}:


    \begin{center}
    \begin{tabular}{|l|l|l|l|l|l|}
     \hline
      \multicolumn{2}{|c|}{} & \multicolumn{4}{c|}{\VAR{zteam}} \\ \cline{3-6}
      \multicolumn{2}{|c|}{} & \VAR{y} = 0 & \VAR{y} = 1 & \VAR{y} = 2 & \VAR{y} = 3 \\ \hline
\multirow{2}{*}{\VAR{yteam}} & \VAR{z} = 0  & 0    & 3    & 6  & 9 \\ \cline{2-6}
                             & \VAR{z} = 1  & 12   & 15 \\
     \cline{0-3}
    \end{tabular}
    \end{center}

    The second split of \VAR{yzteam} for \VAR{x} = 1, \VAR{ydim} = 4 results in
    2 \VAR{yteams} and 4 \VAR{zteams}:

    \begin{center}
    \begin{tabular}{|l|l|l|l|l|l|}
     \hline
      \multicolumn{2}{|c|}{} & \multicolumn{4}{c|}{\VAR{zteam}} \\ \cline{3-6}
      \multicolumn{2}{|c|}{} & \VAR{y} = 0 & \VAR{y} = 1 & \VAR{y} = 2 & \VAR{y} = 3 \\ \hline
\multirow{2}{*}{\VAR{yteam}} & \VAR{z} = 0  & 1    & 4    & 7  & 10 \\ \cline{2-6}
                             & \VAR{z} = 1  & 13 \\
     \cline{0-2}
    \end{tabular}
    \end{center}

    The second split of \VAR{yzteam} for \VAR{x} = 2, \VAR{ydim} = 4 results in
    2 \VAR{yteams} and 4 \VAR{zteams}:

    \begin{center}
    \begin{tabular}{|l|l|l|l|l|l|}
     \hline
      \multicolumn{2}{|c|}{} & \multicolumn{4}{c|}{\VAR{zteam}} \\ \cline{3-6}
      \multicolumn{2}{|c|}{} & \VAR{y} = 0 & \VAR{y} = 1 & \VAR{y} = 2 & \VAR{y} = 3 \\ \hline
\multirow{2}{*}{\VAR{yteam}} & \VAR{z} = 0  & 2    & 5    & 8  & 11 \\ \cline{2-6}
                             & \VAR{z} = 1  & 14 \\
     \cline{0-2}
    \end{tabular}
    \end{center}

    The final number of teams for each dimension are:
    \begin{itemize}
        \item 6 \VAR{xteams}: these are teams where (\VAR{z},\VAR{y}) is fixed and \VAR{x} varies.
        \item 6 \VAR{yteams}: these are teams where (\VAR{x},\VAR{z}) is fixed and \VAR{y} varies.
        \item 12 \VAR{zteams}: these are teams where (\VAR{x},\VAR{y}) is fixed and \VAR{z} varies.
    \end{itemize}

    The expected output is: \\
    \begin{small}
    \texttt{
    (0, 0, 0) is me = 0  \\
    (1, 0, 0) is me = 1  \\
    (2, 0, 0) is me = 2  \\
    (0, 1, 0) is me = 3  \\
    (1, 1, 0) is me = 4  \\
    (2, 1, 0) is me = 5  \\
    (0, 2, 0) is me = 6  \\
    (1, 2, 0) is me = 7  \\
    (2, 2, 0) is me = 8  \\
    (0, 3, 0) is me = 9  \\
    (1, 3, 0) is me = 10 \\
    (2, 3, 0) is me = 11 \\
    (0, 0, 1) is me = 12 \\
    (1, 0, 1) is me = 13 \\
    (2, 0, 1) is me = 14 \\
    (0, 1, 1) is me = 15
    }
    \end{small}
}

\end{apiexamples}

\end{apidefinition}
