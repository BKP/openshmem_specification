\apisummary{
Create two new teams by splitting an existing parent team into two subsets based on a
2D Cartesian space defined by the \VAR{xrange} argument and a \VAR{y} dimension derived from \VAR{xrange}
and the parent team size. These ranges describe the Cartesian space in \emph{x}-
and \emph{y}-dimensions.}

\begin{apidefinition}

\begin{Csynopsis}
int @\FuncDecl{shmem\_team\_split\_2d}@(shmem_team_t parent_team, int xrange,
     shmem_team_config_t *xaxis_config, long xaxis_mask, shmem_team_t *xaxis_team,
     shmem_team_config_t *yaxis_config, long yaxis_mask, shmem_team_t *yaxis_team);
\end{Csynopsis}

\begin{apiarguments}
\apiargument{IN}{parent\_team}{A valid \openshmem team. Any predefined teams, such as
\LibHandleRef{SHMEM\_TEAM\_WORLD}, may be used, or any team created by the user.}

\apiargument{IN}{xrange}{A nonnegative integer representing the number of
elements in the first dimension.}

\apiargument{INOUT}{xaxis\_config}{
  A pointer to the configuration parameters for the new \VAR{x}-axis team.}

\apiargument{IN}{xaxis\_mask}{
  The bitwise mask representing the set of configuration parameters to use
  from \VAR{xaxis\_config}.}

\apiargument{OUT}{xaxis\_team}{A new \ac{PE} team handle representing a \ac{PE}
subset consisting of all the \acp{PE} that have the same coordinate along the \VAR{x}-axis
as the calling \ac{PE}.}

\apiargument{INOUT}{yaxis\_config}{
  A pointer to the configuration parameters for the new \VAR{y}-axis team.}

\apiargument{IN}{yaxis\_mask}{
  The bitwise mask representing the set of configuration parameters to use
  from \VAR{yaxis\_config}.}

\apiargument{OUT}{yaxis\_team}{A new \ac{PE} team handle representing a \ac{PE}
subset consisting of all the \acp{PE} that have the same coordinate along the \VAR{y}-axis
as the calling \ac{PE}.}
\end{apiarguments}

\apidescription{
The \FUNC{shmem\_team\_split\_2d} routine is a collective routine. It creates two
new teams by splitting an existing parent team into up to two subsets based on a
2D Cartesian space. The user provides the size of the \VAR{x} dimension, which is then
used to derive the size of the \VAR{y} dimension based on the size of the parent team.
The size of the \VAR{y} dimension will be equal to $ceiling ( N \div xrange )$, where
\VAR{N} is the size of the parent team. In other words,
$xrange \times yrange \geq N$, so that every \ac{PE} in the parent team has a
unique \VAR{(x,y)} location the 2D Cartesian space.

After the split operation, each of the new teams will contain all \acp{PE} that
have the same coordinate along the \VAR{x}-axis and \VAR{y}-axis, respectively, as the calling
\ac{PE}. The \acp{PE} are numbered in the new teams based on the position of the
\ac{PE} along the given axis.

Any valid \openshmem team can be used as the parent team. This routine must be
called by all \acp{PE} in the parent team. The value of \VAR{xrange} must be
nonnegative and all \acp{PE} in the parent team must pass the same value for
\VAR{xrange}. None of the parameters need to reside in symmetric memory.

The \VAR{xaxis\_config} and \VAR{yaxis\_config} arguments specify team
configuration parameters for the \VAR{x}- and \VAR{y}-axis teams, respectively.
These parameters are described in Section~\ref{subsec:shmem_team_config_t}.
All \acp{PE} that will be in the same resultant team must specify the same
configuration parameters.
The \acp{PE} in the parent team \emph{do not} have to all provide the same
parameters for new teams.

The \VAR{xaxis\_mask} and\VAR{xaxis\_mask} arguments are a bitwise masks
representing the set of configuration parameters to use from
\VAR{xaxis\_config} and \VAR{yaxis\_config}, respectively.
A mask value of \CONST{0} indicates that all the field members of the
configuration parameter argument should be used.
Individual field masks can be combined through a bitwise OR operation
of the following library constants:

{
  \apitablerow{\LibConstRef{SHMEM\_TEAM\_NOCOLLECTIVE}}{
    The team should be created using the value of the
    \VAR{disable\_collectives} member of the respective
    configuration parameter.
  }
  \apitablerow{\LibConstRef{SHMEM\_TEAM\_LOCAL\_LIMIT}}{
    The team should be created using the value of the
    \VAR{return\_local\_limit} member of the respective
    configuration parameter.
  }
  \apitablerow{\LibConstRef{SHMEM\_TEAM\_NUM\_THREADS}}{
    The team should be created using the value of the
    \VAR{num\_threads} member of the respective
    configuration parameter.
  }
}

If \VAR{parent\_team} is an invalid team handle, the behavior is undefined.

If \VAR{parent\_team} compares equal to \LibConstRef{SHMEM\_TEAM\_NULL}, no new
teams will be created, and both \VAR{xaxis\_team} and \VAR{yaxis\_team}
will be assigned the value \LibConstRef{SHMEM\_TEAM\_NULL}.

If either team cannot be created, that team will be assigned the value
\LibConstRef{SHMEM\_TEAM\_NULL}.
}

\apireturnvalues{
  Zero on successful creation of both \VAR{xaxis\_team} and \VAR{yaxis\_team},
  nonzero otherwise.
}

\apinotes{
Since the split may result in a 2D space with more points than there are members of
the parent team, there may be a final, incomplete row of the 2D mapping of the parent
team. This means that the resultant \VAR{x}-axis teams may vary in size by up to 1 \ac{PE},
and that there may be one resultant \VAR{y}-axis team of smaller size than all of the other
\VAR{y}-axis teams.

The following grid shows the 12 teams that would result from splitting a parent team
of size 10 with \VAR{xrange} of 3. The numbers in the grid cells are the \ac{PE} numbers
in the parent team. The rows are the \VAR{y}-axis teams. The columns are the \VAR{x}-axis teams.

\begin{center}
\begin{tabular}{|l|l|l|l|}
 \hline
      & x=0 & x=1 & x=2  \\ \hline
 y=0  & 0   & 1   & 2  \\ \hline
 y=1  & 3   & 4   & 5  \\ \hline
 y=2  & 6   & 7   & 8  \\ \hline
 y=3  & 9     \\
 \cline{0-1}
\end{tabular}
\end{center}

It would be legal, for example, if \acp{PE} 0, 3, 6, 9 specified a different value
for \VAR{xaxis\_config} than all of the other \acp{PE}, as long as the configuration parameters match
for all \acp{PE} in each of the new teams.

See the description of team handles and predefined teams at the top of section
\ref{subsec:team} for more information about team handle semantics and usage.
}

\begin{apiexamples}

\end{apiexamples}

\end{apidefinition}
