\apisummary{
Create two new teams by splitting an existing parent team into two subsets based on a
2D Cartesian space defined by xrange argument and the yrange derived from xrange
and the parent team size. These ranges describe the Cartesian space in X and Y
dimensions.}

\begin{apidefinition}

\begin{Csynopsis}
void @\FuncDecl{shmem\_team\_split\_2d}@(shmem_team_t parent_team, int xrange,
long xaxis_options, shmem_team_t *xaxis_team, long yaxis_options, shmem_team_t *yaxis_team);
\end{Csynopsis}

\begin{apiarguments}
\apiargument{IN}{parent\_team}{A valid \openshmem team. Any predefined teams, such as
\LibHandleRef{SHMEM\_TEAM\_WORLD}, may be used, or any team created by the user.}

\apiargument{IN}{xrange}{A non-negative integer representing the number of
elements in the first dimension.}

\apiargument{IN}{xaxis\_options}{The set of options requested for the new X-axis team.
Multiple options may be requested by combining them with a bitwise OR operation;
otherwise, \CONST{0} can be given if no options are requested.}

\apiargument{OUT}{xaxis\_team}{A new \ac{PE} team handle representing a \ac{PE}
subset consisting of all the \acp{PE} that have the same coordinate along the X-axis 
as the calling \ac{PE}.}

\apiargument{IN}{yaxis\_options}{The set of options requested for the new Y-axis team.
These options do not have to be the same as the options for the new X-axis team.}

\apiargument{OUT}{yaxis\_team}{A new \ac{PE} team handle representing a \ac{PE}
subset consisting of all the \acp{PE} that have the same coordinate along the Y-axis 
as the calling \ac{PE}.}
\end{apiarguments}

\apidescription{
The \FUNC{shmem\_team\_split\_2d} routine is a collective routine. It creates two
new teams by splitting an existing parent team into up to two subsets based on a
2D Cartesian space. The user provides the size of the X dimension, which is then
used to derive the size of the Y dimension based on the size of the parent team.
The size of the Y dimension will be equal to $ceiling ( N \div xrange )$, where
\VAR{N} is the size of the parent team. In other words,
$xrange \times yrange \geq N$, so that every \ac{PE} in the parent team has a
unique \VAR{(x,y)} location the 2D Cartesian space.

After the split operation, each of the new teams will contain all \acp{PE} that
have the same coordinate along the X-axis and Y-axis, respectively, as the calling
\ac{PE}. The \acp{PE} are numbered in the new teams based on the position of the
\ac{PE} along the given axis.

Any valid \openshmem team can be used as the parent team. This routine must be
called by all \acp{PE} in the parent team. The value of \VAR{xrange} must be
non-negative and all \acp{PE} in the parent team must pass the same value for
\VAR{xrange}. None of the parameters need to reside in symmetric memory.

Error checking will be done to detect if the value \LibConstRef{SHMEM\_TEAM\_NULL}
is passed as the parent team handle. In that case, the program will abort with
an informative error message. If an invalid team handle is passed to the
routine, the behavior is undefined.

The options in the table below can be supplied during team split to restrict
team functions and enable performance optimizations. All \acp{PE} that will
be in the same resultant new team must specify the same options. The \acp{PE}
in the parent team \emph{do not} have to all provide the same options for new
teams.

When using a given team, the application must comply with the requirements
of all options set on that team; otherwise, the behavior is undefined.
No options are enabled on \LibHandleRef{SHMEM\_TEAM\_WORLD} or on other
predefined \openshmem teams.

    \apitablerow{\LibConstRef{SHMEM\_TEAM\_NOCOLLECTIVE}}{
                 The new team will not be created with the necessary support
                 structures to enable team based collectives.
                 This will typically allow implementations to speed up team creation
                 and reduce \openshmem library footprint for teams with this option.
                 This option will not prevent the new team from using atomics or
                 other non-collective team based operations.}
}

\apireturnvalues{
None.
}

\apinotes{
Since the split may result in a 2D space with more points than there are members of
the parent team, there may be a final, incomplete row of the 2D mapping of the parent
team. This means that the resultant X-axis teams may vary in size by up to 1 \ac{PE},
and that there may be one resultant Y-axis team of smaller size than all of the other
Y-axis teams.

The following grid shows the 12 teams that would result from splitting a parent team
of size 10 with \VAR{xrange} of 3. The numbers in the grid cells are the \ac{PE} numbers
in the parent team. The rows are the Y-axis teams. The columns are the X-axis teams.

\begin{center}
\begin{tabular}{|l|l|l|l|}
 \hline
      & x=0 & x=1 & x=2  \\ \hline
 y=0  & 0   & 1   & 2  \\ \hline
 y=1  & 3   & 4   & 5  \\ \hline
 y=2  & 6   & 7   & 8  \\ \hline
 y=3  & 9     \\
 \cline{0-1} 
\end{tabular}
\end{center}

It would be legal, for example, if \acp{PE} 0, 3, 6, 9 specified a different value
for \VAR{xaxis\_options} than all of the other \acp{PE}, as long as options match
for all \acp{PE} in each of the new teams.

See the description of team handles and predefined teams at the top of section
\ref{subsec:team} for more information about team handle semantics and usage.
}

\begin{apiexamples}

\end{apiexamples}

\end{apidefinition}
