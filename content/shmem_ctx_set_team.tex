\apisummary{
  Update the team associated with the communication context.
}

\begin{apidefinition}

  \begin{Csynopsis}
int @\FuncDecl{shmem\_ctx\_set\_team}@(shmem_ctx_t ctx, shmem_team_t team);
  \end{Csynopsis}

  \begin{apiarguments}

    \apiargument{IN}{ctx}{
      A handle to a communication context.
    }

    \apiargument{IN}{team}{
      A handle to the specified \ac{PE} team.
    }

  \end{apiarguments}

  \apidescription{
    The \FUNC{shmem\_ctx\_set\_team} routine associates the \ac{PE} team
    identified by the handle \VAR{team} with the communication context
    specified by the handle \VAR{ctx}.
    All subsequent \openshmem operations performed on the specified context
    will operate with respect to the updated \ac{PE} team.

    If \VAR{ctx} is a handle to the default context or
    \VAR{team} is equal to the constant \CONST{SHMEM\_TEAM\_NULL}, then
    the specified context is not updated and a value of \CONST{-1} is returned.

    If \VAR{ctx} is an invalid context, a value of \CONST{-1} is returned.
  }

  \apireturnvalues{
    Zero on success; otherwise, \CONST{-1}.

    \begin{FeedbackRequest}
      Should this routine return nonzero, -1, or negative values
      (e.g., to allow for implementation-defined error codes) on error?
      Will slowing down the critical path of this routine by adding
      input checking adversely affect its use?
    \end{FeedbackRequest}
  }

  \apinotes{
    None.
  }

\end{apidefinition}
