\apisummary{
    shmem\_team\_get\_options returns the options flags describing a given team
}

\begin{apidefinition}

\begin{Csynopsis}
long @\FuncDecl{shmem\_team\_get\_options}@(shmem_team_t team);
\end{Csynopsis}

\begin{apiarguments}
\apiargument{IN}{team}{A valid SHMEM team handle.}
\end{apiarguments}

\apidescription{
\FUNC{shmem\_team\_get\_options} returns a long unsigned value containing
all of the options which describe the given team. Options are requested when
new teams are created in the various \FUNC{shmem\_team\_split\_*} functions.
Whichever of the requested options are applied to the team by the library
implementation will be returned by \FUNC{shmem\_team\_get\_options}.

All processes in the team will get back the same value for the team options.

Error checking will be done to ensure a valid team handle is provided.
All errors are considered fatal and will result in the job aborting
with an informative error message.
}

\apireturnvalues{
The set of options applied to the given team. Multiple options are combined
with a bitwise OR and can be extracted with a bitwise AND. A return value of
\CONST{0} implies that the team uses all default options.
}

\apinotes{
A use case for this function is to determine if a given team will
support collective operations by testing for the \LibConstRef{SHMEM\_TEAM\_NOCOLLECTIVE}
option. When teams are created without support for collectives, they may still use
point to point operations to communicate and synchronize. So programmers may wish
to design frameworks with functions that provide alternative algorithms
for teams based on whether they do or do not support collectives.
}

\end{apidefinition}
