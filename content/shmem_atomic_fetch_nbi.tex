\apisummary{
    The nonblocking atomic fetch routine provides a method for atomically
    fetching the value of a remote data object.
}

\begin{apidefinition}

\begin{C11synopsis}
void @\FuncDecl{shmem\_atomic\_fetch\_nbi}@(TYPE *fetch, const TYPE *source, int pe);
void @\FuncDecl{shmem\_atomic\_fetch\_nbi}@(shmem_ctx_t ctx, TYPE *fetch, const TYPE *source, int pe);
\end{C11synopsis}
where \TYPE{} is one of the extended \ac{AMO} types specified by
Table~\ref{extamotypes}.

\begin{Csynopsis}
void @\FuncDecl{shmem\_\FuncParam{TYPENAME}\_atomic\_fetch\_nbi}@(TYPE *fetch, const TYPE *source, int pe);
void @\FuncDecl{shmem\_ctx\_\FuncParam{TYPENAME}\_atomic\_fetch\_nbi}@(shmem_ctx_t ctx, TYPE *fetch, const TYPE *source, int pe);
\end{Csynopsis}
where \TYPE{} is one of the extended \ac{AMO} types and has a corresponding
\TYPENAME{} specified by Table~\ref{extamotypes}.

\begin{apiarguments}

  \apiargument{IN}{ctx}{A context handle specifying the context on which to
    perform the operation. When this argument is not provided, the operation is
    performed on the default context.}
  \apiargument{OUT}{fetch}{Local data object to be updated.}
  \apiargument{IN}{source}{The remotely accessible data object to be fetched
    from the remote \ac{PE}.}
  \apiargument{IN}{pe}{An integer that indicates the \ac{PE} number from which
    \VAR{source} is to be fetched.}

\end{apiarguments}

\apidescription{
    The nonblocking atomic fetch routines perform a nonblocking fetch of a
    value atomically from a remote data object. This routine returns after
    initiating the operation. The operation is considered complete after a
    subsequent call to \FUNC{shmem\_quiet}. At the completion of
    \FUNC{shmem\_quiet}, contents of the \source{} data object from \ac{PE} has been
    fetched into \VAR{fetch} local data object.
}

\apireturnvalues{
    None.
}

\apinotes{
    None.
}

\end{apidefinition}
