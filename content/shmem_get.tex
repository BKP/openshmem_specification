\apisummary{
    Copies data from a specified \ac{PE}.
}

\begin{apidefinition}

\begin{Csynopsis}
void shmem_double_get(double *dest, const double *source, size_t nelems, int pe);
void shmem_float_get(float *dest, const float *source, size_t nelems, int pe);
void shmem_get32(void *dest, const void *source, size_t  nelems, int pe);
void shmem_get64(void  *dest, const void *source, size_t nelems, int pe);
void shmem_get128(void *dest, const void *source, size_t nelems, int pe);
void shmem_getmem(void *dest, const void *source, size_t nelems, int pe);
void shmem_int_get(int *dest, const int *source, size_t  nelems, int pe);
void shmem_long_get(long *dest, const long *source, size_t nelems, int pe);
void shmem_longdouble_get(long double *dest, const long double *source, size_t nelems, int pe);
void shmem_longlong_get(long long *dest, const long long *source, size_t nelems, int pe);
void shmem_short_get(short *dest, const short *source, size_t nelems, int pe);
\end{Csynopsis}

\begin{Fsynopsis}
INTEGER nelems, pe
CALL SHMEM_CHARACTER_GET(dest, source, nelems, pe)
CALL SHMEM_COMPLEX_GET(dest, source, nelems, pe)
CALL SHMEM_DOUBLE_GET(dest, source, nelems, pe)
CALL SHMEM_GET4(dest, source, nelems, pe)
CALL SHMEM_GET8(dest, source, nelems, pe)
CALL SHMEM_GET32(dest, source, nelems, pe)
CALL SHMEM_GET128(dest, source, nelems, pe)
CALL SHMEM_GETMEM(dest, source, nelems, pe)
CALL SHMEM_INTEGER_GET(dest, source, nelems, pe)
CALL SHMEM_LOGICAL_GET(dest, source, nelems, pe)
CALL SHMEM_REAL_GET(dest, source, nelems, pe)
\end{Fsynopsis}

\begin{apiarguments}
    \apiargument{OUT}{dest}{Local data object to be updated.}
    \apiargument{IN}{source}{Data object on the \ac{PE} identified by \VAR{pe}
        that contains the data to be copied.  This data object must be remotely
        accessible.}
    \apiargument{IN}{nelems}{Number of elements in the \dest{} and \source{}
        arrays. \VAR{nelems} must be of type \VAR{size\_t} for \Clang. If you are
        using \Fortran, it must be a constant, variable, or array element of default
        integer type.}
    \apiargument{IN}{pe}{\ac{PE}  number of the remote \ac{PE}.  \VAR{pe} must
        be of type integer. If you are using \Fortran, it must be a constant,
        variable, or array element of default integer type.}
\end{apiarguments}

\apidescription{
   The get routines provide a method for copying a contiguous symmetric data
   object from a different \ac{PE} to a contiguous data object on the local
   \ac{PE}.  The routines return after the data has been delivered to the
   \dest{} array on the local \ac{PE}. 
}

\apidesctable{
    The  \dest{} and \source{} data objects must conform to typing constraints,
    which are as follows:
}{Routine}{Data type of \VAR{dest} and \VAR{source}}

    \apitablerow{shmem\_getmem}{\Fortran: Any noncharacter type. \Clang: Any
        data  type.  nelems is scaled in bytes.}
    \apitablerow{shmem\_get4, shmem\_get32}{Any  noncharacter type
        that has a storage size equal to \CONST{32} bits.}
    \apitablerow{shmem\_get8, shmem\_get64}{Any noncharacter type that
        has a storage size equal to \CONST{64} bits.}
    \apitablerow{shmem\_get128}{Any  noncharacter type that has a
        storage size equal to \CONST{128} bits.}
    \apitablerow{shmem\_short\_get}{Elements of type short.}
    \apitablerow{shmem\_int\_get}{Elements of type int.}
    \apitablerow{shmem\_long\_get}{Elements of type long.}
    \apitablerow{shmem\_longlong\_get}{Elements of type long long.}
    \apitablerow{shmem\_float\_get}{Elements of type float.}
    \apitablerow{shmem\_double\_get}{Elements of type double.}
    \apitablerow{shmem\_longdouble\_get}{Elements of type long double.}
    \apitablerow{SHMEM\_CHARACTER\_GET}{Elements of type character. \VAR{nelems} is
    the number  of characters  to transfer. The actual character
    lengths of the \source{} and \dest{} variables are ignored.}
    \apitablerow{SHMEM\_COMPLEX\_GET}{Elements of type complex of default
       size.}
    \apitablerow{SHMEM\_DOUBLE\_GET}{\Fortran: Elements of type double precision.}
    \apitablerow{SHMEM\_INTEGER\_GET}{Elements of type integer.}
    \apitablerow{SHMEM\_LOGICAL\_GET}{Elements of type logical.}
    \apitablerow{SHMEM\_REAL\_GET}{Elements of type real.}

\apireturnvalues{
    None.
}

\apinotes{
    See Introduction for a definition of the term remotely accessible. If you are
    using \Fortran, data types must be of default size.  For example, a real
    variable must be declared as \CONST{REAL}, \CONST{REAL*4},  or
    \CONST{REAL(KIND=KIND(1.0))}.
}

\begin{apiexamples}

\apifexample
    {Consider this example for \Fortran.}
    {./example_code/shmem_get_example.f90}
    {}

\end{apiexamples}

\end{apidefinition}
