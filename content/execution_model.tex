An \openshmem program consists of a set of \openshmem processes called \ac{PE}s
that execute in a \ac{SPMD}-like model where each \ac{PE} can take a different
execution path. \newtext{For example,} a \ac{PE} can be implemented using an OS
process \oldtext{or an OS thread} \footnote{\oldtext{Implementing \ac{PE}s using
OS threads requires compiler techniques to implement the \openshmem memory
model.}}.  The \ac{PE}s progress asynchronously, and can communicate/synchronize
via the \openshmem interfaces.  All \ac{PE}s in an \openshmem program should
start by calling the initialization routine  \FUNC{shmem\_init}
\footnote{\textbf{start\_pes} has been deprecated as of Specification 1.2}
before using any of the other \openshmem library routines.  An \openshmem
program finishes execution by returning from the main routine or when any PE
calls \FUNC{shmem\_global\_exit}. When returning from main, \openshmem must
complete all pending communication and release all the resources associated to
the library using an implicit collective synchronization across PEs. The user
has the option to call \FUNC{shmem\_finalize} (before returning from main) to
complete all pending communication and release all the \openshmem library
resources without terminating the program. Calling any \openshmem routine after
\FUNC{shmem\_finalize} leads to undefined behavior.

The \ac{PE}s of the \openshmem program are identified by unique integers.  The
identifiers are integers assigned in a monotonically increasing manner from zero
to the total number of \ac{PE}s minus 1. \ac{PE} identifiers are used for
\openshmem calls (e.g. to specify \OPR{put} or \OPR{get} routines on symmetric
data objects, collective synchronization calls) or to dictate a control flow for
\ac{PE}s using constructs of \Clang{} or \Fortran. The identifiers are fixed for
the life of the \openshmem program.

\subsection{Progress of \openshmem Operations}\label{subsec:progress}

The \openshmem model assumes that computation and communication are naturally
overlapped. \openshmem programs are expected to exhibit progression of
communication both with and without \openshmem calls. Consider a \ac{PE} that is
engaged in a computation with no \openshmem calls. Other \ac{PE}s should be able
to communicate (\OPR{put}, \OPR{get}, \OPR{collective}, \OPR{atomic}, etc) and
complete communication operations with that computationally-bound \ac{PE}
without that \ac{PE} issuing any explicit \openshmem calls. \openshmem
communication calls involving that \ac{PE} should progress regardless of when
that \ac{PE} next engages in an \openshmem call.

\textbf{Note to implementors:}
\begin{itemize}
  \item An \openshmem implementation for hardware that does not provide
      asynchronous communication capabilities may require a software progress
      thread in order to process remotely-issued communication requests without
      explicit program calls to the \openshmem library.  
  \item High performance implementations of \openshmem are expected to leverage
      hardware offload capabilities and provide asynchronous one-sided
      communication without software assistance.
  \item Implementations should avoid deferring the execution of one-sided
      operations until a synchronization point where data is known to be
      available. High-quality implementations should attempt asynchronous delivery
      whenever possible, for performance reasons. Additionally, the \openshmem
      community discourages releasing \openshmem implementations that do not
      provide asynchronous one-sided operations, as these have very limited
      performance value for \openshmem programs.
\end{itemize}

\subsection{Atomicity Guarantees}\label{sec:amo_guarantees}

\openshmem contains a number of routines that operate on symmetric data
atomically (Section \ref{sec:amo}).  These routines guarantee that accesses by
\openshmem's atomic operations will be exclusive, but do not guarantee
exclusivity in combination with other routines, either inside \openshmem or
outside.

For example: during the execution of an atomic remote integer increment
operation on a symmetric variable \VAR{X}, no other \openshmem atomic operation
may access \VAR{X}.  After the increment, \VAR{X} will have increased its value
by \CONST{1} on the destination \ac{PE}, at which point other atomic operations
may then modify that \VAR{X}.  However, access to the symmetric object \VAR{X}
with non-atomic operations, such as one-sided \OPR{put} or \OPR{get} operations,
will \OPR{invalidate} the atomicity guarantees.
