An \openshmem program consists of a set of \openshmem processes called \acp{PE}
that execute in an \ac{SPMD}-like model where each \ac{PE} can take a different
execution path. For example, a \ac{PE} can be implemented using an OS
process. The \acp{PE} may be either single or multithreaded.
The \acp{PE} progress asynchronously, and can communicate/synchronize
via the \openshmem interfaces.  All \acp{PE} in an \openshmem program should
start by calling the initialization routine \FUNC{shmem\_init}%
\footnote{\FUNC{start\_pes} has been deprecated as of \openshmem[1.2]}
or \FUNC{shmem\_init\_thread} before using any of the other \openshmem library routines. 
An \openshmem program concludes its use of the \openshmem library when all \acp{PE} call
\FUNC{shmem\_finalize} or any \ac{PE} calls \FUNC{shmem\_global\_exit}.
During a call to \FUNC{shmem\_finalize}, the \openshmem library must
complete all pending communication and release all the resources associated to
the library using an implicit collective synchronization across \acp{PE}.
Calling any \openshmem routine after \FUNC{shmem\_finalize} leads to undefined
behavior.

The \acp{PE} of the \openshmem program are identified by unique integers.  The
identifiers are integers assigned in a monotonically increasing manner from zero
to one less than the total number of \acp{PE}. \ac{PE} identifiers are used for
\openshmem calls (e.g. to specify \OPR{put} or \OPR{get} routines on symmetric
data objects, collective synchronization calls) or to dictate a control flow for
\acp{PE} using constructs of \Cstd or \Fortran. The identifiers are fixed for
the life of the \openshmem program.

\subsection{Progress of \openshmem Operations}\label{subsec:progress}

The \openshmem model assumes that computation and communication are naturally
overlapped. \openshmem programs are expected to exhibit progression of
communication both with and without \openshmem calls. Consider a \ac{PE} that is
engaged in a computation with no \openshmem calls. Other \acp{PE} should be able
to communicate (\OPR{put}, \OPR{get}, \OPR{atomic}, etc) and
complete communication operations with that computationally-bound \ac{PE}
without that \ac{PE} issuing any explicit \openshmem calls. One-sided \openshmem
communication calls involving that \ac{PE} should progress regardless of when
that \ac{PE} next engages in an \openshmem call.

\textbf{Note to implementors:}
\begin{itemize}
  \item An \openshmem implementation for hardware that does not provide
      asynchronous communication capabilities may require a software progress
      thread in order to process remotely-issued communication requests without
      explicit program calls to the \openshmem library.  
  \item High performance implementations of \openshmem are expected to leverage
      hardware offload capabilities and provide asynchronous one-sided
      communication without software assistance.
  \item Implementations should avoid deferring the execution of one-sided
      operations until a synchronization point where data is known to be
      available. High-quality implementations should attempt asynchronous delivery
      whenever possible, for performance reasons. Additionally, the \openshmem
      community discourages releasing \openshmem implementations that do not
      provide asynchronous one-sided operations, as these have very limited
      performance value for \openshmem programs.
\end{itemize}
