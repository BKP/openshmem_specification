The objective of the \openshmem profiling interface is to ensure an 
easy and flexible usage model for profiling (and other similar) 
tool developers to interface their codes into \openshmem 
implementations on different platforms. Since \openshmem is a 
machine independent standard with different implementations, it is 
unreasonable to expect that the authors and developers of profiling 
tools for \openshmem will have access to the source code that 
implements \openshmem on any particular machine. It is therefore 
necessary to provide a mechanism by which the implementors of such 
tools can collect whatever performance information they wish 
\textit{without} access to the underlying implementation.

The \openshmem profiling interface places the following requirements 
on implementations. 

\begin{enumerate}
\item An \openshmem implementation must provide a mechanism through 
which all of the \openshmem defined functions may be accessible 
with a name shift. This requires, in C and Fortran, an alternate 
entry point name, with the prefix \texttt{pshmem\_} for each 
\openshmem function in each provided language binding and support 
method. For \openshmem inlined functions (e.g. macros), it is also 
required that the \texttt{pshmem\_} version be supplied although it 
is not possible to replace the \texttt{pshmem\_} version with a 
user-defined version at link time.
\item It must be ensured that the \openshmem functions that are not 
replaced as above, may still be linked into an executable image 
without causing name clashes. 
\item Documentation of the implementation of different language 
bindings of the \openshmem interface must indicate if they 
are layered on top of each other. Using this documentation,  
developer can determine whether he/she needs to implement the 
profile interface for each binding or not.
\item In the case where the implementation of different language 
bindings is done through a layered approach using ``wrapper'' 
functions, the wrapper functions must be kept separate from the 
rest of the library. This requirement allows the developers to 
extract these functions from the original \openshmem library 
without bringing along any other code.
\item A no-op routine \FUNC{shmem\_pcontrol} must be provided 
in the library.
\end{enumerate}

Provided that an \openshmem implementation meets the requirements 
above, it is possible for the implementor of the profiling system 
to intercept the \openshmem calls that are made by the user 
program. The information required can be collected before and after 
calling the underlying \openshmem implementation through the name 
shifted entry points. 

\subsection{Control of Profiling}
\label{sec:pshmem_control_profile}
Any user code must be able to control the profiler dynamically 
during runtime. Generally, this capability is used for the 
purposes of

\begin{itemize}
\item Enabling and disabling of profiling based on the current 
state of the execution and calculation
\item Flushing of the trace buffers at non-critical execution 
regions
\item Adding user events to a trace file
\end{itemize}

These functionalities can be achieved through the usage of 
\FUNC{shmem\_pcontrol}.

\subsubsection{\textbf{SHMEM\_PCONTROL}}\label{subsec:shmem_pcontrol}
\apisummary{
  Allows the user to control profiling.
}

\begin{apidefinition}

\begin{Csynopsis}
void @\FuncDecl{shmem\_pcontrol}@(const int level, ...);
\end{Csynopsis}

\begin{apiarguments}

  \apiargument{IN}{level}{The profiling level.}

\end{apiarguments}

\apidescription{
  \FUNC{shmem\_pcontrol} sets the profiling level and any other  
  library defined effects through additional arguments. \openshmem libraries
  make no use of this routine and simply return immediately to the user code.
}

\apireturnvalues{
  None.
}

\apinotes{
  Since \openshmem has no control of the implementation of the profiling code, 
  it is impossible to precisely specify the semantics that will be provided by 
  calls to \FUNC{shmem\_pcontrol}. This vagueness extends to the number of 
  arguments to the function and their datatypes. However, to provide some 
  level of portability of user codes to different profiling libraries, the 
  following \VAR{level} values are recommended.

  \begin{itemize}
  \item \texttt{level <= 0} Profiling is disabled.
  \item \texttt{level == 1} Profiling is enabled at the default level of detail.
  \item \texttt{level == 2} Profiling is enabled and profile buffers are 
  flushed if available.
  \item \texttt{level >= 2} Profiling is enabled with profile library defined 
  effects and additional arguments.
  \end{itemize}

  The default state after \FUNC{shmem\_init} is recommended to have profiling 
  enabled at the default level of detail (\texttt{level == 1}). This allows users
  to link with a profiling library and to obtain profile output without 
  having to modify the user-level source code. 
%  The provision of 
%  \FUNC{shmem\_pcontrol} as a no-op in the standard \openshmem library 
%  supports more detailed profiling with a tool that can still link against 
%  the standard \openshmem library.
}

\end{apidefinition}




\subsection{Example Implementations}
\label{sec:pshmem_example_implementations}

\subsubsection{Profiler}
\label{sec:pshmem_example_profiler}

The following example illustrates how a profiler can measure the
total and average time spent by the \FUNC{shmem\_long\_put} 
function in the profiling library that intercepts the \openshmem 
function calls from the user application.

\lstinputlisting[language={C}, tabsize=2,
      basicstyle=\ttfamily\footnotesize,
      morekeywords={size_t, ptrdiff_t, shmem_ctx_t, _Thread_local}]
      {example_code/pshmem_example.c}

\subsubsection{\openshmem Library}
\label{sec:pshmem_example_library}
To implement the name-shift versions of the \openshmem functions, 
there are various options available. The following two examples 
present two such options that can be implemented in C on a Unix 
system. These two options are dependent on whether the linker 
and compiler support weak symbols. 

If the compiler and linker support weak external symbols, then 
only a single library is required. The following two examples show 
how the name-shifted requirement can be achieved on such platforms. 

\noindent\textbf{Example 1}

\lstinputlisting[language={C}, tabsize=2,
      basicstyle=\ttfamily\footnotesize,
      morekeywords={size_t, ptrdiff_t, shmem_ctx_t, _Thread_local}]
      {example_code/pshmem_weak_symbol_1.c}

The effect of the \texttt{\#pragma} directive is to define the 
external symbol \texttt{shmem\_example} as a weak definition that 
aliases the \FUNC{pshmem\_example} function. This 
means that the linker will allow another definition 
of the symbol (e.g. the profiling library may contain an alternate 
definition). The weak definition is used in the case where no other 
definition for the same function exists. 

\noindent\textbf{Example 2}

\lstinputlisting[language={C}, tabsize=2,
      basicstyle=\ttfamily\footnotesize,
      morekeywords={size_t, ptrdiff_t, shmem_ctx_t, _Thread_local}]
      {example_code/pshmem_weak_symbol_2.c}

In this example, the keyword \texttt{\_\_attribute\_\_} is used to 
declare the function, \FUNC{pshmem\_example} as an alias for 
the original function \FUNC{shmem\_example}.

In the absence of weak symbols, one possible solution would be to 
use the C macro preprocessor as shown in the following example. 

\lstinputlisting[language={C}, tabsize=2,
      basicstyle=\ttfamily\footnotesize,
      morekeywords={size_t, ptrdiff_t, shmem_ctx_t, _Thread_local}]
      {example_code/pshmem_no_weak_symbol.c}

The same source file can then be compiled to produce both versions 
of the library, depending on the state of the 
\texttt{BUILD\_PSHMEM\_INTERFACES} macro symbol.
