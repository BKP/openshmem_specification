The objective of the \openshmem profiling interface is to ensure an 
easy and flexible usage model for profiling (and other similar) 
tool developers to interface their codes into \openshmem 
implementations on different platforms. Since \openshmem is a 
machine-independent standard with different implementations, it is 
unreasonable to expect that the authors and developers of profiling 
tools for \openshmem will have access to the source code that 
implements \openshmem on any particular machine. It is, therefore,  
necessary to provide a mechanism by which the implementors of such 
tools can collect whatever performance information they wish 
\emph{without} access to the underlying implementation.

The \openshmem profiling interface places the following requirements 
on implementations. 

\begin{enumerate}
\item An \openshmem implementation must provide a mechanism through 
which all of the \openshmem defined functions may be accessible 
with a name shift. This requires an alternate 
entry point name, with the prefix \FUNC{pshmem\_} for each 
\openshmem function. For \openshmem inlined functions (e.g. macros), 
it is also required that the \FUNC{pshmem\_} version is supplied 
although it is not possible to replace the \FUNC{shmem\_} version 
with a user-defined version at link time.
\item It must be ensured that the \openshmem functions that are not 
replaced as above, may still be linked into an executable image 
without causing name clashes. 
\item Documentation of the implementation of different language 
bindings of the \openshmem interface must indicate if they 
are layered on top of each other. Using this documentation,   
developers can determine whether they need to implement the 
profile interface for each binding or not. For example, it must 
be noted that the \openshmem \Cstd[11] type-generic interfaces for 
different \ac{RMA} and \ac{AMO} operations cannot have any equivalent
\FUNC{pshmem\_} interfaces because the \Cstd[11] type-generic 
interfaces are implemented as macros.
\item In the case where the implementation of different \ac{API}  
feature sets is implemented through a layered approach using 
``wrapper'' functions, the wrapper functions must be kept separate 
from the rest of the library. This requirement allows the developers 
to extract these functions from the original \openshmem library 
and add them into the profiling library without bringing along any 
other code.
\item A no-op routine, \FUNC{shmem\_pcontrol}, must be provided 
in the \openshmem library.
\item It must be ensured that any \openshmem types or constants that are 
needed by the \FUNC{pshmem\_} interfaces are defined in \HEADER{pshmem.h}.
\end{enumerate}

Provided that an \openshmem implementation meets these requirements, 
it is possible for the implementor of the profiling system 
to intercept the \openshmem calls that are made by the user 
program. The information required can be collected before and after 
calling the underlying \openshmem implementation through the name 
shifted entry points. 

\subsection{Control of Profiling}
\label{sec:pshmem_control_profile}
Any user code must be able to control the profiler dynamically 
during runtime. Generally, this capability is used for the 
purposes of

\begin{itemize}
\item Enabling and disabling of profiling based on the current 
state of the execution and calculation,
\item Flushing of the trace buffers at noncritical execution
regions,
\item Adding user events to a trace file.
\end{itemize}

These functionalities can be achieved through the usage of 
\FUNC{shmem\_pcontrol}.

\subsubsection{\textbf{SHMEM\_PCONTROL}}\label{subsec:shmem_pcontrol}
\apisummary{
  Allows the user to control profiling.
}

\begin{apidefinition}

\begin{Csynopsis}
void @\FuncDecl{shmem\_pcontrol}@(const int level, ...);
\end{Csynopsis}

\begin{apiarguments}

  \apiargument{IN}{level}{The profiling level.}

\end{apiarguments}

\apidescription{
  \FUNC{shmem\_pcontrol} sets the profiling level and any other  
  library defined effects through additional arguments. \openshmem libraries
  make no use of this routine and simply return immediately to the user code.
}

\apireturnvalues{
  None.
}

\apinotes{
  Since \openshmem has no control of the implementation of the profiling code, 
  it is impossible to precisely specify the semantics that will be provided by 
  calls to \FUNC{shmem\_pcontrol}. This vagueness extends to the number of 
  arguments to the function and their datatypes. However, to provide some 
  level of portability of user codes to different profiling libraries, the 
  following \VAR{level} values are recommended.

  \begin{itemize}
  \item \texttt{level <= 0} Profiling is disabled.
  \item \texttt{level == 1} Profiling is enabled at the default level of detail.
  \item \texttt{level == 2} Profiling is enabled and profile buffers are 
  flushed if available.
  \item \texttt{level >= 2} Profiling is enabled with profile library defined 
  effects and additional arguments.
  \end{itemize}

  The default state after \FUNC{shmem\_init} is recommended to have profiling 
  enabled at the default level of detail (\texttt{level == 1}). This allows users
  to link with a profiling library and to obtain profile output without 
  having to modify the user-level source code. 
%  The provision of 
%  \FUNC{shmem\_pcontrol} as a no-op in the standard \openshmem library 
%  supports more detailed profiling with a tool that can still link against 
%  the standard \openshmem library.
}

\end{apidefinition}




\subsection{Example Implementations}
\label{sec:pshmem_example_implementations}

\subsubsection{Profiler}
\label{sec:pshmem_example_profiler}

\SourceExample{example_code/pshmem_example.c}{
  The following example illustrates how a profiler can measure the
  total and average time spent by the \FUNC{shmem\_long\_put}
  function in the profiling library that intercepts the \openshmem
  function calls from the user application.
}

\subsubsection{OpenSHMEM Library}
\label{sec:pshmem_example_library}
To implement the name-shift versions of the \openshmem functions,
there are various options available. The following two examples
present two such options that can be implemented in C on a Unix
system. These two options are dependent on whether the linker
and compiler support weak symbols.

If the compiler and linker support weak external symbols, then
only a single library is required. The following two examples show
how the name-shifted requirement can be achieved on such platforms.

\SourceExample{example_code/pshmem_weak_symbol_1.c}{
  Here, the effect of the \FUNC{\#pragma} directive is to define the
  external symbol \FUNC{shmem\_example} as a weak definition that
  aliases the \FUNC{pshmem\_example} function. This
  means that the linker will allow another definition
  of the symbol (e.g. the profiling library may contain an alternate
  definition). The weak definition is used in the case where no other
  definition for the same function exists.
}

\SourceExample{example_code/pshmem_weak_symbol_2.c}{
  In this example, the keyword \FUNC{\_\_attribute\_\_} is used to
  declare the \FUNC{shmem\_example} function as an alias for
  the original function, \FUNC{pshmem\_example}.
}

In the absence of weak symbols, one possible solution would be to
use the \Cstd macro preprocessor as shown in the following example.

\SourceExample{example_code/pshmem_no_weak_symbol.c}{
  Each of the user-defined functions in the profiling library is
  declared using the \FUNC{SHFN} macro, which name-shifts the function
  depending on the state of the \VAR{BUILD\_PSHMEM\_INTERFACES} macro
  symbol.  The same source file can then be compiled to produce both
  versions of the library.
}

\subsection{Limitations}
\label{sec:pshmem_limitations}

\subsubsection{Multiple Counting}
\label{sec:pshmem_multiple_count}
Since some functions in \openshmem library may be implemented
using more basic \openshmem functions, it is possible for these basic
profiling functions to be called from within an \openshmem function
that was originally called from a profiling routine. For example,
\openshmem collective operations can be implemented using basic
point-to-point operations. Thus, profiling such a collective
operation may lead to counting a profiling function for a
point-to-point operation more than once after being called from the
collective function. It is the developer's responsibility to ensure
the profiling application does not count a function more than once if
that effect is not intended. For a single-threaded profiler, this
can be achieved through a static variable counting the number of
times a function has been profiled. In a multi-threaded environment,
additional synchronizations are needed to manage updates to this
counter and thus, it becomes more complex to accurately profile the
\openshmem functions.

\subsubsection{Separate Build and Link}
\label{sec:pshmem_separate_build_link}
To build the profiling tool with both the default \openshmem
functions as well as the \openshmem functions to be intercepted,
developers must build the multiple instances of the \openshmem
functions separately and link them to provide all the definitions.
This is necessary so that the developers of the profiling library
need only to define those \openshmem functions that they wish
to intercept; references to any other functions will be fulfilled
by the default \openshmem library. The link step can be summarized
as follows. \\

\noindent\texttt{\% cc ... -lmyprof -lpsma -lsma} \\

Here, \texttt{libmyprof.a} contains the profiler functions that
intercept the \openshmem functions to be profiled,
\texttt{libpsma.a} contains the name-shifted \openshmem function
definitions, and \texttt{libsma.a} contains the default \openshmem
function definitions.

\subsubsection{C11 Type-Generic Interfaces}
\label{sec:pshmem_c11_type_generic_interfaces}
\openshmem provides type-generic interfaces through \Cstd[11]
generic selection. These interfaces are defined as macros
and are mapped to \Cstd interface bindings. As a result, the
\Cstd[11] type-generic interfaces cannot be intercepted and
name-shifted \FUNC{pshmem\_} routines are not provided for these
bindings. Furthermore, because no two associations in a \Cstd[11]
\FUNC{\_Generic} selection expression can contain compatible types,
the type name of the \Cstd operation that is invoked may not be
identical to the type name of the original call's arguments (e.g.
\VAR{int32\_t} may map to \VAR{int}).
