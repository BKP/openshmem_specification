\apisummary{
  Allocate a zeroed block of symmetric memory.
}

\begin{apidefinition}

\begin{Csynopsis}
void *shmem_calloc(size_t count, size_t size);
\end{Csynopsis}

\begin{apiarguments}
  \apiargument{IN}{count}{The number of elements to allocate.}
  \apiargument{IN}{size}{The size in bytes of each element to allocate.}
\end{apiarguments}


\apidescription{
  The \FUNC{shmem\_calloc} routine allocates a region of remotely-accessible
  memory for an array of \VAR{count} objects of \VAR{size} bytes each and
  returns a pointer to the lowest byte address of the allocated symmetric
  memory. The space is initialized to all bits zero.

  If the allocation succeeds, the pointer returned shall be suitably
  aligned so that it may be assigned to a pointer to any type of object.
  If the allocation does not succeed, or either \VAR{count} or \VAR{size} is
  \CONST{0}, the return value is a null pointer.

  The values for \VAR{count} and \VAR{size} shall each be equal across
  all \acp{PE} calling \FUNC{shmem\_calloc}; otherwise, the behavior is
  undefined.

  The \FUNC{shmem\_calloc} routine calls \FUNC{shmem\_barrier\_all} on exit.
}

\apireturnvalues{
  The \FUNC{shmem\_calloc} routine returns a pointer to the lowest byte
  address of the allocated space; otherwise, it returns a null pointer.
}

\apinotes{
  None.
}

\end{apidefinition}
