This section specifies the interaction between \openshmem{} interfaces and the
user threads, and also describes the routines that can be used for initializing and 
querying the thread environment. There are four levels of threading supported by
the \openshmem{} implementation.
 
\begin{itemize}
\item {\bf \CONST{SHMEM\_THREAD\_SINGLE}} The \openshmem{} program may not be multithreaded, 
and all \openshmem{} interfaces are invoked by a single thread. 

\item {\bf \CONST{SHMEM\_THREAD\_FUNNELED}}
The \openshmem{} program may be multithreaded, however, the 
program must ensure that only the main thread invokes the \openshmem{}
interfaces.

\item {\bf \CONST{SHMEM\_THREAD\_SERIALIZED}} 
The \openshmem{} program may be multithreaded, however, the 
program must ensure that only one thread invokes the \openshmem{}
interfaces at any given instance i.e., the \openshmem{} interfaces 
are not invoked concurrently by multiple threads.

\item {\bf \CONST{SHMEM\_THREAD\_MULTIPLE}}
The \openshmem{} program may be multithreaded, and any 
thread may invoke the \openshmem{} interfaces.
\end{itemize}

\hspace{-15pt}
{\bf Semantics}

\begin{itemize}[leftmargin=*]

\item
In the \openshmem{} program, a thread calling \FUNC{SHMEM\_INIT} is designated as the main
thread. Multiple threads may not call \FUNC{SHMEM\_INIT}. Similarly,
\FUNC{SHMEM\_FINALIZE} may only be called on the main thread.

\item
Any \openshmem{} operation initiated by a thread is considered an action of the \ac{PE} as a whole.
The symmetric heap and symmetric variables scope
are not impacted by multiple threads invoking the
\openshmem{} interfaces, i.e., 
each \ac{PE} has a single symmetric data segment and symmetric heap that is shared by
all threads within that \ac{PE}.  For example, a thread invoking a memory allocation
routine such as \FUNC{SHMEM\_MALLOC} 
allocates memory that is accessible by all threads of the \ac{PE}. 
The requirement that the same symmetric heap operations must
be executed by all \acp{PE} in the same order also applies in a threaded
environment. Similarly, the completion of collective operations is not impacted by multiple threads. 
For example, the \FUNC{SHMEM\_BARRIER\_ALL} is completed when all \acp{PE} enter and
exit the \FUNC{SHMEM\_BARRIER\_ALL} call, even though only one thread in the \ac{PE} is
participating in the collective call. 

\item Blocking \openshmem{} calls will block the calling thread only, allowing
other threads, if available, to continue the execution. The calling thread will be blocked until the
event on which it is waiting occurs. Once the blocked communication is enabled
and can proceed, then the call will complete and the thread will be marked
runnable, within a finite time. A blocked thread will not prevent progress of
other runnable threads on the same \ac{PE}, and will not prevent them from
executing other \openshmem{} calls. Also, a blocked thread will not prevent the
progress of \openshmem{} calls on other \acp{PE}. 

\item In \CONST{SHMEM\_THREAD\_MULTIPLE} thread level, all \openshmem{} calls are thread-safe, 
i.e., two concurrently running threads may make \openshmem{} calls and the outcome will be as if the calls executed in
some order, even if their execution is interleaved.


\item
In \CONST{SHMEM\_THREAD\_SERIALIZED} and \CONST{SHMEM\_THREAD\_MULTIPLE} thread level, 
if multiple threads call the collective calls, it is the programmer's responsibility to ensure the correct ordering of
collective calls.  The symmetric heap management functions, which are defined to call
\FUNC{SHMEM\_BARRIER\_ALL}(\ref{sec:mem_routines}) before they return 
must be treated as collective operations.


\end{itemize} 
