This section specifies the interaction between the \openshmem interfaces and
user threads.  It also describes the routines that can be used for initializing and
querying the thread environment. There are four levels of threading defined by
the \openshmem specification.

\begin{description}
\item[\CONST{SHMEM\_THREAD\_SINGLE}] \hfill \\
The \openshmem program may not be multithreaded.

\item[\CONST{SHMEM\_THREAD\_FUNNELED}] \hfill \\
The \openshmem program may be multithreaded. However, the program must ensure
that only the main thread invokes the \openshmem interfaces. The main thread
is the thread that invokes either \FUNC{shmem\_init} or \FUNC{shmem\_init\_thread}.

\item[\CONST{SHMEM\_THREAD\_SERIALIZED}] \hfill \\
The \openshmem program may be multithreaded. However, the program must ensure
that the \openshmem interfaces are not invoked concurrently by multiple threads.

\item[\CONST{SHMEM\_THREAD\_MULTIPLE}] \hfill \\
The \openshmem program may be multithreaded and any thread may invoke the \openshmem
interfaces.
\end{description}

\noindent The following semantics apply to the usage of these models:

\begin{enumerate}
\item
In the \CONST{SHMEM\_THREAD\_FUNNELED}, \CONST{SHMEM\_THREAD\_SERIALIZED}, and
\CONST{SHMEM\_THREAD\_MULTIPLE} thread levels, the \FUNC{shmem\_init} and
\FUNC{shmem\_finalize} calls may only be invoked by the same thread.

\item
Any \openshmem operation initiated by a thread is considered an action of the
\ac{PE} as a whole. The symmetric heap and symmetric variables scope are not
impacted by multiple threads invoking the \openshmem interfaces.
Each \ac{PE} has a single symmetric data segment and symmetric heap that is shared by
all threads within that \ac{PE}.  For example, a thread invoking a memory allocation
routine such as \FUNC{shmem\_malloc} allocates memory that is accessible by
all threads of the \ac{PE}. The requirement that the same symmetric heap operations
must be executed by all \acp{PE} in the same order also applies in a threaded
environment. Similarly, the completion of collective operations is not impacted
by multiple threads. For example, \FUNC{shmem\_barrier\_all} is completed when
all \acp{PE} enter and exit the \FUNC{shmem\_barrier\_all} call, even though
only one thread in the \ac{PE} is participating in the collective call.

\item Blocking \openshmem calls will only block the calling thread, allowing
other threads, if available, to continue executing. The calling thread will
be blocked until the event on which it is waiting occurs. Once the blocking call is
completed, the thread is ready to continue execution. A blocked thread
will not prevent progress of other threads on the same \ac{PE} and will not
prevent them from executing other \openshmem calls when the thread level permits.
In addition, a blocked thread will not prevent the progress of \openshmem calls
performed on other \acp{PE}.

\item In the \CONST{SHMEM\_THREAD\_MULTIPLE} thread level, all \openshmem calls are thread-safe.
Any two concurrently running threads may make \openshmem calls and the outcome
will be as if the calls executed in some order, even if their execution is interleaved.

\item In the \CONST{SHMEM\_THREAD\_SERIALIZED} and \CONST{SHMEM\_THREAD\_MULTIPLE} thread levels,
if multiple threads call collective routines, including the symmetric heap
management routines, it is the programmer's responsibility to ensure the
correct ordering of collective calls.

\end{enumerate}
