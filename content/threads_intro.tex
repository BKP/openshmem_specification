This section specifies the interaction between \openshmem{} interfaces and the
user threads, and also describes the routines that can be used for initializing and 
querying the thread environment.
 

\begin{itemize}

\item
In a thread-compliant implementation, an \openshmem{} PE is an OS process that
is multithreaded. Each thread can issue \openshmem{} calls; however the threads
are not separately addressable. The symmetric heap and symmetric variables scope
or completion semantics are not impacted by multiple threads invoking the
\openshmem{} interfaces, i.e., the symmetric heap is a per-process resource. A
thread making a SHMEM\_MALLOC(), SHMEM\_REALLOC(), or SHMEM\_FREE() call affects
the entire process. The requirement that the same symmetric heap operations must
be executed by all processes in the same order also applies in a threaded
environment.
                                    	
\item In a thread-compliant implementation, 
all \openshmem{} calls are thread-safe, i.e., two concurrently running threads
may make \openshmem{} calls and the outcome will be as if the calls executed in
some order, even if their execution is interleaved.

\item Blocking \openshmem{} calls will block the calling thread only, allowing another
thread to execute, if available. The calling thread will be blocked until the
event on which it is waiting occurs. Once the blocked communication is enabled
and can proceed, then the call will complete and the thread will be marked
runnable, within a finite time. A blocked thread will not prevent progress of
other runnable threads on the same \ac{PE}, and will not prevent them from
executing other \openshmem{} calls. Also, a blocked thread will not prevent the
progress of other \openshmem{} calls on other \acp{PE}. 
 
\item
In a thread-compliant implementation, if multiple threads call the collective
calls, it is the programmer's responsibility to ensure the correct ordering of
collective calls.  The symmetric heap management functions SHMEM\_MALLOC(),
SHMEM\_REALLOC(), and SHMEM\_FREE() are all defined to call
SHMEM\_BARRIER\_ALL() before they return and thus must be treated as collective
operations.

\item
\openshmem{} thread calling SHMEM\_INIT() is designated as the main thread.
Multiple threads may not call SHMEM\_INIT(). Similarly, \openshmem{} finalize
may only be called on the main thread.

\end{itemize} 
 
{\bf Clarifications:}
 
\begin{itemize}
\item[]
The \openshmem{} specification can be implemented without support for threads.
The \openshmem{} implementations are not required to be thread complaint.
Regardless of whether \openshmem{} is thread complaint or not, SHMEM\_INIT(),
SHMEM\_FINALIZE(), SHMEM\_GLOBALEXIT(), SHMEM\_INFO\_GET\_NAME(), and
SHMEM\_INFO\_GET\_VERSION() should be thread safe.
 
%\item[]
%Interaction with signals: The outcome is undefined if a thread that executes an
%\openshmem{} call catches a signal. However, a thread of an \openshmem{} process
%may terminate, and may catch signals or be canceled by another thread when not
%executing \openshmem{} calls.

\end{itemize}
