\apisummary{
  Sets the signal value of a remote data object.
}

\begin{apidefinition}

\begin{C11synopsis}
void @\FuncDecl{shmem\_signal\_set}@(shmem_ctx_t ctx, const uint64_t *sig_addr, uint64_t signal, int pe);
\end{C11synopsis}

\begin{Csynopsis}
void @\FuncDecl{shmem\_signal\_set}@(const uint64_t *sig_addr, uint64_t signal, int pe);
void @\FuncDecl{shmem\_ctx\_signal\_set}@(shmem_ctx_t ctx, const uint64_t *sig_addr, uint64_t signal, int pe);
\end{Csynopsis}

\begin{apiarguments}
  \apiargument{IN}{ctx}{
    A context handle specifying the context on which to perform the
    operation. When this argument is not provided, the operation is
    performed on the default context.
  }
  \apiargument{OUT}{sig\_addr}{
    Symmetric address of the signal data object to be updated on the
    remote \ac{PE}.
  }
  \apiargument{IN}{signal}{
    Unsigned 64-bit value that is used for updating the remote
    \VAR{sig\_addr} signal data object.
  }
  \apiargument{IN}{pe}{
    \ac{PE} number of the remote \ac{PE}.
  }
\end{apiarguments}

\apidescription{
  \FUNC{shmem\_signal\_set} writes \VAR{value} into the signal data
  object pointed to by \VAR{sig\_addr} on \ac{PE}~\VAR{pe}.
  The update to the \VAR{sig\_addr} signal object at the calling
  \ac{PE} is expected to satisfy the atomicity guarantees as described
  in Section~\ref{subsec:signal_atomicity}.
}

\apireturnvalues{
  None.
}

\end{apidefinition}
