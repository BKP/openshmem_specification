\apisummary{
    Returns a memory block to the symmetric heap.
}

\begin{apidefinition}

\begin{Fsynopsis}
POINTER (addr, A(1))
INTEGER errcode, abort
CALL SHPDEALLC(addr, errcode, abort)
\end{Fsynopsis}

\begin{apiarguments}
    \apiargument{IN}{addr}{ First word address of the block to deallocate.}
    \apiargument{OUT}{errcode}{Error  code is 0 if no error was detected;
    otherwise, it is a  negative integer code for the type of error.}
    \apiargument{IN}{abort}{Abort code.  Nonzero requests abort on error;
    \CONST{0} requests an error code.}
\end{apiarguments}

\apidescription{
    SHPDEALLC returns a block of memory (allocated using \FUNC{SHPALLOC}) to the
    list of available space in the symmetric heap.  To maintain symmetric heap
    consistency, all \acp{PE} in a program must call \FUNC{SHPDEALLC} with the same
    value of \VAR{addr}; if any \acp{PE}  are missing, the program hangs.
}

\apireturnvalues{}
    \apitablerow{Error Code}{Condition}
    \apitablerow{ \CONST{-1} }{Length is not an integer greater than 0}
    \apitablerow{\CONST{-2}}{No more memory is available from the system (checked if the
    request cannot be satisfied from the available blocks on the symmetric heap).}
    \apitablerow{\CONST{-3}}{Address is outside the bounds of the symmetric heap.}
    \apitablerow{\CONST{-4}}{Block is already free.}
    \apitablerow{\CONST{-5}}{Address is not at the beginning of a block.}

\apinotes{
    None.
}   

\end{apidefinition}
