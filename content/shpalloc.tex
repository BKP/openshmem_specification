\apisummary{
    Allocates a block of memory from the symmetric heap.
}

\begin{apidefinition}

\begin{Fsynopsis}
POINTER (addr, A(1))
INTEGER length, errcode, abort
CALL SHPALLOC(addr, length, errcode, abort)
\end{Fsynopsis}

\begin{apiarguments}
    \apiargument{OUT}{addr}{First word address of the allocated block.}
    \apiargument{IN}{length}{Number of words of memory requested. One word is 32 bits.}
    \apiargument{OUT}{errcode}{Error code is \CONST{0} if no error was detected;
    otherwise, it is a negative integer code for the type of error.}
    \apiargument{IN}{abort}{Abort code; nonzero requests abort on error;
    \CONST{0}  requests an error code.}
\end{apiarguments}

\apidescription{   
    \FUNC{SHPALLOC} allocates a block of memory from the program's symmetric heap
    that is greater than or equal to the size requested. To maintain symmetric heap
    consistency, all \acp{PE} in an program must call \FUNC{SHPALLOC} with the same
    value of length; if any  \acp{PE} are missing, the program will hang.
    
    By using the \Fortran \CONST{POINTER} mechanism in the following manner, 
    array \VAR{A} can be used to refer to the block allocated by \FUNC{SHPALLOC}:
    \CONST{POINTER} (\VAR{addr}, \VAR{A}())
}

\apireturnvalues{}
    \apitablerow{Error Code}{Condition}
    \apitablerow{ \CONST{-1} }{Length is not an integer greater than \CONST{0}}
    \apitablerow{\CONST{-2}}{ No more memory is available from the system (checked if the
    request cannot be satisfied from the available blocks on the symmetric heap).}

\apinotes{  
    The total size of the symmetric heap is determined at job startup.  One may
    adjust the size of the heap using the \CONST{SHMEM\_SYMMETRIC\_SIZE} environment
    variable (if available).	
}

\apiimpnotes{
    The symmetric heap allocation routines always return a pointer to corresponding
    symmetric objects across all PEs. The \openshmem specification does not
    require that the virtual addresses are equal across all \acp{PE}. Nevertheless,
    the implementation must avoid costly address translation operations in the
    communication path, including order $N$ (where $N$ is the number of \acp{PE})
    memory translation tables.  In order to avoid address translations, the
    implementation may re-map the allocated block of memory based on agreed virtual
    address.  Additionally, some operating systems provide an option to disable
    virtual address randomization, which enables predictable allocation of virtual
    memory addresses.
}

\end{apidefinition}
