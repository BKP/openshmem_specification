\apisummary{
    Atomically sets the value of a remote data object.
}

\begin{apidefinition}

\begin{C11synopsis}
void shmem_atomic_set(TYPE *dest, TYPE value, int pe);
\end{C11synopsis}
where \TYPE{} is one of the extended \ac{AMO} types specified by
Table~\ref{extamotypes}.

\begin{Csynopsis}
void shmem_<TYPENAME>_atomic_set(TYPE *dest, TYPE value, int pe);
\end{Csynopsis}
where \TYPE{} is one of the extended \ac{AMO} types and has a corresponding
\TYPENAME{} specified by Table~\ref{extamotypes}.

\begin{Fsynopsis}
INTEGER pe
INTEGER*4 SHMEM_INT4_SET, value_i4
CALL SHMEM_INT4_SET(dest, value_i4, pe)
INTEGER*8 SHMEM_INT8_SET, value_i8
CALL SHMEM_INT8_SET(dest, value_i8, pe)
REAL*4 SHMEM_REAL4_SET, value_r4
CALL SHMEM_REAL4_SET(dest, value_r4, pe)
REAL*8 SHMEM_REAL8_SET, value_r8
CALL SHMEM_REAL8_SET(dest, value_r8, pe)
\end{Fsynopsis}

\begin{apiarguments}

\apiargument{IN}{dest}{The remotely accessible data object to be set on
    the remote \ac{PE}.}
\apiargument{IN}{value}{The value to be atomically written to the remote \ac{PE}.}
\apiargument{IN}{pe}{An integer that indicates the \ac{PE} number on which
    \VAR{dest} is to be updated.}

\end{apiarguments}

\apidescription{
    \FUNC{shmem\_atomic\_set} performs an atomic set operation. It writes the 
    \VAR{value} into \VAR{dest} on \VAR{pe} as an atomic operation.
}

\apireturnvalues{
    None.
}

\apinotes{
    As of OpenSHMEM~1.4, the use of \FUNC{shmem\_set} has been deprecated
    and should be expected to be removed in a future specification.
    Its behavior and call signature is identical to the replacement
    interface, \FUNC{shmem\_atomic\_set}.
}

\end{apidefinition}
