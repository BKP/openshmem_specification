\apisummary{
    The  put routines  provide  a method for copying data from a contiguous local
    data object to a data object on a specified \ac{PE}.
}

\begin{apidefinition}

\begin{C11synopsis}
void @\FuncDecl{shmem\_put}@(TYPE *dest, const TYPE *source, size_t nelems, int pe);
void @\FuncDecl{shmem\_put}@(shmem_ctx_t ctx, TYPE *dest, const TYPE *source, size_t nelems, int pe);
\end{C11synopsis}
where \TYPE{} is one of the standard \ac{RMA} types specified by Table \ref{stdrmatypes}.

\begin{Csynopsis}
void @\FuncDecl{shmem\_\FuncParam{TYPENAME}\_put}@(TYPE *dest, const TYPE *source, size_t nelems, int pe);
void @\FuncDecl{shmem\_ctx\_\FuncParam{TYPENAME}\_put}@(shmem_ctx_t ctx, TYPE *dest, const TYPE *source, size_t nelems, int pe);
\end{Csynopsis}
where \TYPE{} is one of the standard \ac{RMA} types and has a corresponding \TYPENAME{} specified by Table \ref{stdrmatypes}.

\begin{CsynopsisCol}
void @\FuncDecl{shmem\_put\FuncParam{SIZE}}@(void *dest, const void *source, size_t nelems, int pe);
void @\FuncDecl{shmem\_ctx\_put\FuncParam{SIZE}}@(shmem_ctx_t ctx, void *dest, const void *source, size_t nelems, int pe);
\end{CsynopsisCol}
where \SIZE{} is one of \CONST{8, 16, 32, 64, 128}.

\begin{CsynopsisCol}
void @\FuncDecl{shmem\_putmem}@(void *dest, const void *source, size_t nelems, int pe);
void @\FuncDecl{shmem\_ctx\_putmem}@(shmem_ctx_t ctx, void *dest, const void *source, size_t nelems, int pe);
\end{CsynopsisCol}

\begin{apiarguments}
    \apiargument{IN}{ctx}{A context handle specifying the context on which to perform the operation.
      When this argument is not provided, the operation is performed on
      the default context.}
    \apiargument{OUT}{dest}{Data object to be updated on the remote \ac{PE}. This
    data object must be remotely accessible.}
    \apiargument{IN}{source}{Data object containing the data to be copied.}
    \apiargument{IN}{nelems}{Number of elements in the \VAR{dest} and \VAR{source}
    arrays. \VAR{nelems} must be of type \VAR{size\_t} for \Cstd.}
    \apiargument{IN}{pe}{\ac{PE} number of the remote \ac{PE}. \VAR{pe} must be
    of type integer.}
\end{apiarguments}

\apidescription{
    The routines return after the data has been copied out of the \source{} array
    on the local \ac{PE}.  The delivery of data words into the data object on the
    destination \ac{PE} may occur in any order.  Furthermore, two successive put
    routines may deliver data out of order unless a call to \FUNC{shmem\_fence} is
    introduced between the two calls.
    If the context handle \VAR{ctx} does not correspond to a valid context,
    the behavior is undefined.
 }

\apireturnvalues{
    None.
}

\apinotes{
    None.
}

\begin{apiexamples}

\apicexample
    { The following \FUNC{shmem\_put} example is for \Cstd[11] programs:}
    {./example_code/shmem_put_example.c}
    {}
\end{apiexamples}

\end{apidefinition}
