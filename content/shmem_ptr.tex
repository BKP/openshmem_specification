\apisummary{
    Returns a local pointer to a symmetric data object on the specified \ac{PE} in the world team.
}

\begin{apidefinition}

\begin{Csynopsis}
void *@\FuncDecl{shmem\_ptr}@(const void *dest, int pe);
\end{Csynopsis}

\begin{apiarguments}
\apiargument{IN}{dest}{The symmetric address of the remotely accessible data
    object to be referenced.}
\apiargument{IN}{pe}{An integer that indicates the \ac{PE} number on which \dest{} is to
    be accessed.}
\end{apiarguments}

\apidescription{
    \FUNC{shmem\_ptr} returns an address that may be used to directly reference
    \dest{} on the specified \ac{PE} in the world team.  This address can be
    assigned to a pointer.  After that, ordinary loads and stores to \dest{} may
    be performed.  The address returned by \FUNC{shmem\_ptr} is a local address
    to a remotely accessible data object.  Providing this address to an argument
    of an \openshmem routine that requires a symmetric address results in
    undefined behavior.
    
    The \FUNC{shmem\_ptr} routine can provide an efficient means to accomplish
    communication, for example when a sequence of reads and writes to a data
    object on a remote \ac{PE} does not match the access pattern provided in an
    \openshmem data transfer routine like \FUNC{shmem\_put} or
    \FUNC{shmem\_iget}.
}

\apireturnvalues{
    A local pointer to the remotely accessible \dest{} data object is returned
    when it can be accessed using memory loads and stores.  Otherwise, a null
    pointer is returned.
}

\apinotes{
    When calling \FUNC{shmem\_ptr}, \dest{} is the address of the referenced
    symmetric data object on the calling \ac{PE}.
}

\begin{apiexamples}

\apicexample
    {In the following \Cstd[11] example, \ac{PE} 0 uses the \FUNC{shmem\_ptr}
    routine to query a pointer and directly access the \VAR{dest} array on
    \ac{PE} 1:}
    {./example_code/shmem_ptr_example.c}
    {}

\end{apiexamples}

\end{apidefinition}
