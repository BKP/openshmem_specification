\apisummary{
    Returns a pointer to a data object on a specified \ac{PE}.
}

\begin{apidefinition}

\begin{Csynopsis}
void *shmem_ptr(const void *dest, int pe);
\end{Csynopsis}

\begin{Fsynopsis}
POINTER (PTR, POINTEE)
INTEGER pe
PTR = SHMEM_PTR(dest, pe)
\end{Fsynopsis}


\begin{apiarguments}
\apiargument{IN}{dest}{The symmetric data object to be referenced.}
\apiargument{IN}{pe}{An integer that indicates the \ac{PE} number on which \dest{} is to
		 be accessed.  If you are using \Fortran, it must be a  default
		 integer value.}
\end{apiarguments}

\apidescription{
    \FUNC{shmem\_ptr} returns an address that may be used to directly reference
    \dest{} on the specified \ac{PE}.  This address can be assigned to a pointer.
    After that, ordinary loads and stores to this remote address may be performed.
    
    When a sequence of loads (gets) and stores (puts) to a data object on a
    remote \ac{PE} does not match the access pattern provided in an \openshmem data
    transfer routine like \FUNC{shmem\_put32} or \FUNC{shmem\_real\_iget}, the
    \FUNC{shmem\_ptr} routine can provide an efficient means to accomplish the
    communication.
}

\apireturnvalues{
    The return value is a non-NULL address of the \dest{} data object when it is 
    accessible using memory loads and stores in addition to \openshmem{} operations.
    Otherwise, a NULL address is returned.
}

\apinotes{
    When calling \FUNC{shmem\_ptr}, \dest{} is the address of the referenced
    symmetric data object on the calling \ac{PE}.
}

\begin{apiexamples}

\apifexample
    { This  \Fortran  program calls \FUNC{shmem\_ptr} and then \ac{PE} 0 writes to
    the \VAR{BIGD} array on \ac{PE} 1: }
    {./example_code/shmem_ptr_example.f90 }
    {}
    
\apicexample
    {This is the equivalent program written in \Cstd:}
    {./example_code/shmem_ptr_example.c}
    {}

\end{apiexamples}

\end{apidefinition}
