\clearpage %%%%%%%%%%%%%%%%%%%%%%%%%%%%%%%%%%%%%%%%%%%%%%%%%%%%%%%%%%%%

\appendix

%defining pagestyle for annex
\pagestyle{fancy}
\fancyhf{}
\fancyhead[RE, LO]{\leftmark}
\fancyhead[RO, LE]{\thepage}
\fancyfoot[CE, CO]{\thepage}
\renewcommand{\headrulewidth}{0pt}




\chapter{Writing OpenSHMEM Programs}
\section*{Incorporating OpenSHMEM into Programs}\label{sec:writing_programs}

The following section describes how to write a ``Hello World" \openshmem program.
To write a ``Hello World" \openshmem program, the user must:

\begin{itemize}
\item Include the header file \HEADER{shmem.h} for \Cstd.
\item Add the initialization call \hyperref[subsec:shmem_init]{\FUNC{shmem\_init}}.
\item Use \openshmem calls to query the local \ac{PE} number
    (\hyperref[subsec:shmem_my_pe]{\FUNC{shmem\_my\_pe}}) and the total number
    of \acp{PE} (\hyperref[subsec:shmem_n_pes]{\FUNC{shmem\_n\_pes}}).
\item Add the finalization call \hyperref[subsec:shmem_finalize]{\FUNC{shmem\_finalize}}.
\end{itemize}

In \openshmem, the order in which lines appear in the output is not
deterministic because \acp{PE} execute asynchronously in parallel.

\SourceExample{example_code/hello-openshmem.c}{
  \label{openshmem-hello}
  ``Hello World'' example program in \Cstd
}

\ProgramOutput{example_code/hello-openshmem-c.output}{
  Possible ordering of expected output with 4 \acp{PE} from the
  program in Example~\ref{openshmem-hello}
}

\clearpage %%%%%%%%%%%%%%%%%%%%%%%%%%%%%%%%%%%%%%%%%%%%%%%%%%%%%%%%%%%%

Example~\ref{openshmem-hello-symmetric} shows a more complex
\openshmem program that illustrates the use of symmetric data objects.
Note the declaration of the \VAR{static short dest} array and its use as the
remote destination in \hyperref[subsec:shmem_put]{\FUNC{shmem\_put}}.

The \KEYWORD{static} keyword makes the \VAR{dest} array symmetric on all \acp{PE}.
Each \ac{PE} is able to transfer data to a remote \dest{} array by simply
specifying to an OpenSHMEM routine such as \hyperref[subsec:shmem_put]{\FUNC{shmem\_put}}
the local address of the symmetric data object that will receive the data.
This local address resolution aids programmability because the address of the
\dest{} need not be exchanged with the active side (\ac{PE} \CONST{0}) prior to
the \acf{RMA} routine.

Conversely, the declaration of the \VAR{short source} array is asymmetric
(local only).
The \source{} object does not need to be symmetric because \PUT{} handles the
references to the \VAR{source} array only on the active (local) side.

\SourceExample{example_code/writing_shmem_example.c}{
  \label{openshmem-hello-symmetric}
  Example program with symmetric data objects
}

\ProgramOutput{example_code/writing_shmem_example.output}{
  Possible ordering of expected output with 4~\acp{PE} from the
  program in Example~\ref{openshmem-hello-symmetric}
}

\chapter{Compiling and Running Programs}\label{sec:compiling}
The \openshmem Specification does not specify how
\openshmem programs are compiled, linked, and run. This section shows some
examples of how wrapper programs are utilized in the \openshmem Reference
Implementation to compile and launch programs.

\section{Compilation}
\subsection*{Programs written in \Cstd}

The \openshmem Reference Implementation provides a wrapper program, named
\textbf{oshcc}, to aid in the compilation of \Cstd programs.
The wrapper may be called as follows:

\begin{lstlisting}[]
oshcc <compiler options> -o myprogram myprogram.c
\end{lstlisting}
Where the $\langle\mbox{compiler options}\rangle$ are options understood by the
underlying \Cstd compiler called by \textbf{oshcc}.


\subsection*{Programs written in \Cpp}

The \openshmem Reference Implementation provides a wrapper program, named
\textbf{oshc++}, to aid in the compilation of \Cpp programs.
The wrapper may be called as follows:

\begin{lstlisting}[]
oshc++ <compiler options> -o myprogram myprogram.cpp
\end{lstlisting}
Where the $\langle\mbox{compiler options}\rangle$ are options understood by the
underlying \Cpp compiler called by \textbf{oshc++}.


\section{Running Programs}

The \openshmem Reference Implementation provides a wrapper program, named
\textbf{oshrun}, to launch \openshmem programs.
The wrapper may be called as follows:

\begin{lstlisting}[]
oshrun <runner options> -np <#> <program> <program arguments>
\end{lstlisting}
The arguments for \textbf{oshrun} are:

\begin{tabular}{p{0.3\textwidth}p{0.6\textwidth}}
$\langle\mbox{runner options}\rangle$ & {Options passed to the underlying launcher.}\tabularnewline
-np $\langle\mbox{\#}\rangle$ & {The number of \acp{PE} to be used in the execution.}\tabularnewline
$\langle\mbox{program}\rangle$ & {The program executable to be launched.}\tabularnewline
$\langle\mbox{program arguments}\rangle$ & {Flags and other parameters to pass to the program.}\tabularnewline
\end{tabular}




\chapter{Undefined Behavior in OpenSHMEM}\label{sec:undefined}

The \openshmem Specification formalizes the expected behavior of
its library routines.  In cases where routines are improperly used
or the input is not in accordance with the Specification, the behavior
is undefined.

\begin{longtable}{|>{\raggedright}p{0.3\textwidth}|>{\raggedright}p{0.6\textwidth}|}
\hline
\textbf{Inappropriate Usage} & \textbf{Undefined Behavior}\tabularnewline
\hline
\endhead
Uninitialized library & If the \openshmem library is not initialized,
calls to non-initializing \openshmem routines have undefined
behavior.  For example, an implementation may try to continue or may abort
immediately upon an \openshmem call into the uninitialized library.
\tabularnewline
\hline
Multiple calls to initialization routines & In an \openshmem program where
the initialization routines \FUNC{shmem\_init} or \FUNC{shmem\_init\_thread}
have already been called, any subsequent calls to these initialization routines
result in undefined behavior.
\tabularnewline
\hline
Specifying invalid \ac{PE} numbers & For \openshmem routines that accept a
\ac{PE} number as an argument, if the \ac{PE} number is invalid for the
team associated with the operation (either implicitly or explicitly), the
behavior is undefined.  An invalid \ac{PE} number includes those that are
negative or greater than or equal to the size of the associated team.
\tabularnewline
\hline
Use of non-symmetric variables & Some routines require remotely accessible
variables to perform their function.  For example, a \PUT{} to a non-symmetric variable may
be trapped where possible and the library may abort the program.  Another
implementation may choose to continue execution with or without a warning.
\tabularnewline
\hline
Non-symmetric allocation of symmetric memory & The symmetric memory management routines are
collectives. For example, all \acp{PE} in the program must call
\FUNC{shmem\_malloc} with the same \VAR{size} argument.  Program behavior after a
mismatched \FUNC{shmem\_malloc} call is undefined.\tabularnewline
\hline
Use of null pointers with non-zero \VAR{len} specified & In any \openshmem routine
that takes a pointer and \VAR{len} describing the number of elements in that
pointer, a null pointer may not be given unless the corresponding \VAR{len} is also
specified as zero. Otherwise, the resulting behavior is undefined.
The following cases summarize this behavior:
\begin{itemize}
    \item \VAR{len} is 0, pointer is null: supported.
    \item \VAR{len} is not 0, pointer is null: undefined behavior.
    \item \VAR{len} is 0, pointer is non-null: supported.
    \item \VAR{len} is not 0, pointer is non-null: supported.
\end{itemize}
\tabularnewline
\hline
Concurrent use of a team & Teams are not thread-safe objects.
Concurrent use of a team from multiple threads results in undefined
behavior.  Such a situation can arise when one thread is calling a
team-implicit collective (e.g., \FUNC{shmem\_barrier\_all}), which
implicitly operates on the world team, and another calls a team-based
collective (e.g., \FUNC{shmem\_broadcastmem}). \tabularnewline
\hline
Destroying a team with unfreed private contexts & Before destroying a given
team, the user is responsible for destroying all contexts created from that team
with the \LibConstRef{SHMEM\_CTX\_PRIVATE} option enabled; otherwise, the
behavior is undefined.\tabularnewline
\hline
\end{longtable}


\chapter{Interoperability with Other Programming Models}\label{sec:interoperability}

OpenSHMEM routines may be used in conjunction with the routines of other
communication libraries or parallel languages in the same program. This section
describes the interoperability with other programming models, including
clarification of undefined behaviors caused by mixed use of different models,
and advice to \openshmem library users and developers that may improve the portability
and performance of hybrid programs.


\section{\ac{MPI} Interoperability}

\openshmem and \ac{MPI} are two commonly used parallel programming models for
distributed-memory systems. The user can choose to utilize both models in the same program
to efficiently and easily support various communication patterns.

A vendor may implement the \openshmem and \ac{MPI} libraries in different ways. For
instance, one may implement both \openshmem and \ac{MPI} as standalone libraries,
each of which allocates and initializes fully isolated communication
resources.
Another approach
is to implement both \openshmem and \ac{MPI} interfaces within the
same software system in order to share a communication resource when possible.

To improve interoperability and portability in \openshmem + \ac{MPI} hybrid
programming, we clarify the relevant semantics in the following subsections.


\subsection{Initialization}
In order to ensure that a hybrid program can be portably performed with different vendor
implementations, the \openshmem environment of the program must be initialized by
a call to \FUNC{shmem\_init} or \FUNC{shmem\_init\_thread} and be finalized by
a call to \FUNC{shmem\_finalize}; the \ac{MPI} environment of the program must be initialized
by a call to \FUNC{MPI\_Init} or \FUNC{MPI\_Init\_thread} and be finalized by a
call to \FUNC{MPI\_Finalize}.

\apiimpnotes{
Portable implementations of OpenSHMEM and \ac{MPI} must ensure that the initialization
calls can be made in an arbitrary order within a program; the same rule also
applies to the finalization calls. A software runtime that utilizes a shared
communication resource for \openshmem and \ac{MPI} communication may maintain an
internal reference counter in order to ensure that the shared resource is
initialized only once and thus no shared resource is released until the last
finalization call is made.
}


\subsection{Dynamic Process Creation}
\label{subsec:interoperability:mpmd}

\ac{MPI} defines a dynamic process model that allows creation of processes after
an \ac{MPI} application has started (e.g., by calling \FUNC{MPI\_Comm\_spawn}) and
connection to independent processes (e.g., through \FUNC{MPI\_Comm\_accept}
and \FUNC{MPI\_Comm\_connect}).
It provides a mechanism to establish communication
between the newly created processes and the existing \ac{MPI} application (see
\ac{MPI} standard version 3.1, Chapter 10).
Unlike \ac{MPI}, \openshmem starts all processes at once and requires all \acp{PE} to
collectively allocate and initialize resources (e.g., symmetric heap) used by
the \openshmem library before any other \openshmem routine may
be called. \openshmem does not support communication with dynamically created
or connected processes. In such a scenario, \ac{MPI} can be used to communicate
with these processes.


\subsection{Thread Safety}
\label{subsec:interoperability:thread}
Both \openshmem and \ac{MPI} define the interaction with user threads in a program
with routines that can be used for initializing and querying the thread
environment. A hybrid program may request different thread levels
at the initialization calls of \openshmem and \ac{MPI} environments; however, the
returned support level provided by the \openshmem or \ac{MPI} library might be different
from that returned in an non-hybrid program. For instance, the former
initialization call in a hybrid program may initialize a resource with the
requested thread level, but the supported level cannot be updated by a subsequent
initialization call if the underlying software runtime of \openshmem and \ac{MPI}
share the same internal communication resource.
The program should always check the \VAR{provided} thread level returned
at the corresponding initialization call or query the level of thread support
after initialization to portably ensure thread support in each communication
environment.

Both \openshmem and \ac{MPI} define similar thread levels, namely, \VAR{THREAD\_SINGLE},
\VAR{THREAD\_FUNNELED}, \VAR{THREAD\_SERIALIZED}, and \VAR{THREAD\_MULTIPLE}.
When requesting threading support in a hybrid program, however,
the following additional rules are applied if the implementations of \openshmem
and \ac{MPI} share the same internal communication resource.
It is strongly recommended to always follow these rules to ensure program
portability.

\begin{itemize}
    \item The \VAR{THREAD\_SINGLE} thread level requires a single-threaded program.
    Hence, a hybrid program should not request \VAR{THREAD\_SINGLE} at the initialization
    call of either \openshmem or \ac{MPI} but request a different thread level at the
    initialization call of the other model.

    \item The \VAR{THREAD\_FUNNELED} thread level allows only the main thread to
    make communication calls. A hybrid program using the \VAR{THREAD\_FUNNELED}
    thread level in both \openshmem and \ac{MPI} should ensure that the same main thread
    is used in both communication environments.

    \item The \VAR{THREAD\_SERIALIZED} thread level requires the program to ensure
    that communication calls are not made concurrently by multiple threads. If a
    hybrid program uses \VAR{THREAD\_SERIALIZED} in one communication environment
    and \VAR{THREAD\_SERIALIZED} or \VAR{THREAD\_FUNNELED} in the other one, it
    should also guarantee that the \openshmem and \ac{MPI} calls are not made concurrently
    from two distinct threads.
\end{itemize}

\subsection{Mapping Process Identification Numbers}
\label{subsec:interoperability:id}

Similar to the \ac{PE} number in \openshmem, \ac{MPI} defines rank as the
identification number of a process in a communicator. Both the \openshmem \ac{PE}
and the \ac{MPI} rank are unique integers assigned from zero to one less than the total
number of processes. In a hybrid program, the \openshmem
\ac{PE} number in \LibHandleRef{SHMEM\_TEAM\_WORLD}
and the \ac{MPI} rank in \VAR{MPI\_COMM\_WORLD} of a process can be equal.
This feature, however, may be provided by only some of the \openshmem and \ac{MPI}
implementations (e.g., if both environments share the same underlying process
manager) and is not portably guaranteed. A portable program should always
use the standard functions in each model, namely, \FUNC{shmem\_team\_my\_pe} in \openshmem
and \FUNC{MPI\_Comm\_rank} in \ac{MPI}, to query the process identification numbers
in each communication environment and manage the mapping of identifiers in the
program when necessary.

\subsubsection*{Examples}
\label{subsubsec:interoperability:id:example}
The following example demonstrates how to manage the mapping between \openshmem
\ac{PE} numbers and \ac{MPI} ranks in \VAR{MPI\_COMM\_WORLD} in a hybrid \openshmem
and \ac{MPI} program.

\lstinputlisting[language={C}, tabsize=2,
      basicstyle=\ttfamily\footnotesize]
      {example_code/hybrid_mpi_mapping_id.c}

The following example demonstrates an alternative approach for managing the mapping
of process identification numbers in a hybrid program. The program creates a
new MPI communicator, named \VAR{shmem\_comm}, that contains all
processes in \VAR{MPI\_COMM\_WORLD} and uses the same \ac{MPI} rank and
\openshmem \ac{PE} numbering.

\lstinputlisting[language={C}, tabsize=2,
      basicstyle=\ttfamily\footnotesize]
      {example_code/hybrid_mpi_mapping_id_shmem_comm.c}

\subsection{RMA Programming Models}
\label{subsec:interoperability:rma}

\openshmem and \ac{MPI} each define similar one-sided communication models;
however, a portable program should not assume interoperability between these
models.
For instance, \openshmem guarantees the atomicity only of concurrent \openshmem AMO operations
that operate on symmetric data with the same datatype. Access to the same symmetric
object with \ac{MPI} atomic operations, such as an \FUNC{MPI\_Fetch\_and\_op}, may
result in an undefined result. A hybrid program should avoid situations where \ac{MPI} and
\openshmem one-sided operations perform concurrent accesses to the same memory
location; otherwise, the behavior is undefined.

\subsection{Communication Progress}
\label{subsec:interoperability:progress}

\openshmem promises the progression of communication both with and without
\openshmem calls and requires the software progress mechanism in the implementation
(e.g., a progress thread) when the hardware does not provide asynchronous communication
capabilities (see Section \ref{subsec:progress}).
In \ac{MPI}, however, a weak progress semantics is applied. That is,
an \ac{MPI} communication call is guaranteed only to complete in finite time. For
instance, an \FUNC{MPI\_Put} may be completed only when the remote process makes an \ac{MPI}
call that internally triggers the progress of \ac{MPI}, if the underlying hardware
does not support asynchronous communication. A hybrid program
should not assume that the \openshmem library also makes progress for \ac{MPI}.
It can explicitly manage the asynchronous communication of \ac{MPI} in
order to prevent any deadlock or performance degradation.


\chapter{History of OpenSHMEM}\label{sec:openshmem_history}

SHMEM has a long history as a parallel-programming model and has been
extensively used on a number of products since 1993, including the Cray T3D,
Cray X1E, Cray XT3 and XT4, \ac{SGI} Origin, \ac{SGI} Altix, Quadrics-based
clusters, and InfiniBand-based clusters.

\begin{itemize}
\item SHMEM Timeline
  \begin{itemize}
  \item Cray SHMEM
    \begin{itemize}
    \item SHMEM first introduced by Cray Research, Inc.\ in 1993 for Cray T3D
    \item Cray was acquired by \ac{SGI} in 1996
    \item Cray was acquired by Tera in 2000 (MTA)
    \item Platforms: Cray T3D, T3E, C90, J90, SV1, SV2, X1, X2, XE, XMT, XT
    \end{itemize}
  \item \ac{SGI} SHMEM
    \begin{itemize}
    \item \ac{SGI} acquired Cray Research, Inc.\ and SHMEM was integrated into
      \ac{SGI}'s Message Passing Toolkit (MPT)
    \item \ac{SGI} currently owns the rights to SHMEM and \openshmem
    \item Platforms: Origin, Altix 4700, Altix XE, ICE, UV
    \item \ac{SGI} was acquired by Rackable Systems in 2009
    \item \ac{SGI} and \ac{OSSS} signed a
      SHMEM trademark licensing agreement in 2010
    \item \ac{HPE} acquired {SGI} in 2016
    \end{itemize}
  \end{itemize}
\end{itemize}

A listing of \openshmem implementations can be found on
\url{http://www.openshmem.org/}.








\chapter{OpenSHMEM Specification and Deprecated API}\label{sec:dep_api}

\section{Overview}\label{subsec:dep_overview}
\TableIndex{Deprecated API}
For the \openshmem Specification, deprecation is the process of identifying
API that is supported but no longer recommended for use by users.
The deprecated API \textbf{must} be supported until clearly
indicated as otherwise by the Specification.
This chapter records the API or functionality that have been deprecated, the
version of the \openshmem Specification that effected the deprecation, and the
most recent version of the \openshmem Specification in which the feature was
supported before removal.

\begin{center}
\scriptsize
\begin{longtable}{|l|c|c|l|}
    \hline
    \textbf{Deprecated API}
    & \textbf{Deprecated Since}
    & \textbf{Last Version Supported}
    & \textbf{Replaced By} \\
    \hline
    \endhead
    %% Deprecated in 1.1
    Header Directory: \hyperref[subsec:dep_rationale:mpp]{\HEADER{mpp}} & 1.1 & Current & (none) \\ \hline
    %% Deprecated in 1.2
    \CorCpp: \hyperref[subsec:start_pes]{\FuncRef{start\_pes}} & 1.2 & Current & \hyperref[subsec:shmem_init]{\FUNC{shmem\_init}} \\ \hline
    \Fortran: \hyperref[subsec:start_pes]{\FuncRef{START\_PES}} & 1.2 & 1.4 & \hyperref[subsec:shmem_init]{\FUNC{SHMEM\_INIT}} \\ \hline
    \hyperref[subsec:start_pes]{Implicit finalization} & 1.2 & Current & \hyperref[subsec:shmem_finalize]{\FUNC{shmem\_finalize}} \\ \hline
    \CorCpp: \FuncRef{\_my\_pe} & 1.2 & Current & \hyperref[subsec:shmem_my_pe]{\FUNC{shmem\_my\_pe}} \\ \hline
    \CorCpp: \FuncRef{\_num\_pes} & 1.2 & Current & \hyperref[subsec:shmem_n_pes]{\FUNC{shmem\_n\_pes}} \\ \hline
    \Fortran: \FuncRef{MY\_PE} & 1.2 & 1.4 & \hyperref[subsec:shmem_my_pe]{\FUNC{SHMEM\_MY\_PE}} \\ \hline
    \Fortran: \FuncRef{NUM\_PES} & 1.2 & 1.4 & \hyperref[subsec:shmem_n_pes]{\FUNC{SHMEM\_N\_PES}} \\ \hline
    \CorCpp: \FuncRef{shmalloc} & 1.2 & Current & \hyperref[subsec:shfree]{\FUNC{shmem\_malloc}} \\ \hline
    \CorCpp: \FuncRef{shfree} & 1.2 & Current & \hyperref[subsec:shfree]{\FUNC{shmem\_free}} \\ \hline
    \CorCpp: \FuncRef{shrealloc} & 1.2 & Current & \hyperref[subsec:shfree]{\FUNC{shmem\_realloc}} \\ \hline
    \CorCpp: \FuncRef{shmemalign} & 1.2 & Current & \hyperref[subsec:shfree]{\FUNC{shmem\_align}} \\ \hline
    \Fortran: \FuncRef{SHMEM\_PUT} & 1.2 & 1.4 & \hyperref[subsec:shmem_put]{\FUNC{SHMEM\_PUT8} or \FUNC{SHMEM\_PUT64}} \\ \hline
    %% Deprecated in 1.3
    \minitab{
        \CorCpp: \FuncRef{shmem\_clear\_cache\_inv}
        \\ \CorCpp: \FuncRef{shmem\_clear\_cache\_line\_inv}
        \\ \CorCpp: \FuncRef{shmem\_set\_cache\_inv}
        \\ \CorCpp: \FuncRef{shmem\_set\_cache\_line\_inv}
        \\ \CorCpp: \FuncRef{shmem\_udcflush}
        \\ \CorCpp: \FuncRef{shmem\_udcflush\_line}
        } & 1.3 & 1.4 & (none) \\ \hline
    \minitab{
        \Fortran: \FuncRef{SHMEM\_CLEAR\_CACHE\_INV}
        %% Note: At the time of deprecation, the Fortran API did not specify
        %% SHMEM_CLEAR_CACHE_LINE_INV. While this omission is certainly an error,
        %% Fortran was removed in 1.5 so the omission was never corrected.
        \\ \Fortran: \FuncRef{SHMEM\_SET\_CACHE\_INV}
        \\ \Fortran: \FuncRef{SHMEM\_SET\_CACHE\_LINE\_INV}
        \\ \Fortran: \FuncRef{SHMEM\_UDCFLUSH}
        \\ \Fortran: \FuncRef{SHMEM\_UDCFLUSH\_LINE}
        } & 1.3 & 1.4 & (none) \\ \hline
    \LibConstRef{\_SHMEM\_SYNC\_VALUE}         & 1.3 & Current & \hyperref[subsec:library_constants]{\CONST{SHMEM\_SYNC\_VALUE}} \\ \hline
    \LibConstRef{\_SHMEM\_BARRIER\_SYNC\_SIZE} & 1.3 & Current & \hyperref[subsec:library_constants]{\CONST{SHMEM\_BARRIER\_SYNC\_SIZE}} \\ \hline
    \LibConstRef{\_SHMEM\_BCAST\_SYNC\_SIZE}   & 1.3 & Current & \hyperref[subsec:library_constants]{\CONST{SHMEM\_BCAST\_SYNC\_SIZE}} \\ \hline
    \LibConstRef{\_SHMEM\_COLLECT\_SYNC\_SIZE} & 1.3 & Current & \hyperref[subsec:library_constants]{\CONST{SHMEM\_COLLECT\_SYNC\_SIZE}} \\ \hline
    \LibConstRef{\_SHMEM\_REDUCE\_SYNC\_SIZE}  & 1.3 & Current & \hyperref[subsec:library_constants]{\CONST{SHMEM\_REDUCE\_SYNC\_SIZE}} \\ \hline
    \LibConstRef{\_SHMEM\_REDUCE\_MIN\_WRKDATA\_SIZE} & 1.3 & Current & \hyperref[subsec:library_constants]{\CONST{SHMEM\_REDUCE\_MIN\_WRKDATA\_SIZE}} \\ \hline
    \LibConstRef{\_SHMEM\_MAJOR\_VERSION} & 1.3 & Current & \hyperref[subsec:library_constants]{\CONST{SHMEM\_MAJOR\_VERSION}} \\ \hline
    \LibConstRef{\_SHMEM\_MINOR\_VERSION} & 1.3 & Current & \hyperref[subsec:library_constants]{\CONST{SHMEM\_MINOR\_VERSION}} \\ \hline
    \LibConstRef{\_SHMEM\_MAX\_NAME\_LEN} & 1.3 & Current & \hyperref[subsec:library_constants]{\CONST{SHMEM\_MAX\_NAME\_LEN}} \\ \hline
    \LibConstRef{\_SHMEM\_VENDOR\_STRING} & 1.3 & Current & \hyperref[subsec:library_constants]{\CONST{SHMEM\_VENDOR\_STRING}} \\ \hline
    \LibConstRef{\_SHMEM\_CMP\_EQ} & 1.3 & Current & \hyperref[subsec:library_constants]{\CONST{SHMEM\_CMP\_EQ}} \\ \hline
    \LibConstRef{\_SHMEM\_CMP\_NE} & 1.3 & Current & \hyperref[subsec:library_constants]{\CONST{SHMEM\_CMP\_NE}} \\ \hline
    \LibConstRef{\_SHMEM\_CMP\_LT} & 1.3 & Current & \hyperref[subsec:library_constants]{\CONST{SHMEM\_CMP\_LT}} \\ \hline
    \LibConstRef{\_SHMEM\_CMP\_LE} & 1.3 & Current & \hyperref[subsec:library_constants]{\CONST{SHMEM\_CMP\_LE}} \\ \hline
    \LibConstRef{\_SHMEM\_CMP\_GT} & 1.3 & Current & \hyperref[subsec:library_constants]{\CONST{SHMEM\_CMP\_GT}} \\ \hline
    \LibConstRef{\_SHMEM\_CMP\_GE} & 1.3 & Current & \hyperref[subsec:library_constants]{\CONST{SHMEM\_CMP\_GE}} \\ \hline
    %% Deprecated in 1.4
    \EnvVarRef{SMA\_VERSION}         & 1.4 & Current & \hyperref[subsec:environment_variables]{\ENVVAR{SHMEM\_VERSION}} \\ \hline
    \EnvVarRef{SMA\_INFO}            & 1.4 & Current & \hyperref[subsec:environment_variables]{\ENVVAR{SHMEM\_INFO}} \\ \hline
    \EnvVarRef{SMA\_SYMMETRIC\_SIZE} & 1.4 & Current & \hyperref[subsec:environment_variables]{\ENVVAR{SHMEM\_SYMMETRIC\_SIZE}} \\ \hline
    \EnvVarRef{SMA\_DEBUG}           & 1.4 & Current & \hyperref[subsec:environment_variables]{\ENVVAR{SHMEM\_DEBUG}} \\ \hline
    \minitab{\CorCpp: \FuncRef{shmem\_wait}
        \\ \CorCpp: \FuncRef{shmem\_\FuncParam{TYPENAME}\_wait}}
        & 1.4 & Current & See \textbf{Notes} for \hyperref[subsec:shmem_wait_until]{\FUNC{shmem\_wait\_until}} \\ \hline
    \CorCpp: \FuncRef{shmem\_wait\_until} & 1.4 & Current
        & \Cstd[11]: \hyperref[subsec:shmem_wait_until]{\FUNC{shmem\_wait\_until}}, \CorCpp: \hyperref[subsec:shmem_wait_until]{\FUNC{shmem\_long\_wait\_until}} \\ \hline
    \minitab{\Cstd[11]: \FuncRef{shmem\_fetch}
        \\ \CorCpp: \FuncRef{shmem\_\FuncParam{TYPENAME}\_fetch}}
        & 1.4 & Current & \hyperref[subsec:shmem_atomic_fetch]{\FUNC{shmem\_atomic\_fetch}} \\ \hline
    \minitab{\Cstd[11]: \FuncRef{shmem\_set}
        \\ \CorCpp: \FuncRef{shmem\_\FuncParam{TYPENAME}\_set}}
        & 1.4 & Current & \hyperref[subsec:shmem_atomic_set]{\FUNC{shmem\_atomic\_set}} \\ \hline
    \minitab{\Cstd[11]: \FuncRef{shmem\_cswap}
        \\ \CorCpp: \FuncRef{shmem\_\FuncParam{TYPENAME}\_cswap}}
        & 1.4 & Current & \hyperref[subsec:shmem_atomic_compare_swap]{\FUNC{shmem\_atomic\_compare\_swap}} \\ \hline
    \minitab{\Cstd[11]: \FuncRef{shmem\_swap}
        \\ \CorCpp: \FuncRef{shmem\_\FuncParam{TYPENAME}\_swap}}
        & 1.4 & Current & \hyperref[subsec:shmem_atomic_swap]{\FUNC{shmem\_atomic\_swap}} \\ \hline
    \minitab{\Cstd[11]: \FuncRef{shmem\_finc}
        \\ \CorCpp: \FuncRef{shmem\_\FuncParam{TYPENAME}\_finc}}
        & 1.4 & Current & \hyperref[subsec:shmem_atomic_fetch_inc]{\FUNC{shmem\_atomic\_fetch\_inc}} \\ \hline
    \minitab{\Cstd[11]: \FuncRef{shmem\_inc}
        \\ \CorCpp: \FuncRef{shmem\_\FuncParam{TYPENAME}\_inc}}
        & 1.4 & Current & \hyperref[subsec:shmem_atomic_inc]{\FUNC{shmem\_atomic\_inc}} \\ \hline
    \minitab{\Cstd[11]: \FuncRef{shmem\_fadd}
        \\ \CorCpp: \FuncRef{shmem\_\FuncParam{TYPENAME}\_fadd}}
        & 1.4 & Current & \hyperref[subsec:shmem_atomic_fetch_add]{\FUNC{shmem\_atomic\_fetch\_add}} \\ \hline
    \minitab{\Cstd[11]: \FuncRef{shmem\_add}
        \\ \CorCpp: \FuncRef{shmem\_\FuncParam{TYPENAME}\_add}}
        & 1.4 & Current & \hyperref[subsec:shmem_atomic_add]{\FUNC{shmem\_atomic\_add}} \\ \hline
    Entire \Fortran API & 1.4 & 1.4 & \openshmem \Cstd API through \Fortran--\Cstd interoperability \\ \hline
    %% Deprecated in 1.5
    \minitab{
        \LibConstRef{SHMEM\_SYNC\_VALUE}
        \\ \LibConstRef{SHMEM\_SYNC\_SIZE}
        \\ \LibConstRef{SHMEM\_BARRIER\_SYNC\_SIZE}
        \\ \LibConstRef{SHMEM\_ALLTOALL\_SYNC\_SIZE}
        \\ \LibConstRef{SHMEM\_ALLTOALLS\_SYNC\_SIZE}
        \\ \LibConstRef{SHMEM\_BCAST\_SYNC\_SIZE}
        \\ \LibConstRef{SHMEM\_COLLECT\_SYNC\_SIZE}
        \\ \LibConstRef{SHMEM\_REDUCE\_SYNC\_SIZE}
        \\ \LibConstRef{SHMEM\_REDUCE\_MIN\_WRKDATA\_SIZE}
    } & 1.5 & Current & Team-based collectives, \minitab{Section~\ref{subsec:team_collectives}}. \\ \hline
    \CorCpp: Active-set-based \FuncRef{shmem\_sync}
        & 1.5 & Current & Team-based \hyperref[subsec:shmem_sync]{\FUNC{shmem\_sync}} \\ \hline
    \CorCpp: \FuncRef{shmem\_alltoall\{32, 64\}} & 1.5 & Current &
    \hyperref[subsec:shmem_alltoall]{\FUNC{shmem\_alltoall}} \\ \hline
    \CorCpp: \FuncRef{shmem\_alltoalls\{32, 64\}} & 1.5 & Current &
    \hyperref[subsec:shmem_alltoalls]{\FUNC{shmem\_alltoalls}} \\ \hline
    \CorCpp: \FuncRef{shmem\_broadcast\{32, 64\}} & 1.5 & Current &
    \hyperref[subsec:shmem_broadcast]{\FUNC{shmem\_broadcast}} \\ \hline
    \CorCpp: \FuncRef{shmem\_collect\{32, 64\}} & 1.5 & Current &
    \hyperref[subsec:shmem_collect]{\FUNC{shmem\_collect}} \\ \hline
    \CorCpp: \FuncRef{shmem\_fcollect\{32, 64\}} & 1.5 & Current &
    \hyperref[subsec:shmem_collect]{\FUNC{shmem\_fcollect}} \\ \hline
    \CorCpp: \FuncRef{shmem\_\FuncParam{TYPENAME}\_and\_to\_all}
        & 1.5 & Current & \hyperref[subsec:shmem_and_reduce]{\FUNC{shmem\_and\_reduce}} \\ \hline
    \CorCpp: \FuncRef{shmem\_\FuncParam{TYPENAME}\_or\_to\_all}
        & 1.5 & Current & \hyperref[subsec:shmem_or_reduce]{\FUNC{shmem\_or\_reduce}} \\ \hline
    \CorCpp: \FuncRef{shmem\_\FuncParam{TYPENAME}\_xor\_to\_all}
        & 1.5 & Current & \hyperref[subsec:shmem_xor_reduce]{\FUNC{shmem\_xor\_reduce}} \\ \hline
    \CorCpp: \FuncRef{shmem\_\FuncParam{TYPENAME}\_max\_to\_all}
        & 1.5 & Current & \hyperref[subsec:shmem_max_reduce]{\FUNC{shmem\_max\_reduce}} \\ \hline
    \CorCpp: \FuncRef{shmem\_\FuncParam{TYPENAME}\_min\_to\_all}
        & 1.5 & Current & \hyperref[subsec:shmem_min_reduce]{\FUNC{shmem\_min\_reduce}} \\ \hline
    \CorCpp: \FuncRef{shmem\_\FuncParam{TYPENAME}\_sum\_to\_all}
        & 1.5 & Current & \hyperref[subsec:shmem_sum_reduce]{\FUNC{shmem\_sum\_reduce}} \\ \hline
    \CorCpp: \FuncRef{shmem\_\FuncParam{TYPENAME}\_prod\_to\_all}
        & 1.5 & Current & \hyperref[subsec:shmem_prod_reduce]{\FUNC{shmem\_prod\_reduce}} \\ \hline
    \CorCpp: \hyperref[subsec:shmem_barrier]{\FuncRef{shmem\_barrier}}
        & 1.5 & Current & \hyperref[subsec:shmem_quiet]{\FuncRef{shmem\_quiet}} + \hyperref[subsec:shmem_sync]{\FuncRef{shmem\_sync}} \\ \hline
    \minitab{\Cstd[11]: \FuncRef{shmem\_wait\_until(\CTYPE{short} ...)}
        \\ \CorCpp: \FuncRef{shmem\_short\_wait\_until}}
        & 1.5 & Current & (none) \\ \hline
    \minitab{\Cstd[11]: \FuncRef{shmem\_wait\_until(\CTYPE{unsigned short} ...)}
        \\ \CorCpp: \FuncRef{shmem\_ushort\_wait\_until}}
        & 1.5 & Current & (none) \\ \hline
    \minitab{\Cstd[11]: \FuncRef{shmem\_test(\CTYPE{short} ...)}
        \\ \CorCpp: \FuncRef{shmem\_short\_test}}
        & 1.5 & Current & (none) \\ \hline
    \minitab{\Cstd[11]: \FuncRef{shmem\_test(\CTYPE{unsigned short} ...)}
        \\ \CorCpp: \FuncRef{shmem\_ushort\_test}}
        & 1.5 & Current & (none) \\ \hline
    \minitab{Table~\ref{p2psynctypes}: point-to-point synchronization types}
        & 1.5 & Current & Table~\ref{stdamotypes}: standard AMO types \\ \hline
    %% Deprecated in 1.6
    %% Deprecated in 1.7
    %% Notes
    %% - If a hyperref spans more than one line vertically, the clickable box
    %%   will also span more than one line. To prevent this, wrap the hyperref
    %%   in a minitab. Example in 1.5: "Team-based collectives".
    \end{longtable}
\end{center}

\section{Deprecation Rationale}\label{subsec:dep_rationale}

\subsection{Header Directory: \HEADER{mpp}}
\label{subsec:dep_rationale:mpp}
In addition to the default system header paths, \openshmem implementations
must provide all \openshmem-specified header files from the \HEADER{mpp}
header directory such that these headers can be referenced in \CorCpp as
\begin{lstlisting}[language=]
#include <mpp/shmem.h>
#include <mpp/shmemx.h>
\end{lstlisting}
and in \Fortran as
\begin{lstlisting}[language=]
include 'mpp/shmem.fh'
include 'mpp/shmemx.fh'
\end{lstlisting}
for backwards compatibility with \ac{SGI} SHMEM.

\subsection{\CorCpp: \FUNC{start\_pes}}
The \CorCpp routine \FUNC{start\_pes} includes an unnecessary initialization
argument that is remnant of historical \emph{SHMEM} implementations and no
longer reflects the requirements of modern \openshmem implementations.
Furthermore, the naming of \FUNC{start\_pes} does not include the standardized
\shmemprefixLC{} naming prefix. This routine has been deprecated and
\openshmem users are encouraged to use \FUNC{shmem\_init} instead.

\subsection{Implicit Finalization}
Implicit finalization was deprecated and replaced with explicit finalization using the
\FUNC{shmem\_finalize} routine.  Explicit finalization improves portability and
also improves interoperability with profiling and debugging tools.

\subsection{\CorCpp: \FUNC{\_my\_pe}, \FUNC{\_num\_pes}, \FUNC{shmalloc},
    \FUNC{shfree}, \FUNC{shrealloc}, \FUNC{shmemalign}}
The \CorCpp routines \FUNC{\_my\_pe}, \FUNC{\_num\_pes}, \FUNC{shmalloc},
\FUNC{shfree}, \FUNC{shrealloc}, and \FUNC{shmemalign} were deprecated in order
to normalize the \openshmem \ac{API} to use \shmemprefixLC{} as the standard
prefix for all routines.

\subsection{\textit{Fortran}: \FUNC{START\_PES}, \FUNC{MY\_PE}, \FUNC{NUM\_PES}} %% WARNING: Issue #66.
The \Fortran routines \FUNC{START\_PES}, \FUNC{MY\_PE}, and \FUNC{NUM\_PES}
were deprecated in order to minimize the API differences from the deprecation
of \CorCpp routines \FUNC{start\_pes}, \FUNC{\_my\_pe}, and \FUNC{\_num\_pes}.

\subsection{\textit{Fortran}: \FUNC{SHMEM\_PUT}} %% WARNING: Issue #66.
The \Fortran routine \FUNC{SHMEM\_PUT} is defined only for the \Fortran
\ac{API} and is semantically identical to \Fortran routines
\FUNC{SHMEM\_PUT8} and \FUNC{SHMEM\_PUT64}.  Since \FUNC{SHMEM\_PUT8} and
\FUNC{SHMEM\_PUT64} have defined equivalents in the \CorCpp interface,
\FUNC{SHMEM\_PUT} is ambiguous and has been deprecated.

\subsection{SHMEM\_CACHE}
The \FUNC{SHMEM\_CACHE} \ac{API}
\begin{center}
\begin{tabular}{ll}
    \CorCpp: & \Fortran: \\
    \FUNC{shmem\_clear\_cache\_inv}     & \FUNC{SHMEM\_CLEAR\_CACHE\_INV} \\
    \FUNC{shmem\_set\_cache\_inv}       & \FUNC{SHMEM\_SET\_CACHE\_INV} \\
    \FUNC{shmem\_set\_cache\_line\_inv} & \FUNC{SHMEM\_SET\_CACHE\_LINE\_INV} \\
    \FUNC{shmem\_udcflush}              & \FUNC{SHMEM\_UDCFLUSH} \\
    \FUNC{shmem\_udcflush\_line}        & \FUNC{SHMEM\_UDCFLUSH\_LINE} \\
    \FUNC{shmem\_clear\_cache\_line\_inv} \\
\end{tabular}
\end{center}
was originally implemented for systems with cache-management instructions.
This API has largely gone unused on cache-coherent system architectures.
\FUNC{SHMEM\_CACHE} has been deprecated.

\subsection{\CONST{\_SHMEM\_*} Library Constants}
The library constants
\begin{center}
\begin{tabular}{ll}
    \CONST{\_SHMEM\_SYNC\_VALUE}         & \CONST{\_SHMEM\_MAX\_NAME\_LEN} \\
    \CONST{\_SHMEM\_BARRIER\_SYNC\_SIZE} & \CONST{\_SHMEM\_VENDOR\_STRING} \\
    \CONST{\_SHMEM\_BCAST\_SYNC\_SIZE}   & \CONST{\_SHMEM\_CMP\_EQ} \\
    \CONST{\_SHMEM\_COLLECT\_SYNC\_SIZE} & \CONST{\_SHMEM\_CMP\_NE} \\
    \CONST{\_SHMEM\_REDUCE\_SYNC\_SIZE}  & \CONST{\_SHMEM\_CMP\_LT} \\
    \CONST{\_SHMEM\_REDUCE\_MIN\_WRKDATA\_SIZE} & \CONST{\_SHMEM\_CMP\_LE} \\
    \CONST{\_SHMEM\_MAJOR\_VERSION}      & \CONST{\_SHMEM\_CMP\_GT} \\
    \CONST{\_SHMEM\_MINOR\_VERSION}      & \CONST{\_SHMEM\_CMP\_GE} \\
\end{tabular}
\end{center}
do not adhere to the \Cstd standard's reserved identifiers and the \Cpp
standard's reserved names.  These constants were deprecated and replaced
with corresponding constants of prefix \shmemprefix{} that adhere to \CorCpp{}
and \Fortran naming conventions.

\subsection{\ENVVAR{SMA\_*} Environment Variables}\label{subsec:deprecate-sma-env}
The environment variables \ENVVAR{SMA\_VERSION}, \ENVVAR{SMA\_INFO},
\ENVVAR{SMA\_SYMMETRIC\_SIZE}, and \ENVVAR{SMA\_DEBUG}
were deprecated in order to normalize the \openshmem \ac{API} to use
\shmemprefix{} as the standard prefix for all environment variables.

\subsection{\CorCpp: \FUNC{shmem\_wait}}
The \CorCpp interface for \FUNC{shmem\_wait} and \FUNC{shmem\_\FuncParam{TYPENAME}\_wait}
was identified as unintuitive with respect to
the comparison operation it performed.  As \FUNC{shmem\_wait} can be trivially
replaced by \FUNC{shmem\_wait\_until} where \VAR{cmp} is
\CONST{SHMEM\_CMP\_NE}, the \FUNC{shmem\_wait} interface was deprecated in
favor of \FUNC{shmem\_wait\_until}, which makes the comparison operation
explicit and better communicates the developer's intent.

\subsection{\CorCpp: \FUNC{shmem\_wait\_until}}
The \CTYPE{long}-typed \CorCpp routine \FUNC{shmem\_wait\_until} was deprecated
in favor of the \Cstd[11] type-generic interface of the same name or the
explicitly typed \CorCpp routine \FUNC{shmem\_long\_wait\_until}.

\subsection{\textit{C11} and \CorCpp: \FUNC{shmem\_fetch}, \FUNC{shmem\_set}, %% Issue #66.
    \FUNC{shmem\_cswap}, \FUNC{shmem\_swap}, \FUNC{shmem\_finc},
    \FUNC{shmem\_inc}, \FUNC{shmem\_fadd}, \FUNC{shmem\_add}}
The \Cstd[11] and \CorCpp interfaces for
\begin{center}
\begin{tabular}{ll}
    \Cstd[11]: & \CorCpp: \\
    \FUNC{shmem\_fetch} & \FUNC{shmem\_\FuncParam{TYPENAME}\_fetch} \\
    \FUNC{shmem\_set}   & \FUNC{shmem\_\FuncParam{TYPENAME}\_set}   \\
    \FUNC{shmem\_cswap} & \FUNC{shmem\_\FuncParam{TYPENAME}\_cswap} \\
    \FUNC{shmem\_swap}  & \FUNC{shmem\_\FuncParam{TYPENAME}\_swap}  \\
    \FUNC{shmem\_finc}  & \FUNC{shmem\_\FuncParam{TYPENAME}\_finc}  \\
    \FUNC{shmem\_inc}   & \FUNC{shmem\_\FuncParam{TYPENAME}\_inc}   \\
    \FUNC{shmem\_fadd}  & \FUNC{shmem\_\FuncParam{TYPENAME}\_fadd}  \\
    \FUNC{shmem\_add}   & \FUNC{shmem\_\FuncParam{TYPENAME}\_add}   \\
\end{tabular}
\end{center}
were deprecated and replaced with
similarly named interfaces within the \FUNC{shmem\_atomic\_*} namespace
in order to more clearly identify these calls as performing atomic operations.
In addition, the abbreviated names ``cswap'', ``finc'', and ``fadd'' were
expanded for clarity to ``compare\_swap'', ``fetch\_inc'', and ``fetch\_add''.

\subsection{\textit{Fortran} API}\label{subsec:deprecate-fortran} %% WARNING: Issue #66.
The entire \openshmem \Fortran API was deprecated in \openshmem[1.4] and
removed in \openshmem[1.5] because of a general lack of
use and a lack of conformance with legacy \Fortran standards. In lieu of an
extensive update of the \Fortran API, \Fortran users are encouraged to
leverage the \openshmem Specification's \Cstd API through the
\Fortran--\Cstd interoperability initially standardized by \Fortran[2003]%
\footnote{Formally, \Fortran[2003] is known as ISO/IEC~1539-1:2004(E).}.


\subsection{Active-set-based collectives}
With the addition of \openshmem teams, Section~\ref{subsec:team}, the previous
method for performing collective
operations has been superseded by a more readable, flexible method for
organizing and communicating between groups of \acp{PE}. All collective routines
which previously indicated subgroups of \acp{PE} with a list of
parameters to describe the subgroup composition (active set) should be phased
out in favor of using collective operations with a team parameter.

The library constants
\begin{center}
\begin{tabular}{ll}
    \LibConstRef{SHMEM\_SYNC\_VALUE}            & \LibConstRef{SHMEM\_BCAST\_SYNC\_SIZE} \\
    \LibConstRef{SHMEM\_SYNC\_SIZE}             & \LibConstRef{SHMEM\_COLLECT\_SYNC\_SIZE} \\
    \LibConstRef{SHMEM\_BARRIER\_SYNC\_SIZE}    & \LibConstRef{SHMEM\_REDUCE\_SYNC\_SIZE} \\
    \LibConstRef{SHMEM\_ALLTOALL\_SYNC\_SIZE}   & \LibConstRef{SHMEM\_REDUCE\_MIN\_WRKDATA\_SIZE} \\
    \LibConstRef{SHMEM\_ALLTOALLS\_SYNC\_SIZE} \\
\end{tabular}
\end{center}
were deprecated as these constants pertain only to active-set-based collectives.

The \CorCpp active-set-based \FuncRef{shmem\_sync} routine was deprecated and
replaced with the team-based \Cstd[11] \FuncRef{shmem\_sync} or \CorCpp
\FuncRef{shmem\_team\_sync} routine.

The fixed-sized versions of the active-set-based routines
\begin{center}
\begin{tabular}{ll}
    \FuncRef{shmem\_alltoall32} & \FuncRef{shmem\_alltoall64} \\
    \FuncRef{shmem\_alltoalls32} & \FuncRef{shmem\_alltoalls64} \\
    \FuncRef{shmem\_broadcast32} & \FuncRef{shmem\_broadcast64} \\
    \FuncRef{shmem\_collect32} & \FuncRef{shmem\_collect64} \\
    \FuncRef{shmem\_fcollect32} & \FuncRef{shmem\_fcollect64} \\
\end{tabular}
\end{center}
were deprecated. Instead, all team-based collective routines use standard
\Cstd types with the option to use generic \Cstd[11] functions for more portable
and maintainable implementations.

The active-set-based reduction routines
\begin{center}
\begin{tabular}{ll}
    \FuncRef{shmem\_\FuncParam{TYPENAME}\_and\_to\_all} & \FuncRef{shmem\_\FuncParam{TYPENAME}\_max\_to\_all} \\
    \FuncRef{shmem\_\FuncParam{TYPENAME}\_or\_to\_all}  & \FuncRef{shmem\_\FuncParam{TYPENAME}\_min\_to\_all} \\
    \FuncRef{shmem\_\FuncParam{TYPENAME}\_xor\_to\_all} & \FuncRef{shmem\_\FuncParam{TYPENAME}\_sum\_to\_all} \\
                                                        & \FuncRef{shmem\_\FuncParam{TYPENAME}\_prod\_to\_all} \\
\end{tabular}
\end{center}
were deprecated and replaced with team-based reduction routines.


\subsection{\CorCpp: \FUNC{shmem\_barrier}}
Each \openshmem team might
be associated with some number of communication contexts. The \FUNC{shmem\_barrier}
function implies that the default context is quiesced after synchronizing
some active set of \acp{PE}. Since teams may have some number of contexts associated
with the team, it becomes less clear which context would be the ``default'' context
for that particular team. Rather than continue to support \FUNC{shmem\_barrier}
for active-sets or teams, programs should use a call to \FUNC{shmem\_quiet}
followed by a call to \FUNC{shmem\_sync} in order to explicitly
indicate which context to quiesce.

\subsection{\textit{C11} and \CorCpp: \FUNC{shmem\_wait\_until} and \FUNC{shmem\_test} --- \CTYPE{short} and \CTYPE{unsigned short} variants}
The \CTYPE{short} and \CTYPE{unsigned short} type \CorCpp and \textit{C11}
routines for \FUNC{shmem\_wait\_until} and \FUNC{shmem\_test} were deprecated
because point-to-point synchronization routines are only compatible with
\acp{AMO} (as of \openshmem 1.5), and there is no corresponding AMO for
\CTYPE{short} and \CTYPE{unsigned short}.

\subsection{Table~\ref{p2psynctypes}: point-to-point synchronization types}
As of \openshmem 1.5, the point-to-point synchronization routines are only
compatible with \acp{AMO}, so their interfaces are defined via the
standard \ac{AMO} types in Table~\ref{stdamotypes}.

\chapter{Changes to this Document}\label{sec:changelog}

\section{Version 1.5}
Major changes in \openshmem[1.5] include the addition of new team-based
collective functions, \OPR{put-with-signal} functions, nonblocking \ac{AMO}
functions, multiple-element point-to-point synchronization and vector
comparison functions, a \FUNC{shmem\_malloc\_with\_hints} function, a profiling
interface, and the removal of the entire \Fortran API.

The following list describes the specific changes in \openshmem[1.5]:
\begin{itemize}
%
\item Removed \FUNC{SHMEM\_CACHE}.
%
\item Deprecated \CTYPE{short} and \CTYPE{unsigned short} variants for
\FUNC{shmem\_wait\_until} and \FUNC{shmem\_test}.
%
\item Added \FUNC{shmem\_malloc\_with\_hints} interface and corresponding hints
\CONST{SHMEM\_MALLOC\_ATOMICS\_REMOTE} and \CONST{SHMEM\_MALLOC\_SIGNAL\_REMOTE}.
\\ See Section \ref{subsec:shmmallochint} and \ref{subsec:library_constants}.
%
\item Specified that team-based broadcast operations update the \VAR{dest} object on
all \acp{PE}, including the root \ac{PE}.
\\ See Section \ref{subsec:shmem_broadcast}.
%
\item Deprecated active-set-based library constants and collective functions.
\\ See Sections \ref{subsec:library_constants} and \ref{subsec:coll}.
%
\item Added team management functions:
  \FUNC{shmem\_team\_my\_pe},
  \FUNC{shmem\_team\_n\_pes},
  \FUNC{shmem\_team\_get\_config}, \\
  \FUNC{shmem\_team\_translate\_pe},
  \FUNC{shmem\_team\_split\_strided},
  \FUNC{shmem\_team\_split\_2d}, and
  \FUNC{shmem\_team\_destroy}.
\\ See Sections
  \ref{subsec:shmem_team_my_pe},
  \ref{subsec:shmem_team_n_pes},
  \ref{subsec:shmem_team_get_config},
  \ref{subsec:shmem_team_translate_pe},
  \ref{subsec:shmem_team_split_strided},
  \ref{subsec:shmem_team_split_2d}, and
  \ref{subsec:shmem_team_destroy}.
%
\item Added team-based communication-management functions:
  \FUNC{shmem\_team\_create\_ctx} and
  \FUNC{shmem\_ctx\_get\_team}.
\\ See Sections
  \ref{subsec:shmem_team_create_ctx} and
  \ref{subsec:shmem_ctx_get_team}.
%
\item Added team-based collective functions: \FUNC{shmem\_sync},
  \FUNC{shmem\_alltoall[mem]}, \FUNC{shmem\_alltoalls[mem]}, \\
  \FUNC{shmem\_broadcast[mem]}, \FUNC{shmem\_collect[mem]}, \FUNC{shmem\_fcollect[mem]}, and \\
  \FUNC{shmem\_\{and, or, xor, max, min, sum, prod\}\_reduce}.
  \\ See Sections \ref{subsec:shmem_sync},
  \ref{subsec:shmem_alltoall}, \ref{subsec:shmem_alltoalls},
  \ref{subsec:shmem_broadcast}, \ref{subsec:shmem_collect},
  and \ref{subsec:shmem_reductions}.
%
\item Clarified interoperability of \openshmem with other programming models.
\\ See Annex~\ref{sec:interoperability}.
%
\item Clarified restrictions on using pointers to symmetric objects.
\\ See Sections
  \ref{subsec:pointers_to_symmetric_objects} and
  \ref{subsec:invoking_openshmem_operations}.
%
\item Added support for nonblocking \ac{AMO} functions.
\\ See Section \ref{sec:amo-nbi}.
%
\item Added support for blocking \OPR{put-with-signal} functions.
\\ See Section \ref{subsec:shmem_put_signal}.
%
\item Added support for nonblocking \OPR{put-with-signal} functions.
\\ See Section \ref{subsec:shmem_put_signal_nbi}.
%
\item Deprecated Table~\ref{p2psynctypes}: point-to-point synchronization types and names.
%
\item Clarified that point-to-point synchronization routines preserve the
  atomicity of OpenSHMEM \acp{AMO}.
\\ See Section~\ref{subsec:amo_guarantees}.
%
\item Clarified that symmetric variables used as \VAR{ivar} arguments to
  point-to-point synchronization routines must be updated using OpenSHMEM
  \acp{AMO}.
\\ See Section~\ref{subsec:p2p_intro}.
%
\item Removed the entire \openshmem \Fortran API. 
%
\item Added support for multipliers in \VAR{SHMEM\_SYMMETRIC\_SIZE}
environment variables.
\\ See Section \ref{subsec:environment_variables}.
%
\item Added support for a multiple-element point-to-point synchronization API with
  the functions: \FUNC{shmem\_wait\_until\_all}, \FUNC{shmem\_wait\_until\_any},
  \FUNC{shmem\_wait\_until\_some}, \FUNC{shmem\_test\_all},
  \FUNC{shmem\_test\_any}, and \FUNC{shmem\_test\_some}.
  \\See Sections \ref{subsec:shmem_wait_until_all},
  \ref{subsec:shmem_wait_until_any}, \ref{subsec:shmem_wait_until_some},
  \ref{subsec:shmem_test_all}, \ref{subsec:shmem_test_any}, and
  \ref{subsec:shmem_test_some}.
%
\item Added support for vectorized comparison values in the multiple-element
  point-to-point synchronization API with the functions:
  \FUNC{shmem\_wait\_until\_all\_vector}, \FUNC{shmem\_wait\_until\_any\_vector},
  \FUNC{shmem\_wait\_until\_some\_vector}, \\
  \FUNC{shmem\_test\_all\_vector}, \FUNC{shmem\_test\_any\_vector}, and
  \FUNC{shmem\_test\_some\_vector}.
  \\See Sections \ref{subsec:shmem_wait_until_all_vector},
  \ref{subsec:shmem_wait_until_any_vector}, \ref{subsec:shmem_wait_until_some_vector},
  \ref{subsec:shmem_test_all_vector}, \ref{subsec:shmem_test_any_vector}, and
  \ref{subsec:shmem_test_some_vector}.
%
\item Added \openshmem profiling interface.
  \\ See Section~\ref{sec:openshmem_profiling_interface}.
%
\item Specified the validity of communication contexts, added the constant
  \CONST{SHMEM\_CTX\_INVALID}, and clarified the behavior of
  \FUNC{shmem\_ctx\_*} routines on invalid contexts.
  \\ See Section~\ref{sec:ctx}.
%
\item Clarified \ac{PE} active set requirements.
    \\See Section~\ref{subsec:coll}.
%
\item Clarified that when the \VAR{size} argument is zero, symmetric heap
    allocation routines perform no action and return a null pointer; that
    symmetric heap management routines that perform no action do not perform a
    barrier; and that the \VAR{alignment} argument to \FUNC{shmem\_align} must
    be power of two multiple of \CONST{sizeof(void*)}.
    \\See Section~\ref{subsec:shfree}.
%
\item Clarified that the \openshmem lock API provides a non-reentrant mutex and
    that \FUNC{shmem\_clear\_lock} performs a quiet operation on the default
    context.
    \\See Section~\ref{subsec:shmem_lock}
%
\item Clarified the atomicity guarantees of the \openshmem memory model.
    \\See Section~\ref{subsec:amo_guarantees}.
%
\end{itemize}

\section{Version 1.4}
Major changes in \openshmem[1.4] include
multithreading support,
\emph{contexts} for communication management,
\FUNC{shmem\_sync},
\FUNC{shmem\_calloc},
expanded type support,
a new namespace for atomic operations,
atomic bitwise operations,
\FUNC{shmem\_test} for nonblocking point-to-point synchronization,
and \Cstd[11] type-generic interfaces for point-to-point synchronization.

The following list describes the specific changes in \openshmem[1.4]:
\begin{itemize}
%
\item New communication management API, including \FUNC{shmem\_ctx\_create};
    \FUNC{shmem\_ctx\_destroy}; and additional RMA, AMO, and memory ordering
    routines that accept \CTYPE{shmem\_ctx\_t} arguments.
\\See Section \ref{sec:ctx}.
%
\item New API \FUNC{shmem\_sync\_all} and \FUNC{shmem\_sync} to provide \ac{PE}
    synchronization without completing pending communication operations.
    \\See Sections \ref{subsec:shmem_sync_all} and \ref{subsec:shmem_sync}.
%
\item Clarified that the \openshmem extensions header files are required, even when empty.
\\See Section~\ref{subsec:bindings}.
%
\item Clarified that the \FUNC{SHMEM\_GET64} and \FUNC{SHMEM\_GET64\_NBI}
    routines are included in the \Fortran language bindings.\\
    See Sections \ref{subsec:shmem_get} and \ref{subsec:shmem_get_nbi}.
%
\item Clarified that \FUNC{shmem\_init} must be matched with a call to
    \FUNC{shmem\_finalize}.
\\See Sections \ref{subsec:shmem_init} and \ref{subsec:shmem_finalize}.
%
\item Added the \CONST{SHMEM\_SYNC\_SIZE} constant.
\\See Section \ref{subsec:library_constants}.
%
\item Added type-generic interfaces for \FUNC{shmem\_wait\_until}.
\\ See Section \ref{subsec:shmem_wait_until}.
%
\item Removed the \VAR{volatile} qualifiers from the \VAR{ivar} arguments to
\FUNC{shmem\_wait} routines and the \VAR{lock} arguments in the lock API.
\emph{Rationale: Volatile qualifiers were added to several API routines in
\openshmem[1.3]; however, they were later found to be unnecessary.}
\\ See Sections \ref{subsec:shmem_wait_until} and \ref{subsec:shmem_lock}.
%
\item Deprecated the \VAR{SMA\_}* environment variables and added equivalent
\VAR{SHMEM\_}* environment variables.
\\ See Section \ref{subsec:environment_variables}.
%
\item Added the \Cstd[11] \CTYPE{\_Noreturn} function specifier to
\FUNC{shmem\_global\_exit}.
\\ See Section \ref{subsec:shmem_global_exit}.
%
\item Clarified ordering semantics of memory ordering, point-to-point synchronization, and collective
synchronization routines.
%
\item Clarified deprecation overview and added deprecation rationale in Annex F.
\\See Section \ref{sec:dep_api}.
%
\item Deprecated header directory \HEADER{mpp}.
\\See Section \ref{sec:dep_api}.
%
\item Deprecated the \FUNC{shmem\_wait} functions and the \CTYPE{long}-typed \CorCpp \FUNC{shmem\_wait\_until} function.
\\ See Section \ref{subsec:p2p_intro}.
%
\item Added the \FUNC{shmem\_test} functions.
\\ See Section \ref{subsec:p2p_intro}.
%
\item Added the \FUNC{shmem\_calloc} function.
\\ See Section \ref{subsec:shmem_calloc}.
%
\item Introduced the thread safe semantics that define the interaction between
    \openshmem routines and user threads.
\\See Section \ref{subsec:thread_support}.
%
\item Added the new routine \FUNC{shmem\_init\_thread} to initialize the
    \openshmem library with one of the defined thread levels.
\\See Section \ref{subsec:shmem_init_thread}.
%
\item Added the new routine \FUNC{shmem\_query\_thread} to query the thread
    level provided by the \openshmem implementation.
\\See Section \ref{subsec:shmem_query_thread}.
%
\item Clarified the semantics of \FUNC{shmem\_quiet} for a multithreaded
    \openshmem \ac{PE}.
\\See Section \ref{subsec:shmem_quiet}
%
\item Revised the description of \FUNC{shmem\_barrier\_all} for a multithreaded
    \openshmem \ac{PE}.
\\See Section \ref{subsec:shmem_barrier_all}
%
\item Revised the description of \FUNC{shmem\_wait} for a multithreaded
    \openshmem \ac{PE}.
\\See Section \ref{subsec:shmem_wait_until}
%
\item Clarified description for \CONST{SHMEM\_VENDOR\_STRING}.
\\See Section \ref{subsec:library_constants}.
%
\item Clarified description for \CONST{SHMEM\_MAX\_NAME\_LEN}.
\\See Section \ref{subsec:library_constants}.
%
\item Clarified API description for \FUNC{shmem\_info\_get\_name}.
\\See Section \ref{subsec:shmem_info_get_name}.
%
\item Expanded the type support for RMA, AMO, and point-to-point
    synchronization operations.
\\ See Tables \ref{stdrmatypes}, \ref{stdamotypes}, \ref{extamotypes}, and
    \ref{p2psynctypes}
%
\item Renamed AMO operations to use \FUNC{shmem\_atomic\_*} prefix and
      deprecated old AMO routines.
\\ See Section \ref{sec:amo}.
%
\item Added fetching and non-fetching bitwise AND, OR, and XOR atomic
      operations.
\\ See Section \ref{sec:amo}.
%
\item Deprecated the entire \Fortran API.
%
\item Replaced the \CTYPE{complex} macro in complex-typed reductions with the
      \Cstd[99] (and later) type specifier \CTYPE{\_Complex} to remove an
      implicit dependence on \HEADER{complex.h}.
\\ See Section \ref{subsec:shmem_reductions}.
%
\item Clarified that complex-typed reductions in C are optionally supported.
\\ See Section \ref{subsec:shmem_reductions}.
%
\end{itemize}




\section{Version 1.3}
Major changes in \openshmem[1.3] include the addition of
nonblocking \ac{RMA} operations,
atomic \PUT{} and \GET{} operations,
all-to-all collectives,
and \Cstd[11] type-generic interfaces for \ac{RMA} and \ac{AMO} operations.

The following list describes the specific changes in \openshmem[1.3]:
\begin{itemize}
%
\item Clarified implementation of \acp{PE} as threads.
%
\item Added \CTYPE{const} to every read-only pointer argument.
%
\item Clarified definition of \OPR{Fence}.
\\See Section \ref{subsec:programming_model}.
%
\item Clarified implementation of symmetric memory allocation.
\\See Section \ref{subsec:memory_model}.
%
\item Restricted atomic operation guarantees to other atomic operations with the same datatype.
\\See Section \ref{subsec:amo_guarantees}.
%
\item Deprecation of all constants that start with \CONST{\_SHMEM\_*}.
\\See Section \ref{subsec:library_constants}.
%
\item Added a type-generic interface to \openshmem \ac{RMA} and \ac{AMO}
    operations based on \Cstd[11] Generics.
\\See Sections \ref{sec:rma}, \ref{sec:rma_nbi} and \ref{sec:amo}.
%
\item New nonblocking variants of remote memory access, \FUNC{SHMEM\_PUT\_NBI}
    and \FUNC{SHMEM\_GET\_NBI}.
\\See Sections \ref{subsec:shmem_put_nbi} and \ref{subsec:shmem_get_nbi}.
%
\item New atomic elemental read and write operations, \FUNC{SHMEM\_FETCH} and
    \FUNC{SHMEM\_SET}.
\\See Sections \ref{subsec:shmem_atomic_fetch} and \ref{subsec:shmem_atomic_set}
%
\item New alltoall data exchange operations, \FUNC{SHMEM\_ALLTOALL}
    and \FUNC{SHMEM\_ALLTOALLS}.
\\See Sections \ref{subsec:shmem_alltoall} and \ref{subsec:shmem_alltoalls}.
%
\item Added \CTYPE{volatile} to remotely accessible pointer argument in
    \FUNC{SHMEM\_WAIT} and \FUNC{SHMEM\_LOCK}.
\\See Sections \ref{subsec:shmem_wait_until} and \ref{subsec:shmem_lock}.
%
\item Deprecation of \FUNC{SHMEM\_CACHE}.
%
\end{itemize}




\section{Version 1.2}
Major changes in \openshmem[1.2] include
a new initialization routine (\FUNC{shmem\_init}),
improvements to the execution model with an explicit
library-finalization routine (\FUNC{shmem\_finalize}),
an early-exit routine (\FUNC{shmem\_global\_exit}),
namespace standardization,
and clarifications to several API descriptions.

The following list describes the specific changes in \openshmem[1.2]:
\begin{itemize}
%
\item Added specification of \VAR{pSync} initialization for all routines that use it.
%
\item Replaced all placeholder variable names \VAR{target} with \VAR{dest} to
      avoid confusion with \Fortran's \KEYWORD{target} keyword.
%
\item New Execution Model for exiting/finishing \openshmem programs.
\\See Section  \ref{subsec:execution_model}.
%
\item New library constants to support API that query version and name information.
\\See Section \ref{subsec:library_constants}.
%
\item New API \FUNC{shmem\_init} to provide mechanism to start an \openshmem
      program and replace deprecated \FUNC{start\_pes}.
\\See Section \ref{subsec:shmem_init}.
%
\item Deprecation of \FUNC{\_my\_pe} and \FUNC{\_num\_pes} routines.
\\See Sections \ref{subsec:shmem_my_pe} and \ref{subsec:shmem_n_pes}.
%
\item New API \FUNC{shmem\_finalize} to provide collective mechanism to cleanly
      exit an \openshmem program and release resources.
\\See Section \ref{subsec:shmem_finalize}.
%
\item New API \FUNC{shmem\_global\_exit} to provide mechanism to exit an
    \openshmem program.
\\See Section \ref{subsec:shmem_global_exit}.
%
\item Clarification related to the address of the referenced object in
    \FUNC{shmem\_ptr}.
\\See Section \ref{subsec:shmem_ptr}.
%
\item New API to query the version and name information.
\\See Section \ref{subsec:shmem_info_get_version} and \ref{subsec:shmem_info_get_name}.
%
\item \openshmem library API normalization. All \Cstd symmetric memory management
      API begins with  \FUNC{shmem\_}.
\\See Section \ref{subsec:shfree}.
%
\item Notes and clarifications added to \FUNC{shmem\_malloc}.
\\See Section \ref{subsec:shfree}.
%
\item Deprecation of \Fortran API routine \FUNC{SHMEM\_PUT}.
\\See Section \ref{subsec:shmem_put}.
%
\item Clarification related to \FUNC{shmem\_wait}.
\\See Section \ref{subsec:shmem_wait_until}.
%
\item Undefined behavior for null pointers without zero counts added.
\\See Annex \ref{sec:undefined}
%
\item Addition of new Annex for clearly specifying deprecated API and its
      support across versions of the \openshmem Specification.
\\See Annex \ref{sec:dep_api}.
%
\end{itemize}




\section{Version 1.1}
Major changes from \openshmem[1.0] to \openshmem[1.1] include
the introduction of the \HEADER{shmemx.h} header file for non-standard API
extensions,
clarifications to completion semantics and API descriptions in agreement with
the \ac{SGI} SHMEM specification,
and general readabilty and usability improvements to the document structure.

The following list describes the specific changes in \openshmem[1.1]:
\begin{itemize}
%
\item Clarifications of the completion semantics of memory synchronization
      interfaces.
\\See Section \ref{subsec:memory_order}.
%
\item Clarification of the completion semantics of memory load and store
      operations in context of \FUNC{shmem\_barrier\_all} and \FUNC{shmem\_barrier}
      routines.
\\See Section \ref{subsec:shmem_barrier_all} and \ref{subsec:shmem_barrier}.
%
\item Clarification of the completion and ordering semantics of
      \FUNC{shmem\_quiet} and \FUNC{shmem\_fence}.
\\See Section \ref{subsec:shmem_quiet} and \ref{subsec:shmem_fence}.
%
\item Clarifications of the completion semantics of \ac{RMA} and \ac{AMO}
      routines.
\\See Sections \ref{sec:rma} and \ref{sec:amo}
%
\item Clarifications of the memory model and the memory alignment requirements
      for symmetric data objects.
\\See Section \ref{subsec:memory_model}.
%
\item Clarification of the execution model and the definition of a \ac{PE}.
\\See Section \ref{subsec:execution_model}
%
\item Clarifications of the semantics of \FUNC{shmem\_pe\_accessible} and
      \FUNC{shmem\_addr\_accessible}.
\\See Section \ref{subsec:shmem_pe_accessible} and \ref{subsec:shmem_addr_accessible}.
%
\item Added an annex on interoperability with \ac{MPI}.
\\See Annex D.
%
\item Added examples to the different interfaces.
%
\item Clarification of the naming conventions for constant in \Cstd and
      \Fortran.
\\See Section \ref{subsec:library_constants} and \ref{subsec:shmem_wait_until}.
%
\item Added \ac{API} calls: \FUNC{shmem\_char\_p}, \FUNC{shmem\_char\_g}.
\\See Sections \ref{subsec:shmem_p} and \ref{subsec:shmem_g}.
%
\item Removed \ac{API} calls: \FUNC{shmem\_char\_put},
      \FUNC{shmem\_char\_get}.
\\See Sections \ref{subsec:shmem_put} and \ref{subsec:shmem_get}.
%
\item The usage of \CTYPE{ptrdiff\_t}, \CTYPE{size\_t}, and \CTYPE{int} in the
      interface signature was made consistent with the description.
\\See Sections \ref{subsec:coll}, \ref{subsec:shmem_iput}, and \ref{subsec:shmem_iget}.
%
\item Revised \FUNC{shmem\_barrier} example.
\\See Section \ref{subsec:shmem_barrier}.
%
\item Clarification of the initial value of \VAR{pSync} work arrays for
\FUNC{shmem\_barrier}.\\ See Section \ref{subsec:shmem_barrier}.
%
\item Clarification of the expected behavior when multiple \FUNC{start\_pes}
calls are encountered.
\\See Section \ref{subsec:start_pes}.
%
\item Corrected the definition of atomic increment operation.
\\See Section \ref{subsec:shmem_atomic_inc}.
%
\item Clarification of the size of the symmetric heap and when it is set.
\\See Section \ref{subsec:shfree}.
%
\item Clarification of the integer and real sizes for \Fortran \ac{API}.
\\See Sections \ref{subsec:shmem_atomic_add},
      \ref{subsec:shmem_atomic_compare_swap},
      \ref{subsec:shmem_atomic_swap},
      \ref{subsec:shmem_atomic_fetch_inc},
      \ref{subsec:shmem_atomic_inc}, and
      \ref{subsec:shmem_atomic_fetch_add}.
%
\item Clarification of the expected behavior on program \OPR{exit}.
\\See Section \ref{subsec:execution_model}, Execution Model.
%
\item More detailed description for the progress of \openshmem operations
provided.
\\See Section \ref{subsec:progress}.
%
\item Clarification of naming convention for non-standard interfaces and their
inclusion in \HEADER{shmemx.h}.
\\See Section \ref{subsec:bindings}.
%
\item Various fixes to \openshmem code examples across the Specification to
include appropriate header files.
%
\item Removing requirement that implementations should detect size mismatch and
return error information for \FUNC{shmalloc} and ensuring consistent
language.
\\See Sections \ref{subsec:shfree} and Annex \ref{sec:undefined}.
%
\item \Fortran programming fixes for examples.\\ See Sections
\ref{subsec:shmem_reductions} and \ref{subsec:shmem_wait_until}.
%
\item Clarifications of the reuse \VAR{pSync} and \VAR{pWork} across
collectives.
\\See Sections \ref{subsec:coll}, \ref{subsec:shmem_broadcast},
      \ref{subsec:shmem_collect} and \ref{subsec:shmem_reductions}.
%
\item Name changes for UV and ICE for \ac{SGI} systems.
\\See Annex \ref{sec:openshmem_history}.
%
\end{itemize}

%end of setlength command that was started in frontmatter.tex
