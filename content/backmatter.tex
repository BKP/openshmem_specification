\clearpage

\appendix

%defining pagestyle for annex
%\pagestyle{plain} \withlinenumbers
\pagestyle{fancy} \withlinenumbers
\fancyhf{}
\fancyhead[RE, LO]{\leftmark}
\fancyhead[RO, LE]{\thepage}
\fancyfoot[CE,CO]{\thepage}
\renewcommand{\headrulewidth}{0pt}




\chapter{Writing \openshmem Programs}
\section*{Incorporating \openshmem{} into Programs}\label{sec:writing_programs}

In this section, we describe how to write a ``Hello World" \openshmem program.
To write a ``Hello World" \openshmem program we need to: 

\begin{itemize}
\item Add the include file \HEADER{shmem.h} (for \Cstd) or \HEADER{shmem.fh} (for \Fortran).
\item Add the initialization call \FUNC{shmem\_init}, (line 9).
\item Use OpenSHMEM calls to query the the total number of PEs (line 10) and PE
    id (line 11).
\item There is no explicit finalize call; either a return from \texttt{main()}
    (line 13) or an explicit \texttt{exit()} acts as an implicit \openshmem
    finalization.
\item In \openshmem the order in which lines appear in the output is not fixed
    as \acp{PE} execute asynchronously in parallel.
\end{itemize}

\begin{minipage}{\linewidth}
\vspace{0.1in}
\numberedlisting{caption={``Hello World'' example program (C)},label=openshmem-hello,language=OSH2+C}
                {example_code/hello-openshmem.c}
\outputlisting{language=bash,caption={Expected output from the program in Listing~\ref{openshmem-hello} (4 processors)}}
                {example_code/hello-openshmem-c.output}
\vspace{0.1in}
\end{minipage}

\openshmem also has a \Fortran API, so for completeness we will now give the
same program written in \Fortran, in listing~\ref{openshmem-hello-f90}:

\begin{minipage}{\linewidth}
\vspace{0.1in}
\numberedlisting{caption={``Hello World'' example program (Fortran)},label=openshmem-hello-f90,language=OSH2+F}
                {example_code/hello-openshmem.f90}
\outputlisting{language=bash,caption={Expected output from the program in Listing~\ref{openshmem-hello-f90} (4 processors)}}
                {example_code/hello-openshmem-f90.output}
\vspace{0.1in}
\end{minipage}

The example in Listing~\ref{openshmem-hello-symmetric} shows a more complex \openshmem program that illustrates
the use of symmetric data objects.  Note the declaration of the  \VAR{static
short dest} array and its use as the remote destination in \openshmem short
\PUT.  The use of the \VAR{static} keyword results in the \VAR{dest} array being
symmetric on \ac{PE} \CONST{0} and \ac{PE} \CONST{1}.  Each \ac{PE} is able to
transfer data to the \dest{} array by simply specifying the local address of the
symmetric data object which is to receive the data.  This aids programmability,
as the address of the \dest{} need not be exchanged with the active side
(\ac{PE} \CONST{0}) prior to the RMA (Remote Memory Access) routine.
Conversely, the declaration of the \VAR{short source} array is asymmetric.
Because the \PUT{} handles the references to the \VAR{source} array only on the
active (local) side, the asymmetric \source{} object is handled correctly.

\begin{minipage}{\linewidth}
\vspace{0.1in}
\numberedlisting{caption={Symmetric data objects example program},label=openshmem-hello-symmetric,language=OSH2+C}
                {example_code/writing_shmem_example.c}
\outputlisting{language=bash,caption={Expected output from the program in Listing~\ref{openshmem-hello-symmetric} (4 processors)}}
                {example_code/writing_shmem_example.output}
\vspace{0.1in}
\end{minipage}




\chapter{Compiling and Running Programs}\label{sec:compiling}
As of this writing, the \openshmem{} specification is silent regarding how
\openshmem{} programs are compiled, linked and run. This section shows some
examples of how wrapper programs are utilized in the \openshmem{} Reference
Implementation to compile and launch programs.

\section{Compilation}
\subsection*{Programs written in \Cstd}

The \openshmem{} Reference Implementation provides a wrapper program named
\textbf{oshcc}, to aid in the compilation of \Cstd programs, the wrapper
could be called as follows:

\begin{lstlisting}[language=bash]
oshcc <compiler options> -o myprogram myprogram.c
\end{lstlisting}
Where the $\langle\mbox{compiler options}\rangle$ are options understood by the
underlying \Cstd compiler.


\subsection*{Programs written in \Cpp}

The  \openshmem{} Reference Implementation provides a wrapper program named
\textbf{oshCC}, to aid in the compilation of \Cpp programs, the wrapper could
be called as follows:

\begin{lstlisting}[language=bash]
oshCC <compiler options> -o myprogram myprogram.cpp
\end{lstlisting}
Where the $\langle\mbox{compiler options}\rangle$ are options understood by the
underlying \Cpp compiler called by \textbf{oshCC}.


\subsection*{Programs written in \Fortran}

The  \openshmem{} Reference Implementation provides a wrapper program named
\textbf{oshfort}, to aid in the compilation of \Fortran programs, the wrapper
could be called as follows:

\begin{lstlisting}[language=bash]
oshfort <compiler options> -o myprogram myprogram.f
\end{lstlisting}
Where the $\langle\mbox{compiler options}\rangle$ are options understood by the
underlying \Fortran compiler called by \textbf{oshfort}.

\section{Running Programs}

The  \openshmem{} Reference Implementation provides a wrapper program named
\textbf{oshrun}, to launch \openshmem programs, the wrapper could be called as
follows:

\begin{lstlisting}[language=bash]
oshrun <additional options> -np <#> <program> <program arguments>
\end{lstlisting}
The program arguments for \textbf{oshrun} are:

\begin{tabular}{p{0.3\textwidth}p{0.6\textwidth}}
$\langle\mbox{additional options}\rangle$ & {Options passed to the underlying launcher.}\tabularnewline
-np $\langle\mbox{\#}\rangle$ & {The number of \acp{PE} to be used in the execution.}\tabularnewline
$\langle\mbox{program}\rangle$ & {The program executable to be launched.}\tabularnewline
$\langle\mbox{program arguments}\rangle$ & {Flags and other parameters to pass to the program.}\tabularnewline
\end{tabular}





\chapter{Undefined Behavior in \openshmem}\label{sec:undefined}

The specification provides guidelines to the expected behavior of
various library routines.  In cases where routines are improperly used
or the input is not in accordance with the specification, undefined
behavior may be observed.  Depending on the implementation there are
many interpretations of undefined behavior. 

$\;$

$ $%
\begin{tabular}{|>{\raggedright}p{0.3\textwidth}|>{\raggedright}p{0.6\textwidth}|}
\hline 
\textbf{Inappropriate Usage} & \textbf{Undefined Behavior}\tabularnewline
\hline 
\hline 
Uninitialized library & If \openshmem is not initialized through a call to
\FUNC{shmem\_init}, subsequent accesses to \openshmem routines have undefined
results.  An implementation may choose, for example, to try to continue or abort
immediately upon the first call to an uninitialized routine.\tabularnewline
\hline 
Accessing non-existent \acp{PE} & If a communications routine accesses a
non-existent \ac{PE}, then the \openshmem library can choose to handle this
situation in an implementation-defined way.  For example, the library may issue
an error message saying that the \ac{PE} accessed is outside the range of
accessible \acp{PE}, or may exit without a warning.\tabularnewline
\hline 
Use of non-symmetric variables & Some routines require remotely accessible
variables to perform their function.  A \PUT{} to a non-symmetric variable can
be trapped where possible and the library can abort the program.  Another
implementation may choose to continue either with a warning or
silently.\tabularnewline
\hline 
Non-symmetric variables & The symmetric memory management routines are
collectives, which means that all \acp{PE} in the program must issue the same
\FUNC{shmem\_malloc} call with the same size request.  Program behavior after a
mismatched \FUNC{shmem\_malloc} call is undefined.\tabularnewline
\hline 
Use of NULL pointers with non-zero \VAR{len} specified & In any \openshmem routine
that takes a pointer and \VAR{len} describing the number of elements in that
pointer, NULL may not be specified for the pointer unless the corresponding \VAR{len} is also
specified as zero. Otherwise, the resulting behavior is undefined.
The following cases summarize this behavior:
\begin{itemize}
    \item \VAR{len} is 0, pointer is NULL: supported.
    \item \VAR{len} is not 0, pointer is NULL: undefined behavior.
    \item \VAR{len} is 0, pointer is not NULL: supported.
    \item \VAR{len} is not 0, pointer is not NULL: supported.
\end{itemize}
\tabularnewline
\hline 
Multiple calls to \FUNC{shmem\_init} & In an OpenSHMEM program where
\FUNC{shmem\_init} has already be called, any subsequent calls to
\FUNC{shmem\_init} result in undefined behavior.\tabularnewline
\hline 
\end{tabular}







\chapter{Interoperability with other Programming Models}\label{sec:mpi}

\section{\ac{MPI} Interoperability}

\begin{sloppypar} % to prevent constants from running into margins.
%
\openshmem routines can be used in conjunction with \ac{MPI} routines  in the
same program.  For example, on SGI systems, programs that use both \ac{MPI} and
\openshmem routines call \FUNC{MPI\_Init} and \FUNC{MPI\_Finalize} but omit the
call to the \FUNC{shmem\_init} routine.  \openshmem \ac{PE} numbers are equal to
the \ac{MPI} rank within the \CONST{MPI\_COMM\_WORLD} environment variable.
Note that this precludes use of \openshmem routines between processes in
different \CONST{MPI\_COMM\_WORLD}s.  \ac{MPI} processes started using the
\FUNC{MPI\_Comm\_spawn} routine, for example, cannot use \openshmem routines to
communicate with their parent \ac{MPI} processes.
%
\end{sloppypar}
%
On SGI systems where \ac{MPI} jobs use TCP/sockets for inter-host communication,
\openshmem routines can be used to communicate with processes running on the
same host.  The \FUNC{shmem\_pe\_accessible} routine can be used to determine if
a remote \ac{PE} is accessible via \openshmem communication from the local
\ac{PE}. When running an \ac{MPI} program involving multiple executable files,
\openshmem routines can be used to communicate with processes running from the
same or different executable files, provided that the communication is limited
to symmetric data objects.  On these systems, static memory such as a
\Fortran common block or \Cstd global variable, is symmetric between
processes running from the same executable file, but is not symmetric between
processes running from different executable files.  Data allocated from the
symmetric heap (\FUNC{shmem\_malloc} or \FUNC{shpalloc}) is symmetric across the
same or different executable files. The routine \FUNC{shmem\_addr\_accessible}
can be used to determine if a local address is accessible via \openshmem
communication from a remote \ac{PE}.

Another important feature of these systems is that the
\FUNC{shmem\_pe\_accessible} routine returns \CONST{TRUE} only if the remote
\ac{PE} is a process running from the same executable file as the local PE,
indicating that full \openshmem support (static memory and symmetric heap) is
available.  When using \openshmem routines within an \ac{MPI} program, the use
of \ac{MPI} memory placement environment variables is required when using
non-default memory placement options.

\clearpage







\chapter{History of \openshmem}\label{sec:openshmem_history}

SHMEM has a long history as a parallel programming model, having been used
extensively on a number of products since 1993, including Cray T3D, Cray X1E,
the Cray XT3/4, SGI Origin, SGI Altix, clusters based on the Quadrics
interconnect, and to a very limited extent, Infiniband based clusters.

\begin{itemize}
\item A SHMEM Timeline
  \begin{itemize}
  \item Cray SHMEM
    \begin{itemize}
    \item SHMEM first introduced by Cray Research Inc. in 1993 for Cray T3D
    \item Cray is acquired by SGI in 1996
    \item Cray is acquired by Tera in 2000 (MTA)
    \item Platforms: Cray T3D, T3E, C90, J90, SV1, SV2, X1, X2, XE, XMT, XT
    \end{itemize}
  \item SGI SHMEM
    \begin{itemize}
    \item SGI purchases  Cray Research Inc. and SHMEM was integrated into
      SGI's Message Passing Toolkit (MPT)
    \item SGI currently owns the rights to SHMEM and \openshmem
    \item Platforms: Origin, Altix 4700, Altix XE, ICE, UV
    \item SGI was purchased by Rackable Systems in 2009
    \item SGI and Open Source Software Solutions, Inc. (OSSS) signed a
      SHMEM trademark licensing agreement, in 2010
    \end{itemize}
  \item Other Implementations
    \begin{itemize}
    \item Quadrics (Vega UK, Ltd.)
    \item Hewlett Packard
    \item GPSHMEM
    \item IBM
    \item QLogic
    \item Mellanox
    % \item University of Houston
    \item University of Florida
    \end{itemize}
  \end{itemize}
\item OpenSHMEM Implementations 
 \begin{itemize}
  \item SGI \openshmem
  \item University of Houston - \openshmem Reference Implementation
  \item Mellanox ScalableSHMEM
  \item Portals-SHMEM
  \item IBM OpenSHMEM
  \end{itemize}
\end{itemize}








\chapter{\openshmem Specification and Deprecated API}\label{sec:dep_api}

\section{Overview}\label{subsec:dep_overview}
For the \openshmem Specification(s), deprecation is the process of identifying
API that is supported but no longer recommended for use by program users. For
\openshmem library users, said API \textbf{must} be supported until clearly
indicated as otherwise by the Specification. In this chapter we will record the
API that has been deprecated, the \openshmem Specification that effected the
deprecation, and if the feature is supported in the current version of the
specification.  

\begin{center}
\scriptsize
\begin{tabular}{|l|c|c|c|}
    \hline
    \textbf{Deprecated API}
    & \textbf{Deprecated Since}
    & \shortstack{\textbf{Last Version Supported}}
    & \textbf{Replaced By} \\
    \hline
    \CorCpp: \hyperref[subsec:start_pes]{\FUNC{start\_pes}} & 1.2 & Current & \hyperref[subsec:shmem_init]{\FUNC{shmem\_init}} \\ \hline
    \Fortran: \hyperref[subsec:start_pes]{\FUNC{START\_PES}} & 1.2 & Current & \hyperref[subsec:shmem_init]{\FUNC{SHMEM\_INIT}} \\ \hline
    \CorCpp: \FUNC{\_my\_pe} & 1.2 & Current & \hyperref[subsec:shmem_my_pe]{\FUNC{shmem\_my\_pe}} \\ \hline
    \CorCpp: \FUNC{\_num\_pes} & 1.2 & Current & \hyperref[subsec:shmem_n_pes]{\FUNC{shmem\_n\_pes}} \\ \hline
    \Fortran: \FUNC{MY\_PE} & 1.2 & Current & \hyperref[subsec:shmem_my_pe]{\FUNC{SHMEM\_MY\_PE}} \\ \hline
    \Fortran: \FUNC{NUM\_PES} & 1.2 & Current & \hyperref[subsec:shmem_n_pes]{\FUNC{SHMEM\_N\_PES}} \\ \hline
    \CorCpp: \FUNC{shmalloc} & 1.2 & Current & \hyperref[subsec:shfree]{\FUNC{shmem\_malloc}} \\ \hline
    \CorCpp: \FUNC{shfree} & 1.2 & Current & \hyperref[subsec:shfree]{\FUNC{shmem\_free}} \\ \hline
    \CorCpp: \FUNC{shrealloc} & 1.2 & Current & \hyperref[subsec:shfree]{\FUNC{shmem\_realloc}} \\ \hline
    \CorCpp: \FUNC{shmemalign} & 1.2 & Current & \hyperref[subsec:shfree]{\FUNC{shmem\_align}} \\ \hline
    \Fortran: \FUNC{SHMEM\_PUT} & 1.2 & Current & \hyperref[subsec:shmem_put]{\FUNC{SHMEM\_PUT8} or \FUNC{SHMEM\_PUT64}} \\ \hline
    \shortstack[l]{\CorCpp: \hyperref[subsec:shmem_cache]{\FUNC{shmem\_clear\_cache\_inv}}
        \\ \Fortran: \hyperref[subsec:shmem_cache]{\FUNC{SHMEM\_CLEAR\_CACHE\_INV}}}
        & 1.3 & Current & (none) \\ \hline
    \CorCpp: \hyperref[subsec:shmem_cache]{\FUNC{shmem\_clear\_cache\_line\_inv}} & 1.3 & Current & (none) \\ \hline
    \shortstack[l]{\CorCpp: \hyperref[subsec:shmem_cache]{\FUNC{shmem\_set\_cache\_inv}}
        \\ \Fortran: \hyperref[subsec:shmem_cache]{\FUNC{SHMEM\_SET\_CACHE\_INV}}}
        & 1.3 & Current & (none) \\ \hline
    \shortstack[l]{\CorCpp: \hyperref[subsec:shmem_cache]{\FUNC{shmem\_set\_cache\_line\_inv}}
        \\ \Fortran: \hyperref[subsec:shmem_cache]{\FUNC{SHMEM\_SET\_CACHE\_LINE\_INV}}}
        & 1.3 & Current & (none) \\ \hline
    \shortstack[l]{\CorCpp: \hyperref[subsec:shmem_cache]{\FUNC{shmem\_udcflush}}
        \\ \Fortran: \hyperref[subsec:shmem_cache]{\FUNC{SHMEM\_UDCFLUSH}}}
        & 1.3 & Current & (none) \\ \hline
    \shortstack[l]{\CorCpp: \hyperref[subsec:shmem_cache]{\FUNC{shmem\_udcflush\_line}}
        \\ \Fortran: \hyperref[subsec:shmem_cache]{\FUNC{SHMEM\_UDCFLUSH\_LINE}}}
        & 1.3 & Current & (none) \\ \hline
    \_SHMEM\_SYNC\_VALUE & 1.3 & Current & \hyperref[subsec:library_constants]{SHMEM\_SYNC\_VALUE} \\ \hline
    \_SHMEM\_BARRIER\_SYNC\_SIZE & 1.3 & Current & \hyperref[subsec:library_constants]{SHMEM\_BARRIER\_SYNC\_SIZE} \\ \hline
    \_SHMEM\_BCAST\_SYNC\_SIZE & 1.3 & Current & \hyperref[subsec:library_constants]{SHMEM\_BCAST\_SYNC\_SIZE} \\ \hline
    \_SHMEM\_COLLECT\_SYNC\_SIZE & 1.3 & Current & \hyperref[subsec:library_constants]{SHMEM\_COLLECT\_SYNC\_SIZE} \\ \hline
    \_SHMEM\_REDUCE\_SYNC\_SIZE & 1.3 & Current & \hyperref[subsec:library_constants]{SHMEM\_REDUCE\_SYNC\_SIZE} \\ \hline
    \_SHMEM\_REDUCE\_MIN\_WRKDATA\_SIZE & 1.3 & Current & \hyperref[subsec:library_constants]{SHMEM\_REDUCE\_MIN\_WRKDATA\_SIZE} \\ \hline
    \_SHMEM\_MAJOR\_VERSION & 1.3 & Current & \hyperref[subsec:library_constants]{SHMEM\_MAJOR\_VERSION} \\ \hline
    \_SHMEM\_MINOR\_VERSION & 1.3 & Current & \hyperref[subsec:library_constants]{SHMEM\_MINOR\_VERSION} \\ \hline
    \_SHMEM\_MAX\_NAME\_LEN & 1.3 & Current & \hyperref[subsec:library_constants]{SHMEM\_MAX\_NAME\_LEN} \\ \hline
    \_SHMEM\_VENDOR\_STRING & 1.3 & Current & \hyperref[subsec:library_constants]{SHMEM\_VENDOR\_STRING} \\ \hline
    SMA\_VERSION         & 1.4 & Current & \hyperref[subsec:environment_variables]{SHMEM\_VERSION} \\ \hline
    SMA\_INFO            & 1.4 & Current & \hyperref[subsec:environment_variables]{SHMEM\_INFO} \\ \hline
    SMA\_SYMMETRIC\_SIZE & 1.4 & Current & \hyperref[subsec:environment_variables]{SHMEM\_SYMMETRIC\_SIZE} \\ \hline
    SMA\_DEBUG           & 1.4 & Current & \hyperref[subsec:environment_variables]{SHMEM\_DEBUG} \\ \hline
    \end{tabular}
\end{center}

\section{Deprecation Rationale}\label{subsec:dep_rationale}

\subsection{\CorCpp: start\_pes}
The \CorCpp routine \FUNC{start\_pes} includes an unnecessary initialization
argument that is remnant of historical \emph{SHMEM} implementations and no
longer reflects the requirements of modern \openshmem implementations.
Furthermore, the naming of \FUNC{start\_pes} does not include the standardized
\shmemprefixLC{} naming prefix. This routine has been deprecated and
\openshmem users are encouraged to use
\hyperref[subsec:shmem_init]{\FUNC{shmem\_init}} instead.

\subsection{\CorCpp: \_my\_pe, \_num\_pes, shmalloc, shfree, shrealloc, shmemalign}
The \CorCpp routines \FUNC{\_my\_pe}, \FUNC{\_num\_pes}, \FUNC{shmalloc},
\FUNC{shfree}, \FUNC{shrealloc} and \FUNC{shmemalign} were deprecated in order
to normalize the \openshmem \ac{API} to use \shmemprefixLC{} as the standard
prefix for all routines.

\subsection{\Fortran: START\_PES, MY\_PE, NUM\_PES}
The \Fortran routines \FUNC{START\_PES}, \FUNC{MY\_PE}, and \FUNC{NUM\_PES}
were deprecated in order to minimize the API differences from the deprecation
of \CorCpp routines \FUNC{start\_pes}, \FUNC{\_my\_pe}, and \FUNC{\_num\_pes}.

\subsection{\Fortran: SHMEM\_PUT}
The \Fortran function \FUNC{SHMEM\_PUT} is defined only for the \Fortran
\ac{API} and is semantically identical to \Fortran functions
\FUNC{SHMEM\_PUT8} and \FUNC{SHMEM\_PUT64}.  Since \FUNC{SHMEM\_PUT8} and
\FUNC{SHMEM\_PUT64} have defined equivalents in the \CorCpp interface,
\FUNC{SHMEM\_PUT} is ambiguous and has been deprecated.

\subsection{SHMEM\_CACHE}
The \FUNC{SHMEM\_CACHE} \ac{API}
\begin{center}
\begin{tabular}{ll}
    \CorCpp: & \Fortran: \\
    shmem\_clear\_cache\_inv & SHMEM\_CLEAR\_CACHE\_INV \\
    shmem\_set\_cache\_inv & SHMEM\_SET\_CACHE\_INV \\
    shmem\_set\_cache\_line\_inv & SHMEM\_SET\_CACHE\_LINE\_INV \\
    shmem\_udcflush & SHMEM\_UDCFLUSH \\
    shmem\_udcflush\_line & SHMEM\_UDCFLUSH\_LINE \\
    shmem\_clear\_cache\_line\_inv \\
\end{tabular}
\end{center}
was originally implemented for systems with cache management instructions.
This API has largely gone unused on cache-coherent system architectures.
\FUNC{SHMEM\_CACHE} has been deprecated.

\subsection{\_SHMEM\_* Library Constants}
The library constants
\begin{center}
\begin{tabular}{ll}
    \_SHMEM\_SYNC\_VALUE & \_SHMEM\_REDUCE\_MIN\_WRKDATA\_SIZE \\
    \_SHMEM\_BARRIER\_SYNC\_SIZE & \_SHMEM\_MAJOR\_VERSION \\
    \_SHMEM\_BCAST\_SYNC\_SIZE & \_SHMEM\_MINOR\_VERSION \\
    \_SHMEM\_COLLECT\_SYNC\_SIZE & \_SHMEM\_MAX\_NAME\_LEN \\
    \_SHMEM\_REDUCE\_SYNC\_SIZE & \_SHMEM\_VENDOR\_STRING \\
\end{tabular}
\end{center}
do not adhere to the \Cstd standard's reserved identifiers and the \Cpp
standard's reserved names.  These constants have been deprecated and replaced
with corresponding constants of prefix \shmemprefix{} that adhere to \CorCpp{}
and \Fortran naming conventions.

\subsection{SMA\_* Environment Variables}
The environment variables
\begin{center}
\begin{tabular}{ll}
    SMA\_VERSION \\
    SMA\_INFO \\
    SMA\_SYMMETRIC\_SIZE \\
    SMA\_DEBUG \\
\end{tabular}
\end{center}
were deprecated in order to normalize the \openshmem \ac{API} to use
\shmemprefix{} as the standard prefix for all environment variables.






\chapter{Changes to this Document}\label{sec:changelog}

\section{Version 1.4}

\begin{itemize}

\item \newtext{New API \FUNC{shmem\_sync\_all} and \FUNC{shmem\_sync} to provide PE
    synchronization without completing pending communication operations.
    \\See Sections \ref{subsec:shmem_sync_all} and \ref{subsec:shmem_sync}.}
%
\item Clarified that the \openshmem extensions header files are required, even when empty.
\\See Section~\ref{subsec:bindings}.
%
\item Clarified that the \FUNC{SHMEM\_GET64} and \FUNC{SHMEM\_GET64\_NBI}
    routines are included in the Fortran language bindings.\\
    See Sections \ref{subsec:shmem_get} and \ref{subsec:shmem_get_nbi}.
%
\item Added the \VAR{SHMEM\_SYNC\_SIZE} constant.
\\See Section \ref{subsec:library_constants}.
%
\item Added type-generic interfaces for \FUNC{SHMEM\_WAIT}.
\\ See Section \ref{subsec:shmem_wait}.
%
\item Removed the \VAR{volatile} qualifiers from the \VAR{ivar} arguments to
\FUNC{shmem\_wait} routines and the \VAR{lock} arguments in the lock API.
\emph{Rationale: Volatile qualifiers were added to several API routines in
version 1.3 of the OpenSHMEM specification; however, they were later found
to be unnecessary.}
\\ See Sections \ref{subsec:shmem_wait} and \ref{subsec:shmem_lock}.
%
\item Deprecated the \VAR{SMA\_}* environment variables and added equivalent
\VAR{SHMEM\_}* environment variables.
\\ See Section \ref{subsec:environment_variables}.
%
\item Added the \Cstd[11] \textbf{\_Noreturn} function specifier to
\FUNC{shmem\_global\_exit}.
\\ See Section \ref{subsec:shmem_global_exit}.
%
\item Clarified ordering semantics of memory ordering, point-to-point synchronization and collective 
synchronization routines.
%
\item Clarified deprecation overview and added deprecation rationale in Annex F.
\\See Section \ref{sec:dep_api}.
%
\item Added the \FUNC{shmem\_calloc} function.
\\ See Section \ref{subsec:shmem_calloc}.
%
\end{itemize}

\section{Version 1.3}
This section summarizes the changes from the \openshmem specification Version
1.2 to Version 1.3. Many major changes to the specification were introduced in Version 1.3. This includes non-blocking RMA operations, 
generic interfaces for various OpenSHMEM interfaces, atomic \FUNC{Put} and \FUNC{Get} operations,  and Alltoall interfaces. 


The following list describes the specific changes in 1.3:

\begin{itemize}
%
\item Clarified implementation of \acp{PE} as threads.
%
\item Added \textbf{const} to every read-only pointer argument.
%
\item Clarified definition of \OPR{Fence}.
\\See Section \ref{subsec:programming_model}.
%
\item Clarified implementation of symmetric memory allocation.
\\See Section \ref{subsec:memory_model}.
%
\item Restricted atomic operation guarantees to other atomic operations with the same datatype.
\\See Section \ref{subsec:amo_guarantees}.
%
\item Deprecation of all constants that start with \CONST{\_SHMEM\_*}.
\\See Section \ref{subsec:library_constants}.
%
\item Added a type-generic interface to \openshmem \ac{RMA} and \ac{AMO}
	operations based on \Cstd[11] Generics.
\\See Sections \ref{sec:rma}, \ref{sec:rma_nbi} and \ref{sec:amo}.
%
\item New non-blocking variants of remote memory access, \FUNC{SHMEM\_PUT\_NBI}
	and \FUNC{SHMEM\_GET\_NBI}.
\\See Sections \ref{subsec:shmem_put_nbi} and \ref{subsec:shmem_get_nbi}.
%
\item New atomic elemental read and write operations, \FUNC{SHMEM\_FETCH} and
	\FUNC{SHMEM\_SET}.
\\See Sections \ref{subsec:shmem_fetch} and \ref{subsec:shmem_set}
%
\item New alltoall data exchange operations, \FUNC{SHMEM\_ALLTOALL} 
	and \FUNC{SHMEM\_ALLTOALLS}.
\\See Sections \ref{subsec:shmem_alltoall} and \ref{subsec:shmem_alltoalls}.
%
\item Added \textbf{volatile} to remotely accessible pointer argument in 
	\FUNC{SHMEM\_WAIT} and \FUNC{SHMEM\_LOCK}.
\\See Sections \ref{subsec:shmem_wait} and \ref{subsec:shmem_lock}.
%
\item Deprecation of \FUNC{SHMEM\_CACHE}.
\\See Section \ref{subsec:shmem_cache}.
%
\end{itemize}




\section{Version 1.2}
This section summarizes the changes from the \openshmem specification Version
1.1 to Version 1.2.  A major change in this version is that it improves upon the
execution model described in 1.1 by introducing an explicit
\FUNC{shmem\_finalize} library call. This provides a collective mechanism of
exiting an \openshmem program and releasing resources used by the library.  


The following list describes the specific changes in 1.2:

\begin{itemize}
%
\item Added specification of \VAR{pSync} initialization for all routines that use it.
%
\item Replaced all placeholder variable names \VAR{target} with \VAR{dest} to
      avoid confusion with Fortran `target' keyword.
%
\item New Execution Model for exiting/finishing OpenSHMEM programs.
\\See Section  \ref{subsec:execution_model}.
%
\item New library constants to support API that query version and name information.
\\See Section \ref{subsec:library_constants}.
%
\item New API \FUNC{shmem\_init} to provide mechanism to start an \openshmem
      program and replace deprecated \FUNC{start\_pes}.
\\See Section \ref{subsec:shmem_init}.
%
\item Deprecation of \FUNC{\_my\_pe} and \FUNC{\_num\_pes} routines.
\\See Sections \ref{subsec:shmem_my_pe} and \ref{subsec:shmem_n_pes}.
%
\item New API \FUNC{shmem\_finalize} to provide collective mechanism to cleanly
      exit an \openshmem program and release resources.
\\See Section \ref{subsec:shmem_finalize}.
%
\item New API \FUNC{shmem\_global\_exit} to provide mechanism to exit an
    \openshmem program.
\\See Section \ref{subsec:shmem_global_exit}.
%
\item Clarification related to the address of the referenced object in
    \FUNC{shmem\_ptr}.
\\See Section \ref{subsec:shmem_ptr}.
%
\item New API to query the version and name information. 
\\See Section \ref{subsec:shmem_info_get_version} and \ref{subsec:shmem_info_get_name}.
%
\item \openshmem library API normalization. All \Cstd symmetric memory management
      API begins with  \FUNC{shmem\_}.
\\See Section \ref{subsec:shfree}.
%
\item Notes and clarifications added to \FUNC{shmem\_malloc}.
\\See Section \ref{subsec:shfree}.
%
\item Deprecation of Fortran API routine \FUNC{SHMEM\_PUT}.
\\See Section \ref{subsec:shmem_put}. 
%
\item Clarification related to \FUNC{shmem\_wait}.
\\See Section \ref{subsec:shmem_wait}.
%
\item Undefined behavior for null pointers without zero counts added.
\\See Annex \ref{sec:undefined}
%
\item Addition of new Annex for clearly specifying deprecated API and its
      support in the existing specification version.
\\See Annex \ref{sec:dep_api}.
%
\end{itemize}




\section{Version 1.1}
This section summarizes the changes from the \openshmem specification Version
1.0 to the Version 1.1.  A major change in this version is that it provides an
accurate description of \openshmem interfaces so that they are in agreement with
the SGI specification.  This version also explains \openshmem's programming,
memory, and execution model.  The document was thoroughly changed to improve the
readability of specification and usability of interfaces.  The code examples
were added to demonstrate the usability of API. Additionally, diagrams were
added to help understand the subtle semantic differences of various operations.


The following list describes the specific changes in 1.1:

\begin{itemize}
%
\item Clarifications of the completion semantics of memory synchronization 
      interfaces.
\\See Section \ref{subsec:memory_order}.
%
\item Clarification of the completion semantics of memory load and store
      operations in context of \FUNC{shmem\_barrier\_all} and \FUNC{shmem\_barrier}
      routines.
\\See Section \ref{subsec:shmem_barrier_all} and \ref{subsec:shmem_barrier}.
%
\item Clarification of the completion and ordering semantics of
      \FUNC{shmem\_quiet} and \FUNC{shmem\_fence}.
\\See Section \ref{subsec:shmem_quiet} and \ref{subsec:shmem_fence}.
%
\item Clarifications of the completion semantics of \ac{RMA} and \ac{AMO}
      routines.
\\See Sections \ref{sec:rma} and \ref{sec:amo}
%
\item Clarifications of the memory model and the memory alignment requirements
      for symmetric data objects.
\\See Section \ref{subsec:memory_model}.
%
\item Clarification of the execution model and the definition of a \ac{PE}.
\\See Section \ref{subsec:execution_model}
%
\item Clarifications of the semantics of \FUNC{shmem\_pe\_accessible} and
      \FUNC{shmem\_addr\_accessible}.
\\See Section \ref{subsec:shmem_pe_accessible} and \ref{subsec:shmem_addr_accessible}.
%
\item Added an annex on interoperability with \ac{MPI}.
\\See Annex \ref{sec:mpi}.
%
\item Added examples to the different interfaces.
%
\item Clarification of the naming conventions for constant in \Cstd and
      \Fortran.
\\See Section \ref{subsec:library_constants} and \ref{subsec:shmem_wait}.
%
\item Added \ac{API} calls: \FUNC{shmem\_char\_p}, \FUNC{shmem\_char\_g}.
\\See Sections \ref{subsec:shmem_p} and \ref{subsec:shmem_g}. 
%
\item Removed \ac{API} calls: \FUNC{shmem\_char\_put},
      \FUNC{shmem\_char\_get}.
\\See Sections \ref{subsec:shmem_put} and \ref{subsec:shmem_get}. 
%
\item The usage of \VAR{ptrdiff\_t}, \VAR{size\_t}, and \VAR{int} in the
      interface signature was made consistent with the description.
\\See Sections \ref{subsec:coll}, \ref{subsec:shmem_iput}, and \ref{subsec:shmem_iget}.
%
\item Revised \FUNC{shmem\_barrier} example.
\\See Section \ref{subsec:shmem_barrier}. 
%
\item Clarification of the initial value of \VAR{pSync} work arrays for
\FUNC{shmem\_barrier}.\\ See Section \ref{subsec:shmem_barrier}. 
%
\item Clarification of the expected behavior when multiple \FUNC{start\_pes}
calls are encountered has been clarified.
\\See Section \ref{subsec:start_pes}.
%
\item Corrected the definition of atomic increment operation.
\\See Section \ref{subsec:shmem_inc}. 
%
\item Clarification of the size of the symmetric heap and when it is set.
\\See Section \ref{subsec:shfree}.
%
\item Clarification of the integer and real sizes for \Fortran \ac{API}.
\\See Sections \ref{subsec:shmem_add}, \ref{subsec:shmem_cswap},
      \ref{subsec:shmem_swap}, \ref{subsec:shmem_finc}, \ref{subsec:shmem_inc}, and
      \ref{subsec:shmem_fadd}. 
%
\item Clarification of the expected behavior on program \OPR{exit}.
\\See Section \ref{subsec:execution_model}, Execution Model. 
%
\item More detailed description for the progress of \openshmem operations
provided.
\\See Section \ref{subsec:progress}. 
%
\item Clarification of naming convention for non-standard interfaces and their
inclusion in \HEADER{shmemx.h}.
\\See Section \ref{subsec:bindings}. 
%
\item Various fixes to \openshmem code examples across the specification to
include appropriate header files. 
%
\item Removing requirement that implementations should detect size mismatch and
return error information for \FUNC{shmalloc} and ensuring consistent
language.
\\See Sections \ref{subsec:shfree} and Annex \ref{sec:undefined}. 
%
\item Fortran programming fixes for examples.\\ See Sections
\ref{subsec:shmem_reductions} and \ref{subsec:shmem_wait}. 
%
\item Clarifications of the reuse \VAR{pSync} and \VAR{pWork} across
collectives.
\\See Sections \ref{subsec:coll}, \ref{subsec:shmem_broadcast},
      \ref{subsec:shmem_collect} and \ref{subsec:shmem_reductions}.
%
\item Name changes for UV and ICE for SGI systems.
\\See Annex \ref{sec:openshmem_history}. 
%
\end{itemize}





} %end of setlength command that was started in frontmatter.tex
