\openshmem implements \ac{PGAS} by defining remotely accessible data objects as
mechanisms to share information among \openshmem processes or \acp{PE} and
private data objects that are accessible by the \ac{PE} itself. The \ac{API}
allows communication and synchronization operations on both private (local to
the PE initiating the operation) and remotely accessible data objects. The key
feature of \openshmem is that data transfer operations are
\textit{\textbf{one-sided}} in nature. This means that a local \ac{PE} executing
a data transfer routine does not require the participation of the remote \ac{PE}
to complete the routine. This allows for overlap between communication and
computation to hide data transfer latencies, which makes  \openshmem ideal for
unstructured, small/medium size data communication patterns. The \openshmem
library routines have the potential to provide a low-latency, high-bandwidth
communication \ac{API} for use in highly parallelized scalable programs.  

The \openshmem interfaces can be used to implement \ac{SPMD} style programs.
It provides interfaces to start the \openshmem \acp{PE} in parallel, and
communication and synchronization interfaces to access remotely accessible data
objects across \acp{PE}. These interfaces can be leveraged to divide a problem
into multiple sub-problems that can be solved independently or with coordination
using the communication and synchronization interfaces.  The \openshmem
specification defines library calls, constants, variables, and language bindings
for \Cstd and \Fortran.%
\footnote{As of \openshmem[1.4], the \Fortran interface has been deprecated.}.
The \Cpp interface is currently the same as that
for \Cstd. Unlike Unified Parallel C, \Fortran[2008], Titanium, X10 and Chapel, which are all
PGAS languages, \openshmem relies on the user to use the library calls  to
implement the correct semantics of its programming model.

An overview of the \openshmem routines is described below:

\begin{enumerate}

\item \textbf{Library Setup and Query}
\begin{enumerate}
  \item \OPR{Initialization}: The \openshmem library environment is initialized. 
  \item \OPR{Query}: The local \ac{PE} may get the number of \acp{PE} running
      the same program and its unique integer identifier. 
  \item \OPR{Accessibility}: The local \ac{PE} can find out if a remote \ac{PE} is
      executing the same binary, or if a particular symmetric data object can be
      accessed by a remote \ac{PE}, or may obtain a pointer to a symmetric data
      object on the specified remote \ac{PE} on shared memory systems.
\end{enumerate}

\item \textbf{Symmetric Data Object Management}
\begin{enumerate}
  \item \OPR{Allocation}: All executing \acp{PE} must participate in the
      allocation of a symmetric data object with identical arguments.
  \item  \OPR{Deallocation}: All executing \acp{PE} must participate in the
      deallocation of the same symmetric data object with identical arguments.
  \item  \OPR{Reallocation}: All executing \acp{PE} must participate in the
      reallocation of the same symmetric data object with identical arguments.
\end{enumerate}

\item \textbf{Remote Memory Access}
\begin{enumerate}
    \item \PUT: The local \ac{PE} specifies the \source{} data object (private
        or symmetric) that is copied to the symmetric data object on the remote
        \ac{PE}. 
  \item \GET: The local \ac{PE} specifies the symmetric data object on the remote
      \ac{PE} that is copied to a data object (private or symmetric) on the local
      \ac{PE}. 
\end{enumerate}

\item \textbf{Atomics}
\begin{enumerate}
    \item \OPR{Swap}: The \ac{PE} initiating the swap gets the old value of a
        symmetric data object from a remote \ac{PE} and copies a new value to
        that symmetric data object on the remote \ac{PE}.
  \item \OPR{Increment}: The \ac{PE} initiating the increment adds 1 to the
      symmetric data object on the remote \ac{PE}.
  \item \OPR{Add}: The \ac{PE} initiating the add specifies the value to be added
      to the symmetric data object on the remote \ac{PE}.
  \item \OPR{Compare and Swap}: The \ac{PE} initiating the swap gets the old value
      of the symmetric data object based on a value to be compared and copies a
      new value to the symmetric data object on the remote \ac{PE}.
  \item \OPR{Fetch and Increment}: The \ac{PE} initiating the increment adds 1 to
      the symmetric data object on the remote \ac{PE} and returns with the old
      value.
  \item \OPR{Fetch and Add}: The \ac{PE} initiating the add specifies the value to
      be added to the symmetric data object on the remote \ac{PE} and returns with
      the old value.
\end{enumerate}

\item \textbf{Synchronization and Ordering}
\begin{enumerate}
  \item \OPR{Fence}: The \ac{PE} calling fence ensures ordering of   
  \PUT, AMO, and memory store operations
  to symmetric data objects with respect to a specific
      destination \ac{PE}. 
  \item \OPR{Quiet}: The \ac{PE} calling quiet ensures completion of remote access
      operations and stores to symmetric data objects. 
  \item \OPR{Barrier}: All or some \acp{PE} collectively synchronize and ensure
      completion of all remote and local updates prior to any \ac{PE} returning
      from the call.
\end{enumerate}

\item \textbf{Collective Communication}
\begin{enumerate}
  \item \OPR{Broadcast}: The \textit{root} \ac{PE} specifies a symmetric data
      object to be copied to a symmetric data object on one or more remote
      \acp{PE} (not including itself). 
  \item \OPR{Collection}: All \acp{PE} participating in the routine get the result
      of concatenated symmetric objects contributed by each of the \acp{PE} in
      another symmetric data object.
  \item \OPR{Reduction}: All \acp{PE} participating in the routine get the result
      of an associative binary routine over elements of the specified symmetric
      data object on another symmetric data object. 
\end{enumerate}

\item \textbf{Mutual Exclusion}
\begin{enumerate}
  \item \OPR{Set Lock}: The \ac{PE} acquires exclusive access to the region
      bounded by the symmetric \textit{lock} variable.
  \item \OPR{Test Lock}: The \ac{PE} tests the symmetric \textit{lock} variable
      for availability.
  \item \OPR{Clear Lock}: The \ac{PE} which has previously acquired the
      \textit{lock} releases it.
\end{enumerate}

\item \textbf{Data Cache Control \textit{(deprecated)}}
\begin{enumerate}
  \item Implementation of mechanisms to exploit the capabilities of hardware cache
      if available.
\end{enumerate}
\end{enumerate}
