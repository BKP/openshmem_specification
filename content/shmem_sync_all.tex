\color{ForestGreen}
\apisummary{
    Registers the arrival of a \ac{PE} at a barrier and suspends \ac{PE}
    execution until all other \acp{PE} arrive at the barrier.
}

\begin{apidefinition}

\begin{Csynopsis}
void shmem_sync_all(void);
\end{Csynopsis}

\begin{apiarguments}

    \apiargument{None.}{}{}

\end{apiarguments}

\apidescription{
    The \FUNC{shmem\_sync\_all} routine registers the arrival of a \ac{PE} at a
    barrier. Barriers are a fast mechanism for synchronizing all \acp{PE} at
    once.  This routine blocks the \ac{PE} until all \acp{PE} have called
    \FUNC{shmem\_sync\_all}. In a multithreaded \openshmem
    program, only the calling thread is blocked.

    In contrast with the \FUNC{shmem\_barrier\_all} routine,
    \FUNC{shmem\_sync\_all} only ensures completion and visibility of previously issued memory
    stores and does not ensure completion of remote memory updates issued via
    \openshmem routines.
}

\apireturnvalues{
    None.
}

\apinotes{
    The \FUNC{shmem\_sync\_all} routine can be used to portably ensure that
    memory access operations observe remote updates in the order enforced by the
    initiator \acp{PE}, provided that the initiator PE ensures completion of remote
    updates with a call to \FUNC{shmem\_quiet} prior to the call to the
    \FUNC{shmem\_sync\_all} routine.
}

\end{apidefinition}
\color{black}
