\apisummary{
    Registers the arrival of a \ac{PE} at a synchronization point and suspends
    execution until all other \acp{PE} in the default team arrive at a synchronization point. For multithreaded programs, execution is suspended
    as specified by the threading model (Section \ref{subsec:thread_support}).
}

\begin{apidefinition}

\begin{Csynopsis}
void @\FuncDecl{shmem\_sync\_all}@(void);
\end{Csynopsis}

\begin{apiarguments}

    \apiargument{None.}{}{}

\end{apiarguments}

\apidescription{

    This routine blocks the calling \ac{PE} until all \acp{PE} in the
    default team have called \FUNC{shmem\_sync\_all}.

    In a multithreaded \openshmem program, only the calling thread is
    blocked.

    In contrast with the \FUNC{shmem\_barrier\_all} routine,
    \FUNC{shmem\_sync\_all} only ensures completion and visibility of previously issued memory
    stores and does not ensure completion of remote memory updates issued via
    \openshmem routines.
}

\apireturnvalues{
    None.
}

\apinotes{
    The \FUNC{shmem\_sync\_all} routine is equivalent to calling
    \FUNC{shmem\_team\_sync} on the default team.
}

\end{apidefinition}
