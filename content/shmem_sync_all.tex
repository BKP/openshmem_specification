\apisummary{
    Registers the arrival of a \ac{PE} at a barrier and suspends \ac{PE}
    execution until all other \acp{PE} arrive at the barrier.
}

\begin{apidefinition}

\begin{Csynopsis}
void shmem_sync_all(void);
\end{Csynopsis}

\begin{apiarguments}

    \apiargument{None.}{}{}

\end{apiarguments}

\apidescription{
    The \FUNC{shmem\_sync\_all} routine registers the arrival of a \ac{PE} at a
    barrier. Barriers are a fast mechanism for synchronizing all \acp{PE} at
    once.  This routine causes a \ac{PE} to suspend execution until all
    \acp{PE} have called \FUNC{shmem\_sync\_all}.

    In contrast with the \FUNC{shmem\_barrier\_all} routine,
    \FUNC{shmem\_sync\_all} only ensures completion of previously issued memory
    stores and does not ensure completion of remote memory updates issued via
    \openshmem routines.
}

\apireturnvalues{
    None.
}

\apinotes{
    The \FUNC{shmem\_sync\_all} routine can be used to portably ensure that
    memory access operations observe remote updates in the order enforced by
    initiator PEs, provided that the initiator PE ensures completion of remote
    updates with a call to \FUNC{shmem\_quiet} prior to the call to
    \FUNC{shmem\_sync\_all} routine.
}

%\begin{apiexamples}
%
%\apicexample
%    { The following \FUNC{shmem\_barrier\_all} example is for \CorCpp programs:}
%    {./example_code/shmem_barrierall_example.c}
%    {}
%
%\end{apiexamples}

\end{apidefinition}
