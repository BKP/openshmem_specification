\apisummary{
    Performs an atomic increment operation on a remote data object.
}

\begin{apidefinition}

\begin{C11synopsis}
void shmem_inc(TYPE *dest, int pe);
\end{C11synopsis}
where \TYPE{} is one of the standard \ac{AMO} types specified by Table \ref{stdamotypes}.

\begin{Csynopsis}
void shmem_<TYPENAME>_inc(TYPE *dest, int pe);
\end{Csynopsis}
where \TYPE{} is one of the standard \ac{AMO} types and has a corresponding \TYPENAME{} specified by Table \ref{stdamotypes}.

\begin{Fsynopsis}
INTEGER pe
CALL SHMEM_INT4_INC(dest, pe)
CALL SHMEM_INT8_INC(dest, pe)
\end{Fsynopsis}

\begin{apiarguments}

\apiargument{IN}{dest}{The remotely accessible integer data object to be updated
    on the remote \ac{PE}. The type of \dest{} should match that implied in the
    SYNOPSIS section.}
\apiargument{IN}{pe}{An integer that indicates the \ac{PE} number on which
    \dest{} is to be  updated. If you are using \Fortran{}, it must be a default
    integer value.}

\end{apiarguments}

\apidescription{
    These  routines perform  an atomic increment operation on the \VAR{dest} data
    object on \ac{PE}.  
}


\apidesctable{
    If you are using \Fortran, \VAR{dest} must be of the following type:
}{Routine}{Data type of \VAR{dest} and \VAR{source}}

\apitablerow{SHMEM\_INT4\_INC}{\CONST{4}-byte integer}
\apitablerow{SHMEM\_INT8\_INC}{\CONST{8}-byte integer}

\apireturnvalues{
    None.
}

\apinotes{
     The term remotely accessible is defined in Section \ref{subsec:memory_model}.
}

\begin{apiexamples}

\apicexample
    { The following \FUNC{shmem\_inc} example is for \CorCpp{} programs: }
    {./example_code/shmem_inc_example.c}
    {}

\end{apiexamples}

\end{apidefinition}
