\apisummary{
    The nonblocking get routines provide a method for copying data from a
    contiguous remote data object on the specified \ac{PE} to the local data object. 
}

\begin{apidefinition}

\begin{C11synopsis}
void @\FuncDecl{shmem\_get\_nbi}@(TYPE *dest, const TYPE *source, size_t nelems, int pe);
void @\FuncDecl{shmem\_get\_nbi}@(shmem_ctx_t ctx, TYPE *dest, const TYPE *source, size_t nelems, int pe);
\end{C11synopsis}
where \TYPE{} is one of the standard \ac{RMA} types specified by Table \ref{stdrmatypes}.

\begin{Csynopsis}
void @\FuncDecl{shmem\_\FuncParam{TYPENAME}\_get\_nbi}@(TYPE *dest, const TYPE *source, size_t nelems, int pe);
void @\FuncDecl{shmem\_ctx\_\FuncParam{TYPENAME}\_get\_nbi}@(shmem_ctx_t ctx, TYPE *dest, const TYPE *source, size_t nelems, int pe);
\end{Csynopsis}
where \TYPE{} is one of the standard \ac{RMA} types and has a corresponding \TYPENAME{} specified by Table \ref{stdrmatypes}.

\begin{CsynopsisCol}
void @\FuncDecl{shmem\_get\FuncParam{SIZE}\_nbi}@(void *dest, const void *source, size_t  nelems, int pe);
void @\FuncDecl{shmem\_ctx\_get\FuncParam{SIZE}\_nbi}@(shmem_ctx_t ctx, void *dest, const void *source, size_t  nelems, int pe);
\end{CsynopsisCol}
where \SIZE{} is one of \CONST{8, 16, 32, 64, 128}.

\begin{CsynopsisCol}
void @\FuncDecl{shmem\_getmem\_nbi}@(void *dest, const void *source, size_t nelems, int pe);
void @\FuncDecl{shmem\_ctx\_getmem\_nbi}@(shmem_ctx_t ctx, void *dest, const void *source, size_t nelems, int pe);
\end{CsynopsisCol}

\begin{Fsynopsis}
INTEGER nelems, pe
CALL @\FuncDecl{SHMEM\_CHARACTER\_GET\_NBI}@(dest, source, nelems, pe)
CALL @\FuncDecl{SHMEM\_COMPLEX\_GET\_NBI}@(dest, source, nelems, pe)
CALL @\FuncDecl{SHMEM\_DOUBLE\_GET\_NBI}@(dest, source, nelems, pe)
CALL @\FuncDecl{SHMEM\_GET4\_NBI}@(dest, source, nelems, pe)
CALL @\FuncDecl{SHMEM\_GET8\_NBI}@(dest, source, nelems, pe)
CALL @\FuncDecl{SHMEM\_GET32\_NBI}@(dest, source, nelems, pe)
CALL @\FuncDecl{SHMEM\_GET64\_NBI}@(dest, source, nelems, pe)
CALL @\FuncDecl{SHMEM\_GET128\_NBI}@(dest, source, nelems, pe)
CALL @\FuncDecl{SHMEM\_GETMEM\_NBI}@(dest, source, nelems, pe)
CALL @\FuncDecl{SHMEM\_INTEGER\_GET\_NBI}@(dest, source, nelems, pe)
CALL @\FuncDecl{SHMEM\_LOGICAL\_GET\_NBI}@(dest, source, nelems, pe)
CALL @\FuncDecl{SHMEM\_REAL\_GET\_NBI}@(dest, source, nelems, pe)
\end{Fsynopsis}

\begin{apiarguments}
    \apiargument{IN}{ctx}{A context handle specifying the context on which to perform the operation.
        When this argument is not provided, the operation is performed on
        the default context.}
    \apiargument{OUT}{dest}{Local data object to be updated.}
    \apiargument{IN}{source}{Symmetric address of the source data object.
        The type of \source{} should match that implied in the SYNOPSIS section.}
    \apiargument{IN}{nelems}{Number of elements in the \dest{} and \source{}
        arrays. \VAR{nelems} must be of type \VAR{size\_t} for \Cstd. When
        using \Fortran, it must be a constant, variable, or array element of default
        integer type.}
    \apiargument{IN}{pe}{\ac{PE}  number of the remote \ac{PE}.  \VAR{pe} must
        be of type integer. When using \Fortran, it must be a constant,
        variable, or array element of default integer type.}
\end{apiarguments}

\apidescription{
    The get routines provide a method for copying a contiguous symmetric data
   object from a different \ac{PE} to a contiguous data object on the local
   \ac{PE}.   The routines return after posting the operation.  The operation is considered 
    complete after a subsequent call to \FUNC{shmem\_quiet}. 
    At the completion of \FUNC{shmem\_quiet}, the 
    data has been delivered to the \dest{} array on the local \ac{PE}. 
    If the context handle \VAR{ctx} does not correspond to a valid context,
    the behavior is undefined.
}

\apidesctable{
    The  \dest{} and \source{} data objects must conform to typing constraints,
    which are as follows:
}{Routine}{Data type of \VAR{dest} and \VAR{source}}

    \apitablerow{shmem\_getmem\_nbi}{\Fortran: Any noncharacter type. \Cstd:
        Any  data  type.  nelems is scaled in bytes.}
    \apitablerow{shmem\_get4\_nbi, shmem\_get32\_nbi}{Any noncharacter type
        that has a storage size equal to \CONST{32} bits.}
    \apitablerow{shmem\_get8\_nbi}{\Cstd: Any noncharacter type that
        has a storage size equal to \CONST{8} bits.}
    \apitablerow{}{\Fortran: Any noncharacter type that
        has a storage size equal to \CONST{64} bits.}
    \apitablerow{shmem\_get64\_nbi}{Any noncharacter type that
        has a storage size equal to \CONST{64} bits.}
    \apitablerow{shmem\_get128\_nbi}{Any noncharacter type that has a
        storage size equal to \CONST{128} bits.}
    \apitablerow{SHMEM\_CHARACTER\_GET\_NBI}{Elements of type character. \VAR{nelems} is
    the number  of characters  to transfer. The actual character
    lengths of the \source{} and \dest{} variables are ignored.}
    \apitablerow{SHMEM\_COMPLEX\_GET\_NBI}{Elements of type complex of default
       size.}
    \apitablerow{SHMEM\_DOUBLE\_GET\_NBI}{\Fortran: Elements of type double precision.}
    \apitablerow{SHMEM\_INTEGER\_GET\_NBI}{Elements of type integer.}
    \apitablerow{SHMEM\_LOGICAL\_GET\_NBI}{Elements of type logical.}
    \apitablerow{SHMEM\_REAL\_GET\_NBI}{Elements of type real.}

\apireturnvalues{
    None.
}

\apinotes{
    See Section \ref{subsec:memory_model} for a definition of the term
    remotely accessible.
    When using \Fortran, data types must be of default size.  For example, a real
    variable must be declared as \CONST{REAL}, \CONST{REAL*4},  or
    \CONST{REAL(KIND=KIND(1.0))}.
}

\end{apidefinition}
