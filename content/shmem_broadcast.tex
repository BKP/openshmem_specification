\apisummary{
    Broadcasts a block of data from one \ac{PE} to one or more destination
    \acp{PE}.
}

\begin{apidefinition}

\begin{Csynopsis}
void shmem_broadcast32(void *dest, const void *source, size_t nelems, int PE_root, int PE_start, int logPE_stride, int PE_size, long *pSync);
void shmem_broadcast64(void *dest, const void *source, size_t nelems, int PE_root, int PE_start, int logPE_stride, int PE_size, long *pSync);
\end{Csynopsis}

\begin{Fsynopsis}
INTEGER nelems, PE_root, PE_start, logPE_stride, PE_size
INTEGER pSync(SHMEM_BCAST_SYNC_SIZE)
CALL SHMEM_BROADCAST4(dest, source, nelems, PE_root, PE_start, logPE_stride, PE_size, pSync)
CALL SHMEM_BROADCAST8(dest, source, nelems, PE_root, PE_start, logPE_stride, PE_size, pSync)
CALL SHMEM_BROADCAST32(dest, source, nelems, PE_root, PE_start, logPE_stride, PE_size,pSync)
CALL SHMEM_BROADCAST64(dest, source, nelems, PE_root, PE_start, logPE_stride, PE_size,pSync)
\end{Fsynopsis}
 
\begin{apiarguments}

\apiargument{OUT}{dest}{A symmetric data object.} 
\apiargument{IN}{source}{A symmetric data object that can be of any data type
    that is permissible for the \dest{} argument.}
\apiargument{IN}{nelems}{The number of elements in \source.  For
    \FUNC{shmem\_broadcast32} and \FUNC{shmem\_broadcast4}, this is the number of
    32-bit halfwords.  nelems must be of type \VAR{size\_t} in \Cstd.  When
    using \Fortran, it must be a default integer value.}
\apiargument{IN}{PE\_root}{Zero-based ordinal of the \ac{PE}, with respect to
    the \activeset,	from which the data is copied. Must be greater than or equal to
    0 and less than \VAR{PE\_size}. \VAR{PE\_root} must be of type integer.  When using \Fortran, it must be a default integer value.}
\apiargument{IN}{PE\_start}{The lowest \ac{PE} number of the \activeset{} of
    \acp{PE}.  \VAR{PE\_start} must be of type integer.  When using \Fortran,
    it must be a default integer value.}
\apiargument{IN}{logPE\_stride}{ The log (base 2) of the stride between
    consecutive \ac{PE} numbers in the \activeset. \VAR{log\_PE\_stride} must be of
    type integer.  When using \Fortran, it must be a default integer value.}
\apiargument{IN}{PE\_size}{ The number of \acp{PE} in the \activeset.
    \VAR{PE\_size} must be of type integer.  When using \Fortran, it must be a
    default integer value.}
\apiargument{IN}{pSync}{ A symmetric work array.  In \CorCpp, \VAR{pSync} must
    be of type long and size \CONST{SHMEM\_BCAST\_SYNC\_SIZE}. In \Fortran,
    \VAR{pSync} must be of type integer and size \CONST{SHMEM\_BCAST\_SYNC\_SIZE}.
    Every element of this array must be initialized with the value
    \CONST{SHMEM\_SYNC\_VALUE} (in \CorCpp) or \CONST{SHMEM\_SYNC\_VALUE} (in
    \Fortran) before any of the \acp{PE}  in  the  \activeset{} enter
    \FUNC{shmem\_broadcast}.}

\end{apiarguments}

\apidescription{   
    \openshmem broadcast routines are collective routines.  They copy data object
    \source{} on the processor specified by \VAR{PE\_root} and store the values at
    \dest{} on the other \acp{PE} specified by the triplet \VAR{PE\_start},
    \VAR{logPE\_stride}, \VAR{PE\_size}.  The data is not copied to the \dest{} area
    on the root \ac{PE}.
    
    As with all \openshmem collective routines, each of these routines assumes that
    only \acp{PE} in the \activeset{} call the routine.  If a \ac{PE} not in the
    \activeset{} calls an \openshmem collective routine, undefined behavior results.
    
    The values of arguments \VAR{PE\_root}, \VAR{PE\_start}, \VAR{logPE\_stride},
    and \VAR{PE\_size} must be equal on all \acp{PE} in the \activeset.  The same
    \dest{} and \source{} data objects and the same \VAR{pSync} work array must be
    passed to all \acp{PE} in the \activeset.
    
    Before any \ac{PE} calls a broadcast routine, the following
    conditions must be met: (synchronization via a barrier or some other method is often
    needed to ensure this): The \VAR{pSync} array on all \acp{PE} in the
    \activeset{} is not still in use from a prior call to a broadcast routine.  The
    \dest{} array on all \acp{PE} in the \activeset{} is ready to accept the
    broadcast data.
    
    Upon return from a broadcast routine, the following are true for the local
    \ac{PE}: If the current \ac{PE} is not the root \ac{PE}, the \dest{} data object
    is updated. The \source{} data object may be safely reused.  
    The values in the \VAR{pSync} array are restored to the original values.
}

\apidesctable{
The  \dest{}  and \source{} data  objects must conform to certain typing
constraints, which are as follows:
}{Routine}{Data type of \VAR{dest} and \VAR{source}}

\apitablerow{shmem\_broadcast8, shmem\_broadcast64}{Any noncharacter
    type that has an element size of \CONST{64} bits. No \Fortran derived types or
    \CorCpp{} structures are allowed.}
\apitablerow{shmem\_broadcast4, shmem\_broadcast32}{Any noncharacter
    type that has an element size of \CONST{32} bits. No \Fortran
    derived types or \CorCpp{} structures are allowed.}

\apireturnvalues{
    None.
}

\apinotes{
    All \openshmem broadcast routines restore \VAR{pSync} to its original contents.
    Multiple calls to \openshmem routines that use the same \VAR{pSync} array do not
    require that \VAR{pSync} be reinitialized after the first call.
    
    The user must ensure that the \VAR{pSync} array is not being updated by any
    \ac{PE} in the \activeset{} while any of the \acp{PE} participates in processing
    of an \openshmem broadcast routine. Be careful to avoid these situations: If the
    \VAR{pSync} array is initialized at run time, some type of synchronization is
    needed to ensure that all \acp{PE} in the \activeset{} have initialized
    \VAR{pSync} before any of them enter an \openshmem routine called with the
    \VAR{pSync} synchronization array.  A \VAR{pSync} array may be reused on a
    subsequent \openshmem broadcast routine only if none of the \acp{PE} in the
    \activeset{} are still processing a prior \openshmem broadcast routine call that
    used the same \VAR{pSync} array.  In general, this can be ensured only by doing
    some type of synchronization.        
}

\begin{apiexamples}

\apicexample
    {In the following examples, the call to \FUNC{shmem\_broadcast64} copies \source{}
    on \ac{PE} 4 to \dest{} on \acp{PE} 5, 6, and 7. 
    
    \CorCpp{} example:}
    {./example_code/shmem_broadcast_example.c}
    {}

\apifexample
    {\Fortran example:}
    {./example_code/shmem_broadcast_example.f90}
    {}

\end{apiexamples}

\end{apidefinition}
