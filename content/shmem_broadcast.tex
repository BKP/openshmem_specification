\apisummary{
    Broadcasts a block of data from one \ac{PE} to one or more destination
    \acp{PE}.
}

\begin{apidefinition}

%% C11
{\color{Green}
\begin{C11synopsis}
int @\FuncDecl{shmem\_broadcast}@(shmem_team_t team, TYPE *dest, const TYPE *source, size_t nelems, int PE_root);
\end{C11synopsis}
where \TYPE{} is one of the standard \ac{RMA} types specified by Table \ref{stdrmatypes}.
}

%% C/C++
\begin{Csynopsis}
\end{Csynopsis}
{\color{Green}
\begin{CsynopsisCol}
int @\FuncDecl{shmem\_\FuncParam{TYPENAME}\_broadcast}@(shmem_team_t team, TYPE *dest, const TYPE *source, size_t nelems, int PE_root);
\end{CsynopsisCol}
where \TYPE{} is one of the standard \ac{RMA} types and has a corresponding \TYPENAME{} specified by Table \ref{stdrmatypes}.

\begin{CsynopsisCol}
int @\FuncDecl{shmem\_broadcastmem}@(shmem_team_t team, void *dest, const void *source, size_t nelems, int PE_root);
\end{CsynopsisCol}
}
\begin{DeprecateBlock}
\begin{CsynopsisCol}
void @\FuncDecl{shmem\_broadcast32}@(void *dest, const void *source, size_t nelems, int PE_root, int PE_start, int logPE_stride, int PE_size, long *pSync);
void @\FuncDecl{shmem\_broadcast64}@(void *dest, const void *source, size_t nelems, int PE_root, int PE_start, int logPE_stride, int PE_size, long *pSync);
\end{CsynopsisCol}
\end{DeprecateBlock}

\begin{Fsynopsis}
INTEGER nelems, PE_root, PE_start, logPE_stride, PE_size
INTEGER pSync(SHMEM_BCAST_SYNC_SIZE)
CALL @\FuncDecl{SHMEM\_BROADCAST4}@(dest, source, nelems, PE_root, PE_start, logPE_stride, PE_size, pSync)
CALL @\FuncDecl{SHMEM\_BROADCAST8}@(dest, source, nelems, PE_root, PE_start, logPE_stride, PE_size, pSync)
CALL @\FuncDecl{SHMEM\_BROADCAST32}@(dest, source, nelems, PE_root, PE_start, logPE_stride, PE_size,pSync)
CALL @\FuncDecl{SHMEM\_BROADCAST64}@(dest, source, nelems, PE_root, PE_start, logPE_stride, PE_size,pSync)
\end{Fsynopsis}
 
\begin{apiarguments}

\newtext{%
\apiargument{IN}{team}{The team over which to perform the operation.}%
}

\apiargument{OUT}{dest}{A symmetric data object. \newtext{See the table below in this description
    for allowable types.}} 
\apiargument{IN}{source}{A symmetric data object that can be of any data type
    that is permissible for the \dest{} argument.}
\apiargument{IN}{nelems}{The number of elements in \source.
    nelems must be of type \VAR{size\_t} in \Cstd.  When
    using \Fortran, it must be a default integer value.}
\apiargument{IN}{PE\_root}{Zero-based ordinal of the \ac{PE}, with respect to
    the \newtext{team or} active set, from which the data is copied.
    \VAR{PE\_root} must be of type \CTYPE{int}.
    When using \Fortran, it must be a default integer value.}

\begin{DeprecateBlock}
\apiargument{IN}{PE\_start}{The lowest \ac{PE} number of the active set of
    \acp{PE}.  \VAR{PE\_start} must be of type integer.  When using \Fortran,
    it must be a default integer value.}
\apiargument{IN}{logPE\_stride}{ The log (base 2) of the stride between
    consecutive \ac{PE} numbers in the active set. \VAR{log\_PE\_stride} must be of
    type integer.  When using \Fortran, it must be a default integer value.}
\apiargument{IN}{PE\_size}{ The number of \acp{PE} in the active set.
    \VAR{PE\_size} must be of type integer.  When using \Fortran, it must be a
    default integer value.}
\apiargument{IN}{pSync}{
    A symmetric work array of size \CONST{SHMEM\_BCAST\_SYNC\_SIZE}.
    In \CorCpp, \VAR{pSync} must be an array of elements of type \CTYPE{long}.
    In \Fortran, \VAR{pSync} must be an array of elements of default integer type.
    Every element of this array must be initialized with the value
    \CONST{SHMEM\_SYNC\_VALUE} before any of the \acp{PE} in the active set
    enters \FUNC{shmem\_broadcast}.}
\end{DeprecateBlock}

\end{apiarguments}

\apidescription{   
    \openshmem broadcast routines are collective routines \newtext{over an active set or
    existing \openshmem team}. They copy data object
    \source{} on the processor specified by \VAR{PE\_root} and store the values at
    \dest{} on the other \acp{PE} \newtext{particpating in the collective operation.}
    \oldtext{specified by the triplet \VAR{PE\_start}, \VAR{logPE\_stride}, \VAR{PE\_size}.} %%
    The data is not copied to the \dest{} area on the root \ac{PE}.

    {\color{Green}
    The same \dest{} and \source{} data objects and the same value of \VAR{PE\_root} must be
    passed by all \acp{PE} particpating in the collective operation.

    Team-based broadcast routines operate over all \acp{PE} in the provided team argument. All
    \acp{PE} in the provided team must participate in the operation.
    If an invalid team handle or \LibConstRef{SHMEM\_TEAM\_INVALID} is passed to this routine,
    the behavior is undefined.

    As with all team-based \openshmem routines, \ac{PE}
    numbering is relative to the team. The specified root \ac{PE} must be a valid \ac{PE}
    number for the team, between \CONST{0} and \VAR{N-1}, where \VAR{N} is
    the size of the team.

    Active-set-based broadcast routines operate over all \acp{PE} in the active set
    defined by the \VAR{PE\_start}, \VAR{logPE\_stride}, \VAR{PE\_size} triplet.
    }

    As with all \newtext{active-set-based} \oldtext{\openshmem} collective routines,
    each of these routines assumes that
    only \acp{PE} in the active set call the routine.  If a \ac{PE} not in the
    active set calls an \newtext{active-set-based} \oldtext{\openshmem}
    collective routine, the behavior is undefined.
    
    The values of arguments \VAR{PE\_root}, \VAR{PE\_start}, \VAR{logPE\_stride},
    and \VAR{PE\_size} must be the same value on all \acp{PE} in the active set.
    \newtext{The value of \VAR{PE\_root} must be between \CONST{0} and \VAR{PE\_size}.}
    The same \VAR{pSync} work array must be passed by all \acp{PE} in the active set.

    Before any \ac{PE} calls a broadcast routine, the following conditions must be ensured:
    \begin{itemize}
    \item The \dest{} array on all \acp{PE} \newtext{participating in the broadcast}
      \oldtext{in the active set} %%
      is ready to accept the broadcast data.
    \item \newtext{If using active-set-based routines,} the
      \VAR{pSync} array on all \acp{PE} in the
      active set is not still in use from a prior call to a collective
      \openshmem routine.
    \end{itemize}
    Otherwise, the behavior is undefined.
    
    Upon return from a broadcast routine, the following are true for the local
    \ac{PE}:
    \begin{itemize}
    \item If the current \ac{PE} is not the root \ac{PE},
      the \dest{} data object is updated.
    \item The \source{} data object may be safely reused.
    \item \newtext{If using active-set-based routines,}
    the values in the \VAR{pSync} array are restored to the original values.
    \end{itemize}
}

\apidesctable{
The  \dest{}  and \source{} data  objects must conform to certain typing
constraints, which are as follows:
}{Routine}{Data type of \VAR{dest} and \VAR{source}}

\apitablerow{shmem\_broadcastmem}{\Cstd: Any data  type.  nelems is scaled in bytes.}
\apitablerow{shmem\_broadcast8, shmem\_broadcast64}{Any noncharacter
    type that has an element size of \CONST{64} bits. No \Fortran derived types \newtext{nor} \oldtext{or} 
    \CorCpp{} structures are allowed.}
\apitablerow{shmem\_broadcast4, shmem\_broadcast32}{Any noncharacter
    type that has an element size of \CONST{32} bits. No \Fortran derived types \newtext{nor} \oldtext{or} 
    \CorCpp{} structures are allowed.}

\apireturnvalues{
   \newtext{Zero on successful local completion. Nonzero otherwise.}
}

\apinotes{
\newtext{%
    There are no specifically defined error codes for these routines.
    See section \ref{subsec:error_handling} for expected error checking and
    return code behavior specific to implementations. For portable
    error checking and debugging behavior, programs should do their own checks
    for invalid team handles or \LibConstRef{SHMEM\_TEAM\_INVALID}
    }

    All \openshmem broadcast routines restore \VAR{pSync} to its original contents.
    Multiple calls to \openshmem routines that use the same \VAR{pSync} array do not
    require that \VAR{pSync} be reinitialized after the first call.
    
    The user must ensure that the \VAR{pSync} array is not being updated by any
    \ac{PE} in the active set while any of the \acp{PE} participates in processing
    of an \openshmem broadcast routine. Be careful to avoid these situations: If the
    \VAR{pSync} array is initialized at run time, before its first use, some type of synchronization is
    needed to ensure that all \acp{PE} in the active set have initialized
    \VAR{pSync} before any of them enter an \openshmem routine called with the
    \VAR{pSync} synchronization array.  A \VAR{pSync} array may be reused on a
    subsequent \openshmem broadcast routine only if none of the \acp{PE} in the
    active set are still processing a prior \openshmem broadcast routine call that
    used the same \VAR{pSync} array. In general, this can be ensured only by doing
    some type of synchronization.        

    Team handle error checking and integer return codes are currently undefined.
    Implementations may define these behaviors as needed, but programs should
    ensure portability by doing their own checks for invalid team handles and for
    \LibConstRef{SHMEM\_TEAM\_INVALID}.
}

\begin{apiexamples}

\apicexample
    {In the following \Cstd[11] example, the call to \FUNC{shmem\_broadcast} copies \source{}
    on \ac{PE} $0$ to \dest{} on \acp{PE} $1\dots npes-1$.
    
    \CorCpp{} example:}
    {./example_code/shmem_broadcast_example.c}
    {}

\end{apiexamples}

\end{apidefinition}
