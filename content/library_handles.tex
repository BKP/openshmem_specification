\TableIndex{Library Handles}
\TableIndex{Handles}

The \openshmem library provides a set of predefined named constant handles.
All named constants can be used in initialization expressions or assignments,
but not necessarily in array declarations or as labels in \Cstd switch statements.
This implies named constants to be link-time but not necessarily compile-time
constants.

\begin{longtable}{|p{0.45\textwidth}|p{0.5\textwidth}|}
\hline
\textbf{Handle} & \textbf{Description}
\tabularnewline \hline
\endhead
%%
\LibHandleDecl{SHMEM\_TEAM\_WORLD} &
Handle of type \CTYPE{shmem\_team\_t} that corresponds to the world
team that contains all \acp{PE} in the \openshmem program.  All point-to-point
communication operations and collective synchronizations that do not specify a team
are performed on the world team.
See Section~\ref{subsec:team} for more detail about its use.
\tabularnewline \hline
%%
\LibHandleDecl{SHMEM\_TEAM\_SHARED} &
Handle of type \CTYPE{shmem\_team\_t} that corresponds to a team of \acp{PE}
that share a memory domain. When this handle is used by some \ac{PE},
it will refer to the team of all \acp{PE} that would return a non-null
pointer from \FUNC{shmem\_ptr} for symmetric objects on that \ac{PE},
and vice versa. This means that symmetric objects on each \ac{PE} are
directly load/store accessible by all \acp{PE} in the team.
See Section~\ref{subsec:team} for more detail about its use.
\tabularnewline \hline
%%
\LibHandleDecl{SHMEM\_CTX\_DEFAULT} &
Handle of type \CTYPE{shmem\_ctx\_t} that corresponds to the
default communication context.  All point-to-point communication operations
and synchronizations that do not specify a context are performed on the
default context.
See Section~\ref{sec:ctx} for more detail about its use.
\tabularnewline \hline
%%
\end{longtable}
