\apisummary{
    Indicate whether all variables within an array of variables on the local
    \ac{PE} meet the specified test conditions.
}

\begin{apidefinition}

\begin{C11synopsis}
int @\FuncDecl{shmem\_test\_all\_vector}@(TYPE *ivars, size_t nelems, const int *status, int cmp,
    TYPE *cmp_values);
\end{C11synopsis}
where \TYPE{} is one of the point-to-point synchronization types specified by
Table \ref{p2psynctypes}.

\begin{Csynopsis}
int @\FuncDecl{shmem\_\FuncParam{TYPENAME}\_test\_all\_vector}@(TYPE *ivars, size_t nelems, const int *status, int cmp,
    TYPE *cmp_values);
\end{Csynopsis}
where \TYPE{} is one of the point-to-point synchronization types and has a
corresponding \TYPENAME{} specified by Table \ref{p2psynctypes}.

\begin{apiarguments}

  \apiargument{IN}{ivars}{Local address of an array of remotely accessible data
    objects.
    The type of \VAR{ivars} should match that implied in the SYNOPSIS section.}
  \apiargument{IN}{nelems}{The number of elements in the \VAR{ivars} array.}
  \apiargument{IN}{status}{Local address of an optional mask array of length \VAR{nelems}
    that indicates which elements in \VAR{ivars} are excluded from the test set.}
  \apiargument{IN}{cmp}{A comparison operator from Table~\ref{p2p-consts} that
    compares elements of \VAR{ivars} with elements of \VAR{cmp\_values}.}
  \apiargument{IN}{cmp\_values}{Local address of an array of length \VAR{nelems}
    containing values to be compared with the respective objects in \VAR{ivars}.
    The type of \VAR{cmp\_values} should match that implied in the SYNOPSIS section.}

\end{apiarguments}

\apidescription{
    The \FUNC{shmem\_test\_all\_vector} routine indicates whether all
    entries in the test set specified by \VAR{ivars} and \VAR{status} have
    satisfied the test condition at the calling \ac{PE}.  The \VAR{ivars}
    objects at the calling \ac{PE} may be updated by an \ac{AMO} performed by a
    thread located within the calling \ac{PE} or within another \ac{PE}.
    This routine does not
    block and returns zero if not all entries in \VAR{ivars} satisfied the test
    conditions.  This routine compares each element of the
    \VAR{ivars} array in the test set with each respective value in
    \VAR{cmp\_values} according to the comparison operator \VAR{cmp} at the
    calling \ac{PE}.  If \VAR{nelems} is 0, the test set is empty and this
    routine returns 1.

    The optional \VAR{status} is a mask array of length \VAR{nelems} where each element
    corresponds to the respective element in \VAR{ivars} and indicates whether
    the element is excluded from the test set.  Elements of \VAR{status} set to
    0 will be included in the test set, and elements set to 1 will be ignored.  If all elements
    in \VAR{status} are set to 1 or \VAR{nelems} is 0, the test set is empty
    and this routine returns 0.  If \VAR{status} is a null pointer, it is
    ignored and all elements in \VAR{ivars} are included in the test set.  The
    \VAR{ivars}, \VAR{indices}, and \VAR{status} arrays must not overlap in
    memory.

    Implementations must ensure that \FUNC{shmem\_test\_all\_vector} does not
    return 1 before the update of the memory indicated by \VAR{ivars} is fully
    complete.
}

\apireturnvalues{
    \FUNC{shmem\_test\_all\_vector} returns 1 if all variables in \VAR{ivars}
    satisfy the test conditions or if \VAR{nelems} is 0, otherwise this routine
    returns 0.
}

\apinotes{
  None.
}

\end{apidefinition}
