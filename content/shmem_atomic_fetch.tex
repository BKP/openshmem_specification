\apisummary{
    Atomically fetches the value of a remote data object.
}

\begin{apidefinition}

\begin{C11synopsis}
TYPE @\FuncDecl{shmem\_atomic\_fetch}@(const TYPE *source, int pe);
TYPE @\FuncDecl{shmem\_atomic\_fetch}@(shmem_ctx_t ctx, const TYPE *source, int pe);
\end{C11synopsis}
where \TYPE{} is one of the extended \ac{AMO} types specified by
Table~\ref{extamotypes}.

\begin{Csynopsis}
TYPE @\FuncDecl{shmem\_\FuncParam{TYPENAME}\_atomic\_fetch}@(const TYPE *source, int pe);
TYPE @\FuncDecl{shmem\_ctx\_\FuncParam{TYPENAME}\_atomic\_fetch}@(shmem_ctx_t ctx, const TYPE *source, int pe);
\end{Csynopsis}
where \TYPE{} is one of the extended \ac{AMO} types and has a corresponding
\TYPENAME{} specified by Table~\ref{extamotypes}.

\begin{DeprecateBlock}
\begin{C11synopsis}
TYPE @\FuncDecl{shmem\_fetch}@(const TYPE *source, int pe);
\end{C11synopsis}
where \TYPE{} is one of \{\CTYPE{float}, \CTYPE{double}, \CTYPE{int},
\CTYPE{long}, \CTYPE{long long}\}.

\begin{Csynopsis}
TYPE @\FuncDecl{shmem\_\FuncParam{TYPENAME}\_fetch}@(const TYPE *source, int pe);
\end{Csynopsis}
where \TYPE{} is one of \{\CTYPE{float}, \CTYPE{double}, \CTYPE{int},
\CTYPE{long}, \CTYPE{long long}\} and has a corresponding
\TYPENAME{} specified by Table~\ref{extamotypes}.
\end{DeprecateBlock}

\begin{Fsynopsis}
INTEGER pe
INTEGER*4 SHMEM_INT4_FETCH, ires_i4
ires\_i4 = @\FuncDecl{SHMEM\_INT4\_FETCH}@(source, pe)
INTEGER*8 SHMEM_INT8_FETCH, ires_i8
ires\_i8 = @\FuncDecl{SHMEM\_INT8\_FETCH}@(source, pe)
REAL*4 SHMEM_REAL4_FETCH, res_r4
res\_r4 = @\FuncDecl{SHMEM\_REAL4\_FETCH}@(source, pe)
REAL*8 SHMEM_REAL8_FETCH, res_r8
res\_r8 = @\FuncDecl{SHMEM\_REAL8\_FETCH}@(source, pe)
\end{Fsynopsis}

\begin{apiarguments}

  \apiargument{IN}{ctx}{The context on which to perform the operation.
    When this argument is not provided, the operation is performed on
    \CONST{SHMEM\_CTX\_DEFAULT}.}
  \apiargument{IN}{source}{The remotely accessible data object to be fetched from
    the remote \ac{PE}.}
  \apiargument{IN}{pe}{An integer that indicates the \ac{PE} number from which
    \VAR{source} is to be fetched.}

\end{apiarguments}

\apidescription{
    \FUNC{shmem\_atomic\_fetch} performs an atomic fetch operation.
    It returns the contents of the \VAR{source} as an atomic operation.
}

\apireturnvalues{
    The contents at the \VAR{source} address on the remote \ac{PE}.
    The data type of the return value is the same as the type of
    the remote data object.
}

\apinotes{
    None.
}

\end{apidefinition}
