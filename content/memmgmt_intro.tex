\openshmem provides a set of \acp{API} for managing the symmetric heap. The
\acp{API} allow one to dynamically allocate, deallocate, reallocate, and align
symmetric data objects in the symmetric heap.

The symmetric memory allocation routines differ from the private heap
allocation routines in that they must be called by all \acp{PE} in
the world team.  When specified, each of these routines includes at
least one call to a procedure that is semantically equivalent to
\FUNC{shmem\_barrier\_all}.  This ensures that all \acp{PE}
participate in the memory management, and that the memory on other
\acp{PE} can be used as soon as the local \ac{PE} returns.  The
implicit barriers performed by these routines quiet the default
context.  It is the user's responsibility to ensure that no
communication operations involving the given memory block are pending
on other contexts prior to calling the \FUNC{shmem\_free} and
\FUNC{shmem\_realloc} routines.

The total size of the symmetric heap is determined at job startup.  One can
specify the size of the heap using the \ENVVAR{SHMEM\_SYMMETRIC\_SIZE} environment
variable (where available).	

\begin{DeprecateBlock}
  As of \openshmem[1.2] the use of \FUNC{shmalloc}, \FUNC{shmemalign},
  \FUNC{shfree},  and \FUNC{shrealloc} has been deprecated. Although \openshmem
  libraries are required to support the calls, users are encouraged to use
  \FUNC{shmem\_malloc}, \FUNC{shmem\_align}, \FUNC{shmem\_free}, and
  \FUNC{shmem\_realloc} instead.  The behavior and signature  of the routines
  remains unchanged from the deprecated versions.
\end{DeprecateBlock}

\parimpnotes{
  The symmetric heap allocation routines always return the symmetric
  addresses of corresponding symmetric objects across all
  \acp{PE}. The \openshmem specification does not require that the
  virtual addresses are equal across all \acp{PE}. Nevertheless, the
  implementation must avoid costly address translation operations in
  the communication path, including $O(N)$ memory translation tables,
  where $N$ is the number of \acp{PE}.  In order to avoid address
  translations, the implementation may re-map the allocated block of
  memory based on agreed virtual address.  Additionally, some
  operating systems provide an option to disable virtual address
  randomization, which enables predictable allocation of virtual
  memory addresses.
}

