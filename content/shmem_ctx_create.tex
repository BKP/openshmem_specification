\apisummary{
    Create a communication context \newtext{locally}.
}

\begin{apidefinition}

\begin{Csynopsis}
int @\FuncDecl{shmem\_ctx\_create}@(long options, shmem_ctx_t *ctx);
\end{Csynopsis}

\begin{apiarguments}
    \apiargument{IN}{options}{The set of options requested for the given context.
        Multiple options may be requested by combining them with a bitwise
        OR operation; otherwise, \CONST{0} can be given if no options are
        requested.}
    \apiargument{OUT}{ctx}{A handle to the newly created context.}
\end{apiarguments}

\apidescription{
    The \FUNC{shmem\_ctx\_create} routine creates a new communication context
    and returns its handle through the \VAR{ctx} argument.  If the context was
    created successfully, a value of zero is returned
    \newtext{and the context handle pointed to by \VAR{ctx} specifies a valid context};
    otherwise, a nonzero value is returned
    \newtext{and the context handle pointed to by \VAR{ctx} is not modified}.
    An unsuccessful context
    creation call is not treated as an error and the \openshmem library remains
    in a correct state.  The creation call can be reattempted with different
    options or after additional resources become available.

    \newtext{
    A newly created communication context has an initial association with the
    default team.
    All \openshmem routines that operate on this context will do so with
    respect to the associated \ac{PE} team.
    That is, all point-to-point routines operating on this context will use
    team-relative \ac{PE} numbering.
    }

    By default, contexts are {\em shareable} and, when it is allowed by the
    threading model provided by the \openshmem library, they can be used concurrently by
    multiple threads within the PE where they were created.
    %
    The following options can be supplied during context creation to restrict
    this usage model and enable performance optimizations.  When using a given
    context, the application must comply with the requirements of all options
    set on that context; otherwise, the behavior is undefined.
    No options are enabled on the default context.

        \apitablerow{\LibConstRef{SHMEM\_CTX\_SERIALIZED}}{
            The given context is shareable; however, it will not be used by multiple threads
            concurrently.  When the \CONST{SHMEM\_CTX\_SERIALIZED} option is
            set, the user must ensure that operations involving the given
            context are serialized by the application.}

        \apitablerow{\LibConstRef{SHMEM\_CTX\_PRIVATE}}{
            The given context will be used only by the thread that created it.}

        \apitablerow{\LibConstRef{SHMEM\_CTX\_NOSTORE}}{
            Quiet and fence operations performed on the given context are not
            required to enforce completion and ordering of memory store
            operations.
            When ordering of store operations is needed, the application must
            perform a synchronization operation on a context without the
            \CONST{SHMEM\_CTX\_NOSTORE} option enabled.}

}

\apireturnvalues{
    Zero on success and nonzero otherwise.
}

\apinotes{
    None.
}

\end{apidefinition}

