\apisummary{
    Create a communication context.
}

\begin{apidefinition}

\begin{Csynopsis}
int @\FuncDecl{shmem\_ctx\_create}@(long options, shmem_ctx_t *ctx);
\end{Csynopsis}

\begin{apiarguments}
    \apiargument{IN}{options}{The set of options requested for the given context.
        Multiple options may be requested by combining them with a bitwise
        OR operation; otherwise, \CONST{0} can be given if no options are
        requested.}
    \apiargument{OUT}{ctx}{A handle to the newly created context.}
\end{apiarguments}

\apidescription{
    The \FUNC{shmem\_ctx\_create} routine creates a new communication context
    and returns its handle through the \VAR{ctx} argument.
    This context is created from the world team;
    however, the context creation operation is not collective.
    If the context was
    created successfully, a value of zero is returned
    and the context handle pointed to by \VAR{ctx} specifies a valid context;
    otherwise, a nonzero value is returned and \VAR{ctx} is set to
    \LibConstRef{SHMEM\_CTX\_INVALID}.
    An unsuccessful context
    creation call is not treated as an error and the \openshmem library remains
    in a correct state.  The creation call can be reattempted with different
    options or after additional resources become available.

    All \openshmem routines that operate on this context will do so with
    respect to the associated \ac{PE} team.
    That is, all point-to-point routines operating on this context will use
    team-relative \ac{PE} numbering.
}

\apireturnvalues{
    Zero on success and nonzero otherwise.
}

\end{apidefinition}

