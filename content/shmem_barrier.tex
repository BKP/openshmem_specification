\begin{DeprecateBlock}
\apisummary{
    Performs all operations described in the \FUNC{shmem\_barrier\_all} interface
    but with respect to a subset of \acp{PE} defined by the active set.
}

\begin{apidefinition}

\begin{Csynopsis}
void @\FuncDecl{shmem\_barrier}@(int PE_start, int logPE_stride, int PE_size, long *pSync);
\end{Csynopsis}

\begin{apiarguments}

    \apiargument{IN}{PE\_start}{The lowest \ac{PE} number of the active set of \acp{PE}.}
\apiargument{IN}{logPE\_stride}{The log (base 2) of the stride between consecutive
    \ac{PE} numbers in the active set.}
    \apiargument{IN}{PE\_size}{The number of \acp{PE} in the active set.}
\apiargument{IN}{pSync}{
    Symmetric address of a work array of size at least \CONST{SHMEM\_BARRIER\_SYNC\_SIZE}.}

\end{apiarguments}

\apidescription{
    \FUNC{shmem\_barrier} is a collective synchronization routine over an
    active set. Control returns from \FUNC{shmem\_barrier} after all \acp{PE} in
    the active set (specified by \VAR{PE\_start}, \VAR{logPE\_stride}, and
    \VAR{PE\_size}) have called \FUNC{shmem\_barrier}.

    As with all \openshmem collective routines, each of these routines assumes that
    only \acp{PE} in the active set call the routine.  If a \ac{PE} not  in  the
    active set calls an \openshmem collective routine, the behavior is undefined.

    The values of arguments \VAR{PE\_start}, \VAR{logPE\_stride}, and \VAR{PE\_size}
    must be the same value on all \acp{PE} in the active set.  The same work array must be
    passed in \VAR{pSync} to all \acp{PE} in the active set.

    \FUNC{shmem\_barrier} ensures that all previously issued stores and remote
    memory updates, including \acp{AMO} and \ac{RMA} operations, done by any of the
    \acp{PE} in the active set on the default context are complete before returning.

    The same \VAR{pSync} array may be reused on consecutive calls to
    \FUNC{shmem\_barrier} if the same active set is used.
}

\apireturnvalues{
    None.
}

\apinotes{
    As of \openshmem[1.5], \FUNC{shmem\_barrier} has been deprecated.
    No team-based barrier is provided by \openshmem, as a team may have any
    number of communication contexts associated with the team.
    Applications seeking such an idiom should call
    \FUNC{shmem\_ctx\_quiet} on the desired communication context,
    followed by a call to \FUNC{shmem\_team\_sync} on the desired
    team.

    The \FUNC{shmem\_barrier} routine can be used to
    portably ensure that memory access operations observe remote updates in the order
    enforced by initiator \acp{PE}.

    Calls to \FUNC{shmem\_ctx\_quiet} can be performed prior
    to calling the barrier routine to ensure completion of operations issued on
    additional contexts.
}

\begin{apiexamples}

\apicexample
    {The following barrier example is for \Cstd[11] programs:}
    {./example_code/shmem_barrier_example.c}
    {}

\end{apiexamples}

\end{apidefinition}
\end{DeprecateBlock}
