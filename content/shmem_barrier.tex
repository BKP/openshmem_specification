\apisummary{
    Performs all operations described in the \FUNC{shmem\_barrier\_all} interface
    but with respect to a subset of \acp{PE} defined by the active set.
}

\begin{apidefinition}

\begin{Csynopsis}
void shmem_barrier(int PE_start, int logPE_stride, int PE_size, long *pSync);
\end{Csynopsis}

\begin{Fsynopsis}
INTEGER PE_start, logPE_stride, PE_size
INTEGER pSync(SHMEM_BARRIER_SYNC_SIZE)
CALL SHMEM_BARRIER(PE_start, logPE_stride, PE_size, pSync)
\end{Fsynopsis}

\begin{apiarguments}

\apiargument{IN}{PE\_start}{The lowest \ac{PE} number of the active set of \acp{PE}.
    \VAR{PE\_start} must be of type integer.  When using \Fortran, it must be
    a default integer value.}
\apiargument{IN}{logPE\_stride}{The log (base 2) of the stride between consecutive
    \ac{PE} numbers in the active set.  \VAR{logPE\_stride} must be of type integer.
    When using \Fortran, it must be a default integer value.}
\apiargument{IN}{PE\_size}{The number of  \acp{PE} in the active set.  \VAR{PE\_size}
    must be of type integer.  When using  \Fortran, it must be a default
    integer value.}
\apiargument{IN}{pSync}{A symmetric work array. In \CorCpp, \VAR{pSync} must
    be of type long and size \CONST{SHMEM\_BARRIER\_SYNC\_SIZE}.  In \Fortran,
    \VAR{pSync} must be of type integer and size \CONST{SHMEM\_BARRIER\_SYNC\_SIZE}.
    When using \Fortran, it must  be a default  integer type.  Every element
    of this array must be initialized to \CONST{SHMEM\_SYNC\_VALUE} before any of
    the \acp{PE} in the active set enter \FUNC{shmem\_barrier} the first time.}

\end{apiarguments}

\apidescription{
    \FUNC{shmem\_barrier} is a collective synchronization routine over an
    active set.  Control returns from \FUNC{shmem\_barrier} after all \acp{PE} in
    the active set (specified by \VAR{PE\_start}, \VAR{logPE\_stride}, and
    \VAR{PE\_size}) have called \FUNC{shmem\_barrier}.
    
    As with all \openshmem collective routines, each of these routines assumes that
    only \acp{PE} in the active set call the routine.  If a \ac{PE} not  in  the
    active set calls an \openshmem collective routine, undefined behavior results.
    
    The values of arguments \VAR{PE\_start}, \VAR{logPE\_stride}, and \VAR{PE\_size}
    must be equal on all \acp{PE} in the active set.  The same work array must be
    passed in \VAR{pSync} to all \acp{PE} in the active set.
    
    \FUNC{shmem\_barrier} ensures that all previously issued stores and remote
    memory updates, including \acp{AMO} and \ac{RMA} operations, done by any of the
    \acp{PE} in the active set are complete before returning.
    
    The  same  \VAR{pSync} array may be reused on consecutive calls   to
    \FUNC{shmem\_barrier} if the same active \ac{PE} set is used.
}

\apireturnvalues{
    None.
}

\apinotes{
    If the \VAR{pSync} array is initialized at run time, be sure to use some type of
    synchronization, for example, a call to \FUNC{shmem\_barrier\_all}, before
    calling \FUNC{shmem\_barrier} for the first time.
    
    If  the active set  does not change, \FUNC{shmem\_barrier} can  be called
    repeatedly with the same \VAR{pSync} array.  No additional synchronization
    beyond that implied by \FUNC{shmem\_barrier} itself is necessary in this case.

    The \FUNC{shmem\_barrier} routine can be used to
    portably ensure that memory access operations observe remote updates in the order
    enforced by initiator \acp{PE}.
}

\begin{apiexamples}

\apicexample
	{The following barrier example is for C11 programs:}
	{./example_code/shmem_barrier_example.c}
	{}

\end{apiexamples}

\end{apidefinition}
