\apisummary{
    Waits for completion of all outstanding memory store, blocking 
    \PUT{}, \ac{AMO}, and \emph{put-with-signal}, as well as 
    nonblocking \PUT{}, \emph{put-with-signal}, and \GET{} routines 
    to symmetric data objects issued by the calling \ac{PE} at the target \ac{PE}.
}

\begin{apidefinition}

\begin{Csynopsis}
void @\FuncDecl{shmem\_pe\_quiet}@(int target_pe);
void @\FuncDecl{shmem\_ctx\_pe\_quiet}@(shmem_ctx_t ctx, int target_pe);
\end{Csynopsis}

\begin{apiarguments}
    \apiargument{IN}{ctx}{A context handle specifying the context on which to perform the operation.
        When this argument is not provided, the operation is performed on
        the default context.}
	\apiargument{IN}{target\_pe}{The target \ac{PE} where the operations needs
	to be completed}
\end{apiarguments}

\apidescription{
    The \FUNC{shmem\_pe\_quiet} ensures completion of memory store, blocking 
    \PUT{}, \ac{AMO}, and \emph{put-with-signal}, as well as 
    nonblocking \PUT{}, \emph{put-with-signal}, and \GET{} routines 
	on the symmetric data objects issued by the
	calling \ac{PE} to the target \ac{PE} and on the given context.
	
	The completion and visibility semantics of these routines are same as the
	\FUNC{shmem\_quiet} routine, however, it applies only to the target \ac{PE} 
	.i.e., the routines to target \ac{PE} are guaranteed to be completed and 
	visible to all \ac{PE} when \FUNC{shmem\_pe\_quiet} returns.
}
\apireturnvalues{
    None.
}

\apinotes{
}
\end{apidefinition}
