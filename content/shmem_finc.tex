\apisummary{
    Performs an atomic fetch-and-increment  operation on a remote data object.
}

\begin{apidefinition}

\begin{C11synopsis}
TYPE shmem_finc(TYPE *dest, int pe);
\end{C11synopsis}
where \TYPE{} is one of the standard \ac{AMO} types specified by Table \ref{stdamotypes}.

\begin{Csynopsis}
TYPE shmem_<TYPENAME>_finc(TYPE *dest, int pe);
\end{Csynopsis}
where \TYPE{} is one of the standard \ac{AMO} types and has a corresponding \TYPENAME{} specified by Table \ref{stdamotypes}.

\begin{Fsynopsis}
INTEGER pe 
INTEGER*4 SHMEM_INT4_FINC, ires_i4
ires_i4 = SHMEM_INT4_FINC(dest, pe)
INTEGER*8 SHMEM_INT8_FINC, ires_i8
ires_i8 = SHMEM_INT8_FINC(dest, pe)
\end{Fsynopsis}


\begin{apiarguments}

\apiargument{IN}{dest}{The remotely accessible integer data object to be updated
    on the remote \ac{PE}. The type of \dest{} should match that implied in the
    SYNOPSIS section.}
\apiargument{IN}{pe}{An integer that indicates the \ac{PE} number on which
    \dest{} is to be updated.  If you are using \Fortran, it must be a default
    integer value.}

\end{apiarguments}


\apidescription{
   These routines perform a fetch-and-increment operation.  The \dest{} on
   \ac{PE} \VAR{pe} is increased by one and the routine returns the previous
   contents of \dest{} as an atomic operation.
}

\apidesctable{
    If you are using \Fortran, \VAR{dest} must be of the following type:
}{Routine}{Data type of \VAR{dest} and \VAR{source}}

\apitablerow{SHMEM\_INT4\_FINC}{\CONST{4}-byte integer}
\apitablerow{SHMEM\_INT8\_FINC}{\CONST{8}-byte integer}

\apireturnvalues{
    The contents that had been at the \dest{} address on the remote \ac{PE} prior to
    the increment.  The data type of the return value is the same as the \dest.
}

\apinotes{
    None.
}

\begin{apiexamples}

\apicexample
    {The following \FUNC{shmem\_finc} example is for C11 programs:}
    {./example_code/shmem_finc_example.c}
    {}

\end{apiexamples}
	
\end{apidefinition}
