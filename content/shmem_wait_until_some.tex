\apisummary{
    Wait on an array of variables on the local \ac{PE} until at least one variable meets the specified wait condition.
}

\begin{apidefinition}

\begin{C11synopsis}
size_t @\FuncDecl{shmem\_wait\_until\_some}@(TYPE *ivars, size_t nelems, size_t *indices, const int *status,
    int cmp, TYPE cmp_value);
\end{C11synopsis}
where \TYPE{} is one of the point-to-point synchronization types specified by
Table \ref{p2psynctypes}.

\begin{Csynopsis}
size_t @\FuncDecl{shmem\_\FuncParam{TYPENAME}\_wait\_until\_some}@(TYPE *ivars, size_t nelems, size_t *indices,
    const int *status, int cmp, TYPE cmp_value);
\end{Csynopsis}
where \TYPE{} is one of the point-to-point synchronization types and has a
corresponding \TYPENAME{} specified by Table~\ref{p2psynctypes}.

\begin{apiarguments}

  \apiargument{IN}{ivars}{A pointer to an array of remotely accessible data
    objects.}
  \apiargument{IN}{nelems}{The number of elements in the \VAR{ivars} array.}
  \apiargument{OUT}{indices}{An array of indices of length at least
    \VAR{nelems} into \VAR{ivars} that satisfied the wait condition.}
  \apiargument{IN}{status}{An optional mask array of length \VAR{nelems}
    that indicates which elements in \VAR{ivars} are excluded from the wait set.}
  \apiargument{IN}{cmp}{A comparison operator from Table~\ref{p2p-consts}
    that compares elements of \VAR{ivars} with \VAR{cmp\_value}.}
  \apiargument{IN}{cmp\_value}{The value to be compared with the objects
    pointed to by \VAR{ivars}.}

\end{apiarguments}

\apidescription{ 
    The \FUNC{shmem\_wait\_until\_some} routine waits until at least one entry
    in the wait set specified by \VAR{ivars} and \VAR{status} satisfies the
    wait condition at the calling \ac{PE}.  This routine tests all elements of
    \VAR{ivars} in the wait set at least once, and the order in which the
    elements are waited upon is unspecified.

    Upon return, the \VAR{indices} array contains the indices of at least one
    element in the wait set that satisfied the wait condition during the call
    to \FUNC{shmem\_wait\_until\_some}.  The return value of
    \FUNC{shmem\_wait\_until\_some} is equal to the total number of these
    satisfied elements.  For a given return value $N$, the first $N$
    elements of the \VAR{indices} array contain those unique indices that
    satisfied the wait condition.
    These first $N$ elements of \VAR{indices} may be unordered with respect to
    the corresponding indices of \VAR{ivars}.
    The array pointed to by \VAR{indices} must
    be at least \VAR{nelems} long.  If an entry $i$ in \VAR{ivars} within the
    wait set satisfies the wait condition, a series of calls to
    \FUNC{shmem\_wait\_until\_some} must eventually include $i$ in the
    \VAR{indices} array.

    The optional \VAR{status} is a mask array of length \VAR{nelems} where each
    element corresponds to the respective element in \VAR{ivars} and indicates
    whether the element is excluded from the wait set.  Elements of
    \VAR{status} set to 0 will be included in the wait set, and elements set to
    1 will be ignored.  If all elements in \VAR{status} are set to 1 or
    \VAR{nelems} is 0, the wait set is empty and this routine returns 0.
    If \VAR{status} is a null pointer, it is ignored
    and all elements in \VAR{ivars} are included in the wait set.  The
    \VAR{ivars}, \VAR{indices}, and \VAR{status} arrays must not overlap in
    memory.
}


\apireturnvalues{
    \FUNC{shmem\_wait\_until\_some} returns the number of indices returned in
    the \VAR{indices} array. If the wait set is empty, this routine returns 0.
}

\apinotes{
    None.
}

\apiimpnotes{
    Implementations must ensure that \FUNC{shmem\_wait\_until\_some} does not
    return before the update of the memory indicated by the completed indices of \VAR{ivars} is fully
    executed.  Partial updates to the memory must not cause
    \FUNC{shmem\_wait\_until\_some} to return.
}


\begin{apiexamples}
  \apicexample
      {The following \Cstd[11] example demonstrates the use of
      \FUNC{shmem\_wait\_until\_some} to process a simple all-to-all transfer
      of $N$ data elements via a sum reduction.  This pattern is similar to the
      \FUNC{shmem\_wait\_until\_any} example above, but may reduce the number of
      iterations in the while loop.}
      {./example_code/shmem_wait_until_some_all2all_sum.c}
      {}

\end{apiexamples}

\end{apidefinition}
