\apisummary{
    Wait on some number of variables on the local \ac{PE} to change to a
    specified value.
}

\begin{apidefinition}

\begin{C11synopsis}
size_t @\FuncDecl{shmem\_wait\_until\_some}@(TYPE *ivars, size_t nelems, _Bool *status,
    int cmp, TYPE cmp_value);
\end{C11synopsis}
where \TYPE{} is one of the point-to-point synchronization types specified by
Table \ref{p2psynctypes}.

\begin{Csynopsis}
size_t @\FuncDecl{shmem\_\FuncParam{TYPENAME}\_wait\_until\_some}@(TYPE *ivars, size_t nelems, _Bool *status,
    int cmp, TYPE cmp_value);
\end{Csynopsis}
where \TYPE{} is one of the point-to-point synchronization types and has a
corresponding \TYPENAME{} specified by Table~\ref{p2psynctypes}.

\begin{apiarguments}

  \apiargument{OUT}{ivars}{A pointer to an array of remotely accessible data
    objects. The type of \VAR{ivars} should match that implied in the SYNOPSIS
    section.} 
  \apiargument{IN}{nelems}{The number of elements in the \VAR{ivars} array to
    be compared with \VAR{cmp\_value}.}
  \apiargument{INOUT}{status}{A mask array of length \VAR{nelems}, where each
    element corresponds to the respective element in \VAR{ivars} and represents
    the status of each implied condition.  On input, elements of \VAR{status}
    set to 0 will be waited upon, and elements set to 1 will be ignored.  On
    output, each element of \VAR{status} equal to 1 corresponds to a satisfied
    condition in the respective element of \VAR{ivars} (or an ignored
    condition, if initialized to 1 on input).}
    \apiargument{IN}{cmp}{A comparison operator from Table~\ref{p2p-consts}
    that compares elements of \VAR{ivars} with \VAR{cmp\_value}.}
  \apiargument{IN}{cmp\_value}{The value to be compared with the objects
    pointed to by \VAR{ivars}.  The type of \VAR{cmp\_value} should match that
    implied in the SYNOPSIS section.}

\end{apiarguments}

\apidescription{ 
    The \FUNC{shmem\_wait\_until\_some} routine waits for some specified
    symmetric object(s) in the \VAR{ivars} array to change to the value
    \VAR{cmp\_value} according to the numeric comparison operator \VAR{cmp}.
    This routine can be used for point-to-point direct synchronization
    involving multiple data objects, potentially avoiding the overhead
    associated with multiple individual calls to \FUNC{shmem\_wait\_until}.

    On input, the \VAR{status} array indicates the respective elements of
    \VAR{ivars} for which to wait until the condition implied by \VAR{cmp} and
    \VAR{cmp\_value} is satisfied.  For each element of \VAR{status}
    initialized to 0, \FUNC{shmem\_wait\_until\_some} sets that value to 1 if
    the respective \VAR{ivars} element has changed to satisfy the implied
    condition. Elements in \VAR{status} initialized to 1 are ignored and not
    checked for the implied condition.  If one or more elements of
    \VAR{status} are intialized to 0, \FUNC{shmem\_wait\_until\_some} does not
    return until a \ac{PE} changes one of the corresponding elements in
    \VAR{ivars} to satisfy the implied condition.  On output, the \VAR{status}
    array indicates which respective elements in \VAR{ivars} have satisfied the
    implied condition or have been ignored.  \FUNC{shmem\_wait\_until\_some}
    returns the total number of the newly satisfied conditions excluding the
    ignored elements.
}


\apireturnvalues{
    \FUNC{shmem\_wait\_until\_some} returns the number of elements in
    \VAR{ivars} that satisfied the condition implied by \VAR{cmp} and
    \VAR{cmp\_value}, excluding the number of ignored elements indicated by the
    initial values in \VAR{status}.
}

\apinotes{
    None.
}

\apiimpnotes{
    Implementations must ensure that \FUNC{shmem\_wait\_until\_some} does not
    return before the update of the memory indicated by \VAR{ivars} is fully
    complete.  Partial updates to the memory must not cause
    \FUNC{shmem\_wait\_until\_some} to return.
}


\begin{apiexamples}

\apicexample
{The following \CorCpp{} example demonstrates the use of 
\FUNC{shmem\_wait\_until\_some} to process a simple all-to-all transfer of N data elements via a sum reduction.}
{./example_code/shmem_wait_until_some_all2all_sum.c}
{}

\end{apiexamples}

\end{apidefinition}
