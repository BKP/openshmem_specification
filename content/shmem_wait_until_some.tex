\apisummary{
    Wait on some number of variables on the local \ac{PE} to change to a
    specified value.
}

\begin{apidefinition}

\begin{C11synopsis}
size_t @\FuncDecl{shmem\_wait\_until\_some}@(TYPE *ivars, size_t nelems, size_t *indices, int *status,
    int cmp, TYPE cmp_value);
\end{C11synopsis}
where \TYPE{} is one of the point-to-point synchronization types specified by
Table \ref{p2psynctypes}.

\begin{Csynopsis}
size_t @\FuncDecl{shmem\_\FuncParam{TYPENAME}\_wait\_until\_some}@(TYPE *ivars, size_t nelems, size_t *indices,
    int *status, int cmp, TYPE cmp_value);
\end{Csynopsis}
where \TYPE{} is one of the point-to-point synchronization types and has a
corresponding \TYPENAME{} specified by Table~\ref{p2psynctypes}.

\begin{apiarguments}

  \apiargument{OUT}{ivars}{A pointer to an array of remotely accessible data
    objects.}
  \apiargument{IN}{nelems}{The number of elements in the \VAR{ivars} array to
    be compared with \VAR{cmp\_value}.}
  \apiargument{OUT}{indices}{An array of indices of length at least
    \VAR{nelems} into \VAR{ivars} that satisfied the wait condition.}
  \apiargument{INOUT}{status}{An optional mask array of length \VAR{nelems}
    that indicates which elements in \VAR{ivars} are excluded from the wait set.}
  \apiargument{IN}{cmp}{A comparison operator from Table~\ref{p2p-consts}
    that compares elements of \VAR{ivars} with \VAR{cmp\_value}.}
  \apiargument{IN}{cmp\_value}{The value to be compared with the objects
    pointed to by \VAR{ivars}.}

\end{apiarguments}

\apidescription{ 
    The \FUNC{shmem\_wait\_until\_some} routine blocks until at least one
    specified value contained in the array of symmetric object(s), \VAR{ivars},
    satisfies the wait condition at the calling \ac{PE}.  This routine is
    semantically similar to \FUNC{shmem\_wait\_until} in
    Section~\ref{subsec:shmem_wait_until}, but adds support for point-to-point
    synchronization involving multiple data objects.

    After returning, the \VAR{indices} array contains the indices of some
    (but not necessarily all) of the elements in \VAR{ivars} that satisfied the
    wait condition during the call to \FUNC{shmem\_wait\_until\_some}.  The
    return value of \FUNC{shmem\_wait\_until\_some} is equal to the total
    number of these satisfied elements.  If the return value is $N$, then the first
    $N$ elements of the \VAR{indices} array contain those unique indices that
    satisfied the wait condition.

    The optional \VAR{status} is a mask array of length \VAR{nelems} where each
    element corresponds to the respective element in \VAR{ivars} and indicates
    whether the element is excluded from the wait set.  Elements in the wait
    set are those that are under consideration for a blocking wait operation.
    On input, elements of \VAR{status} set to 0 will be included in the wait
    set, and elements set to 1 will be ignored.  On output, each element of
    \VAR{status} equal to 1 corresponds to a satisfied or ignored condition in
    the respective element of \VAR{ivars}.  If \VAR{status} is a null pointer
    on input, it is ignored and no values are set on output.
}


\apireturnvalues{
    \FUNC{shmem\_wait\_until\_some} returns the number of elements in
    \VAR{ivars} that satisfied the condition implied by \VAR{cmp} and
    \VAR{cmp\_value}, excluding the number of ignored elements potentially
    indicated by the initial values in \VAR{status}.
}

\apinotes{
    None.
}

\apiimpnotes{
    Implementations must ensure that \FUNC{shmem\_wait\_until\_some} does not
    return before the update of the memory indicated by \VAR{ivars} is fully
    complete.  Partial updates to the memory must not cause
    \FUNC{shmem\_wait\_until\_some} to return.
}


\begin{apiexamples}
  \apicexample
      {The following \CorCpp{} example demonstrates the use of 
      \FUNC{shmem\_wait\_until\_some} to process a simple all-to-all transfer
      of N data elements via a sum reduction.}
      {./example_code/shmem_wait_until_some_all2all_sum.c}
      {}

\end{apiexamples}

\end{apidefinition}
