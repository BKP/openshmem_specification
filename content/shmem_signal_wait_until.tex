\apisummary{
    Wait for a variable on the local \ac{PE} to change from a signaling
    operation.
}

\begin{apidefinition}

\begin{Csynopsis}
uint64_t @\FuncDecl{shmem\_signal\_wait\_until}@(uint64_t *sig_addr, int cmp, uint64_t cmp_value);
\end{Csynopsis}

\begin{apiarguments}

\apiargument{IN}{sig\_addr}{A pointer to a remotely accessible variable.}
\apiargument{IN}{cmp}{The comparison operator that compares \VAR{sig\_addr} with
    \VAR{cmp\_value}.}
\apiargument{IN}{cmp\_value}{The value against which the object pointed to
    by \VAR{sig\_addr} will be compared.}

\end{apiarguments}

\apidescription{
    \FUNC{shmem\_signal\_wait\_until} operation blocks until the value contained
    in the signal data object, \VAR{sig\_addr}, at the calling \ac{PE} satisfies
    the wait condition. In an \openshmem program with single-threaded or
    multithreaded \acp{PE}, the \VAR{sig\_addr} object at the calling \ac{PE} is
    expected only to be updated as a signal, through the signaling operations
    available in Section~\ref{subsec:shmem_put_signal} and
    Section~\ref{subsec:shmem_put_signal_nbi}.

    This routine can be used to implement point-to-point synchronization between
    \acp{PE} or between threads within the same \ac{PE}. A call to this routine
    blocks until the value of \VAR{sig\_addr} at the calling \ac{PE} satisfies
    the wait condition specified by the comparison operator, \VAR{cmp}, and
    comparison value, \VAR{cmp\_value}.
}

\apireturnvalues{
    Return the contents of the signal data object, \VAR{sig\_addr}, at the
    calling \ac{PE} that satisfies the wait condition.
}


\apinotes{
    None.
}

\apiimpnotes{
    Implementations must ensure that \FUNC{shmem\_signal\_wait\_until} do not
    return before the update of the memory indicated by \VAR{sig\_addr} is fully
    complete. Partial updates to the memory must not cause
    \FUNC{shmem\_signal\_wait\_until} to return.
}

\end{apidefinition}
