\color{ForestGreen}

All \openshmem RMA, AMO, and memory ordering routines are
performed on a communication context.  The communication context defines an
independent ordering and completion environment, allowing users to manage the
overlap of communication with computation and also to manage communication
operations performed by separate threads within a multithreaded \ac{PE}.  For
example, in single-threaded environments, contexts can be used to pipeline
communication and computation.  In multithreaded environments, contexts can
additionally provide thread isolation, eliminating overheads resulting from
thread interference within the communication subsystem.

Context handles are of type \CODE{shmem\_ctx\_t} and are valid for
language-level assignment and comparison.  A handle to the desired context is
passed as an argument in the \CODE{shmem\_ctx\_*} API routines.  API routines
that do not accept a context argument operate on the default context.  The
default context can be used explicitly through the \const{SHMEM\_CTX\_DEFAULT}
handle.

\subsubsection{\textbf{SHMEM\_CTX\_CREATE}}
\label{subsec:shmem_ctx_create}
\apisummary{
    Create a communication context.
}

\begin{apidefinition}

\begin{Csynopsis}
int shmem_ctx_create(long options, shmem_ctx_t *ctx);
\end{Csynopsis}

\begin{apiarguments}
    \apiargument{IN}{options}{The set of options requested for the given context.
        Multiple options may be requested by combining them with a bitwise
        \textit{OR} operation.  Otherwise, $0$ can be given if no options are
        requested.}
    \apiargument{OUT}{ctx}{A handle to the newly created context.}
\end{apiarguments}

\apidescription{
    The \FUNC{shmem\_ctx\_create} routine creates a new communication context
    and returns its handle through the \VAR{ctx} argument.  If the context was
    created successfully, a value of zero is returned; otherwise a nonzero
    value is returned.  An unsuccessful context
    creation call is not treated as an error and the \openshmem library remains
    in a correct state.  The creation call can be reattempted with different
    options or after additional resources become available.

    The following constants can be supplied as context options: \\

        \apitablerow{\CONST{SHMEM\_CTX\_SHARED}}{
            In a multithreaded environment, the given context will be used by
            multiple threads concurrently.}
        \apitablerow{\CONST{SHMEM\_CTX\_PRIVATE}}{
            In a multithreaded environment, the given context will not be used
            by multiple threads concurrently.}
}

\apireturnvalues{
    Zero on success and nonzero otherwise.
}

\apinotes{
    None.
}

%\begin{apiexamples}
%
%    \apicexample
%    {The following example uses \FUNC{shmem\_p}  in a \Clang{} program.}	 
%    {./example_code/shmem_p_example.c}
%    {}
%
%\end{apiexamples}

\end{apidefinition}

\subsubsection{\textbf{SHMEM\_CTX\_DESTROY}}
\label{subsec:shmem_ctx_create}
\apisummary{
    Destroy a communication context.
}

\begin{apidefinition}

\begin{Csynopsis}
void shmem_ctx_destroy(shmem_ctx_t ctx);
\end{Csynopsis}

\begin{apiarguments}
    \apiargument{IN}{ctx}{Handle to the context that will be destroyed.}
\end{apiarguments}

\apidescription{
    \FUNC{shmem\_ctx\_destroy} destroys a context that was created by a call to
    \FUNC{shmem\_ctx\_create}.  This function performs an implicit quiet
    operation on the given context before it is freed.

    \textit{Rationale:} Destroying a context makes it impossible for the user to
    complete communication operations that are pending on that context.  This
    includes nonblocking communication operations, whose local buffers are only
    returned to the user after the operations have been completed.  An implicit
    quiet is performed when freeing a context to avoid this ambiguity.
}

\apireturnvalues{
    None.
}

\apinotes{
    None.
}

%\begin{apiexamples}
%
%    \apicexample
%    {The following example uses \FUNC{shmem\_p}  in a \Clang{} program.}	 
%    {./example_code/shmem_p_example.c}
%    {}
%
%\end{apiexamples}

\end{apidefinition}
\color{black}
