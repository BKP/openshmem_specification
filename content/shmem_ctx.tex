\color{ForestGreen}

All \openshmem RMA, AMO, and memory ordering routines are
performed on a communication context.  The communication context defines an
independent ordering and completion environment, allowing users to manage the
overlap of communication with computation and also to manage communication
operations performed by separate threads within a multithreaded \ac{PE}.  For
example, in single-threaded environments, contexts can be used to pipeline
communication and computation.  In multithreaded environments, contexts can
additionally provide thread isolation, eliminating overheads resulting from
thread interference within the communication subsystem.

Context handles are of type \CODE{shmem\_ctx\_t} and are valid for
language-level assignment and comparison.  A handle to the desired context is
passed as an argument in the \CODE{shmem\_ctx\_*} API routines.  API routines
that do not accept a context argument operate on the default context.  The
default context can be used explicitly through the \const{SHMEM\_CTX\_DEFAULT}
handle.

\subsubsection{\textbf{SHMEM\_CTX\_CREATE}}
\label{subsec:shmem_ctx_create}
\apisummary{
    Create a communication context.
}

\begin{apidefinition}

\begin{Csynopsis}
int shmem_ctx_create(long options, shmem_ctx_t *ctx);
\end{Csynopsis}

\begin{apiarguments}
    \apiargument{IN}{options}{The set of options requested for the given context.
        Multiple options may be requested by combining them with a bitwise
        \textit{OR} operation.  Otherwise, $0$ can be given if no options are
        requested.}
    \apiargument{OUT}{ctx}{A handle to the newly created context.}
\end{apiarguments}

\apidescription{
    The \FUNC{shmem\_ctx\_create} routine creates a new communication context
    and returns its handle through the \VAR{ctx} argument.  If the context was
    created successfully, a value of zero is returned; otherwise a nonzero
    value is returned.  An unsuccessful context
    creation call is not treated as an error and the \openshmem library remains
    in a correct state.  The creation call can be reattempted with different
    options or after additional resources become available.

    By default contexts are {\em shareable} and, when it is allowed by the
    threading model set during initialization, they can be used concurrently by
    multiple threads within the PE where they were created.
    %
    The following options can be supplied during context creation to restrict
    this usage model and enable performance optimizations.  When one or more
    options are provided on a context, the application must comply with the
    requirements of all provided options when using that context.  Otherwise
    the behavior is undefined.

        \apitablerow{\CONST{SHMEM\_CTX\_SERIALIZED}}{
            The given context will not be used by multiple threads
            concurrently.  When the \CONST{SHMEM\_CTX\_SERIALIZED} option is
            set, the user must ensure that operations involving the given
            context are serialized by the application, including operations
            that perform an explicit or implicit barrier.}

        \apitablerow{\CONST{SHMEM\_CTX\_PRIVATE}}{
            The given context will be used only by the thread that created it.
            Contexts with the \CONST{SHMEM\_CTX\_PRIVATE} option set are
            excluded from the quiet operation performed during explicit and
            implicit barrier operations.}
}

\apireturnvalues{
    Zero on success and nonzero otherwise.
}

\apinotes{
    None.
}

%\begin{apiexamples}
%
%    \apicexample
%    {The following example uses \FUNC{shmem\_p}  in a \Clang{} program.}	 
%    {./example_code/shmem_p_example.c}
%    {}
%
%\end{apiexamples}

\end{apidefinition}

\subsubsection{\textbf{SHMEM\_CTX\_DESTROY}}
\label{subsec:shmem_ctx_create}
\apisummary{
    Destroy a communication context.
}

\begin{apidefinition}

\begin{Csynopsis}
void shmem_ctx_destroy(shmem_ctx_t ctx);
\end{Csynopsis}

\begin{apiarguments}
    \apiargument{IN}{ctx}{Handle to the context that will be destroyed.}
\end{apiarguments}

\apidescription{
    \FUNC{shmem\_ctx\_destroy} destroys a context that was created by a call to
    \FUNC{shmem\_ctx\_create}.  It is the user's responsibility to ensure that
    the context is no longer in use before calling this routine.  This function
    performs an implicit quiet operation on the given context before it is freed.
}

\apireturnvalues{
    None.
}

\apinotes{
    Destroying a context makes it impossible for the user to complete
    communication operations that are pending on that context.  This includes
    nonblocking communication operations, whose local buffers are only returned
    to the user after the operations have been completed.  An implicit quiet is
    performed when freeing a context to avoid this ambiguity.

    A context with the \CONST{SHMEM\_CTX\_PRIVATE} option enabled must be
    destroyed by the thread that created it.
}

%\begin{apiexamples}
%
%    \apicexample
%    {The following example uses \FUNC{shmem\_p}  in a \Clang{} program.}	 
%    {./example_code/shmem_p_example.c}
%    {}
%
%\end{apiexamples}

\end{apidefinition}
\color{black}
