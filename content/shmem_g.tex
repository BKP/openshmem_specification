\apisummary{
    Transfers one data item from a remote \ac{PE}
}

\begin{apidefinition}

\begin{Csynopsis}
char shmem_char_g(|\aftergroup\colorswapnt|const|\aftergroup\prevcolor| char *addr, int pe);
short shmem_short_g(|\aftergroup\colorswapnt|const|\aftergroup\prevcolor| short *addr, int pe);
int shmem_int_g(|\aftergroup\colorswapnt|const|\aftergroup\prevcolor| int *addr, int pe);
long shmem_long_g(|\aftergroup\colorswapnt|const|\aftergroup\prevcolor| long *addr, int pe);
long long  shmem_longlong_g(|\aftergroup\colorswapnt|const|\aftergroup\prevcolor| long long *addr, int pe);
float shmem_float_g(|\aftergroup\colorswapnt|const|\aftergroup\prevcolor| float *addr, int pe);
double shmem_double_g(|\aftergroup\colorswapnt|const|\aftergroup\prevcolor| double *addr, int pe);
long double shmem_longdouble_g(|\aftergroup\colorswapnt|const|\aftergroup\prevcolor| long double *addr, int pe);
\end{Csynopsis}

\begin{apiarguments}
    \apiargument{IN}{addr}{The remotely accessible array element or scalar data object.}
    \apiargument{IN}{pe}{The number of the remote \ac{PE} on which \VAR{addr} resides.}
\end{apiarguments}

\apidescription{
  These routines provide a very low latency get capability for single elements
  of most basic types. 
}

\apireturnvalues{
    Returns a single element of type specified in the synopsis.
}

\apinotes{
    None.
}

\begin{apiexamples}

\apicexample
    {The following \FUNC{shmem\_long\_g} example is for \CorCpp{} programs:}
    {./example_code/shmem_g_example.c}
    {}
\end{apiexamples}

\end{apidefinition}
