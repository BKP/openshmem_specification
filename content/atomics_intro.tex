An \ac{AMO} is a one-sided communication mechanism that combines memory update
operations with atomicity guarantees described in Section
\ref{subsec:amo_guarantees}.  Similar to the \ac{RMA} routines, described in
Section \ref{sec:rma}, the \acp{AMO} are performed only on symmetric objects.
\openshmem{} defines the two types of \ac{AMO} routines:
\begin{itemize}
\item % Blocking\\
The \textit{fetching} routines return the original value of, and optionally
update, the remote data object in a single atomic operation.  The routines 
return after the data has been fetched and delivered to the local \ac{PE}.

The \textit{fetching} operations include: \FUNC{shmem\_fetch},
\FUNC{shmem\_cswap}, \FUNC{shmem\_swap}, \FUNC{shmem\_finc},
\FUNC{shmem\_fadd}, \\
\FUNC{shmem\_fand}, \FUNC{shmem\_for}, and \FUNC{shmem\_fxor}.


\item % Non-Blocking\\
The \textit{non-fetching} atomic routines update the remote memory in a single
atomic operation.  A \textit{non-fetching} atomic routine starts the atomic
operation and may return before the operation execution on the remote \ac{PE}.
To force completion for these \textit{non-fetching} atomic routines,
\FUNC{shmem\_quiet}, \FUNC{shmem\_barrier}, or \FUNC{shmem\_barrier\_all} can be
used by an \openshmem{} program. 

The \textit{non-fetching} operations include: \FUNC{shmem\_set},
\FUNC{shmem\_inc}, \FUNC{shmem\_add}, \FUNC{shmem\_and}, \FUNC{shmem\_or},
and \FUNC{shmem\_xor}.
\end{itemize}

Where appropriate compiler support is available, \openshmem{} provides
type-generic atomic memory operation interfaces via \Celev{} generic selection.
The type-generic \ac{AMO} routines each support the ``standard \ac{AMO} types''
listed in Table~\ref{stdamotypes}, except for \FUNC{shmem\_fadd},
\FUNC{shmem\_add}, \FUNC{shmem\_fetch}, \FUNC{shmem\_set}, and
\FUNC{shmem\_swap}, which support the ``extended \ac{AMO} types'' listed in
Table~\ref{extamotypes}.

\begin{table}[h]
  \begin{center}
    \begin{tabular}{|l|l|}
      \hline
      \TYPE              & \TYPENAME  \\ \hline
      int                & int        \\ \hline
      long               & long       \\ \hline
      long long          & longlong   \\ \hline
      unsigned int       & uint       \\ \hline
      unsigned long      & ulong      \\ \hline
      unsigned long long & ulonglong  \\ \hline
      int32\_t           & int32      \\ \hline
      int64\_t           & int64      \\ \hline
      uint32\_t          & uint32     \\ \hline
      uint64\_t          & uint64     \\ \hline
      intptr\_t          & intptr     \\ \hline
      uintptr\_t         & uintptr    \\ \hline
      size\_t            & size       \\ \hline
      ptrdiff\_t         & ptrdiff    \\ \hline
    \end{tabular}
    \caption{Standard \ac{AMO} Types and Names}
    \label{stdamotypes}
  \end{center} 
\end{table}

\begin{table}[h]
  \begin{center}
    \begin{tabular}{|l|l|}
      \hline
      \TYPE              & \TYPENAME  \\ \hline
      float              & float      \\ \hline
      double             & double     \\ \hline
      int                & int        \\ \hline
      long               & long       \\ \hline
      long long          & longlong   \\ \hline
      unsigned int       & uint       \\ \hline
      unsigned long      & ulong      \\ \hline
      unsigned long long & ulonglong  \\ \hline
      int32\_t           & int32      \\ \hline
      int64\_t           & int64      \\ \hline
      uint32\_t          & uint32     \\ \hline
      uint64\_t          & uint64     \\ \hline
      intptr\_t          & intptr     \\ \hline
      uintptr\_t         & uintptr    \\ \hline
      size\_t            & size       \\ \hline
      ptrdiff\_t         & ptrdiff    \\ \hline
    \end{tabular}
    \caption{Extended \ac{AMO} Types and Names}
    \label{extamotypes}
  \end{center} 
\end{table}
