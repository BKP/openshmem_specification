\apisummary{
    shmem\_alltoall is a collective routine where each \ac{PE} exchanges a fixed amount of data with all other \acp{PE} \oldtext{in the active set} \newtext{participating in the collective}.
}

\begin{apidefinition}

%% C11
{\color{Green}
\begin{C11synopsis}
int @\FuncDecl{shmem\_alltoall32}@(void *dest, const void *source, size_t nelems, shmem_team_t team);
int @\FuncDecl{shmem\_alltoall64}@(void *dest, const void *source, size_t nelems, shmem_team_t team);
\end{C11synopsis}
}

\begin{Csynopsis}
\end{Csynopsis}
{\color{Green}
\begin{CsynopsisCol}
int @\FuncDecl{shmem\_team\_alltoall32}@(void *dest, const void *source, size_t nelems, shmem_team_t team);
int @\FuncDecl{shmem\_team\_alltoall64}@(void *dest, const void *source, size_t nelems, shmem_team_t team);
\end{CsynopsisCol}
}

\begin{DeprecateBlock}
\begin{CsynopsisCol}
void @\FuncDecl{shmem\_alltoall32}@(void *dest, const void *source, size_t nelems, int PE_start, int logPE_stride, int PE_size, long *pSync);
void @\FuncDecl{shmem\_alltoall64}@(void *dest, const void *source, size_t nelems, int PE_start, int logPE_stride, int PE_size, long *pSync);
\end{CsynopsisCol}
\end{DeprecateBlock}

\begin{Fsynopsis}
INTEGER pSync(SHMEM_ALLTOALL_SYNC_SIZE)
INTEGER PE_start, logPE_stride, PE_size, nelems
CALL @\FuncDecl{SHMEM\_ALLTOALL32}@(dest, source, nelems, PE_start, logPE_stride, PE_size, pSync)
CALL @\FuncDecl{SHMEM\_ALLTOALL64}@(dest, source, nelems, PE_start, logPE_stride, PE_size, pSync)
\end{Fsynopsis}

\begin{apiarguments}

\apiargument{OUT}{dest}{A symmetric data object large enough to receive
    the combined total of \VAR{nelems} elements from each \ac{PE} in the
    active set.}
\apiargument{IN}{source}{A symmetric data object that contains \VAR{nelems}
    elements of data for each \ac{PE} in the active set, ordered according to
    destination \ac{PE}.}
\apiargument{IN}{nelems}{The number of elements to exchange for each \ac{PE}.
    \VAR{nelems} must be of type size\_t for \CorCpp.  When using
    \Fortran, it must be a default integer value.}

\newtext{%
\apiargument{IN}{team}{A valid \openshmem team handle to a team.}
}

\begin{DeprecateBlock}
\apiargument{IN}{PE\_start}{The lowest \ac{PE} number of the active set of
    \acp{PE}.  \VAR{PE\_start} must be of type integer.  When using \Fortran,
    it must be a default integer value.}
\apiargument{IN}{logPE\_stride}{The log (base 2) of the stride between
    consecutive \ac{PE} numbers in the active set.  \VAR{logPE\_stride} must be of
    type integer.  When using \Fortran, it must be a default integer value.}
\apiargument{IN}{PE\_size}{The number of \acp{PE} in the active set.
    \VAR{PE\_size} must be of type integer.  When using \Fortran, it must
    be a default integer value.}
\apiargument{IN}{pSync}{
    A symmetric work array of size \CONST{SHMEM\_ALLTOALL\_SYNC\_SIZE}.
    In \CorCpp, \VAR{pSync} must be an array of elements of type \CTYPE{long}.
    In \Fortran, \VAR{pSync} must be an array of elements of default integer type.
    Every element of this array must be initialized with the value
    \CONST{SHMEM\_SYNC\_VALUE} before any of the \acp{PE} in the active set
    enter the routine.}
\end{DeprecateBlock}

\end{apiarguments}

\apidescription{
    The \FUNC{shmem\_alltoall} routines are collective routines. Each \ac{PE}
    \oldtext{in the active set} \newtext{participating in the operation}
    exchanges \VAR{nelems} data elements of size
    32 bits (for \FUNC{shmem\_alltoall32}) or 64 bits (for \FUNC{shmem\_alltoall64})
    with all other \acp{PE} \oldtext{in the set}
    \newtext{participating in the operation}. The data being sent and received are
    stored in a contiguous symmetric data object. The total size of each \acp{PE}
    \VAR{source} object and \VAR{dest} object is \VAR{nelems} times the size of
    an element (32 bits or 64 bits) times \oldtext{\VAR{PE\_size}}
    \newtext{\VAR{N}, where \VAR{N} equals the number of \acp{PE} participating
    in the operation}.
    The \VAR{source} object contains oldtext{\VAR{PE\_size}} \VAR{N} blocks of data
    (where the size of each block is defined by \VAR{nelems}) and each block of data
    is sent to a different \ac{PE}.

    \newtext{The same \dest{} and \source{}
    arrays, and same value for \newtext{nelems}
    must be passed by all \acp{PE} that participate in the collective.}

    Given a \ac{PE} \VAR{i} that is the \kth \ac{PE} \oldtext{in the active set}
    \newtext{participating in the operation} and a \ac{PE}
    \VAR{j} that is the \lth \ac{PE} \oldtext{in the active set}
    \newtext{participating in the operation},
    \ac{PE} \VAR{i} sends the \lth block of its \VAR{source} object to
    the \kth block of
    the \VAR{dest} object of \ac{PE} \VAR{j}.

{\color{Green}
    Team-based collect routines operate over all \acp{PE} in the provided team
    argument. All \acp{PE} in the provided team must participate in the collective.

    Active-set-based collective routines operate over all \acp{PE} in the active set
    defined by the \VAR{PE\_start}, \VAR{logPE\_stride}, \VAR{PE\_size} triplet.
}
    As with all \oldtext{\openshmem} \newtext{active-set-based} collective routines,
    this routine assumes that only \acp{PE} in the active set call the routine.
    If a \ac{PE} not in the active set calls an  \oldtext{\openshmem}
    \newtext{active-set-based} collective routine,
    the behavior is undefined.

    The values of arguments \oldtext{\VAR{nelems},} \VAR{PE\_start}, \VAR{logPE\_stride},
    and \VAR{PE\_size} must be equal on all \acp{PE} in the active set. The same
    \oldtext{\VAR{dest} and \VAR{source} data objects, and the same} \VAR{pSync} work
    array must be passed to all \acp{PE} in the active set.

    Before any \ac{PE} calls a \FUNC{shmem\_alltoall} routine,
    the following conditions must be ensured:
    \begin{itemize}
    \item The \VAR{dest} data object on all \acp{PE} in the active set is
      ready to accept the \FUNC{shmem\_alltoall} data.
    \item \newtext{For active-set-based routines}, the \VAR{pSync} array
    on all \acp{PE} in the active set is not still in use from a prior call
    to a \FUNC{shmem\_alltoall} routine.
    \end{itemize}
    Otherwise, the behavior is undefined.

    Upon return from a \FUNC{shmem\_alltoall} routine, the following is true for
    the local PE:
    \begin{itemize}
    \item Its \VAR{dest} symmetric data object is completely updated and
    the data has been copied out of the \VAR{source} data object.
    \item \newtext{For active-set-based routines, }
    the values in the \VAR{pSync} array are restored to the original values.
    \end{itemize}
}

\apidesctable{
The  \dest{}  and \source{} data  objects must conform to certain typing
constraints, which are as follows:
}{Routine}{Data type of \VAR{dest} and \VAR{source}}

\apitablerow{shmem\_alltoall64}{\CONST{64} bits aligned.}
\apitablerow{shmem\_alltoall32}{\CONST{32} bits aligned.}

\apireturnvalues{
    \newtext{Zero on successful local completion. Nonzero otherwise.}
}

\apinotes{
\newtext{%
    There are no specifically defined error codes for these routines.
    See section \ref{subsec:error_handling} for expected error checking and
    return code behavior specific to implementations. For portable
    error checking and debugging behavior, programs should do their own checks
    for invalid team handles or \LibConstRef{SHMEM\_TEAM\_NULL}.
}

    This routine restores \VAR{pSync} to its original contents.  Multiple calls
    to \openshmem\ routines that use the same \VAR{pSync} array do not require
    that \VAR{pSync} be reinitialized after the first call.
    The user must ensure that the \VAR{pSync} array is not being updated by any
    \ac{PE} in the active set while any of the \acp{PE} participates in
    processing of an \openshmem\ \FUNC{shmem\_alltoall} routine. Be careful to
    avoid these situations: If the \VAR{pSync} array is initialized at run time,
    some type of synchronization is needed to ensure that all \acp{PE} in the
    active set have initialized \VAR{pSync} before any of them enter an
    \openshmem\ routine called with the \VAR{pSync} synchronization array.  A
    \VAR{pSync} array may be reused on a subsequent \openshmem\
    \FUNC{shmem\_alltoall} routine only if none of the \acp{PE} in the
    active set are still processing a prior \openshmem\ \FUNC{shmem\_alltoall}
    routine call that used the same \VAR{pSync} array.  In general, this can be
    ensured only by doing some type of synchronization.
}

\begin{apiexamples}

\apicexample
    {This example shows a \FUNC{shmem\_alltoall64} on two long elements among all
    \acp{PE}.}
    {./example_code/shmem_alltoall_example.c}
    {}

\end{apiexamples}

\end{apidefinition}

