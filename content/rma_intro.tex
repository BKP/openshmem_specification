The \ac{RMA} routines described in this section can be used to perform
asynchronous reads from  and writes to symmetric data objects. These operations
are one-sided, meaning that the \ac{PE} invoking an operation provides all
communication parameters and the targeted \ac{PE} is passive. A characteristic
of one-sided communication is that it decouples communication from
synchronization. One-sided communication mechanisms transfer data; however,
they do not synchronize the sender of the data with the receiver of the data.

\openshmem \ac{RMA} routines are performed on the symmetric data objects.  The
initiator \ac{PE} of a call is designated as the \emph{origin} \ac{PE} and the
PE targeted by an operation is designated as the \emph{destination} PE.  The
\source{} and \dest{} designators refer to the data objects that an operation
reads from and writes to.  In the case of the remote update routine, \PUT{},
the origin \ac{PE} provides the \source{} data object and the destination
\ac{PE} provides the \dest{} data object. In the case of the remote read
routine, \GET{}, the origin \ac{PE} provides the \dest{} data object and the
destination \ac{PE} provides the \source{} data object.

Where appropriate compiler support is available, \openshmem provides type-generic 
one-sided communication interfaces via \Cstd[11] generic selection
(\Cstd[11]~\S6.5.1.1\footnote{Formally, the \Cstd[11] specification is ISO/IEC 9899:2011(E).})
for block, scalar, and block-strided put and get communication. 
Such type-generic routines are supported for the ``standard \ac{RMA} types''
listed in Table \ref{stdrmatypes}.

The standard \ac{RMA} types include the exact-width integer types defined in
\HEADER{stdint.h} by \Cstd[99]%
\footnote{Formally, the \Cstd[99] specification is ISO/IEC~9899:1999(E).}%
~\S7.18.1.1 and \Cstd[11]~\S7.20.1.1. When the \Cstd translation environment
does not provide exact-width integer types with \HEADER{stdint.h}, an
\openshmem implemementation is not required to provide support for these types.

\begin{table}[h]
  \begin{center}
    \begin{tabular}{|l|l|}
      \hline
      \TYPE              & \TYPENAME  \\ \hline
      float              & float      \\ \hline
      double             & double     \\ \hline
      long double        & longdouble \\ \hline
      char               & char       \\ \hline
      signed char        & schar      \\ \hline
      short              & short      \\ \hline
      int                & int        \\ \hline
      long               & long       \\ \hline
      long long          & longlong   \\ \hline
      unsigned char      & uchar      \\ \hline
      unsigned short     & ushort     \\ \hline
      unsigned int       & uint       \\ \hline
      unsigned long      & ulong      \\ \hline
      unsigned long long & ulonglong  \\ \hline
      int8\_t            & int8       \\ \hline
      int16\_t           & int16      \\ \hline
      int32\_t           & int32      \\ \hline
      int64\_t           & int64      \\ \hline
      uint8\_t           & uint8      \\ \hline
      uint16\_t          & uint16     \\ \hline
      uint32\_t          & uint32     \\ \hline
      uint64\_t          & uint64     \\ \hline
      size\_t            & size       \\ \hline
      ptrdiff\_t         & ptrdiff    \\ \hline
    \end{tabular}
    \TableCaptionRef{Standard \ac{RMA} Types and Names}
    \label{stdrmatypes}
  \end{center} 
\end{table}
