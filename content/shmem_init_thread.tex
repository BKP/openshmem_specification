\apisummary{
Initializes the OpenSHMEM library, similar to \FUNC{shmem\_init}, and performs any
initialization required for supporting the provided thread level.
}

\begin{apidefinition}

\begin{Csynopsis}
int shmem_init_thread(int requested, int *provided);
\end{Csynopsis}

\begin{apiarguments}
\apiargument{IN}{requested}{The thread level support requested by the user.}
\apiargument{OUT}{provided}{The thread level support provided by the \openshmem implementation.}
\end{apiarguments}

\apidescription{
\FUNC{shmem\_init\_thread} initializes the \openshmem library in the same way as 
\FUNC{shmem\_init}. In addition, \FUNC{shmem\_init\_thread} also performs 
the initialization required for supporting the provided thread level. 
The argument \VAR{requested} is used to specify the desired level of 
thread support. The argument \VAR{provided} returns the support level 
provided by the library. The allowed values for \VAR{provided} and 
\VAR{requested} are \CONST{SHMEM\_THREAD\_SINGLE}, \CONST{SHMEM\_THREAD\_FUNNELED},
\CONST{SHMEM\_THREAD\_SERIALIZED}, or \CONST{SHMEM\_THREAD\_MULTIPLE}.

An \openshmem program is initialized either by \FUNC{shmem\_init} or \FUNC{shmem\_init\_thread}. 
Similar to \FUNC{shmem\_init}, the \FUNC{shmem\_init\_thread} routine may not 
be called multiple times in an \openshmem program. If the call to \FUNC{shmem\_init\_thread} 
is unsuccessful in allocating and initializing resources for the 
\openshmem library, then the behavior of any subsequent call 
to the \openshmem library is undefined.
}

\apireturnvalues{
\FUNC{shmem\_init\_thread} returns 0 upon success; otherwise, it returns a
non-zero value.
}

\apinotes{
The \openshmem library can be initialized either by \FUNC{shmem\_init} 
or \FUNC{shmem\_init\_thread}. If the \openshmem library is initialized 
by \FUNC{shmem\_init}, the library implementation can choose to 
support one of the defined thread levels.
}

\end{apidefinition}
