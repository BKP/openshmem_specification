\apisummary{
Initializes the OpenSHMEM library, similar to \FUNC{shmem\_init}, and performs any
initialization required for supporting the four thread levels.}

%Initializes the \openshmem{} library, similar to \FUNC{shmem\_init}, and in addition performs the initialization required for thread-safe invocation of \openshmem{} functions.}

\begin{apidefinition}

\begin{Csynopsis}
void shmem_init_thread(int requested, int *provided);
\end{Csynopsis}

\begin{apiarguments}
\apiargument{IN}{requested}  {The thread level support requested by the user.
}
\apiargument{OUT}{provided}{The thread level support provided by the \openshmem{} implementation.}
\end{apiarguments}


\apidescription{
\FUNC{shmem\_init\_thread}  initializes the \openshmem{} library similar to
\FUNC{shmem\_init}, and in addition perform the initialization required for
supporting the four thread levels. The argument
\VAR{requested} is used to specify the desired level of thread support.
The function returns the support level provided by the library.
The correct values for \VAR{provided} and \VAR{requested} are 
\CONST{SHMEM\_THREAD\_SINGLE}, \newline 
\CONST{SHMEM\_THREAD\_FUNNELED}, \CONST{SHMEM\_THREAD\_SERIALIZED}, or 
\CONST{SHMEM\_THREAD\_MULTIPLE}.
     
An \openshmem{} program is initialized either by \FUNC{shmem\_init} or
\FUNC{shmem\_init\_thread}. If thread-safe invocation of \openshmem{} functions
is desired, then the \openshmem{} program should be initialized by
\FUNC{shmem\_init\_thread}. Similar to \FUNC{shmem\_init}, the
\FUNC{shmem\_init\_thread} may not be called multiple times in an \openshmem{} program.
}

\apireturnvalues{    
	None
    }

\apinotes{
    None
}		


\end{apidefinition}

 
  
