
\apisummary{
    Collective memory allocation routine with support for providing hints.
}

\begin{apidefinition}

\begin{Csynopsis}
void *@\FuncDecl{shmem\_malloc\_with\_hints}@(size_t size, long hints);
\end{Csynopsis}

\begin{apiarguments}
    \apiargument{IN}{size}{The size, in bytes, of a block to be
        allocated from the symmetric heap. This argument is of type \CTYPE{size\_t}}
    \apiargument{IN}{hints}{A bit array of hints provided by the user to the implementation}
\end{apiarguments}


\apidescription{

    The \FUNC{shmem\_malloc\_with\_hints} routine, like \FUNC{shmem\_malloc}, returns a pointer to a block of at least
    \VAR{size} bytes, which shall be suitably aligned so that it may be
    assigned to a pointer to any type of object.  This space is allocated from
    the symmetric heap (similar to \FUNC{shmem\_malloc}).  When the \VAR{size} is zero, 
    the \FUNC{shmem\_malloc\_with\_hints} routine performs no action and returns a null pointer. 
    
    In addition to the \VAR{size} argument, the \VAR{hints} argument is provided by the user. 
    The \VAR{hints} describes the expected manner in which the \openshmem program may use the allocated memory.
    The valid usage hints are described in Table~\ref{usagehints}. Multiple hints may be requested by combining them with a bitwise \CONST{OR} operation.
	A zero option can be given if no options are requested.
    
    The information provided by the \VAR{hints} is used to optimize for performance by the implementation. 
    If the implementation cannot optimize, the behavior is same as \FUNC{shmem\_malloc}.
    If more than one hint is provided, the implementation will make the best effort to use one or more hints 
    to optimize performance. 
            
    The \FUNC{shmem\_malloc\_with\_hints} routine is provided  so that multiple \acp{PE} in a program can allocate symmetric,
    remotely accessible memory blocks.  When no action is performed, these
    routines return without performing a barrier. Otherwise, the routine will call a barrier on \LibHandleRef{SHMEM\_TEAM\_WORLD} on exit.
    This ensures that all \acp{PE} participate in the memory allocation, and that the memory on other
    \acp{PE} can be used as soon as the local \ac{PE} returns. The implicit barrier performed by this routine will quiet the
    default context.  It is the user's responsibility to ensure that no communication operations involving the given memory block are pending on
    other contexts prior to calling the \FUNC{shmem\_free} and \FUNC{shmem\_realloc} routines.
    The user is also responsible for calling these routines with identical argument(s) on all
    \acp{PE}; if differing \VAR{size}, or \VAR{hints} arguments are used, the behavior of the call
    and any subsequent \openshmem calls is undefined.
}

\apireturnvalues{
    The \FUNC{shmem\_malloc\_with\_hints} routine returns a pointer to the allocated space;
    otherwise, it returns a null pointer.
}

	\begin{longtable}{|p{0.45\textwidth}|p{0.5\textwidth}|}
	\hline
	\textbf{Hints} & \textbf{Usage hint}
	\tabularnewline \hline
	\endhead
	%%
	\newline
	\CONST{0} &
	\newline
	Behavior same as \FUNC{shmem\_malloc}
	\tabularnewline \hline
	
		
	\LibConstDecl{SHMEM\_MALLOC\_ATOMICS\_REMOTE} &
	\newline 
	Memory used for \VAR{atomic} operations
	\tabularnewline \hline
	
	\LibConstDecl{SHMEM\_MALLOC\_SIGNAL\_REMOTE} &
	\newline
	Memory used for \VAR{signal} operations
	\tabularnewline \hline

	\TableCaptionRef{Memory usage hints}
        \label{usagehints}
    \end{longtable}

\apinotes{
		The \openshmem programs should allocate memory with
		\CONST{SHMEM\_MALLOC\_ATOMICS\_REMOTE}, when the majority of
		operations performed on this memory are atomic operations, and origin
		and target \ac{PE} of the atomic operations do not share a memory domain
		.i.e., symmetric objects on the target \ac{PE} is not accessible using
		load/store operations from the origin \ac{PE} or vice versa.
}
\end{apidefinition}
\newpage
