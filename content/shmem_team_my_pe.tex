\apisummary{
    shmem\_team\_my\_pe returns the calling process's virtual rank in the
    provided team.
}

\begin{apidefinition}

\begin{Csynopsis}
int @\FuncDecl{shmemx\_team\_my\_pe}@(shmem_team_t newteam);
\end{Csynopsis}

\begin{apiarguments}
\apiargument{IN}{newteam}{A valid SHMEM team handle.}
\end{apiarguments}

\apidescription{
The shmemx\_team\_my\_pe function returns the calling process's virtual
rank in the provided team. The rank will be a value between 0 and N-1,
for a team of size N. Different members of a team cannot have the same
rank. For the team SHMEM\_TEAM\_WORLD, this will return shmem\_my\_pe.

Error checking will be done to ensure a valid team handle is provided.
All errors are considered fatal, and will result in the job aborting
with an informative error message.
}

\apireturnvalues{
Calling process's virtual rank in the provided team.
}

\apinotes{
By default, SHMEM creates two predefined teams that will be available
for use once the routine start\_pes has been called. These teams can be
referenced in the application by the constants SHMEM\_TEAM\_WORLD and
SHMEM\_TEAM\_NODE. Every PE process is a member of the SHMEM\_TEAM\_WORLD
team, and its rank in SHMEM\_TEAM\_WORLD corresponds to the value of its
global PE rank. The SHMEM\_TEAM\_NODE team only contains the set of PEs
that reside on the same node as the current PE.
}
