\apisummary{
    Wait on an array of variables on the local \ac{PE} until at least one
    variable meets the its specified wait condition.
}

\begin{apidefinition}

\begin{C11synopsis}
size_t @\FuncDecl{shmem\_wait\_until\_some\_vector}@(TYPE *ivars, size_t nelems, size_t *indices,
    const int *status, int cmp, TYPE *cmp_values);
\end{C11synopsis}
where \TYPE{} is one of the standard \ac{AMO} types specified by
Table \ref{stdamotypes}.

\begin{Csynopsis}
size_t @\FuncDecl{shmem\_\FuncParam{TYPENAME}\_wait\_until\_some\_vector}@(TYPE *ivars, size_t nelems, size_t *indices,
    const int *status, int cmp, TYPE *cmp_values);
\end{Csynopsis}
where \TYPE{} is one of the standard \ac{AMO} types and has a
corresponding \TYPENAME{} specified by Table~\ref{stdamotypes}.

\begin{apiarguments}

  \apiargument{IN}{ivars}{Symmetric address of an array of remotely accessible data
    objects.
    The type of \VAR{ivars} should match that implied in the SYNOPSIS section.}
  \apiargument{IN}{nelems}{The number of elements in the \VAR{ivars} array.}
  \apiargument{OUT}{indices}{Local address of an array of indices of length at least
    \VAR{nelems} into \VAR{ivars} that satisfied the wait condition.}
  \apiargument{IN}{status}{Local address of an optional mask array of length \VAR{nelems}
    that indicates which elements in \VAR{ivars} are excluded from the wait set.}
  \apiargument{IN}{cmp}{A comparison operator from Table~\ref{p2p-consts} that
    compares elements of \VAR{ivars} with elements of \VAR{cmp\_values}.}
  \apiargument{IN}{cmp\_values}{Local address of an array of length \VAR{nelems}
    containing values to be compared with the respective objects in \VAR{ivars}.
    The type of \VAR{cmp\_values} should match that implied in the SYNOPSIS section.}

\end{apiarguments}

\apidescription{ 
    The \FUNC{shmem\_wait\_until\_some\_vector} routine waits until
    at least one entry in the wait set specified by \VAR{ivars} and
    \VAR{status} satisfies the wait condition at the calling \ac{PE}.
    The \VAR{ivars} objects at the calling \ac{PE} may be updated by an
    \ac{AMO} performed by a thread located within the calling \ac{PE} or within
    another \ac{PE}.
    This routine compares each element of the \VAR{ivars} array in the
    wait set with each respective value in \VAR{cmp\_values} according to the
    comparison operator \VAR{cmp} at the calling \ac{PE}.  This routine tests
    all elements of \VAR{ivars} in the wait set at least once, and the order in
    which the elements are waited upon is unspecified.

    Upon return, the \VAR{indices} array contains the indices of at least one
    element in the wait set that satisfied the wait condition during the call
    to \FUNC{shmem\_wait\_until\_some\_vector}.  The return value of
    \FUNC{shmem\_wait\_until\_some\_vector} is equal to the total
    number of these satisfied elements.  For a given return value $N$, the
    first $N$ elements of the \VAR{indices} array contain those unique indices
    that satisfied the wait condition.  These first $N$ elements of
    \VAR{indices} may be unordered with respect to the corresponding indices of
    \VAR{ivars}.  The array pointed to by \VAR{indices} must be at least
    \VAR{nelems} long.  If an entry $i$ in \VAR{ivars} within the wait set
    satisfies the wait condition, a series of calls to
    \FUNC{shmem\_wait\_until\_some\_vector} must eventually include
    $i$ in the \VAR{indices} array.

    The optional \VAR{status} is a mask array of length \VAR{nelems} where each
    element corresponds to the respective element in \VAR{ivars} and indicates
    whether the element is excluded from the wait set.  Elements of
    \VAR{status} set to 0 will be included in the wait set, and elements set to
    a nonzero value will be ignored.  If all elements in \VAR{status} are nonzero or
    \VAR{nelems} is 0, the wait set is empty and this routine returns 0.
    If \VAR{status} is a null pointer, it is ignored
    and all elements in \VAR{ivars} are included in the wait set.  The
    \VAR{ivars}, \VAR{indices}, and \VAR{status} arrays must not overlap in
    memory.

    Implementations must ensure that \FUNC{shmem\_wait\_until\_some\_vector}
    does not return before the update of the memory indicated by \VAR{ivars} is
    fully complete.
}


\apireturnvalues{
    \FUNC{shmem\_wait\_until\_some\_vector} returns the number of
    indices returned in the \VAR{indices} array. If the wait set is empty, this
    routine returns 0.
}


\end{apidefinition}
