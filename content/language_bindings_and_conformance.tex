\openshmem provides ISO \Cstd language bindings. Any implementation that
provides \Cstd bindings can claim conformance to the specification. The
\openshmem header file \HEADER{shmem.h} for \Cstd must contain only the
interfaces and constant names defined in this specification.

\openshmem \acp{API} can be implemented as functions, inline functions, or
macros. However, implementing the interfaces using inline functions or macros
could limit the use of external profiling tools and high-level compiler
optimizations. An \openshmem program should avoid defining routine names,
variables, or other identifiers with the prefix ``shmem'' using any combination
of uppercase letters, lowercase letters, and underscores.

All extensions to the \openshmem \ac{API} that are not part of this
specification must be defined in the \HEADER{shmemx.h} header file, with the
following exceptions.

\begin{enumerate}
    \item Extensions to the \openshmem interfaces that add support for
        additional datatypes.
    \item Implementation-specific constants, types, and macros that use a
        consistent, implementation-defined prefix.
    \item Extensions to the type-generic interfaces.
\end{enumerate}

The \HEADER{shmemx.h}
header file must exist, even if no extensions are provided. Any extensions
shall use the \FUNC{shmemx\_} prefix for all routine, variable, and constant
names.
