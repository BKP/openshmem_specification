\apisummary{
    Destroys existing team.
}

\begin{apidefinition}

\begin{Csynopsis}
int @\FuncDecl{shmem\_team\_destroy}@(shmem_team_t team);
\end{Csynopsis}

\begin{apiarguments}
\apiargument{IN}{team}{A valid \openshmem team handle.}
\end{apiarguments}

\apidescription{
The \FUNC{shmem\_team\_destroy} function destroys an existing team. This is a
collective call, in which every member of the team being destroyed needs
to participate. This will free all internal memory structures associated
with the team and invalidate the team handle. Upon return, the team
handle can no longer be used for team API calls.

It is considered erroneous to free \LibHandleRef{SHMEM\_TEAM\_WORLD} or
\LibHandleRef{SHMEM\_TEAM\_NODE}. Error checking will be done to ensure a valid
team handle is provided. Errors will result in a return value less than \CONST{0}.
}

\begin{FeedbackRequest}
\apireturnvalues{
On success, the function will return 0. Otherwise a value less than
\CONST{0} will be returned.
}
\end{FeedbackRequest}

\apinotes{
Note that \openshmem team handles have local semantics only. That is, team
handles should not be stored in shared variables and used across other
processes. If a team handle or its value is used by any \ac{PE} other than
that which created it, the behavior is undefined.
}

\end{apidefinition}
