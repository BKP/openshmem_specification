\apisummary{
    Destroys existing team.
}

\begin{apidefinition}

\begin{Csynopsis}
int @\FuncDecl{shmem\_team\_destroy}@(shmem_team_t *team);
\end{Csynopsis}

\begin{apiarguments}
\apiargument{INOUT}{team}{A pointer to a valid \openshmem team handle.}
\end{apiarguments}

\apidescription{
The \FUNC{shmem\_team\_destroy} routine destroys an existing team. This is a
collective call, in which every member of the team being destroyed needs
to participate. This will free all internal memory structures associated
with the team and invalidate the team handle. Upon return, the team
handle can no longer be used for team API calls.

It is considered erroneous to free \LibHandleRef{SHMEM\_TEAM\_WORLD} or
any other predefined team.

If a pointer to an invalid handle is provided, the behavior is undefined.

If the pointer to \VAR{team} is a null pointer, then no team is destroyed,
and a nonzero value is returned.

After returning from the routine, if the team was successfully destroyed,
the handle will be assigned the value \LibConstRef{SHMEM\_TEAM\_INVALID}.

Team destruction assumes that any resources explicitly created from the team,
such as contexts created from the team, have already been released through
the appropriate function, such as destroying the context. If there are any
objects or resources explicitly created from the team that have not been
explicitly released before \FUNC{shmem\_team\_destroy} is called, behavior is
undefined. 
}

\apireturnvalues{
Zero upon successful destruction of the team, nonzero otherwise.
}

\apinotes{
None.
}

\end{apidefinition}
