\apisummary{
    The nonblocking put routines provide a method for copying data
    from a contiguous local data object to a data object on a specified \ac{PE}. 
}

\begin{apidefinition}

\begin{C11synopsis}
void @\FuncDecl{shmem\_put\_nbi}@(TYPE *dest, const TYPE *source, size_t nelems, int pe);
void @\FuncDecl{shmem\_put\_nbi}@(shmem_ctx_t ctx, TYPE *dest, const TYPE *source, size_t nelems, int pe);
\end{C11synopsis}
where \TYPE{} is one of the standard \ac{RMA} types specified by Table \ref{stdrmatypes}.

\begin{Csynopsis}
void @\FuncDecl{shmem\_\FuncParam{TYPENAME}\_put\_nbi}@(TYPE *dest, const TYPE *source, size_t nelems, int pe);
void @\FuncDecl{shmem\_ctx\_\FuncParam{TYPENAME}\_put\_nbi}@(shmem_ctx_t ctx, TYPE *dest, const TYPE *source, size_t nelems, int pe);
\end{Csynopsis}
where \TYPE{} is one of the standard \ac{RMA} types and has a corresponding \TYPENAME{} specified by Table \ref{stdrmatypes}.

\begin{CsynopsisCol}
void @\FuncDecl{shmem\_put\FuncParam{SIZE}\_nbi}@(void *dest, const void *source, size_t nelems, int pe);
void @\FuncDecl{shmem\_ctx\_put\FuncParam{SIZE}\_nbi}@(shmem_ctx_t ctx, void *dest, const void *source, size_t nelems, int pe);
\end{CsynopsisCol}
where \SIZE{} is one of \CONST{8, 16, 32, 64, 128}.

\begin{CsynopsisCol}
void @\FuncDecl{shmem\_putmem\_nbi}@(void *dest, const void *source, size_t nelems, int pe);
void @\FuncDecl{shmem\_ctx\_putmem\_nbi}@(shmem_ctx_t ctx, void *dest, const void *source, size_t nelems, int pe);
\end{CsynopsisCol}

\begin{Fsynopsis}
CALL @\FuncDecl{SHMEM\_CHARACTER\_PUT\_NBI}@(dest, source, nelems, pe)
CALL @\FuncDecl{SHMEM\_COMPLEX\_PUT\_NBI}@(dest, source, nelems, pe)
CALL @\FuncDecl{SHMEM\_DOUBLE\_PUT\_NBI}@(dest, source, nelems, pe)
CALL @\FuncDecl{SHMEM\_INTEGER\_PUT\_NBI}@(dest, source, nelems, pe)
CALL @\FuncDecl{SHMEM\_LOGICAL\_PUT\_NBI}@(dest, source, nelems, pe)
CALL @\FuncDecl{SHMEM\_PUT4\_NBI}@(dest, source, nelems, pe)
CALL @\FuncDecl{SHMEM\_PUT8\_NBI}@(dest, source, nelems, pe)
CALL @\FuncDecl{SHMEM\_PUT32\_NBI}@(dest, source, nelems, pe)
CALL @\FuncDecl{SHMEM\_PUT64\_NBI}@(dest, source, nelems, pe)
CALL @\FuncDecl{SHMEM\_PUT128\_NBI}@(dest, source, nelems, pe)
CALL @\FuncDecl{SHMEM\_PUTMEM\_NBI}@(dest, source, nelems, pe)
CALL @\FuncDecl{SHMEM\_REAL\_PUT\_NBI}@(dest, source, nelems, pe)
\end{Fsynopsis}

\begin{apiarguments}
  \apiargument{IN}{ctx}{A context handle specifying the context on which to perform the operation.
    When this argument is not provided, the operation is performed on
    the default context.}
  \apiargument{OUT}{dest}{Data object to be updated on the remote \ac{PE}. This
    data object must be remotely accessible.}
  \apiargument{IN}{source}{Data object containing the data to be copied.}
  \apiargument{IN}{nelems}{Number of elements in the \VAR{dest} and \VAR{source}
    arrays. \VAR{nelems} must be of type \VAR{size\_t} for \Cstd. When using
    \Fortran, it must be a constant, variable, or array element of default
    integer type.}
  \apiargument{IN}{pe}{\ac{PE} number of the remote \ac{PE}. \VAR{pe} must be
    of type integer. When using \Fortran, it must be a constant, variable,
    or array element of default integer type.}
\end{apiarguments}

\apidescription{
    The routines return after posting the operation.  The operation is considered 
    complete after a subsequent call to \FUNC{shmem\_quiet}. 
    At the completion of \FUNC{shmem\_quiet}, the data has been copied into the \dest{} array
    on the destination \ac{PE}.
    The delivery of data words into the data object on the
    destination \ac{PE} may occur in any order.
    Furthermore, two successive put
    routines may deliver data out of order unless a call to \FUNC{shmem\_fence} is
    introduced between the two calls.   
    If the context handle \VAR{ctx} does not correspond to a valid context,
    the behavior is undefined.
 }

\apidesctable{
    The \dest{} and \source{} data objects must conform to certain typing
    constraints, which are as follows:}
    {Routine}{Data type of \VAR{dest} and \VAR{source}}
    \apitablerow{shmem\_putmem\_nbi}{\Fortran: Any noncharacter type. \Cstd:
        Any  data  type.  nelems is scaled in bytes.}
    \apitablerow{shmem\_put4\_nbi, shmem\_put32\_nbi}{Any noncharacter type
        that has a storage size equal to \CONST{32} bits.}
    \apitablerow{shmem\_put8\_nbi}{\Cstd: Any noncharacter type that
        has a storage size equal to \CONST{8} bits.}
    \apitablerow{}{\Fortran: Any noncharacter type that
        has a storage size equal to \CONST{64} bits.}
    \apitablerow{shmem\_put64\_nbi}{Any noncharacter type that
        has a storage size equal to \CONST{64} bits.}
    \apitablerow{shmem\_put128\_nbi}{Any noncharacter type that has a
        storage size equal to \CONST{128} bits.}
    \apitablerow{SHMEM\_CHARACTER\_PUT\_NBI}{Elements of type character.  \VAR{nelems}
    is  the number  of	 characters to transfer. The actual character lengths of
    the \source{} and \dest{} variables are ignored. }
    \apitablerow{SHMEM\_COMPLEX\_PUT\_NBI}{Elements of type complex of default size.}
    \apitablerow{SHMEM\_DOUBLE\_PUT\_NBI}{Elements of type double precision. }
    \apitablerow{SHMEM\_INTEGER\_PUT\_NBI}{Elements of type integer.}
    \apitablerow{SHMEM\_LOGICAL\_PUT\_NBI}{Elements of type logical.}
    \apitablerow{SHMEM\_REAL\_PUT\_NBI}{Elements of type real.}

\apireturnvalues{
    None.
}
\apinotes{ None.}

\end{apidefinition}
