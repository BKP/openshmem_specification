\apisummary{
    Controls data cache utilities.
}

\begin{apidefinition}

\begin{DeprecateBlock}
\begin{Csynopsis}
void @\FuncDecl{shmem\_clear\_cache\_inv}@(void);
void @\FuncDecl{shmem\_set\_cache\_inv}@(void);
void @\FuncDecl{shmem\_clear\_cache\_line\_inv}@(void *dest);
void @\FuncDecl{shmem\_set\_cache\_line\_inv}@(void *dest);
void @\FuncDecl{shmem\_udcflush}@(void);
void @\FuncDecl{shmem\_udcflush\_line}@(void *dest);
\end{Csynopsis}
\end{DeprecateBlock}

\begin{apiarguments}

\apiargument{IN}{dest}{A data object that is local to the \ac{PE}.}

\end{apiarguments}

\apidescription{
    \FUNC{shmem\_set\_cache\_inv} enables automatic cache coherency mode.

    \FUNC{shmem\_set\_cache\_line\_inv} enables automatic cache coherency mode for
    the cache line associated with the address of \VAR{dest} only.

    \FUNC{shmem\_clear\_cache\_inv} disables automatic cache coherency mode
    previously enabled by \FUNC{shmem\_set\_cache\ \_inv} or
    \FUNC{shmem\_set\_cache\_line\_inv}.

    \FUNC{shmem\_udcflush} makes the entire user data cache coherent.

    \FUNC{shmem\_udcflush\_line} makes coherent the cache line that corresponds with
    the address specified by \VAR{dest}.
}

\apireturnvalues{
    None.
}

\apinotes{
    These routines have been retained for improved backward compatibility with
    legacy architectures.  They are not required to be supported by implementing
    them as \VAR{no-ops} and where used, they may have no effect on cache line
    states.
}

\begin{apiexamples}

None.

\end{apiexamples}

\end{apidefinition}
