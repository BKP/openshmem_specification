\apisummary{
    shmemx\_team\_translate\_pe -- Translate a given virtual rank of one team
    to its corresponding virtual rank in another team.
}

\begin{apidefinition}

\begin{Csynopsis}
int @\FuncDecl{shmemx\_team\_translate\_pe}@(shmem_team_t team1, int team1_pe,
    shmem_team_t team2);
\end{Csynopsis}

\begin{apiarguments}
\apiargument{IN}{team1}{A valid SHMEM team handle.}
\apiargument{IN}{team1\_pe}{A virtual team rank in team1.}
\apiargument{IN}{team2}{A valid SHMEM team handle.}
\end{apiarguments}

\apidescription{
The shmemx\_team\_translate\_pe function will translate a virtual rank of
one team to its corresponding virtual rank in another team.
Specifically, given the team1\_pe in team1, this function returns that
\ac{PE}'s virtual rank in team2.

If SHMEM\_TEAM\_WORLD is provided as the team2 parameter, this function
acts as a global \ac{PE} rank translator and will return the corresponding
SHMEM\_TEAM\_WORLD rank. This may be useful when performing point-to-
point operations between \acp{PE} in a subset, as point-to-point operations
require the global (SHMEM\_TEAM\_WORLD) rank. This function requires
team1\_pe to be a member of team1. If team1\_pe is not a member of
team2, a value of -1 is returned.

Error checking will be done to ensure valid team handles are provided.
All team handle errors are considered fatal and will result in the job
aborting with an informative error message.
}

\apireturnvalues{
Calling process's virtual rank in the provided team.
}

\apinotes{
By default, SHMEM creates two predefined teams that will be available
for use once the routine start\_pes has been called. These teams can be
referenced in the application by the constants SHMEM\_TEAM\_WORLD and
SHMEM\_TEAM\_NODE. Every \ac{PE} process is a member of the SHMEM\_TEAM\_WORLD
team, and its rank in SHMEM\_TEAM\_WORLD corresponds to the value of its
global \ac{PE} rank. The SHMEM\_TEAM\_NODE team only contains the set of PEs
that reside on the same node as the current PE.
}

\end{apidefinition}
