\newtext{The performance of \openshmem programs that issue many consecutive and
small-sized communication routines might be improved by combining these
operations into fewer messages.
The \emph{bundling routines} provide a convenient interface for indicating to
the \openshmem library that a series of operations on a communication context
are eligible for bundling-related optimization.
The \FUNC{shmem\_bundle\_start} routine indicates the beginning of a bundling
phase, and the \FUNC{shmem\_bundle\_stop} routine indicates the end of a
bundling phase.}

\newtext{The bundling routines are \textit{hints} to the \openshmem library,
and they do not affect the completion or ordering semantics of any \openshmem
routines in the program.
For this reason, routines such as non-blocking RMAs, non-blocking AMOs,
non-blocking \OPR{put-with-signal}, blocking scalar \OPR{puts}, and blocking
non-fetching AMOs are viable candidates for bundling optimizations.
Other routines, such as blocking non-scalar \OPR{puts} and \OPR{gets}, blocking
fetching AMOs, blocking scalar \OPR{gets}, and the memory ordering routines
might require the library to enforce remote completion, reducing the potential
benefit of bundling optimizations.
Because bundling is performed with respect to an \openshmem communication
context, routines not performed on a communication context are ineligible for
bundling optimization.}
