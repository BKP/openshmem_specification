\apisummary{
    Wait on an array of variables on the local \ac{PE} until all variables meet the specified wait condition.
}

\begin{apidefinition}

\begin{C11synopsis}
void @\FuncDecl{shmem\_wait\_until\_all}@(TYPE *ivars, size_t nelems, const int *status, int cmp,
    TYPE cmp_value);
\end{C11synopsis}
where \TYPE{} is one of the point-to-point synchronization types specified by
Table \ref{p2psynctypes}.

\begin{Csynopsis}
void @\FuncDecl{shmem\_\FuncParam{TYPENAME}\_wait\_until\_all}@(TYPE *ivars, size_t nelems, const int *status, int cmp, TYPE cmp_value);
\end{Csynopsis}
where \TYPE{} is one of the point-to-point synchronization types and has a
corresponding \TYPENAME{} specified by Table~\ref{p2psynctypes}.

\begin{apiarguments}

  \apiargument{IN}{ivars}{A pointer to an array of remotely accessible data
    objects.}
  \apiargument{IN}{nelems}{The number of elements in the \VAR{ivars} array.}
  \apiargument{IN}{status}{An optional mask array of length \VAR{nelems}
    that indicates which elements in \VAR{ivars} are excluded from the wait set.}
  \apiargument{IN}{cmp}{A comparison operator from Table~\ref{p2p-consts}
    that compares elements of \VAR{ivars} with \VAR{cmp\_value}.}
  \apiargument{IN}{cmp\_value}{The value to be compared with the objects
    pointed to by \VAR{ivars}.}

\end{apiarguments}

\apidescription{ 
    The \FUNC{shmem\_wait\_until\_all} routine waits until all entries in the
    wait set specified by \VAR{ivars} and \VAR{status} have satisfied the wait condition at the
    calling \ac{PE}.  If \VAR{nelems} is 0, the wait set is empty and this routine returns immediately.
    This routine is semantically similar to
    \FUNC{shmem\_wait\_until} in Section~\ref{subsec:shmem_wait_until}, but
    adds support for point-to-point synchronization involving an array of
    symmetric data objects.

    The optional \VAR{status} is a mask array of length \VAR{nelems} where each
    element corresponds to the respective element in \VAR{ivars} and indicates
    whether the element is excluded from the wait set.  Elements of
    \VAR{status} set to 0 will be included in the wait set, and elements set to
    1 will be ignored.  If all elements in \VAR{status} are set to 1 or
    \VAR{nelems} is 0, the wait set is empty and this routine returns
    \CONST{SIZE\_MAX}.  If \VAR{status} is a null pointer, it is ignored and
    all elements in \VAR{ivars} are included in the wait set.  The \VAR{ivars}
    and \VAR{status} arrays must not overlap in memory.
}


\apireturnvalues{
    None.
}

\apinotes{
    None.
}

\apiimpnotes{
    Implementations must ensure that \FUNC{shmem\_wait\_until\_all} does not
    return before the update of the memory indicated by \VAR{ivars} is fully
    complete.  Partial updates to the memory must not cause
    \FUNC{shmem\_wait\_until\_all} to return.
}


\begin{apiexamples}
  \apicexample
      {The following \Cstd[11] example demonstrates the use of
      \FUNC{shmem\_wait\_until\_all} to implement a simple linear barrier
      synchronization.}
      {./example_code/shmem_wait_until_all.c}
      {}

\end{apiexamples}

\end{apidefinition}
