\apisummary{
  Return the configuration parameters of a given team
}

\begin{apidefinition}

\begin{Csynopsis}
void @\FuncDecl{shmem\_team\_get\_config}@(shmem_team_t team, shmem_team_config_t *team);
\end{Csynopsis}

\begin{apiarguments}
  \apiargument{IN}{team}{A valid \openshmem team handle.}
  \apiargument{OUT}{config}{
    A pointer to the configuration parameters for the new team.}
\end{apiarguments}

\apidescription{
\FUNC{shmem\_team\_get\_config} returns through the \VAR{config} argument
the configuration parameters of the given team, which were specified when the
team was created.

\begin{FeedbackRequest}
A library implementation must apply all requested options to a team, even in
the event that the library does not make optimizations based on these options.
For example, suppose library implementation must always create teams with the same
overhead, no matter if the program disables collective support during team creation.
The library must still enable the \LibConstRef{SHMEM\_TEAM\_NOCOLLECTIVE} option
when it is requested, so that the \openshmem program will be portable across implementations.
\end{FeedbackRequest}

All \acp{PE} in the team will get back the same parameter values for the team options.

If the \VAR{team} argument does not specify a valid team, the behavior is
undefined.
}

\apireturnvalues{
  None.
}

\apinotes{
A use case for this function is to determine whether a given team can
support collective operations by testing for the \LibConstRef{SHMEM\_TEAM\_NOCOLLECTIVE}
option. When teams are created without support for collectives, they may still use
point to point operations to communicate and synchronize. So programmers may wish
to design frameworks with functions that provide alternative algorithms
for teams based on whether they do or do not support collectives.
}

\end{apidefinition}
