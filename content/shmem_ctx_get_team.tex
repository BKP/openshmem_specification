\apisummary{
  Retrieve the team associated with the communication context.
}

\begin{apidefinition}

  \begin{Csynopsis}
int @\FuncDecl{shmem\_ctx\_get\_team}@(shmem_ctx_t ctx, shmem_team_t *team);
  \end{Csynopsis}

  \begin{apiarguments}

    \apiargument{IN}{ctx}{
      A handle to a communication context.
    }

    \apiargument{OUT}{team}{
      A pointer to a handle to the associated \ac{PE} team.
    }

  \end{apiarguments}

  \apidescription{
    The \FUNC{shmem\_ctx\_get\_team} routine returns a handle to the \ac{PE}
    team associated with the specified communication context \VAR{ctx}.
    The team handle is returned through the pointer argument \VAR{team}.

    If \VAR{ctx} is the default context, the returned team is guaranteed
    to be \CONST{SHMEM\_TEAM\_WORLD}.

    If \VAR{ctx} is an invalid context, the argument \VAR{team} is not
    modified and a value of \CONST{-1} is returned.

    If \VAR{team} is a null pointer, a value of \CONST{-1} is returned.
  }

  \apireturnvalues{
    Zero on success; otherwise, \CONST{-1}.

    \begin{FeedbackRequest}
      Should this routine return nonzero, -1, or negative values
      (e.g., to allow for implementation-defined error codes) on error?
      Will slowing down the critical path of this routine by adding
      input checking adversely affect its use?
    \end{FeedbackRequest}
  }

  \apinotes{
    None.
  }

\end{apidefinition}
