\apisummary{
  Retrieve the team associated with the communication context.
}

\begin{apidefinition}

\begin{Csynopsis}
int @\FuncDecl{shmem\_ctx\_get\_team}@(shmem_ctx_t ctx, shmem_team_t *team);
\end{Csynopsis}

\begin{apiarguments}
  \apiargument{IN}{ctx}{
    A handle to a communication context.
  }
  \apiargument{OUT}{team}{
    A pointer to a handle to the associated \ac{PE} team.
  }
\end{apiarguments}

\apidescription{
  The \FUNC{shmem\_ctx\_get\_team} routine returns a handle to the
  team associated with the specified communication context \VAR{ctx}.
  The team handle is returned through the pointer argument \VAR{team}.

  If \VAR{ctx} is the default context or one created by a call to
  \FUNC{shmem\_ctx\_create}, \VAR{team} is assigned the handle value
  \LibHandleRef{SHMEM\_TEAM\_WORLD}.

  If \VAR{ctx} compares equal to \LibConstRef{SHMEM\_CTX\_INVALID},
  then \VAR{team} is assigned the value
  \LibConstRef{SHMEM\_TEAM\_INVALID} and a nonzero value is returned.
  If \VAR{ctx} is otherwise invalid, the behavior is undefined.
}

\apireturnvalues{
  Zero on success; otherwise, nonzero.
}

\begin{apiexamples}

    \apicexample
    {The following example demonstrates the use of contexts for multiple teams in a
    \Cstd[11] program. This example shows contexts being used to communicate within
    a team using team \ac{PE} numbers, and across teams using translated \ac{PE} numbers.}
    {./example_code/shmem_team_context.c}
    {}

\end{apiexamples}

\end{apidefinition}
