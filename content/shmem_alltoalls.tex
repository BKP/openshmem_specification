\apisummary{
    shmem\_alltoalls is a collective routine where each \ac{PE} exchanges a fixed amount of strided data with all other
    \acp{PE} in the \activeset.
}

\begin{apidefinition}

\begin{Csynopsis}
void shmem_alltoalls32(void *dest, const void *source, ptrdiff_t dst, ptrdiff_t sst, size_t nelems, int PE_start, int logPE_stride, int PE_size, long *pSync);
void shmem_alltoalls64(void *dest, const void *source, ptrdiff_t dst, ptrdiff_t sst, size_t nelems, int PE_start, int logPE_stride, int PE_size, long *pSync);
\end{Csynopsis}

\begin{Fsynopsis}
INTEGER pSync(SHMEM_ALLTOALLS_SYNC_SIZE)
INTEGER dst, sst, PE_start, logPE_stride, PE_size
INTEGER nelems 
CALL SHMEM_ALLTOALLS32(dest, source, dst, sst, nelems, PE_start, logPE_stride, PE_size, pSync)
CALL SHMEM_ALLTOALLS64(dest, source, dst, sst, nelems, PE_start, logPE_stride, PE_size, pSync)
\end{Fsynopsis}

\begin{apiarguments}

\apiargument{OUT}{dest}{A symmetric data object large enough to receive 
    the combined total of \VAR{nelems} elements from each \ac{PE} in the
    \activeset.}
\apiargument{IN}{source}{A symmetric data object that contains \VAR{nelems} 
    elements of data for each \ac{PE} in the \activeset{}, ordered according to 
    destination \ac{PE}.}
\apiargument{IN}{dst}{The stride between consecutive elements of the \dest{}
    data object.  The stride is scaled by the element size.  A
    value of \CONST{1} indicates contiguous data.  \VAR{dst} must be of type
    \textit{ptrdiff\_t}.  If you are using \Fortran, it must be a default integer
    value.}
\apiargument{IN}{sst}{The  stride between consecutive elements of the
    \source{} data object.  The stride is scaled by the element size.
    A value of \CONST{1} indicates contiguous data.  \VAR{sst} must be
    of type \textit{ptrdiff\_t}.  If you are using \Fortran, it must be a
    default integer value.}
\apiargument{IN}{nelems}{The number of elements to exchange for each \ac{PE}.
    \VAR{nelems} must be of type size\_t for \CorCpp.  If you are using
    \Fortran, it must be a default integer value.}
\apiargument{IN}{PE\_start}{The lowest \ac{PE} number of the \activeset{} of
    \acp{PE}.  \VAR{PE\_start} must be of type integer.  If you are using \Fortran,
    it must be a default integer value.}
\apiargument{IN}{logPE\_stride}{The log (base 2) of the stride between
    consecutive \ac{PE} numbers in the \activeset.  \VAR{logPE\_stride} must be of
    type integer.  If you are using \Fortran, it must be a default integer value.}
\apiargument{IN}{PE\_size}{The number of \acp{PE} in the \activeset.
    \VAR{PE\_size} must be of type integer.  If you are using \Fortran, it must
    be a default integer value.}
\apiargument{IN}{pSync}{A symmetric work array. In \CorCpp, \VAR{pSync} must be
    of type long and size \CONST{SHMEM\_ALLTOALLS\_SYNC\_SIZE}. In \Fortran,
    \VAR{pSync} must be of type integer and size
    \CONST{SHMEM\_ALLTOALLS\_SYNC\_SIZE}.  If you are using \Fortran, it must be a
    default integer value. Every element of this array must be initialized with
    the value \CONST{SHMEM\_SYNC\_VALUE} before any of the \acp{PE} in the
    \activeset{} enter the routine.}
    
\end{apiarguments}

\apidescription{
    The \FUNC{shmem\_alltoalls} routines are collective routines. Each \ac{PE}
    in the \activeset{} exchanges \VAR{nelems} strided data elements of size
    32 bits (for \FUNC{shmem\_alltoalls32}) or 64 bits (for \FUNC{shmem\_alltoalls64})
    with all other \acp{PE} in the set. Both strides, \VAR{dst} and \VAR{sst}, must be greater
    than or equal to \CONST{1}. The \VAR{sst}*\VAR{j}th block sent from \ac{PE} \VAR{i} to
    \ac{PE} \VAR{j} is placed in the \VAR{dst}*\VAR{i}th block of the \VAR{dest} data object on
    \ac{PE} \VAR{j}.

    As with all \openshmem collective routines, these routines assume
    that only \acp{PE} in the \activeset{} call the routine.  If a \ac{PE} not
    in the \activeset{} calls an \openshmem collective routine, undefined
    behavior results.

    The values of arguments \VAR{dst}, \VAR{sst}, \VAR{nelems}, \VAR{PE\_start},
    \VAR{logPE\_stride}, and \VAR{PE\_size} must be equal on all \acp{PE} in the
    \activeset. The same \VAR{dest} and \VAR{source} data objects, and the same
    \VAR{pSync} work array must be passed to all \acp{PE} in the \activeset.
    
    Before any \ac{PE} calls to a \FUNC{shmem\_alltoalls} routine, the following
    conditions must exist (synchronization via a barrier or some other method is
    often needed to ensure this): The \VAR{pSync} array on all \acp{PE} in the
    \activeset{} is not still in use from a prior call to a
    \FUNC{shmem\_alltoalls} routine.  The \VAR{dest} data object on
    all \acp{PE} in the \activeset{} is ready to accept the
    \FUNC{shmem\_alltoalls} data.
    
    Upon return from a \FUNC{shmem\_alltoalls} routine, the following is true for
    the local PE: Its \VAR{dest} symmetric data object is completely updated and
    the data has been copied out of the \VAR{source} data object.
    The values in the \VAR{pSync} array are restored to the original values.
} 

\apidesctable{
The  \dest{}  and \source{} data  objects must conform to certain typing
constraints, which are as follows:
}{Routine}{Data type of \VAR{dest} and \VAR{source}}

\apitablerow{shmem\_alltoalls64}{\CONST{64} bits aligned.}
\apitablerow{shmem\_alltoalls32}{\CONST{32} bits aligned.}

\apireturnvalues{
    None.
}

\apinotes{
    This routine restores \VAR{pSync} to its original contents.  Multiple calls
    to \openshmem\ routines that use the same \VAR{pSync} array do not require
    that \VAR{pSync} be reinitialized after the first call.
    You must ensure the that the \VAR{pSync} array is not being updated by any
    \ac{PE} in the \activeset{} while any of the \acp{PE} participates in
    processing of an \openshmem\ \FUNC{shmem\_alltoalls} routine. Be careful to
    avoid these situations: If the \VAR{pSync} array is initialized at run time,
    some type of synchronization is needed to ensure that all \acp{PE} in the
    \activeset{} have initialized \VAR{pSync} before any of them enter an
    \openshmem\ routine called with the \VAR{pSync} synchronization array.  A
    \VAR{pSync} array may be reused on a subsequent \openshmem\
    \FUNC{shmem\_alltoalls} routine only if none of the \acp{PE} in the
    \activeset{} are still processing a prior \openshmem\ \FUNC{shmem\_alltoalls}
    routine call that used the same \VAR{pSync} array.  In general, this can be
    ensured only by doing some type of synchronization.        
}

\begin{apiexamples}

\apicexample
    {This example shows a \FUNC{shmem\_alltoalls64} on two long elements among
    all \acp{PE}.}
    {./example_code/shmem_alltoalls_example.c}
    {}

\end{apiexamples}

\end{apidefinition}
