\apisummary{
    shmem\_alltoalls is a collective routine where each \ac{PE} exchanges a fixed amount of strided data with all other \acp{PE} participating in the collective.
}

\begin{apidefinition}

%% C11
\begin{C11synopsis}
int @\FuncDecl{shmem\_alltoalls}@(shmem_team_t team, TYPE *dest, const TYPE *source, ptrdiff_t dst, ptrdiff_t sst, size_t nelems);
\end{C11synopsis}
where \TYPE{} is one of the standard \ac{RMA} types specified by Table \ref{stdrmatypes}.

\begin{Csynopsis}
\end{Csynopsis}
\begin{CsynopsisCol}
int @\FuncDecl{shmem\_\FuncParam{TYPENAME}\_alltoalls}@(shmem_team_t team, TYPE *dest, const TYPE *source, ptrdiff_t dst, ptrdiff_t sst, size_t nelems);
\end{CsynopsisCol}
where \TYPE{} is one of the standard \ac{RMA} types and has a corresponding \TYPENAME{} specified by Table \ref{stdrmatypes}.

\begin{CsynopsisCol}
int @\FuncDecl{shmem\_alltoallsmem}@(shmem_team_t team, void *dest, const void *source, ptrdiff_t dst, ptrdiff_t sst, size_t nelems);
\end{CsynopsisCol}

\begin{DeprecateBlock}
\begin{CsynopsisCol}
void @\FuncDecl{shmem\_alltoalls32}@(void *dest, const void *source, ptrdiff_t dst, ptrdiff_t sst, size_t nelems, int PE_start, int logPE_stride, int PE_size, long *pSync);
void @\FuncDecl{shmem\_alltoalls64}@(void *dest, const void *source, ptrdiff_t dst, ptrdiff_t sst, size_t nelems, int PE_start, int logPE_stride, int PE_size, long *pSync);
\end{CsynopsisCol}
\end{DeprecateBlock}

\begin{apiarguments}

\apiargument{IN}{team}{A valid \openshmem team handle.}%

\apiargument{OUT}{dest}{Symmetric address of a data object large enough to receive 
    the combined total of \VAR{nelems} elements from each \ac{PE} in the
    active set.
    The type of \dest{} should match that implied in the SYNOPSIS section.}
\apiargument{IN}{source}{Symmetric address of a data object that contains \VAR{nelems}
    elements of data for each \ac{PE} in the active set, ordered according to
    destination \ac{PE}.
    The type of \source{} should match that implied in the SYNOPSIS section.}
\apiargument{IN}{dst}{The stride between consecutive elements of the \dest{}
    data object.  The stride is scaled by the element size.  A
    value of \CONST{1} indicates contiguous data.}
\apiargument{IN}{sst}{The  stride between consecutive elements of the
    \source{} data object.  The stride is scaled by the element size.
    A value of \CONST{1} indicates contiguous data.}
    
\begin{DeprecateBlock}
\apiargument{IN}{nelems}{The number of elements to exchange for each \ac{PE}.}
\apiargument{IN}{PE\_start}{The lowest \ac{PE} number of the active set of
    \acp{PE}.}
\apiargument{IN}{logPE\_stride}{The log (base 2) of the stride between
    consecutive \ac{PE} numbers in the active set.}
    \apiargument{IN}{PE\_size}{The number of \acp{PE} in the active set.}
\apiargument{IN}{pSync}{
    Symmetric address of a work array of size \CONST{SHMEM\_ALLTOALLS\_SYNC\_SIZE}.
    Every element of \VAR{pSync} must be initialized with the value
    \CONST{SHMEM\_SYNC\_VALUE} before any of the \acp{PE} in the active set
    enter the routine.}
\end{DeprecateBlock}

\end{apiarguments}

\apidescription{
    The \FUNC{shmem\_alltoalls} routines are collective routines.
    These routines are equivalent in functionality to the corresponding
    \FUNC{shmem\_alltoall} routines except that they add explicit stride values
    for accessing the source and destination data arrays, whereas the array
    access in \FUNC{shmem\_alltoall} is always with a stride of \CONST{1}.

    Each \ac{PE} participating in the operation
    exchanges \VAR{nelems} strided data elements
    with all other \acp{PE} participating in the operation.
    Both strides, \VAR{dst} and \VAR{sst}, must be greater
    than or equal to \CONST{1}.

    The same \dest{} and \source{} arrays and same values for values of
    arguments \VAR{dst}, \VAR{sst}, \VAR{nelems} must be passed by all \acp{PE}
    that participate in the collective.
    
    Given a \ac{PE} \VAR{i} that is the \kth \ac{PE}
    participating in the operation and a \ac{PE}
    \VAR{j} that is the \lth \ac{PE}
    participating in the operation
    \ac{PE} \VAR{i} sends the \VAR{sst}*\lth block of the \VAR{source} data object to
    the \VAR{dst}*\kth block of the \VAR{dest} data object on
    \ac{PE} \VAR{j}.

    See the description of \FUNC{shmem\_alltoall} in
    Section~\ref{subsec:shmem_alltoall} for:
    \begin{itemize}
    \item Data element sizes for the different sized and typed \FUNC{shmem\_alltoalls} variants.
    \item Rules for \ac{PE} participation in the collective routine.
    \item The pre- and post-conditions for symmetric objects.
    \item Typing constraints for \dest{} and \source{} data objects.
    \end{itemize}
    
} 


\apireturnvalues{
   Zero on successful local completion. Nonzero otherwise.
}

\apinotes{
    See notes for \FUNC{shmem\_alltoall} in Section~\ref{subsec:shmem_alltoall}.
}

\begin{apiexamples}

\apicexample
    {This \CorCpp{} example shows a \FUNC{shmem\_int64\_alltoalls} on two 64-bit integers among
    all \acp{PE}.}
    {./example_code/shmem_alltoalls_example.c}
    {}

\end{apiexamples}

\end{apidefinition}
