\apisummary{
shmemx\_team\_split\_3d - partitions an existing parent team into three subgroups,
based on the three-dimensional Cartesian space defined by the triplet (xrange,
yrange, and zrange) describing the size of the Cartesian space in X, Y, and Z
dimensions.
}

\begin{apidefinition}

\begin{Csynopsis}
void @\FuncDecl{shmemx\_team\_split\_3d}@(shmem_team_t parent_team, int xrange,
int yrange, int zrange, shmem_team_t *xaxis_team, shmem_team_t *yaxis_team,
shmem_team_t *zaxis_team);
\end{Csynopsis}

\begin{apiarguments}
\apiargument{IN}{parent\_team}{A valid SHMEM team. The predefined teams
SHMEM\_TEAM\_WORLD or SHMEM\_TEAM\_NODE may be used, or any team created by the
users.}

\apiargument{IN}{xrange}{A non-negative integer representing the number of
elements in the first dimension.}

\apiargument{IN}{yrange}{A non-negative integer representing the number of
elements in the second dimension.}

\apiargument{IN}{zrange}{A non-negative integer representing the number of
elements in the third dimension.}

\apiargument{OUT}{xaxis\_team}{A new \ac{PE} team handle representing a \ac{PE}
subset consisting of all the \acp{PE} that are in the same row in the X-axis.}

\apiargument{OUT}{yaxis\_team}{A new \ac{PE} team handle representing a \ac{PE}
subset consisting of all the \acp{PE} that are in the same column in the Y-axis.}

\apiargument{OUT}{zaxis\_team}{A new \ac{PE} team handle representing a \ac{PE}
subset consisting of all the \acp{PE} that are in the same position in in the
Z-axis.}
\end{apiarguments}

\apidescription{
The shmemx\_team\_split\_3d routine is a collective routine. It
partitions an existing parent team into three subgroups, based on the
three-dimensional Cartesian space defined by the triplet (xrange,
yrange, and zrange) describing the size of the Cartesian space in X,
Y, and Z dimensions. Each subgroup contains all \acp{PE} that are in the same
dimension, along the X-axis, Y-axis and Z-axis. Within each subgroup,
the \acp{PE} are ranked based on the position of the \ac{PE} with respect to its
dimension in three-dimensional Cartesian space.

Any valid \ac{PE} team can be used as the parent team. This routine must be
called by all \acp{PE} in the parent team. The value of the triplets must be
non-negative, and the size of the parent team should be greater than or
equal to the size of the three-dimensional Cartesian space. None of the
parameters need to reside in symmetric memory.

Error checking will be done to ensure a valid team handle is provided.
All errors are considered fatal and will result in the job aborting with
an informative error message.
}

\apireturnvalues{
None.
}

\apinotes{
Note that SHMEM team handles have local semantics only. That is, team
handles should not be stored in shared variables and used across other
processes. Doing so will result in unpredictable behavior.
}

\begin{apiexamples}

\end{apiexamples}

\end{apidefinition}
