\apisummary{
    Destroy a communication context.
}

\begin{apidefinition}

\begin{Csynopsis}
void @\FuncDecl{shmem\_ctx\_destroy}@(shmem_ctx_t ctx);
\end{Csynopsis}

\begin{apiarguments}
    \apiargument{IN}{ctx}{Handle to the context that will be destroyed.}
\end{apiarguments}

\apidescription{
    \FUNC{shmem\_ctx\_destroy} destroys a context that was created by a call to
    \FUNC{shmem\_ctx\_create}.  It is the user's responsibility to ensure that
    the context is not used after it has been destroyed, for example when the
    destroyed context is used by multiple threads.  This function
    performs an implicit quiet operation on the given context before it is freed.
}

\apireturnvalues{
    None.
}

\apinotes{
    It is invalid to pass \CONST{SHMEM\_CTX\_DEFAULT} to this routine.

    Destroying a context makes it impossible for the user to complete
    communication operations that are pending on that context.  This includes
    nonblocking communication operations, whose local buffers are only returned
    to the user after the operations have been completed.  An implicit quiet is
    performed when freeing a context to avoid this ambiguity.

    A context with the \CONST{SHMEM\_CTX\_PRIVATE} option enabled must be
    destroyed by the thread that created it.
}

\begin{apiexamples}

    \apicexample
    {The following example demonstrates the use of contexts in a multithreaded
    \Cstd[11] program that uses OpenMP for threading.  This example shows the
    shared counter load balancing method and illustrates the use of contexts
    for thread isolation.}
    {./example_code/shmem_ctx.c}
    {}

    \apicexample
    {The following example demonstrates the use of contexts in a
    single-threaded \Cstd[11] program that performs a summation reduction where
    the data contained in the \VAR{in\_buf} arrays on all \acp{PE} is reduced into
    the \VAR{out\_buf} arrays on all \acp{PE}.  The buffers are divided into
    segments and processing of the segments is pipelined.  Contexts are used
    to overlap an all-to-all exchange of data for segment \VAR{p} with the
    local reduction of segment \VAR{p-1}.}
    {./example_code/shmem_ctx_pipelined_reduce.c}
    {}

\end{apiexamples}

\end{apidefinition}

