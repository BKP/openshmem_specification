\apisummary{
    Returns the total number of \acp{PE} in the provided team.
}

\begin{apidefinition}

\begin{Csynopsis}
int @\FuncDecl{shmem\_team\_n\_pes}@(shmem_team_t team);
\end{Csynopsis}

\begin{apiarguments}
\apiargument{IN}{team}{A valid \openshmem team handle.}
\end{apiarguments}

\apidescription{
The \FUNC{shmem\_team\_n\_pes} function returns the number of \acp{PE} in the
team. This will always be a value between 1 and N, where N is the total number of
\acp{PE} accessible to the \openshmem program. For the team
\LibHandleRef{SHMEM\_TEAM\_WORLD}, this will return the same value as
\FUNC{shmem\_n\_pes}.

Every team must have a least one member. All \acp{PE} in the team
will get back the same value for the team size.

Error checking will be done to ensure a valid team handle is provided.
Errors will result in a return value less than \CONST{0}.
}

\apireturnvalues{
Total number of \acp{PE} in the provided team, or a value less than
\CONST{0} if the team handle is invalid.
}

\apinotes{
By default, \openshmem creates two predefined teams that will be available
for use once the routine \FUNC{shmem\_init} has been called. These teams can be
referenced in the application by the constants \LibHandleRef{SHMEM\_TEAM\_WORLD} and
\LibHandleRef{SHMEM\_TEAM\_NODE}. Every \ac{PE} is a member of the \LibHandleRef{SHMEM\_TEAM\_WORLD}
team, and its number in \LibHandleRef{SHMEM\_TEAM\_WORLD} corresponds to the value of its
global \ac{PE} number. The \LibHandleRef{SHMEM\_TEAM\_NODE} team contains the set of only those
\acp{PE} that reside on the same node as the current \ac{PE}.
}

\end{apidefinition}
