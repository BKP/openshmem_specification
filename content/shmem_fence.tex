\apisummary{
    Assures ordering of delivery of \PUT{}, \acp{AMO}, and memory store routines
    to symmetric data objects.
}

\begin{apidefinition}

\begin{Csynopsis}
void shmem_fence(void);
\end{Csynopsis}

\begin{Fsynopsis}
CALL SHMEM_FENCE
\end{Fsynopsis}

\begin{apiarguments}
\apiargument{None.}{}{}
\end{apiarguments}

\apidescription{
    This routine assures ordering of delivery of \PUT{}, \acp{AMO}, and memory store
    routines to symmetric data objects.  All \PUT{}, \acp{AMO}, and memory store
    routines to symmetric data objects issued to a particular remote \ac{PE} prior
    to the call to \FUNC{shmem\_fence} are guaranteed to be delivered before any
    subsequent \PUT{}, \acp{AMO}, and memory store routines to symmetric data
    objects to the same \ac{PE}. \FUNC{shmem\_fence} guarantees order of delivery,
    not completion.
}

\apireturnvalues{
    None.
}

\apinotes{
    \FUNC{shmem\_fence} only provides per-\ac{PE} ordering guarantees and does not
    guarantee completion of delivery.  
    \FUNC{shmem\_fence} also does not have an effect on the ordering between memory 
    accesses issued by the target PE. \FUNC{shmem\_wait}, \FUNC{shmem\_wait\_until},
    \FUNC{shmem\_barrier}, \FUNC{shmem\_barrier\_all} routines can be called by the target PE to guarantee 
    ordering of its memory accesses.
    There is a subtle difference between
    \FUNC{shmem\_fence} and \FUNC{shmem\_quiet}, in that, \FUNC{shmem\_quiet}
    guarantees completion of \PUT{}, \acp{AMO}, and memory store routines to
    symmetric data objects which makes the updates visible to all other
    \acp{PE}. 
    
    The \FUNC{shmem\_quiet} routine should be called if completion of PUT{},
    \acp{AMO}, and memory store routines to symmetric data objects is desired
    when multiple remote \acp{PE} are involved.
}

\begin{apiexamples}

\apicexample
    {The following \FUNC{shmem\_fence} example is for C11 programs: }
    {./example_code/shmem_fence_example.c}
    {\VAR{Put1} will be ordered to be delivered before \VAR{put3} and \VAR{put2}
    will be ordered to be delivered before \VAR{put4}.}

\end{apiexamples}

\end{apidefinition}
