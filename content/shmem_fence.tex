\apisummary{
    Ensures ordering of delivery of operations on symmetric data objects.
}

\begin{apidefinition}

\begin{Csynopsis}
void @\FuncDecl{shmem\_fence}@(void);
void @\FuncDecl{shmem\_ctx\_fence}@(shmem_ctx_t ctx);
\end{Csynopsis}

\begin{apiarguments}
    \apiargument{IN}{ctx}{A context handle specifying the context on which to perform the operation.
        When this argument is not provided, the operation is performed on
        the default context.}
\end{apiarguments}

\apidescription{
    This routine ensures ordering of delivery of operations on symmetric data
    objects.
    Table~\ref{mem-order} lists the operations that are ordered
    by the \FUNC{shmem\_fence} routine.
    All operations on symmetric data objects issued to a particular
    \ac{PE} on the given context prior
    to the call to \FUNC{shmem\_fence} are guaranteed to be delivered before any
    subsequent operations on symmetric data
    objects to the same \ac{PE} on the same context. \FUNC{shmem\_fence} guarantees order of delivery,
    not completion. It does not guarantee order of delivery of nonblocking
    \GET{} or values fetched by nonblocking \ac{AMO} routines.
    If \VAR{ctx} has the value \CONST{SHMEM\_CTX\_INVALID}, no operation is
    performed.
}

\apireturnvalues{
    None.
}

\apinotes{
    \FUNC{shmem\_fence} only provides per-\ac{PE} ordering guarantees and does not
    guarantee completion of delivery.
    \FUNC{shmem\_fence} also does not have an effect on the ordering between memory
    accesses issued by the target PE. \FUNC{shmem\_wait\_until}, \FUNC{shmem\_test},
    \FUNC{shmem\_barrier}, \FUNC{shmem\_barrier\_all} routines can be called by the target PE to guarantee
    ordering of its memory accesses.
    There is a subtle difference between
    \FUNC{shmem\_fence} and \FUNC{shmem\_quiet}, in that, \FUNC{shmem\_quiet}
    guarantees completion of all operations on
    symmetric data objects which makes the updates visible to all other
    \acp{PE}.

    The \FUNC{shmem\_quiet} routine should be called if completion of operations
    on symmetric data objects is desired
    when multiple \acp{PE} are involved.

    In an \openshmem program with multithreaded \acp{PE}, it is the
    user's responsibility to ensure ordering between operations issued by the threads
    in a \ac{PE} that target symmetric memory and calls by threads in that \ac{PE} to
    \FUNC{shmem\_fence}. The \FUNC{shmem\_fence} routine can enforce memory store ordering only for the
    calling thread. Thus, to ensure ordering for memory stores performed by a thread that is
    not the thread calling \FUNC{shmem\_fence}, the update must be made visible to the
    calling thread according to the rules of the memory model associated with
    the threading environment.
}

\begin{apiexamples}

\apicexample
    {The following example uses \FUNC{shmem\_fence} in a \Cstd[11] program: }
    {./example_code/shmem_fence_example.c}
    {\VAR{Put1} will be ordered to be delivered before \VAR{put3} and \VAR{put2}
    will be ordered to be delivered before \VAR{put4}.}

\end{apiexamples}

\end{apidefinition}
