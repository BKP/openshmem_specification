\apisummary{
    Assures ordering of delivery of \PUT{}, \ac{AMO}, memory store, and non-blocking \PUT{} routines
    to symmetric data objects.
}

\begin{apidefinition}

\begin{Csynopsis}
void shmem_fence(void);
void shmem_ctx_fence(shmem_ctx_t ctx);
\end{Csynopsis}

\begin{Fsynopsis}
CALL SHMEM_FENCE
\end{Fsynopsis}

\begin{apiarguments}
    \apiargument{IN}{ctx}{Context on which to perform the operation.  When this
        argument is not provided, \CONST{SHMEM\_CTX\_DEFAULT} is used.}
\end{apiarguments}

\apidescription{
    This routine assures ordering of delivery of \PUT{}, \ac{AMO}, memory store, and non-blocking \PUT{}
    routines to symmetric data objects.  All \PUT{}, \ac{AMO}, memory store, and non-blocking \PUT{}
    routines to symmetric data objects issued to a particular remote \ac{PE} 
    on the given context prior
    to the call to \FUNC{shmem\_fence} are guaranteed to be delivered before any
    subsequent \PUT{}, \ac{AMO}, memory store, and non-blocking \PUT{} routines to symmetric data
    objects to the same \ac{PE}. \FUNC{shmem\_fence} guarantees order of delivery,
    not completion.
}

\apireturnvalues{
    None.
}

\apinotes{
    \FUNC{shmem\_fence} only provides per-\ac{PE} ordering guarantees and does not
    guarantee completion of delivery.  
    \FUNC{shmem\_fence} also does not have an effect on the ordering between memory 
    accesses issued by the target PE. \FUNC{shmem\_wait}, \FUNC{shmem\_wait\_until}, \FUNC{shmem\_test},
    \FUNC{shmem\_barrier}, \FUNC{shmem\_barrier\_all} routines can be called by the target PE to guarantee 
    ordering of its memory accesses.
    There is a subtle difference between
    \FUNC{shmem\_fence} and \FUNC{shmem\_quiet}, in that, \FUNC{shmem\_quiet}
    guarantees completion of \PUT{}, \ac{AMO}, memory store, and non-blocking \PUT{} routines to
    symmetric data objects which makes the updates visible to all other
    \acp{PE}. 
    
    The \FUNC{shmem\_quiet} routine should be called if completion of \PUT{},
    \ac{AMO}, memory store, and non-blocking \PUT{} routines to symmetric data objects is desired
    when multiple remote \acp{PE} are involved.

    In an \openshmem program with multithreaded \acp{PE}, it is the
    user's responsibility to ensure ordering between operations issued by the threads
    in a \ac{PE} that target symmetric memory (e.g. \PUT{}, \ac{AMO}, memory stores,
    and nonblocking routines) and calls by threads in that \ac{PE} to
    \FUNC{shmem\_fence}. The \FUNC{shmem\_fence} routine can enforce memory store ordering only for the
    calling thread. Thus, to ensure ordering for memory stores performed by a thread that is
    not the thread calling \FUNC{shmem\_fence}, the update must be made visible to the
    calling thread according to the rules of the memory model associated with
    the threading environment.
}

\begin{apiexamples}

\apicexample
    {The following example uses  \FUNC{shmem\_fence}  in a C11 program: }
    {./example_code/shmem_fence_example.c}
    {\VAR{Put1} will be ordered to be delivered before \VAR{put3} and \VAR{put2}
    will be ordered to be delivered before \VAR{put4}.}

\end{apiexamples}

\end{apidefinition}
