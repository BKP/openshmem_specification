\TableIndex{Environment Variables}

The \openshmem specification provides a set of environment variables that allows
users to configure the \openshmem implementation, and receive information about
the implementation. The implementations of the specification are free to define
additional variables. Currently, the specification defines four environment
variables. All environment variables that start with \VAR{SMA\_*} are
deprecated, but currently supported for backwards compatibility.
If both \VAR{SHMEM\_}- and \VAR{SMA\_}-prefixed environment variables
are set, then the value in the \VAR{SHMEM\_}-prefixed environment variable
establishes the controlling value. Refer to the
\hyperref[subsec:deprecate-sma-env]{\VAR{SMA\_*} Environment Variables}
deprecation rationale for more details.

\medskip{}

\begin{longtable}{|p{0.275\textwidth}|p{0.125\textwidth}|p{0.5\textwidth}|}
\hline
\textbf{Variable} & \textbf{Value} & \textbf{Description}
\tabularnewline\hline
%%
\EnvVarDecl{SHMEM\_VERSION}
    & Any
    & Print the library version at start-up
    \tabularnewline\hline
%%
\EnvVarDecl{SHMEM\_INFO}
    & Any
    & Print helpful text about all these environment variables
    \tabularnewline\hline
%%
\EnvVarDecl{SHMEM\_SYMMETRIC\_SIZE}
    & Non-negative decimal with character suffix or a whole number with optional
    character suffix
    & Controls the size of the symmetric heap per \ac{PE}. The value
      set in this environment variable is interpreted as number
      of bytes, unless the number is followed by a single
      character that acts as a multiplier, where:
      \begin{itemize}
        \item k or K multiplies by \(2^{10}\)  (kilobytes)
        \item m or M multiplies by \(2^{20}\)  (megabytes)
        \item g or G multiplies by \(2^{30}\)  (gigabytes)
        \item t or T multiplies by \(2^{40}\)  (terabytes)
      \end{itemize}
      For example, the string 20m is equivalent to the integer value 20971520,
      or 20 megabytes. Similarly the string 3.1g is equivalent to the integer
      value 3250585. Only one multiplier is recognized and a decimal value is
      always required to be followed by a character suffix, so 20kk
      will not produce the same value as 20m, nor will invalid
      strings such as 20MB or 20.1 produce the desired result. Any unrecognized
      suffix is treated as an error and results in program termination with an
      error report.
    \tabularnewline\hline
%%
\EnvVarDecl{SHMEM\_DEBUG}
    & Any
    & Enable debugging messages
    \tabularnewline\hline
\end{longtable}

\medskip{}
