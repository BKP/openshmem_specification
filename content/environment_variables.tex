\TableIndex{Environment Variables}

The \openshmem specification provides a set of environment variables that allows
users to configure the \openshmem implementation, and receive information about
the implementation. The implementations of the specification are free to define
additional variables. Currently, the specification defines four environment
variables. All environment variables that start with \VAR{SMA\_*} are
deprecated, but currently supported for backwards compatibility.
If both \VAR{SHMEM\_}- and \VAR{SMA\_}-prefixed environment variables
are set, then the value in the \VAR{SHMEM\_}-prefixed environment variable
establishes the controlling value. Refer to the
\hyperref[subsec:deprecate-sma-env]{\VAR{SMA\_*} Environment Variables}
deprecation rationale for more details.

\medskip{}

\begin{longtable}{|p{0.260\textwidth}|p{0.145\textwidth}|p{0.50\textwidth}|}
\hline
\textbf{Variable} & \textbf{Value} & \textbf{Description}
\tabularnewline\hline
%%
\EnvVarDecl{SHMEM\_VERSION}
    & Any
    & Print the library version at start-up
    \tabularnewline\hline
%%
\EnvVarDecl{SHMEM\_INFO}
    & Any
    & Print helpful text about all these environment variables
    \tabularnewline\hline
%%
\EnvVarDecl{SHMEM\_SYMMETRIC\_SIZE}
    & Positive decimal value (integer or floating point) with an optional
    character suffix %that specifies a scaling factor
    & The value set in this environment variable is interpreted as an
    implementation-defined size (in bytes) at least as large as the integer
    ceiling of the product of the numeric prefix and the scaling factor. The
    allowed character suffix for the scaling factor are as follows:
      \begin{itemize}
        \item k or K multiplies by \(2^{10}\)  (kilobytes)
        \item m or M multiplies by \(2^{20}\)  (megabytes)
        \item g or G multiplies by \(2^{30}\)  (gigabytes)
        \item t or T multiplies by \(2^{40}\)  (terabytes)
      \end{itemize}
      For example, string \enquote{20m} is equivalent to the integer value
      20971520, or 20 megabytes. Similarly the string \enquote{3.1m} is
      equivalent to the integer value 3250586. Only one multiplier is
      recognized, so \enquote{20kk} will not produce the same desired result as
      \enquote{20m}. Usage of string \enquote{.5m} will yield the same result as
      the string \enquote{0.5m}. Any unrecognized suffix is treated as an error
      and results in program termination with an error report.
    \tabularnewline\hline
%%
\EnvVarDecl{SHMEM\_DEBUG}
    & Any
    & Enable debugging messages
    \tabularnewline\hline
\end{longtable}

\medskip{}
