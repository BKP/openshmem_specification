\apisummary{
    Determines whether an interoperability feature is supported by the \openshmem
    library implementation.
}
\begin{apidefinition}

\begin{Csynopsis}
int @\FuncDecl{shmem\_query\_interoperability}@(int property);
\end{Csynopsis}

\begin{apiarguments}
    \apiargument{IN}{property}{The interoperability property queried by the user.}
\end{apiarguments}

\apidescription{
    \FUNC{shmem\_query\_interoperability} queries whether an interoperability property
    is supported by the \openshmem library. One of the following properties can be 
    queried in an \openshmem program after finishing the
    initialization call to \openshmem and that of the relevant programming models
    being used in the program. An \openshmem library implementation may extend the
    available properties.

    \begin{itemize}
    \item \VAR{SHMEM\_PROGRESS\_MPI} Query whether the \openshmem
    implementation makes progress for the MPI communication used in the user program.
    \end{itemize}
}

\apireturnvalues{
    The return value is \CONST{1} if \VAR{property} is supported by the \openshmem library;
    otherwise, it is \CONST{0}.
}

\apinotes{
  None.
}

\begin{apiexamples}

\apicexample
    {The following example queries whether the \openshmem library supports asynchronous
progress for MPI. If it returns 1, the library guarantees the MPI nonblocking send
is processed while PE 0 is in the busy wait loop with repeated calls to
\FUNC{shmem\_int\_atomic\_fetch} so that deadlock will not occur.}
    {./example_code/shmem_query_mpi_progress.c}
    {}

\end{apiexamples}

\end{apidefinition}
