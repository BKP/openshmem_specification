\apisummary{
  A structure type representing team configuration arguments
}

\begin{apidefinition}

  \begin{Csynopsis}
typedef struct {
  int disable_collectives;
  int return_local_limit;
  int num_contexts;
} shmem_team_config_t;
\end{Csynopsis}

  \vspace{1.0em}

  \apidescription{
    A team configuration argument acts as both input and output to the
    \FUNC{shmem\_team\_split\_*} routines.
    As an input, it specifies the requested capabilities of the team to be
    created.
    As an output, the configuration argument is conditionally updated on
    whether team creation is successful.
    If successful, the configuration argument is not modified;
    if unsuccessful, it is updated to specify the limiting configuration
    parameter(s).

    The \VAR{disable\_collectives} member allows for teams to be created
    without support for collective communications, which allows implementations
    to reduce team creation overheads for those teams.
    When its value is zero, it specifies that the team should have collectives
    enabled.
    When nonzero, the team will not support collective operations, which
    allows implementations to reduce team creation overheads.

    The \VAR{return\_local\_limit} member controls whether, after a failed
    team creation, the team configuration argument is updated with the
    locally restrictive parameter(s) or the most restrictive parameter(s)
    across the \acp{PE} of the new team.
    When its value is zero, the most restrictive parameters are returned;
    otherwise, the locally restrictive parameters are returned.

    The \VAR{num\_contexts} member specifies the total number of contexts
    created from this team that can simultaneously exist. These contexts
    may be created in any number of threads. A program
    may destroy any number of contexts made from this team and make
    any number of new ones so long as the total existing at any point
    remains less than \VAR{num\_contexts}. Any contexts created from this
    team must be destroyed before the team is destroyed, or the
    behavior is undefined.
    See Section~\ref{sec:ctx} for more on communication contexts and
    Section~\ref{subsec:shmem_team_create_ctx} for team-based context creation.

    When using the configuration structure to create teams, a mask parameter
    controls which fields to use and which to ignore. So, a program does
    not have to set all fields in the config struct; only those for which
    it does not want the default values.

    A configuration mask value is created by combining individual field
    masks with through a bitwise OR operation of the following library constants:
    
  {
  \apitablerow{\LibConstRef{SHMEM\_TEAM\_NOCOLLECTIVE}}{
    The team should be created using the value of the
    \VAR{disable\_collectives} member of the configuration parameter
    \VAR{config}.
  }
  \apitablerow{\LibConstRef{SHMEM\_TEAM\_LOCAL\_LIMIT}}{
    The team should be created using the value of the
    \VAR{return\_local\_limit} member of the configuration parameter
    \VAR{config}.
  }
  \apitablerow{\LibConstRef{SHMEM\_TEAM\_NUM\_CONTEXTS}}{
    The team should be created using the value of the
    \VAR{num\_contexts} member of the configuration parameter \VAR{config}.
  }
  }

   A configuration mask value of \CONST{0} indicates that the team
   should be created with the default values for all configuration
   parameters, as follows:

  {
  \apitablerow{disable\_collectives = \CONST{0}}{
    By default, teams support collective operations
    }
  \apitablerow{return\_local\_limit = \CONST{0}}{
    By default, when team creation fails, the configuration structure returns the most restrictive
    parameter value across all \acp{PE} in the new team
    }
  \apitablerow{num\_contexts = \CONST{0}}{
    By default, no contexts can be created on a new team
    }
  }

  }

  \apinotes{
    None.
  }

\end{apidefinition}
