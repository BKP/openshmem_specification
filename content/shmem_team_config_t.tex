\apisummary{
  A structure type representing team configuration arguments
}

\begin{apidefinition}

  \begin{Csynopsis}
typedef struct {
  int disable_collectives;
  int return_local_limit;
  int num_threads;
} shmem_team_config_t;
\end{Csynopsis}

  \vspace{1.0em}

  \apidescription{
    A team configuration argument acts as both input and output to the
    \FUNC{shmem\_team\_split\_*} routines.
    As an input, it specifies the requested capabilities of the team to be
    created.
    As an output, the configuration argument is conditionally updated on
    whether team creation is successful.
    If successful, the configuration argument is not modified;
    if unsuccessful, it is updated to specify the limiting configuration
    parameter(s).

    The \VAR{disable\_collectives} member allows for teams to be created
    without support for collective communications, which allows implementations
    to reduce team creation overheads for those teams.
    When its value is zero, it specifies that the team should have collectives
    enabled.
    When nonzero, the team will not support collective operations, which
    allows implementations to reduce team creation overheads.

    The \VAR{return\_local\_limit} member controls whether, after a failed
    team creation, the team configuration argument is updated with the
    locally restrictive parameter(s) or the most restrictive parameter(s)
    across the \acp{PE} of the new team.
    When its value is zero, the most restrictive parameters are returned;
    otherwise, the locally restrictive parameters are returned.

    The \VAR{num\_threads} member specifies the number of threads that will
    create contexts from the new team.
    It must have a nonnegative value.
    See Section~\ref{sec:ctx} for more on communication contexts and
    Section~\ref{subsec:shmem_team_create_ctx} for team-based context creation.
  }

  \apinotes{
    None.
  }

\end{apidefinition}
