\apisummary{
    Indicate whether all variables within an array of variables on the local \ac{PE} meet a specified test condition.
}

\begin{apidefinition}

\begin{C11synopsis}
int @\FuncDecl{shmem\_test\_all}@(TYPE *ivars, size_t nelems, const int *status, int cmp, TYPE cmp_value);
\end{C11synopsis}
where \TYPE{} is one of the point-to-point synchronization types specified by
Table \ref{p2psynctypes}.

\begin{Csynopsis}
int @\FuncDecl{shmem\_\FuncParam{TYPENAME}\_test\_all}@(TYPE *ivars, size_t nelems, const int *status, int cmp,
    TYPE cmp_value);
\end{Csynopsis}
where \TYPE{} is one of the point-to-point synchronization types and has a
corresponding \TYPENAME{} specified by Table \ref{p2psynctypes}.

\begin{apiarguments}

  \apiargument{IN}{ivars}{A pointer to an array of remotely accessible data
    objects.}
  \apiargument{IN}{nelems}{The number of elements in the \VAR{ivars} array.}
  \apiargument{IN}{status}{An optional mask array of length \VAR{nelems}
    that indicates which elements in \VAR{ivars} are excluded from the test set.}
  \apiargument{IN}{cmp}{A comparison operator from Table~\ref{p2p-consts}
    that compares elements of \VAR{ivars} with \VAR{cmp\_value}.}
  \apiargument{IN}{cmp\_value}{The value to be compared with the objects
    pointed to by \VAR{ivars}.}

\end{apiarguments}

\apidescription{
    The \FUNC{shmem\_test\_all} routine indicates whether all entries in the
    test set specified by \VAR{ivars} and \VAR{status} have satisfied the test
    condition at the calling \ac{PE}.  The \VAR{ivars} objects at the calling
    \ac{PE} may be updated by an \ac{AMO} performed by a thread located within
    the calling \ac{PE} or within another \ac{PE}.
    This routine does not block and returns zero if
    not all entries in \VAR{ivars} satisfied the test condition.
    \newtext{This routine compares each element of the \VAR{ivars} array in the
    test set with the value \VAR{cmp\_value} according to the comparison
    operator \VAR{cmp} at the calling \ac{PE}}.

    If \VAR{nelems} is 0, the test set is empty and this routine returns 1.

    The optional \VAR{status} is a mask array of length \VAR{nelems} where each element
    corresponds to the respective element in \VAR{ivars} and indicates whether
    the element is excluded from the test set.  Elements of \VAR{status} set to
    0 will be included in the test set, and elements set to 1 will be ignored.  If all elements
    in \VAR{status} are set to 1 or \VAR{nelems} is 0, the test set is empty
    and this routine returns 0.  If \VAR{status} is a null pointer, it is
    ignored and all elements in \VAR{ivars} are included in the test set.  The
    \VAR{ivars}, \VAR{indices}, and \VAR{status} arrays must not overlap in
    memory.

    Implementations must ensure that \FUNC{shmem\_test\_all} does not return 1
    before the update of the memory indicated by \VAR{ivars} is fully complete.
}

\apireturnvalues{
    \FUNC{shmem\_test\_all} returns 1 if all variables in \VAR{ivars} satisfy the test condition or if \VAR{nelems} is 0, otherwise this routine returns 0.
}

\apinotes{
  None.
}

\end{apidefinition}
