\apisummary{
    Waits for completion of outstanding operations on symmetric data objects
    issued by a \ac{PE}.
}

\begin{apidefinition}

\begin{Csynopsis}
void @\FuncDecl{shmem\_quiet}@(void);
void @\FuncDecl{shmem\_ctx\_quiet}@(shmem_ctx_t ctx);
\end{Csynopsis}

\begin{apiarguments}
    \apiargument{IN}{ctx}{A context handle specifying the context on which to
    perform the operation. When this argument is not provided, the operation is
    performed on the default context.}
\end{apiarguments}

\apidescription{
    The \FUNC{shmem\_quiet} routine ensures completion of all operations
    on symmetric data objects issued by the calling \ac{PE} on the given context.
    Table~\ref{mem-order} lists the operations for which the \FUNC{shmem\_quiet}
    routine ensures completion. All operations on symmetric data objects are
    guaranteed to be complete and visible to all \acp{PE} when
    \FUNC{shmem\_quiet} returns. If \VAR{ctx} has the value
    \CONST{SHMEM\_CTX\_INVALID}, no operation is performed.
}


\apireturnvalues{
    None.
}

\apinotes{
    \FUNC{shmem\_quiet} is most useful as a way of ensuring completion of
    several operations on symmetric data objects initiated by the calling
    \ac{PE}. For example, one might use \FUNC{shmem\_quiet} to await delivery
    of a block of data before issuing another \PUT{} or nonblocking
    \PUT{} routine, which sets a completion flag on another \ac{PE}.
    \FUNC{shmem\_quiet} is not usually needed if
    \FUNC{shmem\_barrier\_all} or \FUNC{shmem\_barrier} are called.  The barrier
    routines wait for the completion of outstanding operations to
    symmetric data objects on all \acp{PE}.

    In an \openshmem program with multithreaded \acp{PE}, it is the
    user's responsibility to ensure ordering between operations issued by the
    threads in a \ac{PE} that target symmetric memory and calls by threads in
    that \ac{PE} to \FUNC{shmem\_quiet}. The \FUNC{shmem\_quiet} routine can
    enforce memory store ordering only for the calling thread. Thus, to ensure
    ordering for memory stores performed by a thread that is not the thread
    calling \FUNC{shmem\_quiet}, the update must be made visible to the calling
    thread according to the rules of the memory model associated with the
    threading environment.

    A call to \FUNC{shmem\_quiet} by a thread completes the operations posted
    prior to calling \FUNC{shmem\_quiet}. If the user intends to also complete
    operations issued by a thread that is not the thread calling
    \FUNC{shmem\_quiet}, the user must ensure that the operations are performed
    prior to the call to \FUNC{shmem\_quiet}. This may require the use of a
    synchronization operation provided by the threading package. For example,
    when using POSIX Threads, the user may call the
    \FUNC{pthread\_barrier\_wait} routine to ensure that all threads have issued
    operations before a thread calls \FUNC{shmem\_quiet}.

    \FUNC{shmem\_quiet} does not have an effect on the ordering between memory
    accesses issued by the target PE. \FUNC{shmem\_wait\_until},
    \FUNC{shmem\_test}, \FUNC{shmem\_barrier}, \FUNC{shmem\_barrier\_all} routines
    can be called by the target PE to guarantee ordering of its memory accesses.
}

\begin{apiexamples}

\apicexample
    {The following example uses \FUNC{shmem\_quiet} in a \Cstd[11] program: }
    {./example_code/shmem_quiet_example.c}
    {\VAR{Put1} and \VAR{put2} will be completed and visible before \VAR{put3}
    and \VAR{put4}.}
\end{apiexamples}

\end{apidefinition}
