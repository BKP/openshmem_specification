\color{Green}
\apisummary{
    This nonblocking atomic routine performs an atomic fetch-and-increment
    operation on a remote data object.
}

\begin{apidefinition}

\begin{C11synopsis}
void @\FuncDecl{shmem\_atomic\_fetch\_inc\_nbi}@(TYPE *fetch, TYPE *dest, int pe);
void @\FuncDecl{shmem\_atomic\_fetch\_inc\_nbi}@(shmem_ctx_t ctx, TYPE *fetch, TYPE *dest, int pe);
\end{C11synopsis}
where \TYPE{} is one of the standard \ac{AMO} types specified by
Table~\ref{stdamotypes}.

\begin{Csynopsis}
void @\FuncDecl{shmem\_\FuncParam{TYPENAME}\_atomic\_fetch\_inc\_nbi}@(TYPE *fetch, TYPE *dest, int pe);
void @\FuncDecl{shmem\_ctx\_\FuncParam{TYPENAME}\_atomic\_fetch\_inc\_nbi}@(shmem_ctx_t ctx, TYPE *fetch, TYPE *dest, int pe);
\end{Csynopsis}
where \TYPE{} is one of the standard \ac{AMO} types and has a corresponding
\TYPENAME{} specified by Table~\ref{stdamotypes}.

\begin{apiarguments}

\apiargument{IN}{ctx}{A context handle specifying the context on which to
    perform the operation. When this argument is not provided, the
    operation is performed on the default context.}
\apiargument{OUT}{fetch}{Local data object to be updated.}
\apiargument{OUT}{dest}{The remotely accessible data object to be updated on the
    remote \ac{PE}.}
\apiargument{IN}{pe}{An integer that indicates the \ac{PE} number on which
    \dest{} is to be updated.}

\end{apiarguments}


\apidescription{
   These nonblocking \FUNC{shmem\_atomic\_fetch\_inc\_nbi} routines perform an
   atomic fetch-and-increment operation. This routine returns after posting the
   operation. The operation is considered complete after a subsequent call to
   \FUNC{shmem\_quiet}. At the completion of \FUNC{shmem\_quiet}, the \dest{} on
   \ac{PE} \VAR{pe} is increased by one and the previous contents of \dest{} are
   fetched into the \VAR{fetch} local data object as one atomic operation.
}

\apireturnvalues{
    None.
}

\apinotes{
    None.
}

\end{apidefinition}
\color{Black}
