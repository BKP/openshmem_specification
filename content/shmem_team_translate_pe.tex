\apisummary{
  Translate a given \ac{PE} number from one team to the corresponding
  \ac{PE} number in another team.
}

\begin{apidefinition}

\begin{Csynopsis}
int @\FuncDecl{shmem\_team\_translate\_pe}@(shmem_team_t src_team, int src_pe,
    shmem_team_t dest_team);
\end{Csynopsis}

\begin{apiarguments}
\apiargument{IN}{src\_team}{A valid SHMEM team handle.}
\apiargument{IN}{src\_pe}{A \ac{PE} number in src\_team.}
\apiargument{IN}{dest\_team}{A valid SHMEM team handle.}
\end{apiarguments}

\apidescription{
The \FUNC{shmem\_team\_translate\_pe} routine will translate a given \ac{PE} number
to the corresponding \ac{PE} number in another team.
Specifically, given the \VAR{src\_pe} in \VAR{src\_team}, this routine returns that
\ac{PE}'s number in \VAR{dest\_team}. If \VAR{src\_pe} is not a member of both the
\VAR{src\_team} and \VAR{dest\_team}, a value of \CONST{-1} is returned.


If either of the \VAR{src\_team} or \VAR{dest\_team} handle is invalid, the behavior is undefined.
}

\apireturnvalues{
The specified \ac{PE}'s number in the \VAR{dest\_team}, or a value of \CONST{-1} if any
team handle arguments are invalid or the \VAR{src\_pe} is not in both the source and destination teams.
}

\apinotes{
  If \LibHandleRef{SHMEM\_TEAM\_WORLD} is provided as the
  \VAR{dest\_team} parameter, this routine acts as a global \ac{PE}
  number translator and will return the corresponding
  \LibHandleRef{SHMEM\_TEAM\_WORLD} number.
}

\begin{apiexamples}

    \apicexample
    {The following example demonstrates the use of the team \ac{PE}
    number translation routine. The program makes a new team of all
    of the even number \acp{PE} in the default team. Then, all \acp{PE}
    in the new team acquire their \ac{PE} number in the new team
    and translate it to the \ac{PE} number in the default team.}
    {./example_code/shmem_team_translate_pe.c}
    {}

\end{apiexamples}

\end{apidefinition}
