\apisummary{
    Determines whether a \ac{PE} is accessible via \openshmem's data transfer
    routines.
}

\begin{apidefinition}

\begin{Csynopsis}
int @\FuncDecl{shmem\_pe\_accessible}@(int pe);
\end{Csynopsis}

\begin{Fsynopsis}
LOGICAL LOG, SHMEM_PE_ACCESSIBLE
INTEGER pe
LOG = SHMEM_PE_ACCESSIBLE(pe)
\end{Fsynopsis}

\begin{apiarguments}
    \apiargument{IN}{pe}{Specific \ac{PE} to be checked for accessibility from
    the local \ac{PE}.}
\end{apiarguments}

\apidescription{
    \FUNC{shmem\_pe\_accessible} is a query routine that indicates whether a
    specified \ac{PE} is accessible via \openshmem from the local \ac{PE}. The
    \FUNC{shmem\_pe\_accessible} routine returns a value indicating whether the remote
    \ac{PE} is a process running from the same executable file as the local
    \ac{PE}, thereby indicating whether full support for symmetric data objects,
    which may reside in either static memory or the symmetric heap, is available.
}

\apireturnvalues{
    \CorCpp: The return value is 1 if the specified \ac{PE} is a valid remote \ac{PE}
    for \openshmem routines; otherwise, it is 0.

    \Fortran: The return value is \CONST{.TRUE.} if the specified \ac{PE} is a valid
    remote \ac{PE} for \openshmem routines; otherwise, it is \CONST{.FALSE.}.
}

\apinotes{
    This routine may be particularly useful for hybrid programming with other
    communication libraries (such as \ac{MPI}) or parallel languages.  For
    example, when an \ac{MPI} job uses \ac{MPMD} mode, multiple executable
    \ac{MPI} programs are executed as part of the same MPI job.  In such cases,
    \openshmem support may only be available between processes running from the
    same executable file.  In addition, some environments may allow a hybrid
    job may span multiple network partitions.  In such scenarios, \openshmem
    support may only be available between \acp{PE} within the same partition.
}

\end{apidefinition}
