\apisummary{
    A collective operation that releases all resources used by the \openshmem
    library.  This only terminates the \openshmem portion of a program, not the
    entire program.
}

\begin{apidefinition}

\begin{Csynopsis}
void @\FuncDecl{shmem\_finalize}@(void);
\end{Csynopsis}

\begin{apiarguments}
    \apiargument{None.}{}{}
\end{apiarguments}

\apidescription{
    \FUNC{shmem\_finalize} is a collective operation that ends the \openshmem
    portion of a program previously initialized by \FUNC{shmem\_init} or \FUNC{shmem\_init\_thread} and
    releases all resources used by the \openshmem library. This collective
    operation requires all \acp{PE} to participate in the call. There is an
    implicit global barrier in \FUNC{shmem\_finalize} to ensure that pending
    communications are completed and that no resources are released until all
    \acp{PE} have entered \FUNC{shmem\_finalize}.
    This routine destroys all teams created by the \openshmem program.
    As a result, all shareable contexts are destroyed.
    The user is
    responsible for destroying all contexts with the
    \CONST{SHMEM\_CTX\_PRIVATE} option enabled prior to calling this routine;
    otherwise, the behavior is undefined.
    \FUNC{shmem\_finalize} must be
    the last \openshmem library call encountered in the \openshmem portion of a
    program. A call to \FUNC{shmem\_finalize} will release all resources
    initialized by a corresponding call to \FUNC{shmem\_init} or \FUNC{shmem\_init\_thread}. All processes
    that represent the \acp{PE} will still exist after the
    call to \FUNC{shmem\_finalize} returns, but they will no longer have access
    to resources that have been released.
}

\apireturnvalues{
    None.
}

\apinotes{
    \FUNC{shmem\_finalize} releases all resources used by the \openshmem library
    including the symmetric memory heap and pointers initiated by
    \FUNC{shmem\_ptr}. This collective operation requires all \acp{PE} to
    participate in the call, not just a subset of the \acp{PE}. The
    non-\openshmem portion of a program may continue after a call to
    \FUNC{shmem\_finalize} by all \acp{PE}.
}

\begin{apiexamples}

\apicexample
    {The following finalize example is for \Cstd[11] programs:}
    {./example_code/shmem_finalize_example.c}
    {}

\end{apiexamples}

\end{apidefinition}
