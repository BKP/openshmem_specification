\apisummary{
    Copies one data item to a remote \ac{PE}.
}

\begin{apidefinition}

\begin{Csynopsis}
void shmem_char_p(char *addr, char value, int pe);
void shmem_short_p(short *addr, short value, int pe);
void shmem_int_p(int *addr, int value, int pe);
void shmem_long_p(long *addr, long value, int pe);
void shmem_longlong_p(long long *addr, long long value, int pe);
void shmem_float_p(float *addr, float value, int pe);
void shmem_double_p(double *addr, double value, int pe);
void shmem_longdouble_p(long double *addr, long double value, int pe);
\end{Csynopsis}

\begin{apiarguments}
    \apiargument{IN}{addr}{The remotely accessible array element or scalar data object
    which will receive the data on the remote \ac{PE}.}
    \apiargument{IN}{value}{The value to be transferred to \VAR{addr} on the
    remote \ac{PE}.}
    \apiargument{IN}{pe}{The number of the remote \ac{PE}.}
\end{apiarguments}

\apidescription{
    These routines provide a very low latency put capability for single elements of
    most basic types.
    
    As with \FUNC{shmem\_put}, these routines start the remote transfer and may
    return before the data is delivered to the remote \ac{PE}.  Use
    \FUNC{shmem\_quiet} to force completion of all remote \PUT{} transfers.
}

\apireturnvalues{
    None.
}

\apinotes{
    None.
}

\begin{apiexamples}

    \apicexample
    {The following example uses \FUNC{shmem\_double\_p}  in a \Clang{} program.}	 
    {./example_code/shmem_p_example.c}
    {}

\end{apiexamples}

\end{apidefinition}
