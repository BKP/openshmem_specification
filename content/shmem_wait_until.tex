\apisummary{
    Wait for a variable on the local \ac{PE} to change.
}

\begin{apidefinition}

\begin{C11synopsis}
void @\FuncDecl{shmem\_wait\_until}@(TYPE *ivar, int cmp, TYPE cmp_value);
\end{C11synopsis}
where \TYPE{} is one of the point-to-point synchronization types specified by
Table \ref{p2psynctypes}.

\begin{Csynopsis}
void @\FuncDecl{shmem\_\FuncParam{TYPENAME}\_wait\_until}@(TYPE *ivar, int cmp, TYPE cmp_value);
\end{Csynopsis}
where \TYPE{} is one of the point-to-point synchronization types and has a
corresponding \TYPENAME{} specified by Table~\ref{p2psynctypes}.

\begin{DeprecateBlock}
\begin{CsynopsisCol}
void @\FuncDecl{shmem\_wait\_until}@(long *ivar, int cmp, long cmp_value);
void @\FuncDecl{shmem\_wait}@(long *ivar, long cmp_value);
void @\FuncDecl{shmem\_\FuncParam{TYPENAME}\_wait}@(TYPE *ivar, TYPE cmp_value);
\end{CsynopsisCol}
where \TYPE{} is one of \{\CTYPE{short}, \CTYPE{int}, \CTYPE{long},
\CTYPE{long long}\} and has a corresponding \TYPENAME{} specified by
Table~\ref{p2psynctypes}.
\end{DeprecateBlock}

\begin{apiarguments}

\apiargument{IN}{ivar}{A remotely accessible integer variable. When using \CorCpp,
    the type of \VAR{ivar} should match that implied in the SYNOPSIS section.} 
\apiargument{IN}{cmp}{The compare operator that compares \VAR{ivar} with
  \VAR{cmp\_value}.
  When using \CorCpp, it must be of type \CTYPE{int}.}
\apiargument{IN}{cmp\_value}{\VAR{cmp\_value} must be of type integer.  When
    using \CorCpp, the type of \VAR{cmp\_value} should match that implied in the
    SYNOPSIS section.}

\end{apiarguments}

\apidescription{
    The \FUNC{shmem\_wait} and \FUNC{shmem\_wait\_until} operations block until
    the value contained in the symmetric data object, \VAR{ivar}, at the
    calling \ac{PE} satisfies the wait condition.  The \VAR{ivar} object at the
    calling \ac{PE} may be updated by an \ac{AMO} performed by a thread located
    within the calling \ac{PE} or within another \ac{PE}.

    These routines can be used to implement point-to-point synchronization
    between \acp{PE} or between threads within the same \ac{PE}.  A call to
    \FUNC{shmem\_wait} blocks until the value of
    \VAR{ivar} at the calling \ac{PE} is not equal to \VAR{cmp\_value}.  A call
    to \FUNC{shmem\_wait\_until} blocks until the value of \VAR{ivar} at the
    calling \ac{PE} satisfies the wait condition specified by the comparison
    operator, \VAR{cmp}, and comparison value, \VAR{cmp\_value}.

    Implementations must ensure that \FUNC{shmem\_wait} and
    \FUNC{shmem\_wait\_until} do not return before the update of the memory
    indicated by \VAR{ivar} is fully complete.
}

\apireturnvalues{
    None.
}

\apinotes{
  As of \openshmem[1.4], the \FUNC{shmem\_wait} routine is deprecated;
  however, \FUNC{shmem\_wait} is equivalent to \FUNC{shmem\_wait\_until}
  where \VAR{cmp} is \CONST{SHMEM\_CMP\_NE}.
}

\apiimpnotes{
  Some platforms may allow wait operations to efficiently poll or block on an
  update to \VAR{ivar}.  On others, an atomic read operation may be needed to
  observe updates to \VAR{ivar}.  On platforms where atomic read operations
  incur high overhead, implementations may be able to reduce the number of
  atomic reads performed by using non-atomic reads of \VAR{ivar} to wait for a
  change to occur, followed by an atomic read operation to fetch the updated
  value.
}

\begin{apiexamples}

\apicexample
{The following \CorCpp{} routine waits until the value in \VAR{ivar} is set to
be less than zero by a transfer from a remote PE:}
{./example_code/shmem_wait3_example.c}
{}

\end{apiexamples}

\end{apidefinition}
