%
% Copyright (c) 2011, 2012
%   University of Houston System and Oak Ridge National Laboratory.
% 
% All rights reserved.
% 
% Redistribution and use in source and binary forms, with or without
% modification, are permitted provided that the following conditions
% are met:
% 
% o Redistributions of source code must retain the above copyright notice,
%   this list of conditions and the following disclaimer.
% 
% o Redistributions in binary form must reproduce the above copyright
%   notice, this list of conditions and the following disclaimer in the
%   documentation and/or other materials provided with the distribution.
% 
% o Neither the name of the University of Houston System, Oak Ridge
%   National Laboratory nor the names of its contributors may be used to
%   endorse or promote products derived from this software without specific
%   prior written permission.
% 
% THIS SOFTWARE IS PROVIDED BY THE COPYRIGHT HOLDERS AND CONTRIBUTORS
% ``AS IS'' AND ANY EXPRESS OR IMPLIED WARRANTIES, INCLUDING, BUT NOT
% LIMITED TO, THE IMPLIED WARRANTIES OF MERCHANTABILITY AND FITNESS FOR
% A PARTICULAR PURPOSE ARE DISCLAIMED. IN NO EVENT SHALL THE COPYRIGHT
% HOLDER OR CONTRIBUTORS BE LIABLE FOR ANY DIRECT, INDIRECT, INCIDENTAL,
% SPECIAL, EXEMPLARY, OR CONSEQUENTIAL DAMAGES (INCLUDING, BUT NOT LIMITED
% TO, PROCUREMENT OF SUBSTITUTE GOODS OR SERVICES; LOSS OF USE, DATA, OR
% PROFITS; OR BUSINESS INTERRUPTION) HOWEVER CAUSED AND ON ANY THEORY OF
% LIABILITY, WHETHER IN CONTRACT, STRICT LIABILITY, OR TORT (INCLUDING
% NEGLIGENCE OR OTHERWISE) ARISING IN ANY WAY OUT OF THE USE OF THIS
% SOFTWARE, EVEN IF ADVISED OF THE POSSIBILITY OF SUCH DAMAGE.
%

\chapter{About This Book}

Why this book exists, rationale for HPC and parallel computing,
high-level overview of SHMEM.

Can talk about impact of computing on science and modern life.  How
pushing boundaries means we need ever bigger and faster computing, and
also a way of harnessing huge machines.

Introductory discussion of weather forecasting might be good here
since it is a topic that affects everyone and is easy to use to get a
handle on HPC.

\section{How HPC is Used}

Talk about how computing is used in science and other disciplines.
Give some examples of how computing is used in various ways in
different subjects.  Could be all the way from embedded devices
(leverage TI experience) through traditional laptop/desktop use on to
tablet (ha ha) use and then through to HPC/cloud for imaging,
diagnosis.

Challenges facing those subjects where HPC is concerned.  What does
HPC enable is we go bigger and better with performance delivered?

Need for HPC to solve those problems.

\section{Addressing Challenges}

Time to talk about really really big machines?  History of ``-scale''
suffix and timeline showing performance of top500 leaders.  Current
state of play and hurdles to get to Exascale.
