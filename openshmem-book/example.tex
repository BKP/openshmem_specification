%
% Copyright (c) 2011, 2012
%   University of Houston System and Oak Ridge National Laboratory.
% 
% All rights reserved.
% 
% Redistribution and use in source and binary forms, with or without
% modification, are permitted provided that the following conditions
% are met:
% 
% o Redistributions of source code must retain the above copyright notice,
%   this list of conditions and the following disclaimer.
% 
% o Redistributions in binary form must reproduce the above copyright
%   notice, this list of conditions and the following disclaimer in the
%   documentation and/or other materials provided with the distribution.
% 
% o Neither the name of the University of Houston System, Oak Ridge
%   National Laboratory nor the names of its contributors may be used to
%   endorse or promote products derived from this software without specific
%   prior written permission.
% 
% THIS SOFTWARE IS PROVIDED BY THE COPYRIGHT HOLDERS AND CONTRIBUTORS
% ``AS IS'' AND ANY EXPRESS OR IMPLIED WARRANTIES, INCLUDING, BUT NOT
% LIMITED TO, THE IMPLIED WARRANTIES OF MERCHANTABILITY AND FITNESS FOR
% A PARTICULAR PURPOSE ARE DISCLAIMED. IN NO EVENT SHALL THE COPYRIGHT
% HOLDER OR CONTRIBUTORS BE LIABLE FOR ANY DIRECT, INDIRECT, INCIDENTAL,
% SPECIAL, EXEMPLARY, OR CONSEQUENTIAL DAMAGES (INCLUDING, BUT NOT LIMITED
% TO, PROCUREMENT OF SUBSTITUTE GOODS OR SERVICES; LOSS OF USE, DATA, OR
% PROFITS; OR BUSINESS INTERRUPTION) HOWEVER CAUSED AND ON ANY THEORY OF
% LIABILITY, WHETHER IN CONTRACT, STRICT LIABILITY, OR TORT (INCLUDING
% NEGLIGENCE OR OTHERWISE) ARISING IN ANY WAY OUT OF THE USE OF THIS
% SOFTWARE, EVEN IF ADVISED OF THE POSSIBILITY OF SUCH DAMAGE.
%

\chapter{\openshmem By Example}
\label{chp:example}

In this chapter we introduce the most used features of \openshmem
through example programs.

\section{First Steps}

\subsection{Sequential Code}

Our first example will be the ubiquitous ``hello world'' program.
Here is the standard, sequential version in C that will act as our
template:

\begin{minipage}{\linewidth}
\vspace{0.1in}
\numberedlisting{language=C}{programs/hello-seq.c}
\outputlisting{language=bash,caption={Expected Output}}{programs/hello-seq.output}
\vspace{0.1in}
\end{minipage}

\subsection{Message Passing Interface}

The Message Passing Interface (MPI) has become the \textit{de facto}
parallel and distributed ``go to'' paradigm.  We will show the MPI
``hello world'' in listing~\ref{mpi-hello} as it is likely to be
familiar to many readers and can thus serve as a conceptual
launching-point:

\begin{minipage}{\linewidth}
\vspace{0.1in}
\numberedlisting{label=mpi-hello,language=MPI+C}{programs/hello-mpi.c}
\outputlisting{language=bash,caption={Expected Output (4
    processors)}}{programs/hello-mpi.output}
\vspace{0.1in}
\end{minipage}

Items of interest:

\begin{itemize}
\item the lifetime of the MPI environment is enclosed by
  initialization via \texttt{MPI\_Init()} (line 9) and finalization
  via \texttt{MPI\_Finalize()} (line 13).
\item the MPI environment is queried (lines 10, 11) to discover how
  many processors are taking part in this parallel program, and which
  ``rank'' an individual execution has
\item the order in which lines of output appear will vary from run to
  run because the constituent processes of the MPI program run
  concurrently and not in sequence
\end{itemize}

\subsection{\openshmem}

Now let us have a look at an equivalent OpenSMHEM program in
listing~\ref{openshmem-hello}:

\begin{minipage}{\linewidth}
\vspace{0.1in}
\numberedlisting{label=openshmem-hello,language=OSH+C}{programs/hello-openshmem.c}
\outputlisting{language=bash,caption={Expected Output (4
    processors)}}{programs/hello-openshmem-c.output}
\vspace{0.1in}
\end{minipage}

The programs are quite similar in their overall structure but here are
the main differences in the \openshmem version:

\begin{itemize}
\item the initialization call \texttt{start\_pes()}, (line 9) has a
  single integer argument, 0, which is ignored~\footnote{The unused
    argument is for compatibility with older SHMEM implementations.}.
\item there is no explicit finalize call, either a return from
  \texttt{main()} (line 13) or an explicit \texttt{exit()} acts as an
  implicit \openshmem finalization.
\item as in program ~\ref{mpi-hello} the order in which lines appear
  in the output is not fixed.
\end{itemize}

\openshmem also has a Fortran API, so for completeness we now give the
same program written in Fortran in listing~\ref{openshmem-hello-f90}:

\begin{minipage}{\linewidth}
\vspace{0.1in}
\numberedlisting{label=openshmem-hello-f90,language=OSH+F}{programs/hello-openshmem.f90}
\outputlisting{language=bash,caption={Expected Output (4
    processors)}}{programs/hello-openshmem-f90.output}
\vspace{0.1in}
\end{minipage}

\section{Communication}

Of course, the whole point of MPI and \openshmem is communication.
\openshmem supports both point-to-point communication, in which one PE
exchanges data with another individual PE, and collective
communication, in which multiple PEs engage in a specified operation
as a group.

\subsection{Point-to-Point Communication}
\label{sec:p-to-p}

Let us now have a look at a very simple \openshmem example in
listing~\ref{rotput} in which each PE sends a value to the PE above it
(so 0 sends to 1, 1 to 2, 2 to 3, and so on until the last PE wraps
around back to PE 0).

\begin{minipage}{\linewidth}
\vspace{0.1in}
\numberedlisting{label=rotput,language=OSH+C}{programs/rotate-put.c}
\outputlisting{language=bash,caption={Expected Output (4
    processors)}}{programs/rotate-put.output}
\vspace{0.1in}
\end{minipage}

Points to notice:

\begin{itemize}
\item the target variable ``d'' (line 5) is a global variable to make
  it remotely accessible
\item we use a short form of the ``put'' call for a single value (line
  22)
\item there is a global barrier \emph{after} the ``put'' (line 24):
  there is no need to have a global barrier \emph{before} the ``put''
  because only the local value needs to be ready, and the global
  variable is pre-allocated in the executable.
\end{itemize}

\subsection{Collective Communication}

Here is an example of \openshmem's broadcast: this is a ``1-to-others''
broadcast in which one designated ``root'' PE sends data to the other
PEs in the ``active set''.  The root PE does not update data on
itself, though.

\begin{minipage}{\linewidth}
\vspace{0.1in}
\numberedlisting{language=OSH+C}{programs/broadcast.c}
\outputlisting{language=bash,caption={Expected Output (3 processors,
    output sorted for clarity)}}{programs/broadcast.output}
\vspace{0.1in}
\end{minipage}

Points to notice:

\begin{itemize}
\item collective routines use a symmetric synchronization variable
  (defined on line 6) \todo{cross-ref with more detailed discussion of
  synch. vars}
\item the synchronization variable must be initialized \emph{on all
  PEs} before use (lines 28--31)
\item \texttt{shmem\_barrier\_all} (line 32) ensures the data and the
  synchronization variable are initialized \emph{everywhere} before
  proceeding with the collective routine.
\item the size of the data updated depends on the number of PEs in the
  program overall, so we dynamically allocate symmetric memory (lines
  18--27) (checking the return value of \texttt{shmalloc} omitted for
  brevity)
\item dynamically allocated memory should be released when not needed
  any more (lines 42--43)
\end{itemize}

\section{Atomic Operations}

\subsection{Arithmetic}

There are 2 kinds of atomic arithmetic operations in \openshmem.  In
both cases, one PE initiates an atomic change to a variable on another
PE.

\begin{description}
  \item[{remote increment/addition}] the remote variable is updated
  \item[{fetch \& remote increment/addition}] the remote variable is
    updated, and the \emph{previous} value of the remote variable is
    returned to the caller
\end{description}

\subsubsection{Addition}

Here is an example of an atomic addition: the program in listing~\ref{addprog}
computes the sum from $1$ to $N-1$:

\begin{minipage}{\linewidth}
\vspace{0.1in}
\numberedlisting{label=addprog,language=OSH+C}{programs/add.c}
\outputlisting{language=bash,caption={Expected Output (4 processors)}}{programs/add.output}
\vspace{0.1in}
\end{minipage}

Points to notice:

\begin{itemize}
\item the counter is initialized to 0 via the global declaration (line
  5)
\item all PEs $> 0$ add their PE number \emph{atomically} to the
  variable ``counter'' on PE 0 (line 17)
\item the order in which the additions happen is unknown in advance,
  but the atomic guarantee means that 2 PEs cannot interfere with each
  other's updates
\end{itemize}

\subsubsection{Increment}

The increment operation in listing~\ref{incprog} is a special case of
addition.  The program simply counts the number of $PEs - 1$.

\begin{minipage}{\linewidth}
\vspace{0.1in}
\numberedlisting{label=incprog,language=OSH+C}{programs/inc.c}
\outputlisting{language=bash,caption={Expected Output (4 processors)}}{programs/inc.output}
\vspace{0.1in}
\end{minipage}

Points to notice:

\begin{itemize}
\item the increment call does not need to pass a value to be added
  since it is 1 by definition (line 17)
\end{itemize}
\subsubsection{Fetch and Add}

The fetch-and-add call in listing~\ref{faddprog} adds a value to a
target variable and returns the previous value of that variable to the
caller PE.

\begin{minipage}{\linewidth}
\vspace{0.1in}
\numberedlisting{label=faddprog,language=OSH+C}{programs/fadd.c}
\outputlisting{language=bash,caption={Expected Output (2 processors)}}{programs/fadd.output}
\vspace{0.1in}
\end{minipage}

\subsubsection{Fetch and Increment}

The fetch-and-increment operation in listing~\ref{fincprog} is a
special case of fetch-and-addition.

\begin{minipage}{\linewidth}
\vspace{0.1in}
\numberedlisting{label=fincprog,language=OSH+C}{programs/finc.c}
\outputlisting{language=bash,caption={Expected Output (2 processors)}}{programs/finc.output}
\vspace{0.1in}
\end{minipage}

\subsection{Swaps}

\subsubsection{Unconditional}

An \openshmem swap atomically writes a new value from the caller PE
into a variable on the target PE, and returns the old value to the
caller.

\todo{NEED swap EXAMPLE}

\subsubsection{Conditional}

An \openshmem conditional swap atomically writes a new value from the
caller PE into a variable on the target PE, but only if a condition
value matches.  In either case, the old value of the target variable
is returned to the caller.

\todo{NEED cswap EXAMPLE}

\section{Locks}

Locks in \openshmem are \emph{distributed}.  A lock is built on a
symmetric variable whose value is managed by PEs through the API
below.  The \openshmem lock is effectively a mutual-exclusion device
that allows precisely \emph{one} PE access to a code region at a time.

\subsection{Set Lock}

The

\begin{verbatim}
shmem_set_lock (long *L)
\end{verbatim}

% \apilisting{shmem\_set\_lock (long *L)}

call allows the calling PE to ``grab'' a symmetric variable and to
then block access to a code region until or unless the lock is cleared
(v.i.).

\texttt{shmem\_set\_lock} returns, and the calling PE proceeds, only
when no other PE has locked the named symmetric variable.

The lock variable must be initialized to ``0'' before first use.
After that, the lock API calls must be used to manage the variable.


\subsection{Clear Lock}

A PE that holds a lock on the symmetric variable ``L'' can release
that lock by calling

\begin{verbatim}
shmem_clear_lock (long *L)
\end{verbatim}

Before release, all pending \openshmem communications initiated in the
locked region are completed.

\subsection{Test Lock}

A PE can check whether a lock is available by calling

\begin{verbatim}
shmem_test_lock (long *L)
\end{verbatim}

If the lock on ``L'' is currently held by another PE,
\texttt{shmem\_test\_lock} returns 1 (true) immediately and does not
set the lock.

If the lock on ``L'' is currently clear, \texttt{shmem\_test\_lock}
claims the lock and returns 0 (false).

\subsection{Use of Locks}

% \todo{want a good example of where locks would be useful.
%   E.g.\ managing a distributed data structure of some kind}

Consider a program that implements a distributed data structure: each
PE manages part of the overall structure and any PE can update values
on other PEs.  An example of such a structure is a distributed hash
table.  In such a table, a hash function maps keys to a particular
table index, where associated values are then stored (maybe as a
linked list or tree).

When updates are performed, the presence of multiple writers
(i.e.\ multiple PEs that could be updating entries arbitrarily at the
same time) means that 2 PEs could collide and corrupt the table by
writing new data on top of each other.

Locks in \openshmem provide a way for the PE that is performing the
update to indicate that it ``owns'' that table entry on the other PE
and has exclusive access to write (and/or read) the locked index until
it releases the lock.

\subsection{Use of Ordering}

Talk about fence and quiet for sentinels that flag arrival of data.
There's an example of quiet in the man pages that should be expanded.

Think this is the right place to discuss ordering since it has a
similar purposes to locks, namely putting a boundary around a region
with a separate advisory variable.
