\openshmem provides ISO \Clang and \Fortran \textit{90} language bindings.
Any implementation that provides both \Clang and \Fortran bindings can claim
conformance to the specification. An implementation that provides e.g.\ only a
\Clang interface may claim to conform to the \openshmem specification with
respect to the \Clang language, but not to \Fortran, and should make this
clear in its documentation. The \openshmem header files for \Clang and
\Fortran must contain only the interfaces and constant names defined in this
specification.

\openshmem \ac{API}s can be implemented as either routines or macros. However,
implementing the interfaces using macros is strongly discouraged as this could
severely limit the use of external profiling tools and high-level compiler
optimizations. An \openshmem program should avoid defining routine names,
variables, or identifiers with the prefix \shmemprefix (for \Clang and
\Fortran), \shmemprefixC (for \Clang) or with \openshmem \ac{API} names.

All \openshmem extension \ac{API}s that are not part of this specification must
be defined in the \FUNC{shmemx.h} include file. These extensions shall use
\FUNC{shmemx\_} prefix for all routine, variable, and constant names.
