\apisummary{
    Performs an atomic add operation on a remote symmetric data object.
}

\begin{apidefinition}

\begin{Csynopsis}
void shmem_int_add(int *dest, int value, int pe);
void shmem_long_add(long *dest, long value, int pe);
void shmem_longlong_add(long long *dest, long long value, int pe);
\end{Csynopsis}

\begin{Fsynopsis}
INTEGER pe
INTEGER*4  value_i4
CALL SHMEM_INT4_ADD(dest, value_i4, pe)
INTEGER*8 value_i8
CALL SHMEM_INT8_ADD(dest, value_i8, pe)
\end{Fsynopsis}

\begin{apiarguments}
    \apiargument{OUT}{dest}{The remotely accessible integer data object to be
        updated  on the remote \ac{PE}.  If you are using \CorCpp, the type of
        \dest{} should match that implied in the SYNOPSIS section.}
    \apiargument{IN}{value}{The value to be atomically added to \dest. If you
        are using \CorCpp, the type of \VAR{value} should match that  implied  in
        the SYNOPSIS  section.  If you are using \Fortran, it must be of type
        integer with an element size of \dest.}
    \apiargument{IN}{pe}{An integer that indicates the \ac{PE} number upon which
        \dest{} is to be updated.  If you are using \Fortran, it must be a default
        integer value.}
\end{apiarguments}

\apidescription{
    The \FUNC{shmem\_add} routine performs an atomic add operation. It adds
    \VAR{value} to \dest{} on \ac{PE} \VAR{pe} and atomically updates the \dest{}
    without returning the value.
 } 

\apidesctable{
    If you are using \Fortran, \VAR{dest} must be of the following type:
}{Routine}{Data type of \VAR{dest} and \VAR{source}}

\apitablerow{SHMEM\_INT4\_ADD}{\CONST{4}-byte integer}
\apitablerow{SHMEM\_INT8\_ADD}{\CONST{8}-byte integer}

\apireturnvalues{
    None.
}

\apinotes{
    The term remotely accessible is defined in the Introduction.
}

\begin{apiexamples}

\apicexample
    {}
    {./example_code/shmem_add_example.c}
    {}

\end{apiexamples}

\end{apidefinition}
