\apisummary{
    Returns the number of the calling \ac{PE}.
}

\begin{apidefinition}

\begin{Csynopsis}
int shmem_my_pe(void);
\end{Csynopsis}

\begin{Fsynopsis}
INTEGER SHMEM_MY_PE, ME
ME = SHMEM_MY_PE()
ME = MY_PE ()
\end{Fsynopsis}

\begin{apiarguments}
    \apiargument{None.}{}{}
\end{apiarguments}

\apidescription{
    This routine returns the \ac{PE} number of the calling \ac{PE}.  It accepts no
    arguments.  The result is an integer between \CONST{0} and \VAR{npes} -
    \CONST{1}, where \VAR{npes} is the total number of \ac{PE}s executing the
    current program.
}

\apireturnvalues{
    Integer - Between \CONST{0} and \VAR{npes} - \CONST{1}
}

\apinotes{
    Each \ac{PE} has a unique number or identifier. As of \openshmem Specification
    1.2 the use of \FUNC{\_my\_pe} has been deprecated. Although \openshmem
    libraries are required to support the call, users are encouraged to use
    \FUNC{shmem\_my\_pe} instead.
}

\begin{apiexamples}

\apicexample
    {The following \FUNC{shmem\_my\_pe} example is for \CorCpp{} programs:}
    {./example_code/shmem_mype_example.c}
    {}

\end{apiexamples}

\end{apidefinition}
