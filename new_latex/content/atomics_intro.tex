\ac{AMO} is a one-sided communication mechanism that combines memory update
operations with atomicity guarantees described in Section
\ref{sec:amo_guarantees}.  Similar to the \ac{RMA} routines, described in
Section \ref{sec:rma}, the \acp{AMO} are performed only on symmetric objects.
\openshmem{} defines the two types of \ac{AMO} routines:
\begin{itemize}
\item % Blocking\\
The \textit{fetch-and-operate} routines combine memory update and fetch
operations in a single atomic operation.  The routines return after the data has
been fetched and delivered to the local \ac{PE}.

The \textit{fetch-and-operate} operations include: \FUNC{SHMEM\_CSWAP},
\FUNC{SHMEM\_SWAP}, \FUNC{SHMEM\_FINC}, and\\ \FUNC{SHMEM\_FADD}.

\item % Non-Blocking\\
The \textit{non-fetch} atomic routines update the remote memory in a single
atomic operation.  A \textit{non-fetch} atomic routine starts the atomic
operation and may return before the operation execution on the remote \ac{PE}.
To force completion for these \textit{non-fetch} atomic routines,
\FUNC{shmem\_quiet}, \FUNC{shmem\_barrier}, or \FUNC{shmem\_barrierall} can be
used by an \openshmem{} program. 

The \textit{non-fetch} operations include: \FUNC{SHMEM\_INC} and
\FUNC{SHMEM\_ADD}.
\end{itemize}
