\apisummary{
    Determines whether a \ac{PE} is accessible via \openshmem's data transfer
    routines.
}

\begin{apidefinition}

\begin{Csynopsis}
int shmem_pe_accessible(int pe);
\end{Csynopsis}

\begin{Fsynopsis}
LOGICAL LOG, SHMEM_PE_ACCESSIBLE
INTEGER pe
LOG = SHMEM_PE_ACCESSIBLE(pe)
\end{Fsynopsis}

\begin{apiarguments}
    \apiargument{IN}{pe}{Specific \ac{PE} to be checked for accessibility from
    the local \ac{PE}.}
\end{apiarguments}

\apidescription{
    \FUNC{shmem\_pe\_accessible} is  a  query routine  that indicates  whether  a
    specified \ac{PE} is accessible via \openshmem from the local \ac{PE}. The
    \FUNC{shmem\_pe\_accessible} routine returns \CONST{TRUE} only if  the  remote
    \ac{PE} is a process  running from the same executable  file as the local
    \ac{PE}, indicating that full \openshmem support for symmetric data objects
    (that reside in the static memory and symmetric heap) is available, otherwise it
    returns \CONST{FALSE}.  This routine may be particularly useful for hybrid
    programming with other communication libraries (such as a \ac{MPI}) or parallel
    languages.  For example, on  SGI Altix  series  systems, \openshmem is
    supported  across multiple partitioned hosts and InfiniBand connected hosts.
    When running multiple executable MPI programs using \openshmem on an Altix, full
    \openshmem support is available between processes running from the same
    executable file. However, \openshmem support between processes of different
    executable  files  is  supported only for data objects on the symmetric heap,
    since static data objects are  not symmetric  between  different executable
    files.        
}

\apireturnvalues{
    \Clang: The return value is 1 if the specified \ac{PE} is a valid remote \ac{PE}
    for \openshmem routines; otherwise, it is 0. \\ \\

    \Fortran: The return value is \CONST{.TRUE.} if the specified \ac{PE} is a valid
    remote \ac{PE} for \openshmem routines; otherwise, it is \CONST{.FALSE.}.	 
}

\apinotes{ None. }

\end{apidefinition}
