\bAPI{SHMEM\_BARRIER\_ALL}{Registers the arrival of a processing element \ac{PE} at a barrier and suspends \ac{PE} execution until all other \ac{PE}s arrive at the barrier and all local and remote memory updates are completed.}
\synC   %Synopisis for C API
void barrier(void);
void shmem_barrier_all(void);
%*\synCE    %DO NOT DELETE. THIS LINE IS NOT A COMMENT
\synF   %Synopsis for FORTRAN API
CALL BARRIER
CALL SHMEM_BARRIER_ALL
%*\synFE   %DO NOT DELETE. THIS LINE IS NOT A COMMENT  

% Arguments table. If no arguments you can use \argRow{NONE}{}{} 
\desB{  
    \argRow{None.}{}{} 
}
%API description
{   
    The \FUNC{shmem\_barrier\_all} function registers the arrival of a \ac{PE} at a
    barrier. Barriers are a fast mechanism for synchronizing all \ac{PE}s at
    once. This routine causes a \ac{PE} to suspend execution until all \ac{PE}s have
    called \FUNC{shmem\_barrier\_all}. This function must be used with \ac{PE}s started
    by \FUNC{start\_pes}.

    Prior to synchronizing with other \ac{PE}s, \FUNC{shmem\_barrier\_all} ensures
    completion of all previously issued local memory stores and remote
    memory updates issued via shared memory routine calls such as
    \FUNC{shmem\_put32}.
}
{
%API Description Table. 
\desR{
    %Return Values     
    {None.}
}
% Notes. If there are no notes, this field can be left empty.
\notesB{None.}
}
%Example
\exampleB{
    %For each example, you can enter it as an item.
    \exampleITEM
    { The following \FUNC{shmem\_barrier\_all} example is for C/C++ programs:}
    {./EXAMPLES/shmem_barrierall_example.c}
    {} 
}  	
\eAPI 
%%%%%%%%%%%%%%%%%%%%%%%%%%%%%%%%%%%%%%%%%%%%%%%%%%%%%%%%%%%%%%%%%%%%%%%%%%%%%%
%
% Registers  the  arrival of a processing element (PE) at a barrier and suspends PE execution until all other PEs arrive at the barrier and all local and remote memory updates are completed.
% 
% SYNOPSIS
%        C or C++:
% 
% 	  void barrier(void);
% 
% 	  void shmem_barrier_all(void);
% 
%        Fortran:
% 
% 	  CALL BARRIER
% 
% 	  CALL SHMEM_BARRIER_ALL
% 
% DESCRIPTION
% 
% Arguments
%        None.
% 
% API Description
% 
%        The  shmem_barrier_all  function	 registers  the	 arrival  of a PE at a
%        barrier.	 Barriers are a fast mechanism for synchronizing  all  PEs  at
%        once.  This routine causes a PE to suspend execution until all PEs have
%        called shmem_barrier_all.  This function must be used with PEs  started
%        by start_pes().
% 
%        Prior  to  synchronizing	 with  other  PEs,  shmem_barrier_all  ensures
%        completion of all previously issued  local  memory  stores  and	remote
%        memory	updates	 issued	 via  shared  memory  routine  calls  such  as
%        shmem_put32().
% 
% Return Value
%        None.
% 
% EXAMPLES
% 	The following shmem_barrier_all example is for C/C++ programs:
% 
%        \lstinputlisting[language=C]{shmem_barrierall_example.c}
