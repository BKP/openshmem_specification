\bAPI{SHMEM\_PUT}{The  put routines  provide  a method for copying data from a contiguous local data object to a data object on a specified PE.}
\synC   %Synopisis for C API
 void shmem_double_put(double target, const double *source, size_t len, int pe);
 void shmem_float_put(float *target, const float *source, size_t len, int pe);
 void shmem_int_put(int *target, const int *source, size_t len, int pe);
 void shmem_long_put(long *target, const long *source, size_t len, int pe);
 void shmem_longdouble_put(long double *target, const long double *source, size_t len, int pe);
 void shmem_longlong_put(long long *target, const long long *source, size_t len, int pe);
 void shmem_put32(void *target, const void *source, size_t len, int pe);
 void shmem_put64(void *target, const void *source, size_t len, int pe);
 void shmem_put128(void *target, const void *source, size_t len, int pe);
 void shmem_putmem(void *target, const void *source, size_t len, int pe);
 void shmem_short_put(short*target, const short*source, size_t len, int pe);
%*\synCE    %DO NOT DELETE. THIS LINE IS NOT A COMMENT

\synF   %Synopsis for FORTRAN API
 CALL SHMEM_CHARACTER_PUT (target, source, len, pe)
 CALL SHMEM_COMPLEX_PUT (target, source, len, pe)
 CALL SHMEM_DOUBLE_PUT (target, source, len, pe)
 CALL SHMEM_INTEGER_PUT (target, source, len, pe)
 CALL SHMEM_LOGICAL_PUT (target, source, len, pe)
 CALL SHMEM_PUT (target, source, len, pe)
 CALL SHMEM_PUT4 (target, source, len, pe)
 CALL SHMEM_PUT8 (target, source, len, pe)
 CALL SHMEM_PUT32 (target, source, len, pe)
 CALL SHMEM_PUT64 (target, source, len, pe)
 CALL SHMEM_PUT128 (target, source, len, pe)
 CALL SHMEM_PUTMEM (target, source, len, pe)
 CALL SHMEM_REAL_PUT (target, source, len, pe)
 %*\synFE   %DO NOT DELETE. THIS LINE IS NOT A COMMENT  

% Arguments table. If no arguments you can use \argRow{NONE}{}{} 
\desB{  
       \argRow{IN}{target}{Data object to be updated on the remote PE. This data object must be remotely accessible.}
       \argRow{OUT}{source}{Data object containing the data to be copied.}
       \argRow{IN}{len}{Number of elements in the target and source arrays. len must be of type integer. If you are using Fortran, it must be a constant, variable, or array element of default integer type.}
        \argRow{IN}{pe}{PE number of the remote PE. pe must be of type integer. If you are using Fortran, it must be a constant, variable, or array element of default integer type.}
 }
 %API description
 {   The  routines  return  after the data has been copied out of the source
       array on the local PE.
       The delivery of data words into the data object on the  destination  PE
       may  occur  in  any order.  Furthermore,	 two successive put operations
       may deliver  data  out  of  order  unless  a  call  to  shmem\_fence  is
       introduced between the two calls.   
 }
 %API Description Table. 
{
 % If there is no Description Table and return, this field can be 
  \desTB { 
    The  target  and source data  objects must conform to certain typing
    constraints, which are as follows: } 
    {
       \cRow{shmem\_putmem}{ Fortran:  Any noncharacter type. C:	 Any data  type.   len	is  scaled  in bytes.} 
       \cRow{shmem\_put4, shmem\_put32}{Any  noncharacter type that has a storage size equal to 32 bits. }
       \cRow{shmem\_put4, shmem\_put32}{Any  noncharacter type that has a storage size equal to 32 bits.}
       \cRow{shmem\_put8,  shmem\_put64}{Any noncharacter type that has a  storage size equal to 64 bits.}
       \cRow{shmem\_put8,  shmem\_put64}{Any noncharacter type that has a  storage size equal to 64 bits. }
       \cRow{shmem\_put128}{Any  noncharacter type that has a storage size equal to 128 bits. }
       \cRow{shmem\_double\_put}{Elements of type double.}
       \cRow{shmem\_longdouble\_put}{Elements of type long double.}
       \cRow{SHMEM\_CHARACTER\_PUT}{Elements of type character.  len  is  the number  of	 characters to transfer. The actual character lengths of the source and target variables are ignored. }
       \cRow{SHMEM\_COMPLEX\_PUT}{Elements of type complex of default size.}
       \cRow{SHMEM\_DOUBLE\_PUT}{Elements of type double precision. }
       \cRow{SHMEM\_INTEGER\_PUT}{Elements of type integer.}
       \cRow{SHMEM\_LOGICAL\_PUT}{Elements of type logical.}
       \cRow{SHMEM\_REAL\_PUT}{Elements of type real.}
      } 
 %Return Values     
\desR{None.}

% Notes. If there are no notes, this field can be left empty.
\notesB{    If you are using Fortran, data types must  be  of  default  size.   For
       example, a  real  variable  must  be  declared as  REAL,  REAL*4,  or
       REAL(KIND=4).       
}
} %end of DesB
%Example
\exampleB{
%For each example, you can enter it as an item.
                  \exampleITEM
                  { The following shmem\_put example is for  programs:}
                 {./EXAMPLES/shmem_put_example.c}
                 {} 
                 }  	
\eAPI 
