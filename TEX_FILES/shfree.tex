\bAPI{SHMALLOC, SHFREE, SHREALLOC, SHMEMALIGN}{Symmetric heap memory management functions.}
%SYNOPSIS
\synC
void *shmalloc(size_t size);
void shfree(void *ptr);
void *shrealloc(void *ptr, size_t size);
void *shmemalign(size_t alignment, size_t size);
extern long malloc_error; %*\synCE

%DESCRIPTION
%Arguments
\desB{
       \argRow{IN}{size}{In bytes, to request a block to be allocated from the symmetric heap. This argument is of type \VAR{size\_t}}
       \argRow{IN}{ptr}{Points to a block within the symmetric heap.}
       \argRow{IN}{alignment}{Byte alignment of the block allocated from the symmetric heap.}
}
%API Description
{
       The \FUNC{shmalloc} function returns a pointer to a block  of  at  least  size
       bytes  suitably	aligned for any use.  This space is allocated from the
       symmetric heap (in contrast to \FUNC{malloc},  which  allocates  from  the
       private heap).

       The  \FUNC{shmemalign} function  allocates a block in the symmetric heap that
       has a byte alignment specified by the alignment argument.

       The \FUNC{shfree} function causes the block to which  \VAR{ptr} points to be
       deallocated, that is, made available for further allocation.  If ptr is
       a null pointer, no action occurs; otherwise, if the argument  does  not
       match a pointer  earlier returned by a symmetric heap function, or if
       the space has already been deallocated, \VAR{malloc\_error} is set to indicate
       the error, and \FUNC{shfree} returns.

       The  \FUNC{shrealloc}  function changes the size of the block to which ptr
       points to the size (in bytes) specified by size. The contents  of  the
       block  are unchanged up to the lesser of the new and old sizes.	If the
       new size is larger, the value of the newly  allocated  portion  of  the
       block  is  indeterminate.   If  \VAR{ptr} is a \CONST{NULL} pointer, the \FUNC{shrealloc}
       function behaves like the \FUNC{shmalloc} function for the specified size.  If
       size  is \CONST{0} and ptr is not a \CONST{NULL} pointer, the block to which it points
       is freed.  Otherwise, if ptr does not match a pointer earlier  returned
       by  a  symmetric	 heap  function,  or  if  the  space  has already been
       deallocated, the \VAR{malloc\_error} variable is set to	 indicate  the error,
       and  \FUNC{shrealloc}  returns	a  \CONST{NULL} pointer.   If the  space  cannot  be
       allocated, the block to which ptr points is unchanged.

       The \FUNC{shmalloc}, \FUNC{shfree}, and \FUNC{shrealloc}  functions  are  provided  so  that
       multiple \ac{PE}s  in  an  application  can allocate  symmetric,  remotely
       accessible memory blocks.  These memory blocks can then be  used  with
       \openshmem communication  routines.   Each of  these  functions call the
       \FUNC{shmem\_barrier\_all} function before returning; this ensures  that  all
       \ac{PE}s  participate in the memory allocation, and that the memory on other
       \ac{PE}s can be used	as  soon  as  the  local  \ac{PE}  returns.	 The  user  is
       responsible  for	 calling these functions with identical argument(s) on
       all \ac{PE}s; if differing size arguments are used, subsequent calls may not
       return the same symmetric heap address on all \ac{PE}s.
}
%API Description Table
{
		%Return Value
		\desR{
					 The \FUNC{shmalloc} function returns a pointer to the allocated space (which
					 should  be  identical on all \ac{PE}s); otherwise, it returns a \CONST{NULL} pointer
					 (with \VAR{malloc\_error} set).

					 The \FUNC{shfree} function returns no value.

					 The \FUNC{shrealloc} function returns a pointer to the allocated space (which
					 may   have   moved);   otherwise,  it  returns  a  null	pointer (with
					 \FUNC{malloc\_error} set).
		}
		%NOTES
		\notesB{      
					 The total size of the symmetric heap is determined at job startup.  One
					 can  adjust  the size  of  the heap  using   the \CONST{SMA\_SYMMETRIC\_SIZE}
					 environment  variable (where available).	

					 The \FUNC{shmalloc}, \FUNC{shfree}, and \FUNC{shrealloc} functions differ from  the  private
					 heap  allocation functions in that all \ac{PE}s in an application must call
					 them (a barrier is used to ensure this).
		}
}

%EXAMPLES
%\rcomment{Tommy: Code will not compile without example, need example for shfree. \\}
%
%\exampleB{
%		\exampleITEM
%		{The following shmalloc example is for C/C++ programs:}
%		{./EXAMPLES/shmem_npes_example.c}
%		{}

\eAPI
