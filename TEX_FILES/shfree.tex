\bAPI{SHMEM\_MALLOC, SHMEM\_FREE, SHMEM\_REALLOC, SHMEM\_ALIGN}{Symmetric heap memory management functions.}
%SYNOPSIS
\synC
void *shmalloc(size_t size);
void *shmem_malloc(size_t size);
void shfree(void *ptr);
void shmem_free(void *ptr);
void *shrealloc(void *ptr, size_t size);
void *shmem_realloc(void *ptr, size_t size);
void *shmemalign(size_t alignment, size_t size);
void *shmem_align(size_t alignment, size_t size);%*\synCE

%DESCRIPTION
%Arguments
\desB{
       \argRow{IN}{size}{In bytes, to request a block to be allocated from the symmetric heap. This argument is of type \VAR{size\_t}}
       \argRow{IN}{ptr}{Points to a block within the symmetric heap.}
       \argRow{IN}{alignment}{Byte alignment of the block allocated from the symmetric heap.}
}
%API Description
{
       The \FUNC{shmalloc} and \FUNC{shmem\_malloc}  functions return a pointer to a block of at least size
       bytes suitably aligned for any use.  This space is allocated from the
       symmetric heap (in contrast to \FUNC{malloc}, which allocates from the
       private heap).

       The  \FUNC{shmemalign} and \FUNC{shmem\_align} functions allocate a block in the symmetric heap that
       has a byte alignment specified by the alignment argument.

       The \FUNC{shfree} and \FUNC{shmem\_free} functions cause the block to which \VAR{ptr} points to be
       deallocated, that is, made available for further allocation.  If ptr is
       a null pointer, no action occurs. 
              
       The  \FUNC{shrealloc} and  \FUNC{shmem\_realloc} functions change the size of the block to which ptr
       points to the size (in bytes) specified by size.  The contents of the
       block are unchanged up to the lesser of the new and old sizes. If the
       new size is larger, the value of the newly allocated portion of the
       block is indeterminate.  
       If \VAR{ptr} is a \CONST{NULL} pointer, the \FUNC{shrealloc}/\FUNC{shmem\_realloc} function behaves like the \FUNC{shmalloc}/\FUNC{shmem\_malloc} function for the specified size.  If size  is \CONST{0} and ptr is not a \CONST{NULL} pointer, the block to which it points is freed. If the space cannot be allocated, the block to which ptr points is unchanged.

       The \FUNC{shmalloc}, \FUNC{shfree}, and \FUNC{shrealloc} functions are provided  so that multiple \ac{PE}s in a program can allocate symmetric, remotely
       accessible memory blocks.  These memory blocks can then be used with
       \openshmem communication routines.  Each of these functions call the
       \FUNC{shmem\_barrier\_all} function before returning; this ensures that all
       \ac{PE}s participate in the memory allocation, and that the memory on other
       \ac{PE}s can be used	as  soon as the local \ac{PE} returns.  The user is
       responsible for calling these functions with identical argument(s) on
       all \ac{PE}s; if differing size arguments are used, the behavior of the call and any subsequent \openshmem calls becomes undefined.
       %subsequent calls may not
       %return the same symmetric heap address on all \ac{PE}s.
}
%API Description Table
{
		%Return Value
		\desR{
					 The \FUNC{shmalloc} function returns a pointer to the allocated space (which should be identical on all \ac{PE}s); otherwise, it returns a \CONST{NULL} pointer.

					 The \FUNC{shfree} function returns no value.

					 The \FUNC{shrealloc} function returns a pointer to the allocated space (which may have moved); otherwise, it returns a null pointer.
		}
		%NOTES
		\notesB{ As of Specification 1.2 the use of \FUNC{shmalloc}, \FUNC{shmemalign}, \FUNC{shfree},  and \FUNC{shrealloc} has been deprecated. Although OpenSHMEM libraries are required to support the calls, program users are encouraged to use \FUNC{shmem\_malloc}, \FUNC{shmem\_align}, \FUNC{shmem\_free}, and  \FUNC{shmem\_realloc} instead.
					 
The total size of the symmetric heap is determined at job startup.  One can adjust the size of the heap using the \CONST{SMA\_SYMMETRIC\_SIZE} environment  variable (where available).	

The \FUNC{shmem\_malloc}, \FUNC{shmem\_free}, and \FUNC{shmem\_realloc} functions differ from the private heap allocation functions in that all \ac{PE}s in a program must call them (a barrier is used to ensure this).

}		
}

%EXAMPLES
%\rcomment{Tommy: Code will not compile without example, need example for shfree. \\}
%
%\exampleB{
%		\exampleITEM
%		{The following shmalloc example is for C/C++ programs:}
%		{./EXAMPLES/shmem_npes_example.c}
%		{}

\eAPI
