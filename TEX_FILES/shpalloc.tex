       Allocates a block of memory from the symmetric heap

SYNOPSIS
       POINTER (addr, A(1))
       INTEGER (length, errcode, abort)
       CALL SHPALLOC(addr, length, errcode, abort)

DESCRIPTION

Arguments

	IN       addr	      First word address of the allocated block (output).
       IN	length	      Number  of  words of memory requested (input).  One word
		      is 32 bits.
       IN	errcode	      Error code is 0 if no error was detected; otherwise,  it
		      is  a  negative  integer	code  for  the	type  of error
		      (output).
       IN	abort	      Abort code; nonzero requests abort on error; 0  requests
		      an error code (input).
API Description

       SHPALLOC	 allocates a block of memory from the program's symmetric heap
       that is greater than or equal  to  the  size  requested.	  To  maintain
       symmetric  heap	consistency,  all PEs in an program must call SHPALLOC
       with the same value of length; if any  processing  elements  (PEs)  are
       missing, the program will hang.


       By using the Fortran POINTER mechanism in the following manner, you can
       use array A to refer to the block allocated by SHPALLOC: POINTER (addr,
       A())

Return Value

       Error conditions are as follows:

       Error Code     Condition
	     -1	      Length is not an integer greater than 0.
	     -2	      No more memory is available from the system (checked  if
		      the  request  cannot  be	satisfied  from	 the available
		      blocks on the symmetric heap).
NOTES
       The total size of the symmetric heap is determined at job startup.  One
       may  adjust  the	 size  of  the	heap  using   the   SMA_SYMMETRIC_SIZE
       environment  variable (if available).	


