       Determines whether an address is accessible via OpenSHMEM data transfers operations from the specified  remote  processing
       element (PE).

SYNOPSIS
       C/C++:

	  int shmem_addr_accessible(void *addr, int pe);

       Fortran:

	  LOGICAL LOG, SHMEM_ADDR_ACCESSIBLE

	  INTEGER pe

	  LOG = SHMEM_ADDR_ACCESSIBLE(addr, pe)

DESCRIPTION

Arguments

       IN	addr	Data object on the local PE.
       IN	pe	Integer id of a remote PE.

API Description

       shmem_addr_accessible  is  a  query  function  that indicates whether a
       local address is accessible via SHMEM  operations  from	the  specified
       remote PE.

       This function verifies that the data object is symmetric and accessible
       with respect to a remote PE via OpenSHMEM  data  transfer  functions.   The
       specified address addr is a data object on the local PE.

       On  SGI	Altix series systems, for multiple executable MPI applications
       that use SHMEM functions, it is important to note that  static  memory,
       such  as	 a  Fortran  common  block  or C global variable, is symmetric
       between processes running from the same executable  file,  but  is  not
       symmetric  between  processes  running from different executable files.
       Data allocated from  the	 symmetric  heap  (shmalloc  or	 shpalloc)  is
       symmetric across the same or different executable files.

Return Value

       C/C++:	 The  return value is 1 if addr is a symmetric data object and
		 accessible via OpenSHMEM operations from the specified remote PE;
		 otherwise,it is 0.

       Fortran:	 The return value is .TRUE. if addr is a symmetric data object
		 and accessible via OpenSHMEM operations from the specified remote
		 PE; otherwise, it is .FALSE..

NOTES

EXAMPLES


