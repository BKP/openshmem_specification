       Returns  a	 pointer  to  a	 data  object  on  a specified
       processing element (PE).

SYNOPSIS
       C or C++:

	  void *shmem_ptr(void *target, int pe);

       Fortran:

	  POINTER (PTR, POINTEE)
	  INTEGER pe
	  PTR = SHMEM_PTR(target, pe)

DESCRIPTION

Arguments

	IN       target	 The symmetric data object to be referenced.

       IN	pe	 An integer that indicates the PE number on which target is to
		 be accessed.  If you are using Fortran, it must be a  default
		 integer value.

API Description

       shmem_ptr returns an address that may be	 used  to  directly  reference
       target on the specified PE.  This address can be assigned to a pointer.
       After that, ordinary loads and stores to this  remote  address  may  be
       performed.

       When a sequence of loads (gets) and stores (puts) to a data object on a
       remote PE does not match the access pattern provided in	a OpenSHMEM data
       transfer	  routine   like  shmem_put32()  or  shmem_real_iget(),  the
       shmem_ptr function can provide an efficient  means  to  accomplish  the
       communication.

Return Value

       shmem_ptr returns a pointer to the data object on the specified	remote
       PE.   If target is not remotely accessible, a NULL pointer is returned.

NOTES
       The shmem_ptr function is available  only  on  systems  where  ordinary
       memory  loads  and  stores  are	used  to  implement OpenSHMEM put and get
       operations. When calling shmem_ptr, you pass the address on  the	calling	 PE  for  a  symmetric
       array on the remote PE.

EXAMPLES

       This  Fortran  program calls shmem_ptr and then PE 0 writes to the BIGD
       array on PE 1:

	\lstinputlisting[language=C]{shmem_ptr_example.f90}

       This is the equivalent program written in C:

	\lstinputlisting[language=C]{shmem_ptr_example.c}
