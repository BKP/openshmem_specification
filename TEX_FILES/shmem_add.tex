\bAPI{SHMEM\_ADD}{Performs an atomic add operation on a remote symmetric data object.}
\synC
void shmem_int_add(int *target, int value, int pe);
void shmem_long_add(long *target, long value, int pe);
void shmem_longlong_add(long long *target, long long value, int pe); %*\synCE
\synF
INTEGER pe
INTEGER*4  value_i4
CALL SHMEM_INT4_ADD(target, value_i4, pe)
INTEGER*8 value_i8
CALL SHMEM_INT8_ADD(target, value_i8, pe) %*\synFE

%DESCRIPTION
 
%Arguments
\desB{
 	\argRow{OUT}{target}{The remotely accessible integer data object to be updated  on the remote \ac{PE}.  If you are using \CorCpp, the type of \target{} should match that implied in the SYNOPSIS section.}
	\argRow{IN}{value}{The value to be atomically added to \target. If you are using \CorCpp, the type of value should match that  implied  in  the SYNOPSIS  section.  If you are using \Fortran, it must be of type integer with an element size of \target.}
	\argRow{IN}{pe}{An integer that indicates the \ac{PE} number upon which \target{} is to be updated.  If you are using \Fortran, it must be a default integer value.}
}
%API Description
{The \FUNC{shmem\_add} routine performs an atomic add operation. It adds value
 to \target{} on \ac{PE} \VAR{pe} and atomically increments the \target{} without returning the value.
 } 
{
 \hfill \\
\desTB {If you are using \Fortran, \VAR{target} must be of the following type:}
{
\cRow{SHMEM\_INT4\_ADD}{\CONST{4}-byte integer}
\cRow{SHMEM\_INT8\_ADD}{\CONST{8}-byte integer}
}
  \desR{None.}
   \notesB{The term remotely accessible is defined in the Introduction.}
}

\exampleB{
		\exampleITEM
		{}
		{./EXAMPLES/shmem_fadd_example.c}
		{}
}

\eAPI
