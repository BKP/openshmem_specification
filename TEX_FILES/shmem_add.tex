\bAPI{SHMEM\_ADD}{Performs an atomic add operation on a remote symmetric data object.}

%SYNOPSIS
\synC
	  void shmem_int_add(int *target, int value, int pe);
	  void shmem_long_add(long *target, long value, int pe);
	  void shmem_longlong_add(long long *target, long long value, int pe);
%*\synCE

\synF
	  INTEGER pe
	  CALL SHMEM_INT4_ADD(target, value, pe)
	  CALL SHMEM_INT8_ADD(target, value, pe)
%*\synFE

%DESCRIPTION
 
%Arguments
\desB{
       \argRow{OUT}{target}{The remotely accessible integer data object to be updated  on
	        	 the  remote  \ac{PE}. If you are using C/C++, the type of target
		         should match that implied in the SYNOPSIS  section.   If  you
			 			 are  using  the  Fortran compiler, it must be of type integer
			 			 with an element size of 4  bytes  for	SHMEM\_INT4\_ADD	and  8
			 			 bytes for SHMEM\_INT8\_ADD.}
			 \argRow{IN}{value}{The value to be atomically added to target. If you are using
			 				 C/C++, the type of value should match	that  implied  in  the
			 				 SYNOPSIS  section.   If  you are using Fortran, it must be of
			 				 type integer with an element size of target.}
       \argRow{IN}{pe}{An integer that indicates the \ac{PE} number upon which target  is
			 					to  be	 updated. If	you  are  using	 Fortran, it must be a
			 					default integer value.}
}
%API Description
{
	\desTB {The shmem\_add routine performs an atomic add operation.	It adds value
       to target on Processing Element (\ac{PE}) pe and atomically increments the
       target without returning the value.}
	{
		\cRow{}{}
	}
	
	%Return Value
  {None.}

	%NOTES
	{The term remotely accessible is defined in the Introduction.}
}

\exampleB{
		\exampleITEM
		{}
		{./EXAMPLES/shmem_fadd_example.c}
		{}
}

\eAPI
