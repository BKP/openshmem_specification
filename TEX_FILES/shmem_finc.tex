\bAPI{SHMEM\_FINC}{Performs an atomic fetch-and-increment  operation on a remote data object.}
\synC   %Synopisis for C API

int shmem_int_finc(int *target, int pe);
long shmem_long_finc(long *target, int pe);
long long shmem_longlong_finc(long long *target, int pe);
%*\synCE    %DO NOT DELETE. THIS LINE IS NOT A COMMENT

\rcomment{Manju: Do we need the pe, ires4, ires8 ? }

\synF   %Synopsis for FORTRAN API
INTEGER pe 
INTEGER(KIND=4) SHMEM_INT4_FINC, target4
INTEGER(KIND=8) SHMEM_INT8_FINC, target8
ires4 = SHMEM_INT4_FINC(target4, pe)
ires8 = SHMEM_INT8_FINC(target8, pe)
%*\synFE   %DO NOT DELETE. THIS LINE IS NOT A COMMENT  

% Arguments table. If no arguments you can use \argRow{NONE}{}{} 
\desB{  
	\argRow{IN}{target}{The remotely accessible integer data object to be updated on
		 the remote \ac{PE}.	 The type of target should match that  implied
		 in the SYNOPSIS section.}
    \argRow{IN}{pe}{An integer that indicates the \ac{PE} number on which target is to
		 be updated. If you are using Fortran, it must be  a  default
		 integer value.}
 }
%API description
{
   These functions perform a fetch-and-increment operation. The target on
   \ac{PE} \VAR{pe} is increased by one and the function returns
   the previous contents of target as an atomic operation.
}
%API Description Table.
{
    %Return Values     
\desR{The contents that had been at the target address on the remote \ac{PE} prior to the increment. The data type of the return value is the same as the target.} 
% Notes. If there are no notes, this field can be left empty.
\notesB{None.}
}

\rcomment{Manju: Seems like the indent for notes and return depends on whether
there is a newline before (API Description Table) or not. We might have to fix
it}

\exampleB{
       \exampleITEM
	    {The following \FUNC{shmem\_finc} example is for C/C++ programs:}
        {./EXAMPLES/shmem_finc_example.c}
        {}
}
\eAPI



	
