\bAPI{SHMEM\_IPUT}{Copies strided data to a specified \ac{PE}.}
\label{subsec:shmem_iput}
\synC
void shmem_double_iput(double *dest, const double *source, ptrdiff_t tst, ptrdiff_t sst, size_t nelems, int pe);
void shmem_float_iput(float *dest, const float *source, ptrdiff_t tst, ptrdiff_t sst, size_t nelems, int pe);
void shmem_int_iput(int *dest, const int *source, ptrdiff_t tst, ptrdiff_t sst, size_t nelems, int pe);
void shmem_iput32(void *dest, const void *source, ptrdiff_t tst, ptrdiff_t sst, size_t nelems, int pe);
void shmem_iput64(void *dest, const void *source, ptrdiff_t tst, ptrdiff_t sst, size_t nelems, int pe);
void shmem_iput128(void *dest, const void *source, ptrdiff_t tst, ptrdiff_t sst, size_t nelems, int pe);
void shmem_long_iput(long *dest, const long *source, ptrdiff_t tst, ptrdiff_t sst, size_t nelems, int pe);
void shmem_longdouble_iput(long double *dest, const long double *source, ptrdiff_t tst, ptrdiff_t sst, size_t nelems, int pe);
void shmem_longlong_iput(long long *dest, const long long *source, ptrdiff_t tst, ptrdiff_t sst, size_t nelems, int pe);
void shmem_short_iput(short *dest, const short *source, ptrdiff_t tst, ptrdiff_t sst, size_t nelems, int pe); %*\synCE    %DO NOT DELETE. THIS LINE IS NOT A COMMENT
\synF   %Synopsis for FORTRAN API

INTEGER tst, sst, nelems, pe
CALL SHMEM_COMPLEX_IPUT(dest, source, tst, sst, nelems, pe)
CALL SHMEM_DOUBLE_IPUT(dest, source, tst, sst, nelems, pe)
CALL SHMEM_INTEGER_IPUT(dest, source, tst, sst, nelems, pe)
CALL SHMEM_IPUT4(dest, source, tst, sst, nelems, pe)
CALL SHMEM_IPUT8(dest, source, tst, sst, nelems, pe)
CALL SHMEM_IPUT32(dest, source, tst, sst, nelems, pe)
CALL SHMEM_IPUT64(dest, source, tst, sst, nelems, pe)
CALL SHMEM_IPUT128(dest, source, tst, sst, nelems, pe)
CALL SHMEM_LOGICAL_IPUT(dest, source, tst, sst, nelems, pe)
CALL SHMEM_REAL_IPUT(dest, source, tst, sst, nelems, pe) %*\synFE   %DO NOT DELETE. THIS LINE IS NOT A COMMENT  

\desB{
\argRow{OUT}{dest}{Array to be updated on the remote \ac{PE}. This data object  must
		 be remotely accessible.}
\argRow{IN}{source}{Array containing the data to be copied.}
\argRow{IN}{tst}{The stride between consecutive elements of the \dest{} array.
		 The stride is scaled by the element size of the \dest{} array.
		 A  value of \CONST{1} indicates contiguous data.  \VAR{tst} must be of type
		 \textit{ptrdiff\_t}.  If you are using \Fortran, it must be a default integer value.}
\argRow{IN}{sst}{The  stride between consecutive elements of the \source{} array.
		 The stride is scaled by the element size of the \source{} array.
		 A  value of \CONST{1} indicates contiguous data.  \VAR{sst} must be of type
		 \textit{ptrdiff\_t}.  If you are using \Fortran, it must be a default
		 integer value.}
\argRow{IN}{nelems}{Number of elements in the \dest{} and \source{} arrays.  \VAR{nelems} must
		 be of type \VAR{size\_t} for \Clang.  If you are using \Fortran, it must be  a
		 constant, variable, or array element of default integer type.}
\argRow{IN}{pe}{\ac{PE} number of the remote \ac{PE}.  \VAR{pe} must be of type integer.   If you  are  using  \Fortran, it must be a constant, variable, or
		 array element of default integer type.}
}
{

       The \FUNC{iput} routines provide a method  for  copying strided data elements (specified by \VAR{sst}) of an array from a \source{} array on the local \ac{PE} to locations specified by stride \VAR{tst} on a \dest{} array on specified remote \ac{PE}. Both strides, \VAR{tst} and \VAR{sst} must be greater than or equal to \CONST{1}. The routines return when the data has been copied out of the \VAR{source} array on the local \ac{PE} but not necessarily before the data has been delivered to the remote data object.
}{
    \desTB{
       The \dest{} and \source{} data objects must conform to typing  constraints,
       which are as follows:}
      {
     \cRow{shmem\_iput32, shmem\_iput4}{Any  noncharacter type that has a storage
				     size equal to \CONST{32} bits.}
     \cRow{shmem\_iput64, shmem\_iput8}{Any noncharacter type that has a  storage
				     size equal to \CONST{64} bits.}
     \cRow{shmem\_iput128}{Any  noncharacter type that has a storage
				     size equal to \CONST{128} bits.}
     \cRow{shmem\_short\_iput}{Elements of type short.}
     \cRow{shmem\_int\_iput}{Elements of type short.}
     \cRow{shmem\_long\_iput}{Elements of type long.}
     \cRow{shmem\_longlong\_iput}{Elements of type long long.}
     \cRow{shmem\_float\_iput}{Elements of type float.}
     \cRow{shmem\_double\_iput}{Elements of type float.}
     \cRow{shmem\_longdouble\_iput}{Elements of type long double.}
     \cRow{SHMEM\_COMPLEX\_IPUT}{Elements of type complex of default size.}
     \cRow{SHMEM\_DOUBLE\_IPUT}{Elements of type double precision.}
     \cRow{SHMEM\_INTEGER\_IPUT}{Elements of type integer.}
     \cRow{SHMEM\_LOGICAL\_IPUT}{Elements of type logical.}
     \cRow{SHMEM\_REAL\_IPUT}{Elements of type real.}
        }
\desR{None.}
\notesB{
       If you are using \Fortran, data types must be of default size.  For
       example, a real variable must be declared as  \CONST{REAL}, \CONST{REAL*4} or
       \CONST{REAL(KIND=KIND(1.0))}.  See Introduction for a definition of the term remotely accessible.
}
}
\exampleB{
       \exampleITEM
       {Consider the following \FUNC{shmem\_long\_iput}  example  for \CorCpp{}
       programs.} {./EXAMPLES/shmem_iput_example.c}{}
}
\eAPI
