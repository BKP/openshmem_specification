       Copies  strided  data	 to  a specified processing element (PE).

SYNOPSIS

       C or C++:

	  void	shmem_double_iput(double  *target,   const   double   *source,
	  ptrdiff_t tst, ptrdiff_t sst, size_t nelems, int pe);

	  void	shmem_float_iput(float *target, const float *source, ptrdiff_t
	  tst, ptrdiff_t sst, size_t nelems, int pe);

	  void shmem_int_iput(int *target, const int *source,  ptrdiff_t  tst,
	  ptrdiff_t sst, size_t nelems, int pe);

	  void	shmem_iput32(void  *target, const void *source, ptrdiff_t tst,
	  ptrdiff_t sst, size_t nelems, int pe);

	  void shmem_iput64(void *target, const void *source,  ptrdiff_t  tst,
	  ptrdiff_t sst, size_t nelems, int pe);

	  void	shmem_iput128(void *target, const void *source, ptrdiff_t tst,
	  ptrdiff_t sst, size_t nelems, int pe);

	  void shmem_long_iput(long *target,  const  long  *source,  ptrdiff_t
	  tst, ptrdiff_t sst, size_t nelems, int pe);

	  void	shmem_longdouble_iput(long  double  *target, const long double
	  *source, ptrdiff_t tst, ptrdiff_t sst, size_t nelems, int pe);

	  void shmem_longlong_iput(long long *target, const long long *source,
	  ptrdiff_t tst, ptrdiff_t sst, size_t nelems, int pe);

	  void	shmem_short_iput(short *target, const short *source, ptrdiff_t
	  tst, ptrdiff_t sst, size_t nelems, int pe);

       Fortran:

	  INTEGER tst, sst, nelems, pe

	  CALL SHMEM_COMPLEX_IPUT(target, source, tst, sst, nelems, pe)

	  CALL SHMEM_DOUBLE_IPUT(target, source, tst, sst, nelems, pe)

	  CALL SHMEM_INTEGER_IPUT(target, source, tst, sst, nelems, pe)

	  CALL SHMEM_IPUT4(target, source, tst, sst, nelems, pe)

	  CALL SHMEM_IPUT8(target, source, tst, sst, nelems, pe)

	  CALL SHMEM_IPUT32(target, source, tst, sst, nelems, pe)

	  CALL SHMEM_IPUT64(target, source, tst, sst, nelems, pe)

	  CALL SHMEM_IPUT128(target, source, tst, sst, nelems, pe)

	  CALL SHMEM_LOGICAL_IPUT(target, source, tst, sst, nelems, pe)

	  CALL SHMEM_REAL_IPUT(target, source, tst, sst, nelems, pe)

DESCRIPTION

Arguments

 	OUT      target	 Array to be updated on the remote PE.	This data object  must
		 be remotely accessible.

       IN	source	 Array containing the data to be copied.

       IN	tst	 The  stride between consecutive elements of the target array.
		 The stride is scaled by the element size of the target array.
		 A  value of 1 indicates contiguous data.  tst must be of type
		 integer.  If you are using Fortran,  it  must	be  a  default
		 integer value.

       IN	sst	 The  stride between consecutive elements of the source array.
		 The stride is scaled by the element size of the source array.
		 A  value of 1 indicates contiguous data.  sst must be of type
		 integer.  If you are using Fortran,  it  must	be  a  default
		 integer value.

       IN	nelems	 Number of elements in the target and source arrays.  nelems must
		 be of type integer.  If you are using Fortran, it must	 be  a
		 constant, variable, or array element of default integer type.

       IN	pe	 PE number of the remote PE.  pe must be of type integer.   If
		 you  are  using  Fortran, it must be a constant, variable, or
		 array element of default integer type.

API Description

       The iput routines provide  a method  for  copying 
       strided	data elements of an array  from  the  local PE  to strided locations 
       of a symmetric array on a
       different PE.  The routines return when the data has been copied out of
       the  source  array  on the local PE but not necessarily before the data
       has been delivered to the remote data object.

       The target and source data objects must conform to typing  constraints,
       which are as follows:

       Routine			     Data Type of target and source

       shmem_iput32, shmem_iput4     Any  noncharacter type that has a storage
				     size equal to 32 bits.

       shmem_iput64, shmem_iput8     Any noncharacter type that has a  storage
				     size equal to 64 bits.

       shmem_iput128		     Any  noncharacter type that has a storage
				     size equal to 128 bits.

       shmem_short_iput		     Elements of type short.

       shmem_int_iput		     Elements of type int.

       shmem_long_iput		     Elements of type long.

       shmem_longlong_iput	     Elements of type long long.

       shmem_float_iput		     Elements of type float.

       shmem_double_iput	     Elements of type double.

       shmem_longdouble_iput	     Elements of type long double.

       SHMEM_COMPLEX_IPUT	     Elements of type complex of default size.

       SHMEM_DOUBLE_IPUT (Fortran)   Elements of type double precision.

       SHMEM_INTEGER_IPUT	     Elements of type integer.

       SHMEM_LOGICAL_IPUT	     Elements of type logical.

       SHMEM_REAL_IPUT		     Elements of type real.

       If  you	are  using  Fortran,  data types must be of default size.  For
       example,	 a  real  variable  must  be  declared	as  REAL,  REAL*4   or
       REAL(KIND=4).

Return Value
  
	None.

NOTES
       See Introduction for a definition of the term remotely accessible.

EXAMPLES
       Consider	  the  following  simple  shmem_long_iput  example  for	 C/C++
       programs.

	\lstinputlisting[language=C]{shmem_iput_example.c}
