\bAPI{SHMEM\_IGET}{Copies strided data from a specified processing element (PE).}
\synC   %Synopisis for C API

 void	shmem_double_iget(double  *target, const double *source, ptrdiff_t tst, ptrdiff_t sst, size_t nelems, int pe);
 void	shmem_float_iget(float *target, const float *source, ptrdiff_t tst, ptrdiff_t sst, size_t nelems, int pe);
 void shmem_iget32(void *target, const void *source,  ptrdiff_t  tst, ptrdiff_t sst, size_t nelems, int pe);
 void	shmem_iget64(void  *target, const void *source, ptrdiff_t tst, ptrdiff_t sst, size_t nelems, int pe);
 void shmem_iget128(void *target, const void *source, ptrdiff_t  tst, ptrdiff_t sst, size_t nelems, int pe);
 void	shmem_int_iget(int  *target, const int *source, ptrdiff_t tst, ptrdiff_t sst, size_t nelems, int pe);
 void shmem_long_iget(long *target,  const  long  *source,  ptrdiff_t tst, ptrdiff_t sst, size_t nelems, int pe);
 void	shmem_longdouble_iget(long  double  *target, const long double *source, ptrdiff_t tst, ptrdiff_t sst, size_t nelems, int pe);
 void shmem_longlong_iget(long long *target, const long long *source, ptrdiff_t tst, ptrdiff_t sst, size_t nelems, int pe);
 void	shmem_short_iget(short *target, const short *source, ptrdiff_t tst, ptrdiff_t sst, size_t nelems, int pe);
%*\synCE    %DO NOT DELETE. THIS LINE IS NOT A COMMENT
\synF   %Synopsis for FORTRAN API

 INTEGER tst, sst, nelems, pe
 CALL SHMEM_COMPLEX_IGET(target, source, tst, sst, nelems, pe)
 CALL SHMEM_DOUBLE_IGET(target, source, tst, sst, nelems, pe)
 CALL SHMEM_IGET4(target, source, tst, sst, nelems, pe)
 CALL SHMEM_IGET8(target, source, tst, sst, nelems, pe)
 CALL SHMEM_IGET32(target, source, tst, sst, nelems, pe)
 CALL SHMEM_IGET64(target, source, tst, sst, nelems, pe)
 CALL SHMEM_IGET128(target, source, tst, sst, nelems, pe)
 CALL SHMEM_INTEGER_IGET(target, source, tst, sst, nelems, pe)
 CALL SHMEM_LOGICAL_IGET(target, source, tst, sst, nelems, pe)
 CALL SHMEM_REAL_IGET(target, source, tst, sst, nelems, pe)
%*\synFE   %DO NOT DELETE. THIS LINE IS NOT A COMMENT  

% Arguments table. If no arguments you can use \argRow{NONE}{}{} 
\desB{
    \argRow{OUT}{target}{Array to be updated on the local PE. }
    \argRow{IN}{source}{Array containing the data to be copied on the remote PE.}
    \argRow{IN}{tst}{The stride between consecutive elements of the target
    array. The stride is scaled by the element size of the target array.
		 A  value of 1 indicates contiguous data.  tst must be of type
		 integer.  If you are calling  from  Fortran,  it  must	 be  a
		 default integer value.}
     \argRow{IN}{sst}{The  stride between consecutive elements of the source array.
		 The stride is scaled by the element size of the source array.
		 A  value of 1 indicates contiguous data.  sst must be of type
		 integer.  If you are calling  from  Fortran,  it  must	 be  a
		 default integer value.}
     \argRow{IN}{nelems}{Number of elements in the target and source arrays.  nelems must
		 be of type integer.  If you are using Fortran, it must	 be  a
		 constant, variable, or array element of default integer type.}
    \argRow{IN}{pe}{PE number of the remote PE.  pe must be of type integer. If
		 you  are  using  Fortran, it must be a constant, variable, or
		 array element of default integer type.}
}
%API description
{
       The iget routines provide  a method  for  copying strided data elements from a
       symmetric array from a specified remote PE to strided locations on a local  array.
       The routines return when the data has been copied into the local target
       array.}
%This newline is required 
%API Description Table.
{
     \desTB{The target and source data objects must conform to typing  constraints,
       which are as follows:}
       {
       \cRow{shmem\_iget32, shmem\_iget4}{Any  noncharacter type that has a storage
				     size equal to 32 bits.}
       \cRow{shmem\_iget64, shmem\_iget8}{Any noncharacter type that has a  storage
				     size equal to 64 bits.}
      \cRow{shmem\_iget128}{Any noncharacter type that has a storage
				     size equal to 128 bits.}
       \cRow{shmem\_short\_iget}{Elements of type short.}
      \cRow{ shmem\_int\_iget}{Elements of type int.}
       \cRow{shmem\_long\_iget}{Elements of type long.}
       \cRow{shmem\_longlong\_iget}{Elements of type long long.}
       \cRow{shmem\_float\_iget}{Elements of type float.}
      \cRow{shmem\_double\_iget}{Elements of type double.}
       \cRow{shmem\_longdouble\_iget}{Elements of type long double.}
      \cRow{SHMEM\_COMPLEX\_IGET}{Elements of type complex of default size.}
      \cRow{SHMEM\_DOUBLE\_IGET}{Fortran: Elements of type double precision.}
      \cRow{SHMEM\_INTEGER\_IGET}{Elements of type integer.}
      \cRow{SHMEM\_LOGICAL\_IGET}{Elements of type logical.}
      \cRow{SHMEM\_REAL\_IGET}{Elements of type real.}
       }
    %Return Values     
      \desR{None.}
      \notesB{If you are using Fortran, data types must be of default size. For
       example,	a real  variable  must be  declared as  REAL,  REAL*4,  or
       REAL(KIND=4).}
}
\exampleB{
       \exampleITEM
     {The  following  simple  example	uses  shmem\_logical\_iget  in a Fortran
       program.} 
	{./EXAMPLES/shmem_iget_example.f90}
    {}
}
\eAPI
