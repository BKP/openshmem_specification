\bAPI{SHMEM\_IGET}{Copies strided data from a specified \ac{PE}.}
\label{subsec:shmem_iget}
\synC   %Synopisis for C API

void shmem_double_iget(double *dest, const double *source, ptrdiff_t dst, ptrdiff_t sst, size_t nelems, int pe);
void shmem_float_iget(float *dest, const float *source, ptrdiff_t dst, ptrdiff_t sst, size_t nelems, int pe);
void shmem_iget32(void *dest, const void *source, ptrdiff_t dst, ptrdiff_t sst, size_t nelems, int pe);
void shmem_iget64(void *dest, const void *source, ptrdiff_t dst, ptrdiff_t sst, size_t nelems, int pe);
void shmem_iget128(void *dest, const void *source, ptrdiff_t  dst, ptrdiff_t sst, size_t nelems, int pe);
void shmem_int_iget(int *dest, const int *source, ptrdiff_t dst, ptrdiff_t sst, size_t nelems, int pe);
void shmem_long_iget(long *dest, const  long  *source,  ptrdiff_t dst, ptrdiff_t sst, size_t nelems, int pe);
void shmem_longdouble_iget(long double *dest, const long double *source, ptrdiff_t dst, ptrdiff_t sst, size_t nelems, int pe);
void shmem_longlong_iget(long long *dest, const long long *source, ptrdiff_t dst, ptrdiff_t sst, size_t nelems, int pe);
void shmem_short_iget(short *dest, const short *source, ptrdiff_t dst, ptrdiff_t sst, size_t nelems, int pe); %*\synCE    %DO NOT DELETE. THIS LINE IS NOT A COMMENT
\synF   %Synopsis for FORTRAN API

INTEGER dst, sst, nelems, pe
CALL SHMEM_COMPLEX_IGET(dest, source, dst, sst, nelems, pe)
CALL SHMEM_DOUBLE_IGET(dest, source, dst, sst, nelems, pe)
CALL SHMEM_IGET4(dest, source, dst, sst, nelems, pe)
CALL SHMEM_IGET8(dest, source, dst, sst, nelems, pe)
CALL SHMEM_IGET32(dest, source, dst, sst, nelems, pe)
CALL SHMEM_IGET64(dest, source, dst, sst, nelems, pe)
CALL SHMEM_IGET128(dest, source, dst, sst, nelems, pe)
CALL SHMEM_INTEGER_IGET(dest, source, dst, sst, nelems, pe)
CALL SHMEM_LOGICAL_IGET(dest, source, dst, sst, nelems, pe)
CALL SHMEM_REAL_IGET(dest, source, dst, sst, nelems, pe) %*\synFE   %DO NOT DELETE. THIS LINE IS NOT A COMMENT  

% Arguments table. If no arguments you can use \argRow{None}{}{} 
\desB{
    \argRow{OUT}{dest}{Array to be updated on the local \ac{PE}. }
    \argRow{IN}{source}{Array containing the data to be copied on the remote \ac{PE}.}
    \argRow{IN}{dst}{The stride between consecutive elements of the \dest{}
    array. The stride is scaled by the element size of the \dest{} array.
		 A  value of \CONST{1} indicates contiguous data. \VAR{dst} must be of type
		 \textit{ptrdiff\_t}.  If you are calling  from  \Fortran,  it  must be a default integer value.}
     \argRow{IN}{sst}{The stride between consecutive elements of the \source{} array.
		 The stride is scaled by the element size of the \source{} array.
		 A  value of \CONST{1} indicates contiguous data.  \VAR{sst} must be of type
		 \textit{ptrdiff\_t}.  If you are calling  from  \Fortran,  it  must be a default integer value.}
     \argRow{IN}{nelems}{Number of elements in the \dest{} and \source{} arrays.  \VAR{nelems} must
		 be of type \VAR{size\_t} for \Clang. If you are using \Fortran, it must be  a
		 constant, variable, or array element of default integer type.}
    \argRow{IN}{pe}{\ac{PE} number of the remote \ac{PE}.  \VAR{pe} must be of type integer. If you  are  using  \Fortran, it must be a constant, variable, or
		 array element of default integer type.}
}
%API description
{
       The \FUNC{iget} routines provide a method for copying strided data elements from a
       symmetric array from a specified remote \ac{PE} to strided locations on a local array.
       The routines return when the data has been copied into the local \VAR{dest}
       array.}
%This newline is required 
%API Description Table.
{
\hfill \\
     \desTB{The \VAR{dest} and \VAR{source} data objects must conform to typing  constraints, which are as follows:}
       {
       \cRow{shmem\_iget32, shmem\_iget4}{Any  noncharacter type that has a storage
				     size equal to \CONST{32} bits.}
       \cRow{shmem\_iget64, shmem\_iget8}{Any noncharacter type that has a  storage
				     size equal to \CONST{64} bits.}
      \cRow{shmem\_iget128}{Any noncharacter type that has a storage
				     size equal to \CONST{128} bits.}
       \cRow{shmem\_short\_iget}{Elements of type short.}
      \cRow{ shmem\_int\_iget}{Elements of type int.}
       \cRow{shmem\_long\_iget}{Elements of type long.}
       \cRow{shmem\_longlong\_iget}{Elements of type long long.}
       \cRow{shmem\_float\_iget}{Elements of type float.}
      \cRow{shmem\_double\_iget}{Elements of type double.}
       \cRow{shmem\_longdouble\_iget}{Elements of type long double.}
      \cRow{SHMEM\_COMPLEX\_IGET}{Elements of type complex of default size.}
      \cRow{SHMEM\_DOUBLE\_IGET}{\Fortran: Elements of type double precision.}
      \cRow{SHMEM\_INTEGER\_IGET}{Elements of type integer.}
      \cRow{SHMEM\_LOGICAL\_IGET}{Elements of type logical.}
      \cRow{SHMEM\_REAL\_IGET}{Elements of type real.}
       }
    %Return Values     
      \desR{None.}
      \notesB{If you are using \Fortran, data types must be of default size. For
       example, a real variable must be declared as \CONST{REAL}, \CONST{REAL*4}, or \CONST{REAL(KIND=KIND(1.0))}.}
}
\exampleB{
       \exampleITEMF
     {The following example uses \FUNC{shmem\_logical\_iget}  in a \Fortran{}
       program.} 
	{./EXAMPLES/shmem_iget_example.f90}
    {}
}
\eAPI
