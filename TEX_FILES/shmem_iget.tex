        Copies strided data from a specified processing
       element (PE).

SYNOPSIS
       C or C++:

	  void	shmem_double_iget(double  *target,   const   double   *source,
	  ptrdiff_t tst, ptrdiff_t sst, size_t nelems, int pe);

	  void	shmem_float_iget(float *target, const float *source, ptrdiff_t
	  tst, ptrdiff_t sst, size_t nelems, int pe);

	  void shmem_iget32(void *target, const void *source,  ptrdiff_t  tst,
	  ptrdiff_t sst, size_t nelems, int pe);

	  void	shmem_iget64(void  *target, const void *source, ptrdiff_t tst,
	  ptrdiff_t sst, size_t nelems, int pe);

	  void shmem_iget128(void *target, const void *source, ptrdiff_t  tst,
	  ptrdiff_t sst, size_t nelems, int pe);

	  void	shmem_int_iget(int  *target, const int *source, ptrdiff_t tst,
	  ptrdiff_t sst, size_t nelems, int pe);

	  void shmem_long_iget(long *target,  const  long  *source,  ptrdiff_t
	  tst, ptrdiff_t sst, size_t nelems, int pe);

	  void	shmem_longdouble_iget(long  double  *target, const long double
	  *source, ptrdiff_t tst, ptrdiff_t sst, size_t nelems, int pe);

	  void shmem_longlong_iget(long long *target, const long long *source,
	  ptrdiff_t tst, ptrdiff_t sst, size_t nelems, int pe);

	  void	shmem_short_iget(short *target, const short *source, ptrdiff_t
	  tst, ptrdiff_t sst, size_t nelems, int pe);

       Fortran:

	  INTEGER tst, sst, nelems, pe

	  CALL SHMEM_COMPLEX_IGET(target, source, tst, sst, nelems, pe)

	  CALL SHMEM_DOUBLE_IGET(target, source, tst, sst, nelems, pe)

	  CALL SHMEM_IGET4(target, source, tst, sst, nelems, pe)

	  CALL SHMEM_IGET8(target, source, tst, sst, nelems, pe)

	  CALL SHMEM_IGET32(target, source, tst, sst, nelems, pe)

	  CALL SHMEM_IGET64(target, source, tst, sst, nelems, pe)

	  CALL SHMEM_IGET128(target, source, tst, sst, nelems, pe)

	  CALL SHMEM_INTEGER_IGET(target, source, tst, sst, nelems, pe)

	  CALL SHMEM_LOGICAL_IGET(target, source, tst, sst, nelems, pe)

	  CALL SHMEM_REAL_IGET(target, source, tst, sst, nelems, pe)

DESCRIPTION

Arguments

	OUT       target	 Array to be updated on the local PE.

       IN	source	 Array containing the data to be copied on the remote PE.

       IN	tst	 The  stride between consecutive elements of the target array.
		 The stride is scaled by the element size of the target array.
		 A  value of 1 indicates contiguous data.  tst must be of type
		 integer.  If you are calling  from  Fortran,  it  must	 be  a
		 default integer value.

       IN	sst	 The  stride between consecutive elements of the source array.
		 The stride is scaled by the element size of the source array.
		 A  value of 1 indicates contiguous data.  sst must be of type
		 integer.  If you are calling  from  Fortran,  it  must	 be  a
		 default integer value.

       IN	nelems	 Number of elements in the target and source arrays.  nelems must
		 be of type integer.  If you are using Fortran, it must	 be  a
		 constant, variable, or array element of default integer type.

       IN	pe	 PE number of the remote PE.  pe must be of type integer.   If
		 you  are  using  Fortran, it must be a constant, variable, or
		 array element of default integer type.

API Description

       The iget routines provide  a method  for  copying strided data elements from a
       symmetric array from a specified remote PE to strided locations on a local  array.
       The routines return when the data has been copied into the local target
       array.


       The target and source data objects must conform to typing  constraints,
       which are as follows:

       Routine			     Data Type of target and source

       shmem_iget32, shmem_iget4     Any  noncharacter type that has a storage
				     size equal to 32 bits.

       shmem_iget64, shmem_iget8     Any noncharacter type that has a  storage
				     size equal to 64 bits.

       shmem_iget128		     Any  noncharacter type that has a storage
				     size equal to 128 bits.

       shmem_short_iget		     Elements of type short.

       shmem_int_iget		     Elements of type int.

       shmem_long_iget		     Elements of type long.

       shmem_longlong_iget	     Elements of type long long.

       shmem_float_iget		     Elements of type float.

       shmem_double_iget	     Elements of type double.

       shmem_longdouble_iget	     Elements of type long double.

       SHMEM_COMPLEX_IGET	     Elements of type complex of default size.

       SHMEM_DOUBLE_IGET (Fortran)   Elements of type double precision.

       SHMEM_INTEGER_IGET	     Elements of type integer.

       SHMEM_LOGICAL_IGET	     Elements of type logical.

       SHMEM_REAL_IGET		     Elements of type real.

       If  you	are  using  Fortran,  data types must be of default size.  For
       example,	 a  real  variable  must  be  declared	as  REAL,  REAL*4,  or
       REAL(KIND=4).

Return Value

	None.

NOTES

EXAMPLES
       The  following  simple  example	uses  shmem_logical_iget  in a Fortran
       program.	 

	\lstinputlisting[language=C]{shmem_iget_example.f90}
