\bAPI{SHMEM\_INIT}{Called at the beginning of an \openshmem program to initialize the execution environment.}
\synC   %Synopisis for C API 

void start_pes(int npes); 
void shmem_init(void); 
%*\synCE    %DO NOT DELETE. THIS LINE IS NOT A COMMENT

\synF   %Synopsis for FORTRAN API

CALL START_PES(npes) 
CALL SHMEM_INIT() 

%*\synFE   %DO NOT DELETE. THIS LINE IS NOT A COMMENT  

% Arguments table. If no arguments you can use \argRow{None}{}{} 
\desB{  
       \argRow{npes}{Unused}{ Should be set to \CONST{0}.}
}
 %API description
 {   
     The \FUNC{start\_pes}/\FUNC{shmem\_init} routine initializes the \openshmem execution environment.  An \openshmem application must call \FUNC{start\_pes}/\FUNC{shmem\_init} before calling any other \openshmem routine.
 }
 %API Description Table. 
{
 %Return Values     
\desR{ None. }
\notesB{ If any other \openshmem call occurs before \FUNC{start\_pes}/\FUNC{shmem\_init}, the behavior is undefined.
Although it is recommended to set \VAR{npes} to \CONST{0} for \FUNC{start\_pes}, this is not mandated.  The value is ignored.
Calling \FUNC{start\_pes}/\FUNC{shmem\_init} more than once has no subsequent effect.
}
}%end of DesB
% Notes. If there are no notes, this field can be left empty.
 \notesB{As of \openshmem Specification 1.2 the use of \FUNC{start\_pes} has been deprecated. Although \openshmem libraries are required to support the call, application developers are encouraged to use \FUNC{shmem\_init} instead.
}
%Example
\exampleB{
%For each example, you can enter it as an item.
                  \exampleITEMF
                  { This is a simple program that calls \FUNC{start\_pes}:}
                 {./EXAMPLES/shmem_startpes_example.f90}
 {} 
}  	
\eAPI 
