\bAPI{SHMEM\_PE\_ACCESSIBLE}{Determines whether a processing element (PE) is accessible via \openshmem's data transfer operations.}
\synC     
int shmem_pe_accessible(int pe);
%*\synCE    %DO NOT DELETE. THIS LINE IS NOT A COMMENT

\synF
LOGICAL LOG, SHMEM_PE_ACCESSIBLE
INTEGER pe
LOG = SHMEM_PE_ACCESSIBLE(pe)
%*\synFE   %DO NOT DELETE. THIS LINE IS NOT A COMMENT

\desB{
\argRow{IN}{pe}{Specific pe that needs to be checked if accessible from the local PE.}
}
{
       shmem\_pe\_accessible is  a  query function  that indicates  whether  a
       specified PE is accessible via \openshmem from the local PE. For example, on  SGI	Altix  series  systems, \openshmem	is  supported  across multiple
       partitioned hosts and InfiniBand connected hosts. When running multiple executable MPI applications using \openshmem on an Altix,
       full \openshmem support is available between processes running from the same
       executable file. However, \openshmem support between processes of different
       executable  files  is  supported only for data objects on the symmetric
       heap, since static data objects are  not symmetric  between  different
       executable  files. The shmem\_pe\_accessible function on Altix returns
       TRUE only if  the  remote  PE  is  a  process  running  from  the  same
       executable  file	 as  the  local PE, indicating that full \openshmem support
       (static memory and symmetric heap) is available.
}
{
\desR{C:	 The return value is 1 if the specified PE is a	 valid	remote
		 PE for \openshmem functions; otherwise,it is 0. \\ \\
	  Fortran:	 The  return  value  is .TRUE. if the specified PE is a valid
		 remote PE for \openshmem functions; otherwise, it is .FALSE..	 
		 }

\notesB{}
}

\eAPI