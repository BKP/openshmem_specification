\bAPI{SHMEM\_QUIET}{Waits  for  completion of all outstanding \PUT{} (remote writes) and \acp{AMO} issued by a \ac{PE}.}
\synC   %Synopisis for C API

void shmem_quiet(void);
%*\synCE    %DO NOT DELETE. THIS LINE IS NOT A COMMENT

\synF   %Synopsis for FORTRAN API

CALL SHMEM_QUIET
 %*\synFE   %DO NOT DELETE. THIS LINE IS NOT A COMMENT

% Arguments table. If no arguments you can use \argRow{None}{}{}
\desB{
	\argRow{None.}{}{}
	}
%API description
 { 
 The \FUNC{shmem\_quiet} routine ensures completion of \PUT{} operations (remote write), \ac{AMO} and load/store operations on symmetric data issued by the calling \ac{PE}.  
 All \PUT{} and \acp{AMO} issued by the local \ac{PE} to any \ac{PE} prior to the call to \FUNC{shmem\_quiet} are 
 guaranteed to be visible to all other \ac{PE}s when \FUNC{shmem\_quiet} returns.
 %SP: Removing confusing parts as according to SGI they are complete at the end of quiet.
 %no later than any subsequent memory load or 
 %store, \PUT{}  or  \GET{}, \acp{AMO},  or synchronization operations that follow the call to \FUNC{shmem\_quiet}.
}
 %API Description Table. 
{
 %Return Values
\desR{None.}
}%end of DesB

% Notes. If there are no notes, this field can be left empty.
\notesB{ 
       \FUNC{shmem\_quiet} is most useful as a way of ensuring delivery of
       several \PUT{} and \ac{AMO} operations. For example,  you might use \FUNC{shmem\_quiet} to
       await delivery of a block of data before issuing  another \PUT{}, which
       sets a completion flag on another \ac{PE}.

       \FUNC{shmem\_quiet} is not usually needed if \FUNC{shmem\_barrier\_all}  or
        \FUNC{shmem\_barrier} are called.  The barrier routines  all wait  for  the
       completion of outstanding remote writes (\PUT{} and \ac{AMO}).
}
%Example
\exampleB{
%For each example, you can enter it as an item.
        \exampleITEMF
	{The  following  simple  example uses  \FUNC{shmem\_quiet}  in a \Fortran{}
       program:}	 
       {./EXAMPLES/shmem_quiet_example.f90}
       {}
}
\eAPI

