\bAPI{SHMEM\_QUIET}{Waits  for  completion of all outstanding remote writes issued by a PE.}

\synC   %Synopisis for C API

void shmem_quiet(void);
%*\synCE    %DO NOT DELETE. THIS LINE IS NOT A COMMENT

\synF   %Synopsis for FORTRAN API

CALL SHMEM_QUIET
 %*\synFE   %DO NOT DELETE. THIS LINE IS NOT A COMMENT

% Arguments table. If no arguments you can use \argRow{NONE}{}{}
\desB{
	\argRow{None.}{}{}
	}
%API description
 { 
 The shmem\_quiet routine ensures ordering of put (remote write) operations.  All put operations  issued  to any \ac{PE} prior to the call to shmem\_quiet are guaranteed to be visible to all other \ac{PE}s no later than any   subsequent	  memory   load	 or  store,  remote  put  or  get,  or synchronization operations that follow the call to shmem\_quiet.
}
 %API Description Table. 
{
{}
 %Return Values
\desR{None.}
}%end of DesB

% Notes. If there are no notes, this field can be left empty.
\notesB{ 
       shmem\_quiet is most useful as a way of ensuring ordering of delivery of
       several put  operations. For example,  you might use shmem\_quiet to
       await delivery of a block of data before issuing  another  put, which
       sets a completion flag on another \ac{PE}.

       shmem\_quiet is not	usually needed  if   \FUNC{shmem\_barrier\_all}  or
        \FUNC{shmem\_barrier} are called.  The barrier routines  all wait  for  the
       completion of outstanding remote writes (puts).
}
%Example
\exampleB{
%For each example, you can enter it as an item.
        \exampleITEMF
	{The  following  simple  example uses  shmem\_quiet  in a Fortran
       program:}	 
       {./EXAMPLES/shmem_quiet_example.f90}
       {}
}
\eAPI

