\bAPI{SHMEM\_INC}{Performs an atomic fetch-and-increment  operation on a remote data object.}
\synC   %Synopisis for C API
 
 void shmem_int_inc(int *target, int pe);
 void shmem_long_inc(long *target, int pe);
 void shmem_longlong_inc(long long *target, int pe);
%*\synCE    %DO NOT DELETE. THIS LINE IS NOT A COMMENT

\synF   %Synopsis for FORTRAN API

 INTEGER pe
 INTEGER(KIND=4) target4
 INTEGER(KIND=8) target8
 CALL SHMEM_INT4_INC(target4, pe)
 CALL SHMEM_INT8_INC(target8, pe)
%*\synFE   %DO NOT DELETE. THIS LINE IS NOT A COMMENT  

% Arguments table. If no arguments you can use \argRow{NONE}{}{} 
\desB{  
   	\argRow{IN}{target}{The remotely accessible integer data object to be updated on
	 the remote PE. The type of target should match that implied
	 in the SYNOPSIS section. }
   \argRow{IN}{pe}{An integer that indicates the PE number on which target is to
	 be  updated. If you are using Fortran, it must be a default
	 integer value.}
 }
%API description
{
   These  functions	 perform  an  atomic increment operation on the target data object on processing element (PE) pe.
}
{
%API Description Table.
\desR{
    %Return Values     
    None.
}

% Notes. If there are no notes, this field can be left empty.
\notesB{
     The term remotely accessible is defined in the Introduction.
}
} % end of DesB

\exampleB{
       \exampleITEM
       { The following shmem\_int\_inc example is for C/C++ programs: }
       {./EXAMPLES/shmem_inc_example.c}
       {}
}
\eAPI
