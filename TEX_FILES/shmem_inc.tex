\bAPI{SHMEM\_INC}{Performs an atomic increment operation on a remote data object.}
\synC   %Synopisis for C API

void shmem_int_inc(int *target, int pe);
void shmem_long_inc(long *target, int pe);
void shmem_longlong_inc(long long *target, int pe); %*\synCE    %DO NOT DELETE. THIS LINE IS NOT A COMMENT

\synF   %Synopsis for FORTRAN API

INTEGER pe
CALL SHMEM_INT4_INC(target, pe)
CALL SHMEM_INT8_INC(target, pe) %*\synFE   %DO NOT DELETE. THIS LINE IS NOT A COMMENT  

% Arguments table. If no arguments you can use \argRow{None}{}{} 
\desB{  
   	\argRow{IN}{target}{The remotely accessible integer data object to be updated on
	 the remote \ac{PE}. The type of \target{} should match that implied
	 in the SYNOPSIS section.}
   \argRow{IN}{pe}{An integer that indicates the \ac{PE} number on which \target{} is to
	 be  updated. If you are using \Fortran{}, it must be a default
	 integer value.}
 }
%API description
{
   These  functions perform  an atomic increment operation on the \VAR{target} data object on \ac{PE}.
}
{
%API Description Table.
 \hfill \\
\desTB {If you are using \Fortran, \VAR{target} must be of the following type:}
{
\cRow{SHMEM\_INT4\_INC}{\CONST{4}-byte integer}
\cRow{SHMEM\_INT8\_INC}{\CONST{8}-byte integer}
}

\desR{
    %Return Values     
    None.
}
% Notes. If there are no notes, this field can be left empty.
\notesB{
     The term remotely accessible is defined in the Introduction.
}
} % end of DesB

\exampleB{
       \exampleITEM
       { The following \FUNC{shmem\_int\_inc} example is for \CorCpp{} programs: }
       {./EXAMPLES/shmem_inc_example.c}
       {}
}
\eAPI
