\bAPI{SHMEM\_COLLECT, SHMEM\_FCOLLECT}{Concatenates blocks of data from multiple \ac{PE}s to an array in every \ac{PE}.}
\label{subsec:shmem_collect}
\synC   %Synopisis for C API

void shmem_collect32(void *target, const void *source, size_t nelems, int PE_start, int logPE_stride, int PE_size, long *pSync);
void shmem_collect64(void *target, const void *source, size_t nelems, int PE_start, int logPE_stride, int PE_size, long *pSync);
void shmem_fcollect32(void *target, const void *source,	size_t nelems, int PE_start, int logPE_stride, int PE_size, long *pSync);
void shmem_fcollect64(void *target, const void *source,	size_t nelems, int PE_start, int logPE_stride, int PE_size, long *pSync); %*\synCE    %DO NOT DELETE. THIS LINE IS NOT A COMMENT
\synF   %Synopsis for FORTRAN API

INTEGER nelems
INTEGER PE_start, logPE_stride, PE_size
INTEGER pSync(SHMEM_COLLECT_SYNC_SIZE)
CALL SHMEM_COLLECT4(target, source, nelems, PE_start, logPE_stride, PE_size, pSync)
CALL SHMEM_COLLECT8(target, source, nelems, PE_start, logPE_stride, PE_size, pSync)
CALL SHMEM_COLLECT32(target, source, nelems, PE_start, logPE_stride, PE_size, pSync)
CALL SHMEM_COLLECT64(target, source, nelems, PE_start, logPE_stride, PE_size, pSync)
CALL SHMEM_FCOLLECT4(target, source, nelems, PE_start, logPE_stride, PE_size, pSync)
CALL SHMEM_FCOLLECT8(target, source, nelems, PE_start, logPE_stride, PE_size, pSync)
CALL SHMEM_FCOLLECT32(target, source, nelems, PE_start, logPE_stride, PE_size, pSync)
CALL SHMEM_FCOLLECT64(target, source, nelems, PE_start, logPE_stride, PE_size, pSync) %*\synFE   %DO NOT DELETE. THIS LINE IS NOT A COMMENT  

% Arguments table. If no arguments you can use \argRow{None}{}{} 
\desB{  
\argRow{OUT}{target}{A symmetric array.	The  target  argument  must  be	 large
	    enough to accept the concatenation of the source arrays on
	    all \ac{PE}s.  The data types are as follows:
	    For \FUNC{shmem\_collect8}, \FUNC{shmem\_collect64}, \FUNC{shmem\_fcollect8}, and
	    \FUNC{shmem\_fcollect64}, any data type with an element size of 64
	    bits.  \Fortran{} derived types, \Fortran{} character type,  and
	    \CorCpp{}  structures  are not permitted.  For \FUNC{shmem\_collect4},
	    \FUNC{shmem\_collect32}, \FUNC{shmem\_fcollect4}, and \FUNC{shmem\_fcollect32},
	    any	 data  type  with an element size of \CONST{32} bits.  \Fortran{}
	    derived  types,  \Fortran{}   character   type,   and \CorCpp{}
	    structures are not permitted.}
\argRow{IN}{source}{A symmetric   data  object  that	can  be of  any  type
	    permissible for the target argument.}
\argRow{IN}{nelems}{The number of elements in the source array.	 nelems must be
	    of	type  integer.	If you are using \Fortran, it must be a
	    default integer value.}
\argRow{IN}{PE\_start}{The lowest virtual \ac{PE} number of the \activeset{} of \ac{PE}s.
	    \VAR{PE\_start} must be of type integer. If you are using
	    \Fortran, it must be a default integer value.}
\argRow{IN}{logPE\_stride}{The log (base \CONST{2}) of the stride between consecutive virtual
	    \ac{PE}	numbers in the \activeset. \VAR{logPE\_stride} must be of
	    type integer.  If you are using  \Fortran,  it  must be  a
	    default integer value.}
\argRow{IN}{PE\_size}{The number of \ac{PE}s in the \activeset. \VAR{PE\_size} must be of
	    type integer.  If you are using  \Fortran, it must be a
	    default integer value.}
\argRow{IN}{pSync}{A symmetric  work array.  In \CorCpp, \VAR{pSync} must be of type
	    long and size \CONST{\_SHMEM\_COLLECT\_SYNC\_SIZE}.  In \Fortran, \VAR{pSync}
	    must  be of type integer and size \CONST{SHMEM\_COLLECT\_SYNC\_SIZE}.
	    If you are using \Fortran, it must  be  a  default  integer
	    value. Every  element  of this array must be initialized
	    with the value \CONST{\_SHMEM\_SYNC\_VALUE} in \CorCpp{} or
	    \CONST{SHMEM\_SYNC\_VALUE} in \Fortran{} before any of the \ac{PE}s in the
	    \activeset{} enter \FUNC{shmem\_barrier}.}
}
%API description
{   
\OSH{} collect and fcollect routines concatenate
\VAR{nelems} \CONST{64}-bit or \CONST{32}-bit data items from the source array into the target
array, over the set of \ac{PE}s defined by \VAR{PE\_start}, \VAR{log2PE\_stride}, and
\VAR{PE\_size}, in processor number order. The resultant target array
contains the contribution from \ac{PE} \VAR{PE\_start} first, then the contribution
from \ac{PE} \VAR{PE\_start} + \VAR{PE\_stride} second, and so on. The collected result
is written to the target array for all \ac{PE}s in the \activeset.

The \FUNC{fcollect} routines require that \VAR{nelems} be the same value in all
participating \ac{PE}s, while the collect routines allow \VAR{nelems} to vary from
\ac{PE} to \ac{PE}.

As with all \openshmem collective routines, each of these routines assumes
that only \ac{PE}s in the \activeset{} call the routine. If a \ac{PE} not in the
\activeset{} calls a \openshmem collective routine, undefined behavior
results.

The values of arguments \VAR{PE\_start}, \VAR{logPE\_stride}, and \VAR{PE\_size} must be
equal on all \ac{PE}s in the \activeset. The same target and source arrays
and the same \VAR{pSync} work array must be passed to all \ac{PE}s in the \activeset.

Upon return from a collective routine, the following are true for the
local \ac{PE}: The target array is updated. The values in the \VAR{pSync} array
are restored to the original values.
}
{
{
%Return Values     
\desR{None.}
}
% Notes. If there are no notes, this field can be left empty.
\notesB{
All \openshmem collective routines reset the values in \VAR{pSync} before they
return, so a particular \VAR{pSync} buffer need only be initialized the first
time it is used.

You  must ensure that the \VAR{pSync} array is not being updated on any \ac{PE} in
the \activeset{} while any of the \ac{PE}s participate in processing of a
\openshmem collective routine. Be careful to avoid these situations: If the
\VAR{pSync} array is initialized at run time, some type of synchronization is
needed to ensure that all \ac{PE}s in the working set have initialized \VAR{pSync}
before any of them  enter a \openshmem routine called with the \VAR{pSync}
synchronization	array. A \VAR{pSync} array can be reused on a subsequent
\openshmem collective routine only if none of the \ac{PE}s in the \activeset{}  are
still processing a  prior \openshmem collective routine call that used the
same \VAR{pSync} array. In general, this may be ensured only by doing some
type of synchronization. However, in the special case of SHMEM
routines being called with the same \activeset, you can allocate two
\VAR{pSync} arrays and alternate between them on successive calls.

The collective routines operate on active \ac{PE} sets that have a
non-power-of-two \VAR{PE\_size} with some performance degradation. They
operate with no performance degradation when \VAR{nelems} is a
non-power-of-two value.
}
}
%Example
\exampleB{
    \exampleITEM{The following \FUNC{shmem\_collec}t example is for \CorCpp{} programs:}
    {./EXAMPLES/shmem_collect_example.c}
    {}
    %For each example, you can enter it as an item.
    \exampleITEMF{The following \FUNC{SHMEM\_COLLECT} example is for \Fortran{} programs:}
    {./EXAMPLES/shmem_collect_example.f90}
    {}
}  	
\eAPI 

%%%%%%%%%%%%%%%%%%%%%%%%%%%%%%%%%%%%%%%%%%%%%%%%%%%%%%%%%%%%%%%%%%%%%%%%%%%%%%
%        Concatenates blocks of data from multiple processing
%        elements (PEs) to an array in every PE.
% 
% SYNOPSIS
%        C or C++:
% 
% 	  void shmem_collect32(void *target, const void *source, size_t nelems,
% 	  int PE_start, int logPE_stride, int PE_size, long *pSync);
% 
% 	  void shmem_collect64(void *target, const void *source, size_t nelems,
% 	  int PE_start, int logPE_stride, int PE_size, long *pSync);
% 
% 	  void	shmem_fcollect32(void  *target,	 const	void  *source,	size_t
% 	  nelems, int PE_start, int logPE_stride, int PE_size, long *pSync);
% 
% 	  void	shmem_fcollect64(void  *target,	 const	void  *source,	size_t
% 	  nelems, int PE_start, int logPE_stride, int PE_size, long *pSync);
% 
%        Fortran:
% 
% 	  INTEGER nelems
% 	  INTEGER PE_start, logPE_stride, PE_size
% 	  INTEGER pSync(SHMEM_COLLECT_SYNC_SIZE)
% 
% 	  CALL SHMEM_COLLECT4(target, source, nelems,  PE_start,	 logPE_stride,
% 	  PE_size, pSync)
% 
% 	  CALL	SHMEM_COLLECT8(target,	source, nelems, PE_start, logPE_stride,
% 	  PE_size, pSync)
% 
% 	  CALL SHMEM_COLLECT32(target, source, nelems, PE_start,	 logPE_stride,
% 	  PE_size, pSync)
% 
% 	  CALL	SHMEM_COLLECT64(target, source, nelems, PE_start, logPE_stride,
% 	  PE_size, pSync)
% 
% 	  CALL SHMEM_FCOLLECT4(target, source, nelems, PE_start,	 logPE_stride,
% 	  PE_size, pSync)
% 
% 	  CALL	SHMEM_FCOLLECT8(target, source, nelems, PE_start, logPE_stride,
% 	  PE_size, pSync)
% 
% 	  CALL SHMEM_FCOLLECT32(target, source, nelems, PE_start, logPE_stride,
% 	  PE_size, pSync)
% 
% 	  CALL SHMEM_FCOLLECT64(target, source, nelems, PE_start, logPE_stride,
% 	  PE_size, pSync)
% 
% DESCRIPTION
% 
% Arguments
% 
% 	OUT       target	    A symmetric array.	The  target  argument  must  be	 large
% 		    enough to accept the concatenation of the source arrays on
% 		    all PEs.  The data types are as follows:
% 		    For shmem_collect8, shmem_collect64, shmem_fcollect8,  and
% 		    shmem_fcollect64, any data type with an element size of 64
% 		    bits.  Fortran derived types, Fortran character type,  and
% 		    C/C++  structures  are not permitted.  For shmem_collect4,
% 		    shmem_collect32,  shmem_fcollect4,	and  shmem_fcollect32,
% 		    any	 data  type  with an element size of 32 bits.  Fortran
% 		    derived  types,  Fortran   character   type,   and	 C/C++
% 		    structures are not permitted.
% 
%        IN	source	    A	symmetric   data  object  that	can  be	 of  any  type
% 		    permissible for the target argument.
% 
%        IN	nelems	    The number of elements in the source array.	 nelems must be
% 		    of	type  integer.	If you are using Fortran, it must be a
% 		    default integer value.
% 
%        IN	PE_start	    The lowest virtual PE number of the	 active	 set  of  PEs.
% 		    PE_start  must  be	of  type  integer.   If	 you are using
% 		    Fortran, it must be a default integer value.
% 
%        IN	logPE_stride The log (base 2) of the stride between consecutive virtual
% 		    PE	numbers	 in  the  active set.  logPE_stride must be of
% 		    type integer.  If you are using  Fortran,  it  must	 be  a
% 		    default integer value.
% 
%        IN	PE_size	    The	 number	 of PEs in the active set.  PE_size must be of
% 		    type integer.  If you are using  Fortran,  it  must	 be  a
% 		    default integer value.
% 
%        IN	pSync	    A  symmetric  work array.  In C/C++, pSync must be of type
% 		    long and size _SHMEM_COLLECT_SYNC_SIZE.  In Fortran, pSync
% 		    must  be of type integer and size SHMEM_COLLECT_SYNC_SIZE.
% 		    If you are using Fortran, it must  be  a  default  integer
% 		    value.   Every  element  of this array must be initialized
% 		    with   the	 value	 _SHMEM_SYNC_VALUE   in	   C/C++    or
% 		    SHMEM_SYNC_VALUE  in  Fortran before any of the PEs in the
% 		    active set enter shmem_barrier().
% 
% API Description
% 
%        OpenSHMEM collect  and  fcollect	 routines  concatenate
%        nelems  64-bit or  32-bit data items from the source array into the target
%        array, over the set of PEs  defined  by	PE_start,  log2PE_stride,  and
%        PE_size,	 in  processor	number	order.	 The  resultant	 target	 array
%        contains the contribution from PE PE_start first, then the contribution
%        from  PE	 PE_start + PE_stride second, and so on.  The collected result
%        is written to the target array for all PEs in the active set.
% 
%        The fcollect routines require that nelems	 be  the  same	value  in  all
%        participating  PEs, while the collect routines allow nelems to vary from
%        PE to PE.
% 
%        As  with	 all OpenSHMEM collective routines, each of these routines assumes
%        that only PEs in the active set call the routine.  If a PE not  in  the
%        active  set  calls  a  OpenSHMEM  collective	 routine,  undefined  behavior
%        results.
% 
%        The values of arguments PE_start, logPE_stride,	and  PE_size  must  be
%        equal  on all PEs in the active set.  The same target and source arrays
%        and the same pSync work array must be passed to all PEs in  the	active
%        set.
% 
%        Upon  return  from a collective routine, the following are true for the
%        local PE: The target array is updated.  The values in the  pSync	 array
%        are restored to the original values.
% 
% Return Value
% 
% 	None.
% 
% NOTES
% 
%        All  OpenSHMEM  collective  routines	 reset the values in pSync before they
%        return, so a particular pSync buffer need only be initialized the first
%        time it is used.
% 
%        You  must ensure that the pSync array is not being updated on any PE in
%        the active set while any of the PEs  participate	 in  processing	 of  a
%        OpenSHMEM collective routine.  Be careful to avoid these situations: If the
%        pSync array is initialized at run time, some type of synchronization is
%        needed to ensure that all PEs in the working set have initialized pSync
%        before any of  them  enter  a  OpenSHMEM  routine  called  with  the	 pSync
%        synchronization	array.	 A  pSync  array can be reused on a subsequent
%        OpenSHMEM collective routine only if none of the PEs in the active set  are
%        still  processing  a  prior OpenSHMEM collective routine call that used the
%        same pSync array.  In general, this may be ensured only by  doing  some
%        type  of	 synchronization.   However,  in  the  special	case  of SHMEM
%        routines being called with the same active set, you  can	 allocate  two
%        pSync arrays and alternate between them on successive calls.
% 
%        The  collective	routines  operate  on  active  PE  sets	 that  have  a
%        non-power-of-two	 PE_size  with	some  performance  degradation.	  They
%        operate	 with	no   performance   degradation	 when	nelems	is   a
%        non-power-of-two value.
% 
% EXAMPLES
% 	\lstinputlisting[language=C]{shmem_collect_example.c}
% 
% 	\lstinputlisting[language=C]{shmem_collect_example.f90}
