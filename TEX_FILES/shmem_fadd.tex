\bAPI{SHMEM\_FADD}{Performs an atomic fetch-and-add operation on a remote data object.}
\synC   %Synopisis for C API

int shmem_int_fadd(int *target, int value, int pe);
long shmem_long_fadd(long *target, long value, int pe);
long long shmem_longlong_fadd(long long *target, long long value, int pe); %*\synCE    %DO NOT DELETE. THIS LINE IS NOT A COMMENT
\synF   %Synopsis for FORTRAN API

INTEGER pe
INTEGER*4 SHMEM_INT4_FADD, ires_i4, value_i4
ires_i4 = SHMEM_INT4_FADD(target_i4, value_i4, pe)
INTEGER*8 SHMEM_INT8_FADD, ires_i8, value_i8
ires_i8 = SHMEM_INT8_FADD(target_i8, value_i8, pe) %*\synFE   %DO NOT DELETE. THIS LINE IS NOT A COMMENT  

% Arguments table. If no arguments you can use \argRow{None}{}{} 
\desB{  
\argRow{OUT}{target}{The remotely accessible integer data object to be updated on
	 the remote \ac{PE}.	 The type of \VAR{target} should match that implied
	 in the SYNOPSIS section. For Fortran it should be a 4-byte integer for \FUNC{SHMEM\_INT4\_FADD} and a 8-byte integer for \FUNC{SHMEM\_INT8\_FADD}.}
\argRow{IN}{value}{The \VAR{value} to be atomically added to \VAR{target}.  The type of \VAR{value} should match that implied in the SYNOPSIS section.}
\argRow{IN}{pe}{An integer that indicates the \ac{PE} number on which \VAR{target} is to be updated.  If you are using \Fortran, it must be a default integer value.}
}
%API description
{
\FUNC{shmem\_fadd} functions perform an atomic fetch-and-add operation.  An
atomic fetch-and-add operation fetches the old \VAR{target} and adds \VAR{value} to
\VAR{target} without the possibility of another atomic operation on the \VAR{target}
between the time of the fetch and the update.  These routines add \VAR{value}
to \VAR{target} on \VAR{pe} and return the previous
contents of \VAR{target} as an atomic operation.
}
%API Description Table.
{
%Return Values     
\desR{The contents that had been at the \VAR{target} address on the remote \ac{PE} prior to the atomic addition operation.  The data type of return value is the same as the \VAR{target}.}
% Notes. If there are no notes, this field can be left empty.
\notesB{None.}
}

\exampleB{
       \exampleITEM
	{The following \FUNC{shmem\_fadd} example is for \CorCpp{} programs:}
        {./EXAMPLES/shmem_fadd_example.c}
        {}
}
\eAPI
%%%%%%%%%%%%%%%%%%%%%%%%%%%%%%%%%%%%%%%%%%%%%%%%%%%%%%%%%%%%%%%%%%%%%%%%%%
%       Performs an atomic fetch-and-add operation	 on  a
%       remote data object.
%
%SYNOPSIS
%
%       C or C++:
%
%	  int shmem_int_fadd(int *target, int value, int pe);
%
%	  long shmem_long_fadd(long *target, long value, int pe);
%
%	  long	long  shmem_longlong_fadd(long	long *target, long long value,
%	  int pe);
%
%       Fortran:
%
%	  INTEGER pe
%
%	  INTEGER(KIND=4) SHMEM_INT4_FADD, ires, target, value
%	  ires = SHMEM_INT4_FADD(target, value, pe)
%
%	  INTEGER(KIND=8) SHMEM_INT8_FADD, ires, target, value
%	  ires = SHMEM_INT8_FADD(target, value, pe)
%
%
%DESCRIPTION
%
%Arguments
%
%	OUT       target	 The  remotely accessible integer data object to be updated on
%		 the remote PE.	 The type of target should match that  implied
%		 in the SYNOPSIS section.
%
%       IN	value	 The  value  to	 be  atomically	 added to target.  The type of
%		 value should match that implied in the SYNOPSIS section.
%
%       IN	pe	 An integer that indicates the PE number on which target is to
%		 be  updated.	If you are using Fortran, it must be a default
%		 integer value.
%API Description
%
%       shmem_fadd functions perform an	atomic	fetch-and-add  operation.   An
%       atomic fetch-and-add operation fetches the old target and adds value to
%       target without the  possibility	of  another atomic operation on the target
%       between the time of the fetch and the update.  These routines add value
%       to target on  Processing	 Element  (PE)	pe  and	 return	 the  previous
%       contents of target as an atomic operation.
%
%Return Value
%
%       The contents that had been at the target address on the remote PE prior
%       to the atomic addition operation. The data type of return value is the same as the target.
%
%
%NOTES
%
%EXAMPLE
%
%	\lstinputlisting[language=C]{shmem_fadd_example.c}
