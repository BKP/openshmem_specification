\bAPI{SHMEM\_P}{Copies one data item to a remote \ac{PE}.}
\synC 
void shmem_char_p(char *addr, char value, int pe);
void shmem_short_p(short *addr, short value, int pe);
void shmem_int_p(int *addr, int value, int pe);
void shmem_long_p(long *addr, long value, int pe);
void shmem_longlong_p(long long *addr, long long value, int pe);
void shmem_float_p(float *addr, float value, int pe);
void shmem_double_p(double *addr, double value, int pe);
void shmem_longdouble_p(long double *addr, long double value, int pe); %*\synCE    %DO NOT DELETE. THIS LINE IS NOT A COMMENT

\desB{
   \argRow{IN}{addr}{The remotely accessible array element or scalar data object
		 which will receive the data on the remote \ac{PE}.}
   \argRow{IN}{value}{The value to be transferred to \VAR{addr} on the remote \ac{PE}.}
   \argRow{IN}{pe}{The number of the remote \ac{PE}.}
}
{     These routines provide a very low latency put capability for single
       elements of most basic types.

       As with \FUNC{shmem\_put}, these functions start the remote transfer and may
       return	before the   data  is delivered  to the  remote \ac{PE}.  Use
       \FUNC{shmem\_quiet} to force completion of all remote \PUT{} transfers.
}
{
\desR{None.}
\notesB{None.}
}
\exampleB {
\exampleITEM{The following simple example uses \FUNC{shmem\_double\_p}  in a \Clang{} program.}	 
    	               {./EXAMPLES/shmem_p_example.c}{}
	               }
\eAPI
