\bAPI{SHMEM\_GET}{Copies data from a specified \ac{PE}.}
\synC   %Synopisis for C API

void shmem_double_get(double *dest, const double  *source, size_t nelems, int pe);
void shmem_float_get(float *dest, const float *source, size_t nelems, int pe);
void shmem_get32(void *dest, const void *source, size_t  nelems,  int pe);
void shmem_get64(void  *dest, const void *source, size_t nelems, int pe);
void shmem_get128(void *dest, const void *source, size_t nelems,  int pe);
void shmem_getmem(void *dest, const void *source, size_t nelems, int pe);
void shmem_int_get(int *dest, const int *source, size_t  nelems,  int pe);
void shmem_long_get(long *dest, const long *source, size_t nelems, int pe);
void shmem_longdouble_get(long double *dest, const long double *source, size_t nelems, int pe);
void shmem_longlong_get(long long *dest, const long long *source, size_t nelems, int pe);
void shmem_short_get(short *dest, const short *source, size_t nelems, int pe); %*\synCE    %DO NOT DELETE. THIS LINE IS NOT A COMMENT
\synF   %Synopsis for FORTRAN API

INTEGER nelems, pe
CALL SHMEM_CHARACTER_GET(dest, source, nelems, pe)
CALL SHMEM_COMPLEX_GET(dest, source, nelems, pe)
CALL SHMEM_DOUBLE_GET(dest, source, nelems, pe)
CALL SHMEM_GET4(dest, source, nelems, pe)
CALL SHMEM_GET8(dest, source, nelems, pe)
CALL SHMEM_GET32(dest, source, nelems, pe)
CALL SHMEM_GET128(dest, source, nelems, pe)
CALL SHMEM_GETMEM(dest, source, nelems, pe)
CALL SHMEM_INTEGER_GET(dest, source, nelems, pe)
CALL SHMEM_LOGICAL_GET(dest, source, nelems, pe)
CALL SHMEM_REAL_GET(dest, source, nelems, pe) %*\synFE   %DO NOT DELETE. THIS LINE IS NOT A COMMENT  

% Arguments table. If no arguments you can use \argRow{None}{}{} 
\desB{  
	 \argRow{OUT}{dest}{Local data object to be updated.}
     \argRow{IN}{source}{Data object on the \ac{PE} identified by \VAR{pe} that contains the data to be copied.  This data object must be remotely accessible.}
     \argRow{IN}{nelems}{Number of elements in the \dest{} and \source{} arrays. \VAR{nelems} must be of type \VAR{size\_t} for \Clang. If you are using \Fortran, it must be a constant, variable, or array element of default integer type.}
       \argRow{IN}{pe}{\ac{PE}  number of the remote \ac{PE}.  \VAR{pe} must be of type integer. If you are using \Fortran, it must be a constant, variable, or array element of default integer type.}
 }
%API description
{
   The get routines provide a method for copying a
   contiguous symmetric data object from a different \ac{PE} to a contiguous data object
   on the local \ac{PE}.  The routines return after the data has been
   delivered to the \dest{} array on the local \ac{PE}. 
}
%API Description Table.
{
    \desTB{The  \dest{} and \source{} data objects must conform to typing constraints,
       which are as follows:}
       {
     \cRow{shmem\_getmem}{\Fortran: Any noncharacter type. \Clang: Any  data  type.
    nelems is scaled in bytes.}
      \cRow{ shmem\_get4, shmem\_get32}{Any  noncharacter type that has a storage
				     size equal to \CONST{32} bits.}
       \cRow{shmem\_get8, shmem\_get64}{Any noncharacter type that has a  storage
				     size equal to \CONST{64} bits.}
      \cRow{shmem\_get128}{Any  noncharacter type that has a storage
				     size equal to \CONST{128} bits.}
      \cRow{shmem\_short\_get}{Elements of type short.}
       \cRow{shmem\_int\_get}{Elements of type int.}
       \cRow{shmem\_long\_get}{Elements of type long.}
      \cRow{shmem\_longlong\_get}{Elements of type long long.}
      \cRow{shmem\_float\_get}{Elements of type float.}
      \cRow{shmem\_double\_get}{Elements of type double.}
      \cRow{shmem\_longdouble\_get}{Elements of type long double.}
      \cRow{SHMEM\_CHARACTER\_GET}{Elements of type character. \VAR{nelems} is the
				     number  of characters  to transfer. The
				     actual character lengths of the \source{}
				     and \dest{} variables are ignored.}
       \cRow{SHMEM\_COMPLEX\_GET}{Elements of type complex of default
       size.}
      \cRow{SHMEM\_DOUBLE\_GET}{\Fortran: Elements of type double precision.}
      \cRow{SHMEM\_INTEGER\_GET}{Elements of type integer.}
       \cRow{SHMEM\_LOGICAL\_GET}{Elements of type logical.}
       \cRow{SHMEM\_REAL\_GET}{Elements of type real.}
       }
    %Return Values    
      \desR{None.}
    \notesB{
        See Introduction for a definition of the term remotely accessible. If you are using \Fortran, data types must be of default size.  For
       example, a real variable must be declared as \CONST{REAL},  \CONST{REAL*4},  or
       \CONST{REAL(KIND=KIND(1.0))}.
    }
}
\exampleB{
       \exampleITEMF
        {Consider this example for \Fortran.}
        {./EXAMPLES/shmem_get_example.f90}
        {}
}
\eAPI
