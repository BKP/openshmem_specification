\bAPI{SHMEM\_BROADCAST}{Broadcasts a block of data from one \ac{PE} to one or more target \ac{PE}s.}
\synC   %Synopisis for C API

 void shmem_broadcast32(void *target, const void *source, size_t nlong, int PE_root, int PE_start, int logPE_stride, int PE_size, long *pSync);
 void shmem_broadcast64(void *target, const void *source, size_t nlong, int PE_root, int PE_start, int logPE_stride, int PE_size, long *pSync);
%*\synCE    %DO NOT DELETE. THIS LINE IS NOT A COMMENT
\synF   %Synopsis for FORTRAN API

 INTEGER nlong, PE_root, PE_start, logPE_stride, PE_size
 INTEGER pSync(SHMEM_BCAST_SYNC_SIZE)
 CALL SHMEM_BROADCAST4(target, source, nlong, PE_root, PE_start, logPE_stride, PE_size, fIpSync)
 CALL SHMEM_BROADCAST8(target, source, nlong, PE_root, PE_start, logPE_stride, PE_size, pSync)
 CALL SHMEM_BROADCAST32(target, source, nlong, PE_root, PE_start, logPE_stride, PE_size, pSync)
 CALL SHMEM_BROADCAST64(target, source, nlong, PE_root, PE_start, logPE_stride, PE_size, pSync)
%*\synFE   %DO NOT DELETE. THIS LINE IS NOT A COMMENT  

% Arguments table. If no arguments you can use \argRow{NONE}{}{} 
\desB{  
\argRow{OUT}{target}{A symmetric data object.} 
\argRow{IN}{source}{A symmetric data object that can be of any data type that is permissible for the target argument.}
\argRow{IN}{nlong}{The number of elements in source. For \FUNC{shmem\_broadcast32} and \FUNC{shmem\_broadcast4}, this is the number of 32-bit
		   halfwords. nlong must be of type integer. If you are using Fortran, it must be a default integer value.}
\argRow{IN}{PE\_root}{Zero-based ordinal of the \ac{PE}, with respect to the active
		   set,	from which the data is copied. Must be greater than
		   or equal to 0 and less than \VAR{PE\_size}. \VAR{PE\_root} must be of
		   type integer. If you are using Fortran, it must be a default integer value.}
\argRow{IN}{PE\_start}{The lowest virtual \ac{PE} number of the active set of \ac{PE}s.
		   \VAR{PE\_start} must be of type integer. If	you are using
		   Fortran, it must be a default integer value.}
\argRow{IN}{logPE\_stride}{
		   The log (base 2) of the stride between consecutive virtual
		   \ac{PE} numbers in the active set. \VAR{log\_PE\_stride} must be of
		   type integer. If you are using Fortran, it must be a
		   default integer value.}

\argRow{IN}{PE\_size}{
		   The number of \ac{PE}s in the active set. \VAR{PE\_size} must be of
		   type integer. If you are using Fortran, it must be a
		   default integer value.}

\argRow{IN}{pSync}{
		   A symmetric work array. In C/C++, pSync must be of type
		   long and size \CONST{\_SHMEM\_BCAST\_SYNC\_SIZE}. In Fortran, \VAR{pSync}
		   must be of type integer and size \CONST{SHMEM\_BCAST\_SYNC\_SIZE}.
		   Every element of this array must be initialized with the
		   value \CONST{\_SHMEM\_SYNC\_VALUE} (in C/C++) or \CONST{SHMEM\_SYNC\_VALUE} (in
		   Fortran) before any of the \ac{PE}s  in  the  active  set	 enter
		   \FUNC{shmem\_barrier}.}
}
%API description
{   
\OSH{} broadcast routines are collective routines.
They copy data object source on the processor specified by \VAR{PE\_root} and
store the values at target on the other \ac{PE}s specified by the triplet
\VAR{PE\_start}, \VAR{logPE\_stride}, \VAR{PE\_size}. The data is not copied to the target
area on the root \ac{PE}.

As with all \OSH{} collective routines, each of these routines assumes
that only \ac{PE}s in the active set call the routine. If a \ac{PE} not in the
active set calls a \OSH{} collective routine, undefined behavior
results.

The values of arguments \VAR{PE\_root}, \VAR{PE\_start}, \VAR{logPE\_stride}, and \VAR{PE\_size}
must be equal on all \ac{PE}s in the active set. The same target and source
data objects and the same pSync work array must be passed to all \ac{PE}s in
the active set.

Before any \ac{PE} calls a broadcast	routine, you must ensure that the
following conditions exist (synchronization via a barrier or some other
method is often needed to ensure this): The \VAR{pSync} array on all \ac{PE}s in
the active set is not still in use from a prior call to a broadcast
routine. The target array on all \ac{PE}s in the active set is ready to
accept the broadcast data.

Upon return from a broadcast routine, the following are true for the
local \ac{PE}: If the current \ac{PE} is not the root \ac{PE}, the target data	object
is updated. The values in the pSync array are restored to the original
values.
}
%API Description Tabl
{
\desTB { 
The  target  and source data  objects must conform to certain typing
constraints, which are as follows: } 
{
\cRow{shmem\_broadcast8, shmem\_broadcast64}{Any noncharacter type that has an
		      element size of 64 bits. No Fortran
		      derived types or C/C++ structures are
		      allowed.}
\cRow{shmem\_broadcast32}{Any noncharacter type that has an
		      element size of 32 bits. No Fortran
		      derived types or C/C++ structures are
		      allowed.}
\cRow{shmem\_broadcast4}{Any noncharacter type that has an
		      element size of 32 bits.}
}
%Return Values     
\desR{None.}

% Notes. If there are no notes, this field can be left empty.
\notesB{
       All \OSH{} broadcast routines restore pSync to its original contents.
       Multiple	calls to \OSH{} routines that use the same \VAR{pSync} array do not
       require that \VAR{pSync} be reinitialized after the first call.

       You must ensure the that the \VAR{pSync} array is not being updated by any \ac{PE}
       in the active set while any of the \ac{PE}s participates in processing of a
       \OSH{} broadcast routine. Be careful to avoid these situations: If the
       \VAR{pSync} array is initialized at run time, some type of synchronization is
       needed to ensure that all \ac{PE}s in the working set have initialized \VAR{pSync}
       before any of them enter a \OSH{} routine called with the \VAR{pSync}
       synchronization array. A \VAR{pSync} array may be reused on a subsequent
       \OSH{} broadcast routine only if none of the \ac{PE}s in the active set are
       still processing a prior \OSH{} broadcast routine call that used the
       same \VAR{pSync} array. In general, this can be ensured only by doing some
       type of synchronization. However, in the special case of \OSH{}
       routines being called with the same active set, you can allocate two
       \VAR{pSync} arrays and alternate between them on successive calls.
}
}
%Example
\exampleB{
    %For each example, you can enter it as an item.
    \exampleITEM
    {In the following examples, the call to \FUNC{shmem\_broadcast64} copies	source
    on \ac{PE} 4 to target on \ac{PE}s 5, 6, and 7. C/C+ example:}
    {./EXAMPLES/shmem_broadcast_example.c}
    {}
    \exampleITEMF
    {Fortran example:}
    {./EXAMPLES/shmem_broadcast_example.f90}
    {}
}  	
\eAPI 

%%%%%%%%%%%%%%%%%%%%%%%%%%%%%%%%%%%%%%%%%%%%%%%%%%%%%%%%%%%%%%%%%%%%%%%%%%%%%%
% 
% Broadcasts a block  of  data from  one  processing element (PE) to one or more target PEs.
% 
% SYNOPSIS
%        C or C++:
% 
% 	  void	shmem_broadcast32(void	*target,  const	 void  *source, size_t
% 	  nlong, int PE_root, int PE_start,  int  logPE_stride,	 int  PE_size,
% 	  long *pSync);
% 
% 	  void	shmem_broadcast64(void	*target,  const	 void  *source, size_t
% 	  nlong, int PE_root, int PE_start,  int  logPE_stride,	 int  PE_size,
% 	  long *pSync);
% 
%        Fortran:
% 
% 	  INTEGER nlong, PE_root, PE_start, logPE_stride, PE_size
% 
% 	  INTEGER pSync(SHMEM_BCAST_SYNC_SIZE)
% 
% 	  CALL	SHMEM_BROADCAST4(target,  source,  nlong,  PE_root,  PE_start,
% 	  logPE_stride, PE_size, fIpSync)
% 
% 	  CALL	SHMEM_BROADCAST8(target,  source,  nlong,  PE_root,  PE_start,
% 	  logPE_stride, PE_size, pSync)
% 
% 	  CALL	SHMEM_BROADCAST32(target,  source,  nlong,  PE_root, PE_start,
% 	  logPE_stride, PE_size, pSync)
% 
% 	  CALL SHMEM_BROADCAST64(target,  source,  nlong,  PE_root,  PE_start,
% 	  logPE_stride, PE_size, pSync)
% 
% DESCRIPTION
% 
% Arguments
% 
%        OUT	target	   A symmetric data object with	 one  of  the  following  data
% 			   types:
% 
% 			   Routine		 Data Type and Language
% 
% 			   shmem_broadcast8, 
% 			   shmem_broadcast64	 Any  noncharacter  type  that	has an
% 						 element size of 64 bits.  No  Fortran
% 						 derived types or C/C++ structures are
% 						 allowed.
% 
% 			   shmem_broadcast32	 Any noncharacter  type	 that  has  an
% 						 element  size of 32 bits.  No Fortran
% 						 derived types or C/C++ structures are
% 						 allowed.
% 
% 			   shmem_broadcast4	 Any  noncharacter  type  that	has an
% 						 element size of 32 bits.
% 
%        IN	source	   A symmetric data object that can be of any data  type  that
% 			   is permissible for the target argument.
% 
%        IN	nlong	   The	number	of  elements in source.	 For shmem_broadcast32
% 		   and	shmem_broadcast4,  this	 is  the  number   of	32-bit
% 		   halfwords.	nlong  must  be	 of  type integer.  If you are
% 		   using Fortran, it must be a default integer value.
% 
% 	IN       PE_root	   Zero-based ordinal of the PE, with respect  to  the	active
% 		   set,	 from  which the data is copied.  Must be greater than
% 		   or equal to 0 and less than PE_size.	 PE_root  must	be  of
% 		   type	 integer.   If	you  are  using	 Fortran, it must be a
% 		   default integer value.
% 
%        IN	PE_start	   The lowest virtual PE number of  the	 active	 set  of  PEs.
% 		   PE_start  must  be  of  type	 integer.   If	you  are using
% 		   Fortran, it must be a default integer value.
% 
%        IN	logPE_stride
% 		   The log (base 2) of the stride between consecutive  virtual
% 		   PE  numbers	in  the	 active set.  log_PE_stride must be of
% 		   type integer.  If you are  using  Fortran,  it  must	 be  a
% 		   default integer value.
% 
%        IN	PE_size	   The	number	of  PEs in the active set.  PE_size must be of
% 		   type integer.  If you are  using  Fortran,  it  must	 be  a
% 		   default integer value.
% 
%        IN	pSync	   A  symmetric	 work  array.  In C/C++, pSync must be of type
% 		   long and size _SHMEM_BCAST_SYNC_SIZE.   In  Fortran,	 pSync
% 		   must	 be  of	 type  integer and size SHMEM_BCAST_SYNC_SIZE.
% 		   Every element of this array must be	initialized  with  the
% 		   value  _SHMEM_SYNC_VALUE (in C/C++) or SHMEM_SYNC_VALUE (in
% 		   Fortran) before any of the PEs  in  the  active  set	 enter
% 		   shmem_barrier().
% 
% API Description
% 
%        OpenSHMEM broadcast routines are collective routines.
%        They copy data object source on the processor specified by PE_root  and
%        store  the  values  at target on the other PEs specified by the triplet
%        PE_start, logPE_stride, PE_size.	 The data is not copied to the	target
%        area on the root PE.
% 
%        As  with	 all OpenSHMEM collective routines, each of these routines assumes
%        that only PEs in the active set call the routine.  If a PE not  in  the
%        active  set  calls  a  OpenSHMEM  collective	 routine,  undefined  behavior
%        results.
% 
%        The  values  of	arguments PE_root, PE_start, logPE_stride, and PE_size
%        must be equal on all PEs in the active set.  The same target and source
%        data objects and the same pSync work array must be passed to all PEs in
%        the active set.
% 
%        Before any PE calls a broadcast	routine,  you  must  ensure  that  the
%        following conditions exist (synchronization via a barrier or some other
%        method is often needed to ensure this): The pSync array on all  PEs  in
%        the  active  set	 is  not still in use from a prior call to a broadcast
%        routine.	 The target array on all PEs in the active  set	 is  ready  to
%        accept the broadcast data.
% 
%        Upon  return  from  a broadcast routine, the following are true for the
%        local PE: If the current PE is not the root PE, the target data	object
%        is updated.  The values in the pSync array are restored to the original
%        values.
% 
% Return Value
%        
% 	None.
% 
% NOTES
% 
%        All OpenSHMEM broadcast routines restore pSync to  its  original  contents.
%        Multiple	 calls	to OpenSHMEM routines that use the same pSync array do not
%        require that pSync be reinitialized after the first call.
% 
%        You must ensure the that the pSync array is not being updated by any PE
%        in  the active set while any of the PEs participates in processing of a
%        OpenSHMEM broadcast routine.	 Be careful to avoid these situations: If  the
%        pSync array is initialized at run time, some type of synchronization is
%        needed to ensure that all PEs in the working set have initialized pSync
%        before  any  of	them  enter  a	SHMEM  routine	called	with the pSync
%        synchronization array.  A pSync array may be  reused  on	 a  subsequent
%        OpenSHMEM  broadcast	 routine only if none of the PEs in the active set are
%        still processing a prior OpenSHMEM broadcast routine	 call  that  used  the
%        same  pSync  array.  In general, this can be ensured only by doing some
%        type of	synchronization.   However,  in	 the  special  case  of	 SHMEM
%        routines	 being	called	with the same active set, you can allocate two
%        pSync arrays and alternate between them on successive calls.
% 
% EXAMPLES
% 
%        In the following examples, the call to shmem_broadcast64 copies	source
%        on PE 4 to target on PEs 5, 6, and 7.
% 
%        \lstinputlisting[language=C]{shmem_broadcast_example.c}
% 
%        \lstinputlisting[language=C]{shmem_broadcast_example.f90}
