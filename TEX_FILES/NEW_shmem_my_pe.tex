\bAPI{SHMEM\_MY\_PE}{Returns the number of the calling \ac{PE}.}

%Synopsis C
\synC
int shmem_my_pe(void);
int _my_pe (void);
%*\synCE

%Synopsis F
\synF 
INTEGER SHMEM_MY_PE, ME
MYPE = SHMEM_MY_PE()
ME = MY_PE ()
%*\synFE

%DESCRIPTION

%Arguments
\desB{
	\argRow{NONE}{}{}
}
%API Description
{
	This function returns the processing element (\ac{PE}) number of the calling
  \ac{PE}.   It accepts no arguments.	The result is an integer between \CONST{0} and
  \VAR{npes} - \CONST{1}, where \VAR{npes} is the total number of \ac{PE}s executing  the  current
  program.
}
%API Description Table.
{
%	\desTB{ }
%	{
%				\cRow{}{}
%	}

	%Return Value       
  \desR{Integer - Between \CONST{0} and \VAR{npes} - \CONST{1}}

	%NOTES
	\notesB{
		For \openshmem Specification 1.1 the use of \FUNC{\_my\_pe} has been deprecated. Although \openshmem libraries are required to support the call, application developers are encouraged to use \FUNC{shmem\_my\_pe} instead.
     }
}
%EXAMPLES
\exampleB{
		\exampleITEM
       {The following \FUNC{shmem\_my\_pe} example is for C/C++ programs:}
			 {./EXAMPLES/shmem_mype_example.c}
			 {}
}
\eAPI
