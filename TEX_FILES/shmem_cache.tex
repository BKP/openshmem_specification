\bAPI{SHMEM\_CACHE}{Controls data cache utilities.}
\synC   %Synopisis for C API
void shmem_clear_cache_inv(void);
void shmem_set_cache_inv(void);
void shmem_clear_cache_line_inv(void *target);
void shmem_set_cache_line_inv(void *target);
void shmem_udcflush(void);
void shmem_udcflush_line(void *target);
%*\synCE    %DO NOT DELETE. THIS LINE IS NOT A COMMENT
\synF   %Synopsis for FORTRAN API
CALL SHMEM_CLEAR_CACHE_INV
CALL SHMEM_SET_CACHE_INV
CALL SHMEM_SET_CACHE_LINE_INV(target)
CALL SHMEM_UDCFLUSH
CALL SHMEM_UDCFLUSH_LINE(target)
%*\synFE   %DO NOT DELETE. THIS LINE IS NOT A COMMENT  

% Arguments table. If no arguments you can use \argRow{NONE}{}{} 
\desB{  
\argRow{IN}{target}{A data object that is local to the processing element (PE).
		 target can be of any noncharacter type. If you are using
		 Fortran, it can be of any kind.}
}
%API description
{   
shmem\_set\_cache\_inv enables automatic cache coherency mode.

shmem\_set\_cache\_line\_inv enables automatic cache coherency mode for the
cache line associated with the address of target only.

shmem\_clear\_cache\_inv disables automatic cache coherency mode
previously enabled by shmem\_set\_cache\_inv or shmem\_set\_cache\_line\_inv.

shmem\_udcflush makes the entire user data cache coherent.

shmem\_udcflush\_line makes coherent the cache line that corresponds with
the address specified by target.
}
%API Description Tabl
{
\desTB {}{}
    %Return Values     
    {None.}
}
% Notes. If there are no notes, this field can be left empty.
{
These routines have been retained for improved backward compatability
with legacy architectures. They are not required to be supported and where 
provided they may have no effect on cacheline states.
}
%Example
\exampleB{
    %For each example, you can enter it as an item.
None.
}  	
\eAPI 

%%%%%%%%%%%%%%%%%%%%%%%%%%%%%%%%%%%%%%%%%%%%%%%%%%%%%%%%%%%%%%%%%%%%%%%%%%%%%%
%        Controls data cache utilities.
% 
% SYNOPSIS
%        C or C++:
% 
% 	  void shmem_clear_cache_inv(void);
% 
% 	  void shmem_set_cache_inv(void);
% 
% 	  void shmem_clear_cache_line_inv(void *target);
% 
% 	  void shmem_set_cache_line_inv(void *target);
% 
% 	  void shmem_udcflush(void);
% 
% 	  void shmem_udcflush_line(void *target);
% 
%        Fortran:
% 
% 	  CALL SHMEM_CLEAR_CACHE_INV
% 
% 	  CALL SHMEM_SET_CACHE_INV
% 
% 	  CALL SHMEM_SET_CACHE_LINE_INV(target)
% 
% 	  CALL SHMEM_UDCFLUSH
% 
% 	  CALL SHMEM_UDCFLUSH_LINE(target)
% 
% DESCRIPTION
% 
% Arguments
% 
% 	IN       target	 A  data  object that is local to the processing element (PE).
% 		 target can be of any noncharacter type.   If  you  are	 using
% 		 Fortran, it can be of any kind.
% 
% API Description
% 
%        shmem_set_cache_inv enables automatic cache coherency mode.
% 
%        shmem_set_cache_line_inv enables automatic cache coherency mode for the
%        cache line associated with the address of target only.
% 
%        shmem_clear_cache_inv   disables	  automatic   cache   coherency	  mode
%        previously  enabled by shmem_set_cache_inv or shmem_set_cache_line_inv.
% 
%        shmem_udcflush makes the entire user data cache coherent.
% 
%        shmem_udcflush_line makes coherent the cache line that corresponds with
%        the address specified by target.
% 
% Return Value
% 
% 	None.
% 
% NOTES
%        These  routines	have been retained for improved backward compatability
%        with legacy architectures.  They are not	 required to be supported and where 
%        provided they may have no effect on cacheline states.
% 
