\bAPI{SHPDEALLC}{Returns a memory block to the symmetric heap.}
\synF   %Synopsis for FORTRAN API
       POINTER (addr, A(1))
       INTEGER errcode, abort
       CALL SHPDEALLC(addr, errcode, abort)
 %*\synFE   %DO NOT DELETE. THIS LINE IS NOT A COMMENT  

% Arguments table. If no arguments you can use \argRow{NONE}{}{} 
\desB{  
       \argRow{IN}{addr}{ First word address of the block to deallocate.}
       \argRow{OUT}{errcode}{Error  code is 0 if no error was detected; otherwise, it is a  negative  integer  code  for  the  type  of	 error.}
       \argRow{IN}{abort}{Abort code.  Nonzero requests abort on error; 0 requests
		      an error code.}
 }
%API description
 {
       SHPDEALLC  returns  a block of memory (allocated using SHPALLOC) to the
       list of available space in the symmetric heap.  To  maintain  symmetric
       heap  consistency, all PEs in a program must call
       SHPDEALLC with the same value of addr; if  any  PEs  are	 missing,  the
       program hangs.
  }
 %Return Values     
\desR{Error conditions are as follows: }
%       \begin{tabular}{p{2cm} p{6cm}}
%       \textbf{Code}&\textbf{Condition}
%       \end{tabular}\\
%       \begin{tabular}{p{2cm} p{6cm}}
%       {-3}	&      {Address is outside the bounds of the symmetric heap.}
%      \end{tabular}\\
%       \begin{tabular}{p{2cm} p{6cm}}
%       {       -4}	 &     {Block is already free.}
%       \end{tabular}\\
%       \begin{tabular}{p{2cm} p{6cm}}
%       {-5}	  &    {Address is not at the beginning of the block.}
%       \end{tabular}\\
\notesB{   }
} %end of DesB
%Example
\exampleB{}
\eAPI 