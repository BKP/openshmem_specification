\bAPI{SHPDEALLC}{Returns a memory block to the symmetric heap.}
\synF   %Synopsis for FORTRAN API

POINTER (addr, A(1))
INTEGER errcode, abort
CALL SHPDEALLC(addr, errcode, abort) %*\synFE   %DO NOT DELETE. THIS LINE IS NOT A COMMENT  

% Arguments table. If no arguments you can use \argRow{None}{}{} 
\desB{  
       \argRow{IN}{addr}{ First word address of the block to deallocate.}
       \argRow{OUT}{errcode}{Error  code is 0 if no error was detected; otherwise, it is a  negative  integer  code  for  the  type  of error.}
       \argRow{IN}{abort}{Abort code.  Nonzero requests abort on error; \CONST{0} requests
		      an error code.}
 }
%API description
 {
       SHPDEALLC  returns  a block of memory (allocated using \FUNC{SHPALLOC}) to the
       list of available space in the symmetric heap.  To  maintain  symmetric
       heap  consistency, all \ac{PE}s in a program must call
       \FUNC{SHPDEALLC} with the same value of \VAR{addr}; if  any \ac{PE}s  are missing, the
       program hangs.
  }
%API Description Table. 
{
 %Return Values     
\desR{ 
\cRow{Error Code} {Condition}
\cRow{ \CONST{-1} } {Length is not an integer greater than 0}
\cRow{\CONST{-2}} { No more memory is available from the system (checked if the  request  cannot  be	satisfied from the available blocks on the symmetric heap).}
\cRow{\CONST{-3}}{	 Address is outside the bounds of the symmetric heap.}
\cRow{\CONST{-4}}{Block is already free.}
\cRow{\CONST{-5}}{Address is not at the beginning of a block.}
}%end of DesB
\notesB{None.}
}   
\eAPI 
