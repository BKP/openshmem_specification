\bAPI{SHMEM\_FENCE}{Assures ordering of delivery of \PUT{}, \acp{AMO}, and store operations to symmetric data objects.}
\synC   %Synopisis for C API

void shmem_fence(void); %*\synCE    %DO NOT DELETE. THIS LINE IS NOT A COMMENT

\synF   %Synopsis for FORTRAN API

CALL SHMEM_FENCE %*\synFE   %DO NOT DELETE. THIS LINE IS NOT A COMMENT  

% Arguments table. If no arguments you can use \argRow{None}{}{} 
\desB{  
       \argRow{None.}{}{}
 }
%API description
{
This function assures ordering of delivery of \PUT{}, \acp{AMO}, and store operations to symmetric data objects.
%This  function ensures ordering of \PUT{}, \acp{AMO} and store operations on symmetric data objects. 
All \PUT{}, \acp{AMO}, and store operations to symmetric data objects issued to a particular remote \ac{PE} prior to the call to \FUNC{shmem\_fence} are guaranteed to be ordered to be delivered before any subsequent \PUT{}, \acp{AMO}, and store operations to symmetric data objects to the same \ac{PE}.
% which follow the  call  to
% \FUNC{shmem\_fence}.
}
{
%API Description Table.
\desR{
    %Return Values     
    None.
}
% Notes. If there are no notes, this field can be left empty.
\notesB{
 \FUNC{shmem\_fence} only provides per-\ac{PE} ordering guarantees and does not guarantee completion of delivery.  There is a subtle difference between  \FUNC{shmem\_fence} and \FUNC{shmem\_quiet}, in that, \FUNC{shmem\_quiet} guarantees completion of \PUT{}, \acp{AMO}, and memory store operations to symmetric data objects which makes the updates visible to all other \acp{PE}. 
 
 The \FUNC{shmem\_quiet} function should be called if completion of PUT{}, \acp{AMO}, and store operations to symmetric data objects is desired when multiple remote \ac{PE}s are involved.
}
} % end of DesB

\exampleB{
       \exampleITEM
       { The following \FUNC{shmem\_fence} example is for \CorCpp{} programs: }
       {./EXAMPLES/shmem_fence_example.c}
       {\VAR{Put1} will be ordered to be delivered before \VAR{put3} and \VAR{put2} will be ordered to be delivered before
       \VAR{put4}.}
}
\eAPI
