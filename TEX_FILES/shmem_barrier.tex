\bAPI{SHMEM\_BARRIER}{Performs all operations described in the \FUNC{shmem\_barrier\_all} interface but with respect to a subset of \acp{PE} defined by the \activeset{}.} %Eric: Wish list: Add a ~/ref{} to barrier_all API.
\synC
void shmem_barrier(int PE_start, int logPE_stride, int PE_size, long *pSync); %*\synCE
\synF
INTEGER PE_start, logPE_stride, PE_size
INTEGER pSync(SHMEM_BARRIER_SYNC_SIZE)
CALL SHMEM_BARRIER(PE_start, logPE_stride, PE_size, pSync) %*\synFE

%DESCRIPTION

%Arguments
\desB{
 	\argRow{IN}{PE\_start}{The lowest virtual \ac{PE} number of  the \activeset{}  of  \ac{PE}s.
				   \VAR{PE\_start}  must  be  of  type	 integer.   If you  are using
		   		   \Fortran, it must be a default integer value.}
	\argRow{IN}{logPE\_stride}{The log (base 2) of the stride between consecutive  virtual
			   	   \ac{PE} numbers in the \activeset.  \VAR{logPE\_stride} must be of type
		   		   integer.  If you are using \Fortran, it must	be  a  default
		   		   integer value.}
	\argRow{IN}{PE\_size}{The number of  \ac{PE}s in the \activeset.  \VAR{PE\_size} must be of
				   type integer.  If you are  using  \Fortran, it must be a default integer value.}
	\argRow{IN}{pSync}{	A  symmetric work  array.  In \CorCpp, \VAR{pSync} must be of type
						long and size \CONST{\_SHMEM\_BARRIER\_SYNC\_SIZE}.  In  \Fortran, \VAR{pSync}
						must	 be  of type integer and size \CONST{SHMEM\_BARRIER\_SYNC\_SIZE}.
						If you are using \Fortran, it	 must  be  a  default  integer
						type.  Every element of this array must be initialized to 0
						before any of the \ac{PE}s in the \activeset{} enter shmem\_barrier
						the first time.}
} 
%API Description
{
       \FUNC{shmem\_barrier} is a collective synchronization routine over an \activeset.  Control returns
       from \FUNC{shmem\_barrier} after all  \ac{PE}s  in  the  \activeset{} (specified  by
       \VAR{PE\_start}, \VAR{logPE\_stride}, and \VAR{PE\_size}) have called \FUNC{shmem\_barrier}.

       As  with all \openshmem collective routines, each of these routines assumes
       that only \ac{PE}s in the \activeset{} call the routine.  If a \ac{PE} not  in  the
       \activeset{}  calls a \openshmem collective	 routine,  undefined  behavior
       results.

       The  values of arguments  \VAR{PE\_start}, \VAR{logPE\_stride}, and \VAR{PE\_size} must be
       equal on all \ac{PE}s in the \activeset.  The same work array must be passed
       in \VAR{pSync} to all \ac{PE}s in the \activeset.

       \FUNC{shmem\_barrier}  ensures  that  all  previously  issued  stores and
       remote memory updates, including \acp{AMO} and \ac{RMA} operations, 
       done by any of the \ac{PE}s  in  the \activeset{} are complete before returning.
%    For  example, \FUNC{shmem\_put} and \FUNC{shmem\_int\_add}  

       The  same  \VAR{pSync} array may be reused on consecutive calls   to
       \FUNC{shmem\_barrier} if the same active \ac{PE} set is used.
}
%API Description Table
{
		%Return Values
		\desR{None.}
	
		%NOTES
		\notesB{
					 If the \VAR{pSync} array is initialized at run time, be sure to use some type
					 of synchronization, for example, a call to \FUNC{shmem\_barrier\_all}, before
					 calling \FUNC{shmem\_barrier} for the first time.

					 If  the \activeset{}  does  not change, \FUNC{shmem\_barrier} can  be called
					 repeatedly with the same \VAR{pSync} array.   No  additional  synchronization
					 beyond  that implied by \FUNC{shmem\_barrier} itself is necessary in this case.
		}
}
%EXAMPLES
\exampleB{
	\exampleITEM
	{The following barrier example is for \CorCpp{} programs:}
	{./EXAMPLES/shmem_barrier_example.c}
	{}
}

\eAPI
