Performs a barrier operation on a subset of processing
       elements (PEs)

SYNOPSIS
       C or C++:

	  void shmem_barrier(int PE_start, int logPE_stride, int PE_size, long
	  *pSync);

       Fortran:

	  INTEGER PE_start, logPE_stride, PE_size
	  INTEGER pSync(SHMEM_BARRIER_SYNC_SIZE)

	  CALL SHMEM_BARRIER(PE_start, logPE_stride, PE_size, pSync)

DESCRIPTION

Arguments

 	IN      PE_start	   The lowest virtual PE number of  the	 active	 set  of  PEs.
				   PE_start  must  be  of  type	 integer.   If	you  are using
		   		   Fortran, it must be a default integer value.

       IN	logPE_stride       The log (base 2) of the stride between consecutive  virtual
			   	   PE numbers in the active set.  logPE_stride must be of type
		   		   integer.  If you are using Fortran, it must	be  a  default
		   		   integer value.

	IN       PE_size	   The	number	of  PEs in the active set.  PE_size must be of
				   type integer.  If you are  using  Fortran,  it  must	 be  a
		  		   default integer value.

	IN       pSync		   A  symmetric	 work  array.  In C/C++, pSync must be of type
				   int and size _SHMEM_BARRIER_SYNC_SIZE.  In  Fortran,	 pSync
				   must	 be  of type integer and size SHMEM_BARRIER_SYNC_SIZE.
		  		   If you are using Fortran, it	 must  be  a  default  integer
		  		   type.  Every element of this array must be initialized to 0
		  		   before any of the PEs in the active set enter shmem_barrier
		  		   the first time.
   

API Description

       shmem_barrier is a collective synchronization routine.  Control returns
       from shmem_barrier after all  PEs  in  the  active  set	(specified  by
       PE_start, logPE_stride, and PE_size) have called shmem_barrier.

       As  with	 all OpenSHMEM collective routines, each of these routines assumes
       that only PEs in the active set call the routine.  If a PE not  in  the
       active  set  calls  a  OpenSHMEM  collective	 routine,  undefined  behavior
       results.

       The  values  of	arguments  PE_start, logPE_stride, and PE_size must be
       equal on all PEs in the active set.  The same work array must be passed
       in pSync to all PEs in the active set.

       shmem_barrier  ensures  that  all  previously  issued  local stores and
       previously issued remote memory updates done by any of the PEs  in  the
       active  set  (by	 using	OpenSHMEM  calls,  for  example  shmem_put) are
       complete before returning.

       The  same  pSync	 array	may  be	 reused	 on   consecutive   calls   to
       shmem_barrier if the same active PE set is used.

Return Values
  
	None.

NOTES

       If the pSync array is initialized at run time, be sure to use some type
       of synchronization, for example, a call to shmem_barrier_all(), before
       calling shmem_barrier for the first time.

       If  the	active	set  does  not	change,	 shmem_barrier	can  be called
       repeatedly with the same pSync array.   No  additional  synchronization
       beyond  that implied by shmem_barrier itself is necessary in this case.

EXAMPLES
	The following barrier example is for C/C++ programs:

       \lstinputlisting[language=C]{shmem_barrier_example.c}

